% !TEX TS-program = pdflatex
% !TeX program = pdflatex
% !TEX encoding = UTF-8
% !TEX spellcheck = fr

\documentclass[xcolor=table, usenames,dvipsnames]{beamer}


%\usepackage{fullpage}
%\usepackage[left=2.8cm,right=2.2cm,top=2 cm,bottom=2 cm]{geometry}
\setbeamersize{text margin left=10pt,text margin right=10pt}
\usepackage{amsmath,amssymb} 
\usepackage[T1]{fontenc}
\usepackage[utf8]{inputenc}
\usepackage[english,french]{babel}
\usepackage{txfonts}
\usepackage[]{graphicx}
\usepackage{multirow}
\usepackage{hyperref}
\usepackage{colortbl}
\usepackage{listings}
\usepackage{wrapfig}
\usepackage{multicol}

\hypersetup{
	colorlinks,
	urlcolor = blue
}

%\renewcommand{\baselinestretch}{1.5}

\def\supit#1{\raisebox{0.8ex}{\small\it #1}\hspace{0.05em}}

\AtBeginSection{%
	\begin{frame}
		\sectionpage
	\end{frame}
}

\newcommand{\rottext}[2]{%
	\rotatebox{90}{%
	\begin{minipage}{#1}%
		\raggedleft#2%
	\end{minipage}%
	}%
}

\usepackage{longtable}
\usepackage{tabu}


\institute{ %
École  nationale Supérieure d'Informatique (ESI, ex. INI), Algérie
}
\author[ \textbf{\footnotesize  \insertframenumber/\inserttotalframenumber} \hspace*{\fill} ESI (2019-2020)] %
{ARIES Abdelkrime}
%\titlegraphic{\includegraphics[height=1cm]{../img/esi-logo.png}%\hspace*{4.75cm}~


\date{Année unniversitaire: 2019/2020} %\today

\usetheme{Warsaw} % Antibes Boadilla Warsaw

\beamertemplatenavigationsymbolsempty

%\setbeamertemplate{headline}{}

\definecolor{lightblue}{HTML}{D0D2FF}
\definecolor{lightyellow}{HTML}{FFFFAA}
\definecolor{darkblue}{HTML}{0000BB}
\definecolor{olivegreen}{HTML}{006600}
\definecolor{violet}{HTML}{6600CC}

\newcommand{\keyword}[1]{\textcolor{red}{\bfseries\itshape #1}}
\newcommand{\expword}[1]{\textcolor{olivegreen}{#1}}
\newcommand{\optword}[1]{\textcolor{violet}{\bfseries #1}}

\makeatletter
\newcommand\mysphere{%
	\parbox[t]{10pt}{\raisebox{0.2pt}{\beamer@usesphere{item projected}{bigsphere}}}}
\makeatother

%\let\oldtabular\tabular
%\let\endoldtabular\endtabular
%\renewenvironment{tabular}{\rowcolors{2}{white}{lightblue}\oldtabular\rowcolor{blue}}{\endoldtabular}


\NoAutoSpacing %french autospacing after ":"
%\usepackage{calligra}

\title[BWEB : 06- Tableurs et Excel] %
{Bureautique et Web \\Chapitre 06 : Tableurs \\ \slshape\small  Introduction \& Microsoft Excel}  

\changegraphpath{../img/tableurs/}

\begin{document}

\begin{frame}
\frametitle{Tableurs}
\framesubtitle{Les utilisations des tableurs}
	\begin{itemize}
		\item Modélisation et planification
		\item Comptes d'entreprise et budgétisation
		\item Factures et Salaires
		\item Prédictions / Simulations
		\item Calculs et analyses statistiques
		\item Création de graphiques
		\item Collectez des données de différentes sources
	\end{itemize}
\end{frame}
	
\begin{frame}
\frametitle{Tableurs}
\framesubtitle{Les avantages des tableurs}
\begin{itemize}
	\item Les calculs sont corrects et automatiques
	\item L'information est organisée et facile à accéder
	\item Les données peuvent être facilement triées et filtrées
	\item Les données peuvent être analysées rapidement
	\item Les rapports peuvent être rendus plus visuels en utilisant des tableaux et des graphiques
\end{itemize}
\end{frame}

\begin{frame}
\frametitle{Tableurs}
\framesubtitle{Pourquoi ce cours}
\begin{minipage}{0.64\textwidth}
	\begin{itemize}
		\item Il faut maitriser un outil de tableur à cause de ces avantages
		\item Pour créer des statistiques sans erreurs 
		\item Beaucoup d'utilisateurs gaspillent leurs temps à régler des problème : il faut savoir utiliser son outil
	\end{itemize}
\end{minipage}
%
\begin{minipage}{0.35\textwidth}
	\begin{table}
		\hgraphpage{pblm.pdf}
		\caption{Les problèmes rencontrés par les utilisateurs de Excel d'après \cite{chambers}}
	\end{table}
\end{minipage}
\end{frame}

\begin{frame}
\frametitle{Présentation}
\framesubtitle{Tableurs}

\begin{multicols}{2}
	%	\small
	\tableofcontents
\end{multicols}
\end{frame}

%===================================================================================
\section{Tableurs}
%===================================================================================

\subsection{Éléments d'un tableur}

\begin{frame}
\frametitle{Tableurs : Éléments d'un tableur}
\framesubtitle{Structure}

\begin{minipage}{0.49\textwidth}
	\begin{itemize}
		\item Un tableur contient une ou plusieurs feuilles de calcule 
		\item Une feuille de calcule est un grand tableau avec des lignes et des colonnes 
		\item Les lignes et les colonnes ont des références ; en général, les lignes sont référencées par des nombres et les colonnes par des lettres
		\item L'intersection entre une ligne et une colonne est une cellule
	\end{itemize}
\end{minipage}
%
\begin{minipage}{0.5\textwidth} 
	\hgraphpage{tableur.png} 
\end{minipage}

\end{frame}

\begin{frame}
\frametitle{Tableurs : Éléments d'un tableur}
\framesubtitle{Contenu des cellules}

\begin{itemize}
	\item Différents types de données : 
	\begin{itemize}
		\item Nombre
		\item Texte
		\item Date 
		\item ...
	\end{itemize}

	\item Formules : en générale, se commencent par \keyword{=}. Contiennent :
	\begin{itemize}
		\item Constantes : une valeur parmi les différentes types de données
		\item Références : pour récupérer le contenu d'une autre cellule
		\item Fonctions : les tableurs fournissent des fonctions prédéfinies 
	\end{itemize}
	
\end{itemize}


\end{frame}

\begin{frame}
\frametitle{Tableurs : Éléments d'un tableur}
\framesubtitle{Références}

\begin{itemize}
	\item Référence d'une feuille : pour référencer une feuille de calcule ; en générale, on utilise le nom de la feuille suivi par \keyword{!}
	\item Référence d'une cellule : Une cellule est référencée par la référence de sa colonne suivie par la référence de sa ligne ; exemple \expword{D3} pour dire colonne 4 ligne 3. Deux types de références : 
	\begin{itemize}
		\item Référence relative : si on recopie une formule vers une autre cellule, les références dans cette formule sont mises à jours selon la cellule initiale et destinataire
		\item Référence absolue : si on recopie une formule vers une autre cellule, les références dans cette formule restent les mêmes
	\end{itemize}
	\item Référence d'une plage : on peut référencer une plage de cellule ; en générale, on la référence de cellule de départ suivie par \keyword{:} suivie par le référence de cellule de destination (ex. \expword{C2:D6})
\end{itemize}
\end{frame}

\subsection{Cas d'utilisation}


\subsection{Outils}

\begin{frame}
\frametitle{Tableurs}
\framesubtitle{Outils}

\begin{itemize}
	\item intégrés dans des suites bureautiques
	\item spécialisés : pas beaucoup. \href{https://pyspread.gitlab.io/}{Pyspread} est un outil qui permet l'insertion des expressions Python dans le cellules.
	\item sur le cloud : beaucoup d'outils de suites bureautiques fournissent un service cloud comme \expword{Microsoft Office}
\end{itemize}

\end{frame}

\begin{frame}
\frametitle{Tableurs : Outils}
\framesubtitle{Suites bureautiques}

\def\arraystretch{.5}

\begin{tabular}{p{.3\textwidth}cp{.5\textwidth}}%p{.3\textwidth}

\hline

\vgraphpage[.8cm, valign=t]{freeoffice-logo.png} &
\vgraphpage[.8cm, valign=t]{freeoffice-planmaker-logo.png} &
FreeOffice PlanMaker  

\url{https://www.freeoffice.com/}  \\
\hline

\vgraphpage[.8cm, valign=t]{libreoffice-logo.png} &
\vgraphpage[.8cm, valign=t]{libreoffice-spreadsheet-logo.png} & 
LibreOffice Spreadsheet 

\url{https://www.libreoffice.org/}  \\
\hline

\vgraphpage[.8cm, valign=t]{msoffice-logo.png} &
\vgraphpage[.8cm, valign=t]{msoffice-excel-logo.png} & 
Microsoft Office Excel 

\url{https://www.office.com/}  \\
\hline

\vgraphpage[.7cm, valign=t]{onlyoffice-logo.png} & &
%	\graphintable{.8cm}{onlyoffice-documents-logo.png} & 
OnlyOffice Spreadsheets 

\url{https://www.onlyoffice.com/}  \\
\hline

\vgraphpage[.8cm, valign=t]{wps-logo.png} & 
\vgraphpage[.8cm, valign=t]{wps-spreadsheets-logo.png} & 
WPS Spreadsheets

\url{https://www.wps.com/}  \\
\hline


\end{tabular}
\end{frame}

\begin{frame}
\frametitle{Tableurs : Outils}
\framesubtitle{Services cloud}

\def\arraystretch{.5}

\begin{tabular}{p{.15\textwidth}p{.75\textwidth}}%p{.3\textwidth}

\hline

\vgraphpage[.9cm, valign=t]{google-sheets-logo.png} &
Google Sheets 

\url{https://docs.google.com/spreadsheets/}  \\
\hline

\vgraphpage[.9cm, valign=t]{airtable-logo.png} &
Airtable

\url{https://airtable.com/}  \\
\hline


\end{tabular}
\end{frame}


%===================================================================================
\section{Microsoft Excel}
%===================================================================================

\begin{frame}
\frametitle{Microsoft Excel}
%\framesubtitle{}
\begin{itemize}
	\item logiciel de tableur
	\item fait partie de la suite bureautique \optword{Microsoft office}
	\item disponible sur Windows, Mac OS, iOS et Android
	\item il contient des fonctionnalités de Word telles que la mise en forme, etc.
\end{itemize}
\end{frame}

%\begin{frame}
%\frametitle{Microsoft Excel}
%\framesubtitle{Aperçu}
%
%\end{frame}

\subsection{Données et opérations}

\begin{frame}
\frametitle{Excel : Données}
\framesubtitle{Opérations sur les feuilles, les lignes et les colonnes}

\begin{minipage}{0.40\textwidth}
	Les feuilles
	
	\vgraphpage[.6\textheight]{excel-feuille-op.png}
\end{minipage}
\begin{minipage}{0.28\textwidth}
	Les lignes
	
	\vgraphpage[.6\textheight]{excel-ligne-op.png}
\end{minipage}
\begin{minipage}{0.25\textwidth}
	Les colonnes
	
	\vgraphpage[.6\textheight]{excel-col-op.png}
\end{minipage}

\end{frame}

\begin{frame}[t]
\frametitle{Excel : Données}
\framesubtitle{Formats}

\begin{minipage}{0.5\textwidth}
	\begin{itemize}
		\item Onglet : \optword{Accueil}
		\item Groupe : \optword{Nombre}
		\item Objectif : formater les cellules sous forme de texte, nombre, pourcentage, etc.
		\item pour un nombre, on peut spécifier le nombre de chiffres après la virgule
	\end{itemize}
\end{minipage}
%
\begin{minipage}{0.49\textwidth}
	\hgraphpage[.49\textwidth]{excel-donnees-formats-barre.png}
	%
	\hgraphpage[.49\textwidth]{excel-donnees-formats-types.png}
	%
%	\hgraphpage{excel-donnees-formats-options.png}
\end{minipage}

\end{frame}

\begin{frame}
\frametitle{Excel : Données}
\framesubtitle{Séries de données}

\begin{minipage}{0.34\textwidth}
	\begin{itemize}
		\item pour ne pas saisir les données une à une
		\item utile pour les données séquentielles 
		\item plusieurs types : \expword{numériques}, \expword{alpha-numériques} et \expword{chronologique}
	\end{itemize}
\end{minipage}
%
\begin{minipage}{0.65\textwidth}
	\hgraphpage{excel-series.png}
\end{minipage}

%\vgraphpage[2cm]{excel-series.png}

\end{frame}

\begin{frame}
\frametitle{Excel : Données}
\framesubtitle{Séries de données (Comment saisir?)}

\begin{tabular}{p{.6\textwidth}p{.35\textwidth}}
	écrire la valeur dans une cellule et cliquer sur le coin bas à droite &
	\hgraphpage[.25\textwidth, valign=t]{excel-series-etape1.png} \\
	glisser en bas ou à gauche ; un menu apparaitra &
	\hgraphpage[.25\textwidth, valign=t]{excel-series-etape2.png} \\
	choisir \optword{Incrémenter une série} &
	\hgraphpage[.35\textwidth, valign=t]{excel-series-etape3.png} \\
\end{tabular}

\end{frame}

%\begin{frame}
%\frametitle{Excel : Données}
%\framesubtitle{Importer les données}
%
%
%
%\end{frame}

\begin{frame}
\frametitle{Excel : Données}
\framesubtitle{Trier et filtrer les colonnes (1)}

\begin{minipage}{0.7\textwidth}
	\begin{itemize}
		\item Onglet : \optword{Accueil}
		\item Groupe : \optword{Édition}
		\item Option : \optword{Trier et filtrer}
		\item Objectif : trier ou filtrer les données selon les valeurs d'une ou plusieurs colonnes
		\item Sélectionner une ou plusieurs colonnes consécutives et appliquer le trie ou le filtre
		\item Si on applique le filtre, des petites flèches apparaitront en haut des colonnes
	\end{itemize}
\end{minipage}
%
\begin{minipage}{0.29\textwidth}
	\hgraphpage{excel-trier-filtrer.png}
\end{minipage}

\end{frame}

\begin{frame}
\frametitle{Excel : Données}
\framesubtitle{Trier et filtrer les colonnes (2)}

\begin{minipage}{0.5\textwidth}
	Filtrage des textes : 
	
	\hgraphpage{excel-filtrer-texte.png}
\end{minipage}
%
\begin{minipage}{0.49\textwidth}
	Filtrage des nombres :
	
	\hgraphpage{excel-filtrer-num.png}
\end{minipage}

\end{frame}

\begin{frame}
\frametitle{Excel : Données}
\framesubtitle{Validation des données}

\begin{minipage}{0.5\textwidth}
	\begin{itemize}
		\item Onglet : \optword{Données}
		\item Groupe : \optword{Outils de données}
		\item Option : \optword{Validation des données}
		\item Objectif : vérifier les données saisies par l'utilisateur
		\item appliquée sur une ou plusieurs cellules
		\item on peut ajouter un message de saisi
		\item on peut ajouter un message d'erreur si la valeur introduite est erronée
	\end{itemize}
\end{minipage}
%
\begin{minipage}{0.49\textwidth}
	\hgraphpage{excel-outils-data.png}
	
	\hgraphpage{excel-validation.png}
\end{minipage}

\end{frame}

\begin{frame}
\frametitle{Excel : Données}
\framesubtitle{Validation des données : Listes de choix (1)}

\begin{minipage}{0.60\textwidth}
	Afin d'appliquer des listes de sélection sur des cellules
	\begin{itemize}
		\item Créer une liste des choix dans une colonne (exemple: \expword{colonne F})
		\item Sélectionner les cellules sur lesquelles on veut appliquer cette liste (exemple: \expword{colonne C})
		\item Appliquer l'option \optword{Validation des données}
		\item Autoriser \optword{Liste}
		\item Source : la liste de choix 
	\end{itemize}
\end{minipage}
%
\begin{minipage}{0.39\textwidth}
	\hgraphpage{excel-validation-liste1.png} 
	
	\hgraphpage{excel-validation-liste2.png}
\end{minipage}

\end{frame}

\begin{frame}
\frametitle{Excel : Données}
\framesubtitle{Validation des données : Listes de choix (2)}

\begin{minipage}{0.60\textwidth}
	\begin{itemize}
		\item On peut ajouter un message de saisi
		\item Lorsqu'on clique sur la cellule ce message s'affichera 
		\item Aussi, on peut ajouter un message d'erreur (si l'utilisateur insère une valeur hors les valeurs de la liste) 
	\end{itemize}
\end{minipage}
%
\begin{minipage}{0.39\textwidth}
	\hgraphpage{excel-validation-liste3.png} 
	
	\hgraphpage[.45\textwidth]{excel-validation-liste4.png}
	\hgraphpage[.4\textwidth]{excel-validation-liste5.png}
\end{minipage}

\end{frame}

\begin{frame}
\frametitle{Excel : Données}
\framesubtitle{Représentation graphique}

\begin{minipage}{0.5\textwidth}
	\begin{itemize}
		\item Onglet : \optword{Insertion}
		\item Groupe : \optword{Graphiques}
		\item Objectif : sélectionner des données et insérer des graphiques
	\end{itemize}
\end{minipage}
%
\begin{minipage}{0.49\textwidth}
	\hgraphpage[.48\textwidth]{excel-graphiques.png}
	\hgraphpage[.48\textwidth]{excel-graphiques2.png}
	
	\hgraphpage[.70\textwidth]{excel-graphiques3.png}
\end{minipage}

\end{frame}

\begin{frame}
\frametitle{Excel : Représentation graphique}
\framesubtitle{Un peu d'humour}

\begin{center}
	\vgraphpage{graphique-humour.png}
\end{center}

\end{frame}

\subsection{Formules et Référencement}

\begin{frame}
\frametitle{Excel : Formules et Référencement}
\framesubtitle{Formules}

\begin{minipage}{0.60\textwidth}
	\begin{itemize}
		\item Cliquer sur une cellule 
		\item Pour ajouter une formule, insérer \keyword{=}
		\item Cliquer sur la première cellule 
		\item Ajouter une opération ; par exemple : \expword{+}
		\item Cliquer sur la deuxième cellule 
		\item Appuyer sur Entrer
	\end{itemize}
\end{minipage}
%
\begin{minipage}{0.39\textwidth}
	\hgraphpage{excel-formule.png} 
\end{minipage}

\begin{itemize}
	\item Une formule se commence TOUJOURS par \keyword{=}, elle peut contenir :
	\begin{itemize}
		\item des adresses des cellules ; par exemple : \expword{C2}
		\item des opérateurs ; par exemple : \expword{+}
		\item des constantes ; par exemple : \expword{5}
		\item des fonctions ; par exemple : \expword{SOMME} 
	\end{itemize}
\end{itemize}

\end{frame}

\begin{frame}
\frametitle{Excel : Formules et Référencement}
\framesubtitle{Référencement relatif}

\begin{minipage}{0.40\textwidth}
	\begin{itemize}
		\item Si on glisse ou on recopie une cellule contenant une formule, on remarque que les adresses sont mises à jour dans les cellules destinataires
	\end{itemize}
\end{minipage}
%
\begin{minipage}{0.59\textwidth}
	\hgraphpage{excel-adr-rel.png} 
\end{minipage}

\begin{itemize}
%	\item Si on glisse ou on recopier une cellule avec une formule, on remarque que les adresses se changent dans les cellules destinataires
	\item On appelle ça \keyword{adresse relative}
	\item Par exemple, dans la cellule \expword{E2} on a la formule \expword{=C2+D2+5}
	\item Si on recopie ça dans \expword{E3}, elle aura la formule \expword{=C3+D3+5} (décalage d'une ligne)
	\item Si on recopie ça dans \expword{F3}, elle aura la formule \expword{=D3+E3+5} (décalage d'une ligne et d'une colonne)
\end{itemize}

\end{frame}

\begin{frame}
\frametitle{Excel : Formules et Référencement}
\framesubtitle{Référencement absolu}

\begin{minipage}{0.40\textwidth}
	\begin{itemize}
		\item On veut fixer une adresse ; c-à-d, prévenir sa mise à jour lorsqu'on recopie la formule dans une autre cellule
		%		\item On appelle ça \keyword{adresse relative}
		%		\item Par exemple, dans la cellule \expword{}
	\end{itemize}
\end{minipage}
%
\begin{minipage}{0.59\textwidth}
	\hgraphpage{excel-adr-abs.png} 
\end{minipage}

\begin{itemize}
	\item On doit utiliser une \keyword{adresse absolue}
	\item Pour rendre une adresse absolue, on doit utiliser un dollar (\keyword{\$}) avant la colonne (\expword{\$D5}), avant la ligne (\expword{D\$5}) ou avant les deux (\expword{\$D\$5})
	\item Une autre solution est de renommer une cellule ; par exemple, renomer la cellule \expword{D5} par \expword{\_bonus} et utiliser cette référence
\end{itemize}

\end{frame}

\begin{frame}
\frametitle{Excel : Formules et Référencement}
\framesubtitle{Référencement d'une feuille de calcul}

\begin{minipage}{0.40\textwidth}
	\begin{itemize}
		\item On veut référencer des cellules dans une autre feuille
		\item Pour ce faire, on écrit le nom de la feuille suivi par point d'exclamation (\keyword{!}) suivi par l'adresse de la cellule (relative ou absolue)
	\end{itemize}
\end{minipage}
%
\begin{minipage}{0.59\textwidth}
	\hgraphpage{excel-adr-feuille.png} 
\end{minipage}

\end{frame}

\subsection{Fonctions}

\begin{frame}
\frametitle{Excel : Fonctions}
\framesubtitle{Fonctions mathématiques (1)}
\begin{minipage}{0.81\textwidth}
\begin{itemize}
	\item \optword{SOMME} : calcule la somme entre plusieurs valeurs (ex. \expword{SOMME(E3;E6;2)}) ou une plage de cellules (ex. \expword{SOMME(E3:E6)})
	\item \optword{SOMME.SI} : calcule la somme avec une condition. Par exemple : 
	\begin{itemize}
		\item \expword{SOMME.SI(E3:E6;"<10")} calcule la somme des nombres inférieures à \expword{10} dans la plage \expword{E3:E6}
		\item \expword{SOMME.SI(D3:D6;"BWEB";E3:E6)} calcule uniquement la somme des valeurs de la plage \expword{E3:E6}, dans laquelle les cellules correspondantes de la plage \expword{D3:D6} contiennent le mot « \expword{BWEB} ».
	\end{itemize}
\end{itemize}
\end{minipage}
%
\begin{minipage}{0.18\textwidth}
	\hgraphpage{excel-somme.png} 
	
	\hgraphpage{excel-somme-si.png} 
\end{minipage}
\end{frame}

\begin{frame}
\frametitle{Excel : Fonctions}
\framesubtitle{Fonctions mathématiques (2)}
\begin{minipage}{0.81\textwidth}
	\begin{itemize}
		\item \optword{PRODUIT} : comme \optword{SOMME}, mais elle fait le produit.
		\item \optword{ABS} : rend la valeur absolue d'un nombre
		\item \optword{ARRONDI} : calcule l'arrondi d'un nombre à un nombre spécifié de chiffres.
		\item \optword{RACINE} : donne la racine carrée d'un nombre. Si le nombre est négatif, elle retourne une erreur \keyword{\#NOMBRE!}
		\item \optword{PUISSANCE} : renvoie la valeur du nombre élevé à une puissance
		\item Il y a d'autres fonctions mathématiques ...
	\end{itemize}
\end{minipage}
%
\begin{minipage}{0.18\textwidth}
	\hgraphpage[.52\textwidth]{excel-produit.png} 
	\hgraphpage[.38\textwidth]{excel-abs.png} 
	
	\hgraphpage[.60\textwidth]{excel-arrondi.png} 
	\hgraphpage[.60\textwidth]{excel-racine.png}
	\hgraphpage[.60\textwidth]{excel-puissance.png}
\end{minipage}
\end{frame}

\begin{frame}
\frametitle{Excel : Fonctions}
\framesubtitle{Fonctions statistiques}

\begin{minipage}{0.69\textwidth}
	\begin{itemize}
		\item \optword{MOYENNE} : calcule la moyenne entre plusieurs valeurs ou une plage
		\item \optword{MOYENNE.SI} : comme \optword{SOMME.SI} mais pour la moyenne
		\item \optword{MEDIANE} : renvoie la valeur médiane
		\item \optword{MAX} : renvoie le plus grand nombre
		\item \optword{MIN} : renvoie le plus petit nombre
		\item \optword{VARA} : calcule la variance
		\item \optword{NBVAL} : compte le nombre de cellules qui ne sont pas vides dans une plage.
	\end{itemize}
\end{minipage}
%
\begin{minipage}{0.3\textwidth} 
	
	\hgraphpage[.63\textwidth]{excel-moyenne-si.png} 
	
	\hgraphpage[.35\textwidth]{excel-mediane.png}
	\hgraphpage[.29\textwidth]{excel-max.png}
	
	\hgraphpage[.29\textwidth]{excel-min.png}
	\hgraphpage[.3\textwidth]{excel-vara.png}
	\hgraphpage[.32\textwidth]{excel-nbval.png}
\end{minipage}

\end{frame}

\begin{frame}
\frametitle{Excel : Fonctions}
\framesubtitle{Fonctions logiques}

\begin{minipage}{0.69\textwidth}
	\begin{itemize}
		\item \optword{SI} : permet d'établir des comparaisons logiques entre une valeur et le résultat attendu. Elle retourne le deuxième argument si la condition est vrai ; le troisième argument sinon.
		\item \optword{ET} : renvoie \keyword{VRAI} si tous ses arguments sont \keyword{VRAI}
		\item \optword{OU} : renvoie \keyword{VRAI} si un des arguments est \keyword{VRAI}
		\item \optword{NON} : inverse la logique de cet argument
	\end{itemize}
\end{minipage}
%
\begin{minipage}{0.3\textwidth} 
	
	\hgraphpage[.63\textwidth]{excel-si.png} 
	
	\hgraphpage[.48\textwidth]{excel-et.png}
	\hgraphpage[.48\textwidth]{excel-ou.png}
	
	\hgraphpage[.35\textwidth]{excel-non.png}
\end{minipage}

\end{frame}

\begin{frame}
\frametitle{Excel : Fonctions}
\framesubtitle{Fonctions de texte}

\begin{minipage}{0.69\textwidth}
	\begin{itemize}
		\item \optword{CONCATENER} : permet de joindre plusieurs chaînes au sein d’une seule chaîne
		\item \optword{MINUSCULE} : convertit toutes les lettres majuscules d'une chaîne de texte en lettres minuscules
		\item \optword{MAJUSCULE} : convertit toutes les lettres minuscules d'une chaîne de texte en lettres majuscules
		\item \optword{SUBSTITUE} : remplace un sous-texte par un autre dans une chaîne de texte
	\end{itemize}
\end{minipage}
%
\begin{minipage}{0.3\textwidth} 
	
	\hgraphpage{excel-concatener.png} 
	\hgraphpage[.48\textwidth]{excel-minuscule.png}
	\hgraphpage[.48\textwidth]{excel-majuscule.png}
	\hgraphpage{excel-substitue.png}
\end{minipage}

\end{frame}

\begin{frame}
\frametitle{Excel : Fonctions}
\framesubtitle{Fonctions de date}

\begin{minipage}{0.69\textwidth}
	\begin{itemize}
		\item \optword{DATE} : renvoie le numéro de série séquentiel qui représente une date particulière
		\item \optword{JOURS360} : renvoie le nombre de jours compris entre deux dates sur la base d’une année de 360 jours (12 mois de 30 jours)
		\item \optword{AUJOURDHUI} : retourne la date d'aujourd'hui
		\item \optword{ANNEE} : retourne l'année à partir d'une date
	\end{itemize}
\end{minipage}
%
\begin{minipage}{0.3\textwidth} 
	
	\hgraphpage[.48\textwidth]{excel-date.png} 
	\hgraphpage[.48\textwidth]{excel-jours360.png}
	\hgraphpage[.38\textwidth]{excel-aujourdhui.png}
	\hgraphpage[.48\textwidth]{excel-annee.png}
\end{minipage}

\end{frame}

\begin{frame}
\frametitle{Excel : Fonctions}
\framesubtitle{Fonctions de recherche et de référence}

\begin{minipage}{0.69\textwidth}
	\begin{itemize}
		\item \optword{RECHERCHEV} : rechercher des éléments dans un tableau ou une plage par ligne. Par exemple : 
		\begin{itemize}
			\item \expword{RECHERCHEV("ALGO";D2:F4;2)} sélectionne une matrice de \expword{D2} à \expword{F4}
			\item elle cherche \expword{"ALGO"} dans la première  colonne ; c-à-d, dans la plage \expword{D2:D4}
			\item elle retourne la valeur correspondante dans la colonne \expword{2} ; c-à-d, dans la plage \expword{F2:F4}
			\item par défaut, cette fonction cherche une valeur approximative. Si on veut qu'elle cherche une valeur exacte, on doit ajouter \keyword{FAUX} aux arguments (ex. \expword{RECHERCHEV("ALGO";D2:F4;2;FAUX)})
		\end{itemize} 
		\item Il y a d'autres fonctions comme \optword{INDIRECT}, \optword{INDEX}, \optword{ADRESSE}, etc.
	\end{itemize}
\end{minipage}
%
\begin{minipage}{0.3\textwidth} 
	
	\hgraphpage{excel-recherchev.png} 
\end{minipage}

\end{frame}

\subsection{Formatage, impression et protection}

\begin{frame}
\frametitle{Excel : Formatage et impression}
\framesubtitle{Contenu d'une cellule}

\begin{minipage}{0.49\textwidth}
	\begin{itemize}
		\item Problème : le contenu de la cellule déborde ou il ne s'affiche pas complètement
		\item Clique droite sur la cellule
		\item Choisir \optword{Format de cellule}
%		\item Dans \optword{}
	\end{itemize}
\end{minipage}
%
\begin{minipage}{0.5\textwidth} 
	
	\hgraphpage{excel-cellule-format.png} 
	
%	\hgraphpage[.6\textwidth]{excel-mfc2.png} 
\end{minipage}

\begin{itemize}
	\item Dans \optword{Contrôle du texte}, vous pouvez choisir : 
	\begin{itemize}
		\item Renvoyer à la ligne le reste du texte 
		\item Ajuster la taille du texte 
		\item Fusionner automatiquement avec la cellule suivante
	\end{itemize}
\end{itemize}

\end{frame}

\begin{frame}
\frametitle{Excel : Formatage et impression}
\framesubtitle{Mise en forme conditionnelle}

\begin{minipage}{0.69\textwidth}
	\begin{itemize}
		\item Onglet : \optword{ACCUEIL} 
		\item Groupe : \optword{Style} 
		\item Option : \optword{Mise en forme conditionnelle} 
		\item Objectif : Changer le styles des cellules selon leurs valeurs
		\item On peut appliquer plusieurs règles sur une plage de donnée
		\item Par exemple, sur la plage \expword{E2:E5}, colorier la cellule en rouge si sa valeur est inférieure à \expword{10} et colorier la en vert si sa valeur est supérieure à \expword{15}
	\end{itemize}
\end{minipage}
%
\begin{minipage}{0.3\textwidth} 
	
	\hgraphpage{excel-mfc.png} 
	
	\hgraphpage[.6\textwidth]{excel-mfc2.png} 
\end{minipage}

\end{frame}

\begin{frame}
\frametitle{Excel : Formatage et impression}
\framesubtitle{Mise en page}

\begin{minipage}{0.54\textwidth}
	\begin{itemize}
		\item Onglet : \optword{Mise en page}
		\item Groupe : \optword{Mise en page}
		\item Définir la zone d'impression 
		\begin{itemize}
			\item Sélectionner la plage que vous veuillez imprimer
			\item Choisir l'option \optword{ZoneImp} et sélectionner \optword{Définir}
		\end{itemize}
		\item Si la liste prend plusieurs pages et vous voulez répéter la ligne des titres sur chaque page 
		\begin{itemize}
			\item Choisir l'option \optword{Imprimer les titres}
			\item Dans \optword{Lignes à répéter en haut} sélectionner une ou plusieurs lignes
		\end{itemize}
	\end{itemize}
\end{minipage}
%
\begin{minipage}{0.45\textwidth}
	\hgraphpage{excel-mep-barre.png}
	
	\hgraphpage{excel-mep.png}
\end{minipage}

\end{frame}

\begin{frame}
\frametitle{Excel : Formatage et impression}
\framesubtitle{Protection}

\begin{minipage}{0.54\textwidth}
	\begin{itemize}
		\item Onglet : \optword{Révision}
		\item Groupe : \optword{Modifications}
		\item On peut protéger
		\begin{itemize}
			\item tout le classeur (le fichier)
			\item une feuille de calcul
		\end{itemize}
	\end{itemize}
\end{minipage}
%
\begin{minipage}{0.45\textwidth}
	\hgraphpage{excel-protection-barre.png}
	
	\hgraphpage{excel-protection.png}
\end{minipage}

\end{frame}


\insertbibliography{Bweb06}{*}


\end{document}

