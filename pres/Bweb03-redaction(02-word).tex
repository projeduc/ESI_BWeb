% !TEX TS-program = pdflatex
% !TeX program = pdflatex
% !TEX encoding = UTF-8
% !TEX spellcheck = fr

\documentclass[xcolor=table]{beamer}


%\usepackage{fullpage}
%\usepackage[left=2.8cm,right=2.2cm,top=2 cm,bottom=2 cm]{geometry}
\setbeamersize{text margin left=10pt,text margin right=10pt}
\usepackage{amsmath,amssymb} 
\usepackage[T1]{fontenc}
\usepackage[utf8]{inputenc}
\usepackage[english,french]{babel}
\usepackage{txfonts}
\usepackage[]{graphicx}
\usepackage{multirow}
\usepackage{hyperref}
\usepackage{colortbl}
\usepackage{listings}
\usepackage{wrapfig}
\usepackage{multicol}

\hypersetup{
	colorlinks,
	urlcolor = blue
}

%\renewcommand{\baselinestretch}{1.5}

\def\supit#1{\raisebox{0.8ex}{\small\it #1}\hspace{0.05em}}

\AtBeginSection{%
	\begin{frame}
		\sectionpage
	\end{frame}
}

\newcommand{\rottext}[2]{%
	\rotatebox{90}{%
	\begin{minipage}{#1}%
		\raggedleft#2%
	\end{minipage}%
	}%
}

\usepackage{longtable}
\usepackage{tabu}


\institute{ %
École  nationale Supérieure d'Informatique (ESI, ex. INI), Algérie
}
\author[ \textbf{\footnotesize  \insertframenumber/\inserttotalframenumber} \hspace*{\fill} ESI (2019-2020)] %
{ARIES Abdelkrime}
%\titlegraphic{\includegraphics[height=1cm]{../img/esi-logo.png}%\hspace*{4.75cm}~


\date{Année unniversitaire: 2019/2020} %\today

\usetheme{Warsaw} % Antibes Boadilla Warsaw

\beamertemplatenavigationsymbolsempty

%\setbeamertemplate{headline}{}

\definecolor{lightblue}{HTML}{D0D2FF}
\definecolor{lightyellow}{HTML}{FFFFAA}
\definecolor{darkblue}{HTML}{0000BB}
\definecolor{olivegreen}{HTML}{006600}
\definecolor{violet}{HTML}{6600CC}

\newcommand{\keyword}[1]{\textcolor{red}{\bfseries\itshape #1}}
\newcommand{\expword}[1]{\textcolor{olivegreen}{#1}}
\newcommand{\optword}[1]{\textcolor{violet}{\bfseries #1}}

\makeatletter
\newcommand\mysphere{%
	\parbox[t]{10pt}{\raisebox{0.2pt}{\beamer@usesphere{item projected}{bigsphere}}}}
\makeatother

%\let\oldtabular\tabular
%\let\endoldtabular\endtabular
%\renewenvironment{tabular}{\rowcolors{2}{white}{lightblue}\oldtabular\rowcolor{blue}}{\endoldtabular}


\NoAutoSpacing %french autospacing after ":"

\title[BWEB : 03- Rédaction (Microsoft Word)] %
{Bureautique et Web \\Chapitre 03 : Rédaction d'un document numérique \\ \slshape\small  Microsoft Word}  

\changegraphpath{..//img/Bweb03-redaction/word/}

\begin{document}


\begin{frame}
\frametitle{Rédaction d'un document numérique}
\framesubtitle{Microsoft Word}

\begin{itemize}
	\item WYSIWYG : What You See Is What You Get
	\item fait parti de la suite bureautique : Microsoft office
	\item c'est le programme de traitement de texte le plus populaire au monde
	\item facile à configurer et à utiliser 
\end{itemize}

\end{frame}

\begin{frame}
\frametitle{Rédaction d'un document numérique}
\framesubtitle{Microsoft Word}

\begin{center}
	\vgraphpage{word.png}
\end{center}

\end{frame}

\begin{frame}
\frametitle{Rédaction d'un document numérique}
\framesubtitle{Microsoft : Plan}

\begin{multicols}{2}
%	\small
	\tableofcontents
\end{multicols}
\end{frame}

%===================================================================================
\section{Préparer un document}
%===================================================================================

%\begin{frame}
%\frametitle{Préparer un document}
%
%\end{frame}

\subsection{Pages}

\begin{frame}[t]
\frametitle{Préparer un document : Pages}
\framesubtitle{Taille}

\begin{minipage}{0.44\textwidth}
	\begin{itemize}
		\item Onglet : \optword{Mise en page}
		\item Groupe : \optword{Mise en page}
		\item Plusieurs tailles : A4, etc.
		\item On peut définir des tailles personnalisées
	\end{itemize}
\end{minipage}
\begin{minipage}{0.55\textwidth}
	\hgraphpage{taille.png}
\end{minipage}

\end{frame}


\begin{frame}[t]
\frametitle{Préparer un document : Pages}
\framesubtitle{Orientation}

\begin{minipage}{0.44\textwidth}
	\begin{itemize}
		\item Onglet : \optword{Mise en page}
		\item Groupe : \optword{Mise en page}
		\item Deux orientations : \optword{Portrait} ou \optword{Paysage}.
		\item Pour changer l'orientation après une page, insérer un saut de section (page suivante)
	\end{itemize}
\end{minipage}
\begin{minipage}{0.55\textwidth}
	\hgraphpage{orientation.png}
	\vspace{1cm}%pour pousser l'image vers le haut du slide
\end{minipage}

\end{frame}

\begin{frame}[t]
\frametitle{Préparer un document : Pages}
\framesubtitle{Marges}

\begin{minipage}{0.44\textwidth}
	\begin{itemize}
		\item Onglet : \optword{Mise en page}
		\item Groupe : \optword{Mise en page}
		\item Des marges prédéfinies
		\item On peut ajouter des marges personnalisées
	\end{itemize}
\end{minipage}
\begin{minipage}{0.55\textwidth}
	\hgraphpage{marges.png}
\end{minipage}

\end{frame}

\begin{frame}[t]
\frametitle{Préparer un document : Pages}
\framesubtitle{Colonnes}

\begin{minipage}{0.44\textwidth}
	\begin{itemize}
		\item Onglet : \optword{Mise en page}
		\item Groupe : \optword{Mise en page}
		\item Sélectionner le texte qu'on veuille mettre en colonnes
		\item Choisir le nombre de colonnes 
		\item Dans les articles scientifiques, la double-colonnes est très utilisée
	\end{itemize}
\end{minipage}
\begin{minipage}{0.55\textwidth}
	\hgraphpage{colonnes.png}
\end{minipage}

\end{frame}

\begin{frame}[t]
\frametitle{Préparer un document : Pages}
\framesubtitle{Saute de page et section}

\begin{minipage}{0.44\textwidth}
	\begin{itemize}
		\item Onglet : \optword{Mise en page}
		\item Groupe : \optword{Mise en page}
		\item On insère un saut de page par exemple après la table de matière, liste des tableaux, etc.
		\item On insère un saut de section pour séparer les chapitres : avoir des différentes entêtes, différentes numérotations, etc.
	\end{itemize}
\end{minipage}
\begin{minipage}{0.55\textwidth}
	\hgraphpage{sautes.png}
\end{minipage}

\end{frame}

\begin{frame}[t]
\frametitle{Préparer un document : Pages}
\framesubtitle{Page de garde}

\begin{minipage}{0.49\textwidth}
	\begin{itemize}
		\item Onglet : \optword{Insertion}
		\item Groupe : \optword{Pages}
		\item On peut choisir une page de garde
		\item ou, naviguer d'autre sur le site \url{office.com}
	\end{itemize}
\end{minipage}
\begin{minipage}{0.50\textwidth}
	\hgraphpage{garde.png}
\end{minipage}

\end{frame}

\begin{frame}[t]
\frametitle{Préparer un document : Pages}
\framesubtitle{Bordures}

\begin{minipage}{0.49\textwidth}
	\begin{itemize}
		\item Onglet : \optword{Création}
		\item Groupe : \optword{Arrière-plan de la page}
		\item On peut ajouter des bordures de page :
		\begin{itemize}
			\item Encadrement
			\item Ombre
			\item 3D
		\end{itemize}
	\end{itemize}
\end{minipage}
\begin{minipage}{0.50\textwidth}
	\hgraphpage{bordures.png}
\end{minipage}

\end{frame}

\begin{frame}[t]
\frametitle{Préparer un document : Pages}
\framesubtitle{Filigrane}

\begin{minipage}{0.49\textwidth}
	\begin{itemize}
		\item Onglet : \optword{Création}
		\item Groupe : \optword{Arrière-plan de la page}
		\item Le filigrane (Watermark) est un texte transparent qui apparait sur les pages
		\item On peut : 
		\begin{itemize}
			\item choisir un filigrane prédéfinie
			\item naviguer d'autres sur le site \url{office.com}
			\item supprimer le filigrane actuel 
			\item créer un filigrane personnalisé
		\end{itemize}
	\end{itemize}
\end{minipage}
\begin{minipage}{0.50\textwidth}
	\hgraphpage{filigrane1.png}
\end{minipage}

\end{frame}

\begin{frame}[t]
\frametitle{Préparer un document : Pages}
\framesubtitle{Filigrane}

\begin{minipage}{0.49\textwidth}
	\begin{itemize}
		\item Onglet : \optword{Création}
		\item Groupe : \optword{Arrière-plan de la page}
		\item On peut choisir une image ou un texte 
	\end{itemize}
\end{minipage}
\begin{minipage}{0.50\textwidth}
	\hgraphpage{filigrane2.png}
\end{minipage}

\end{frame}

\begin{frame}[t]
\frametitle{Préparer un document : Pages}
\framesubtitle{Arrière-plan}

\begin{minipage}{0.49\textwidth}
	\begin{itemize}
		\item Onglet : \optword{Création}
		\item Groupe : \optword{Arrière-plan de la page}
		\item Pour choisir la couleur des pages
		\item on peut choisir d'autres couleurs 
		\item ou, ajouter un effet de remplissage
	\end{itemize}
\end{minipage}
\begin{minipage}{0.50\textwidth}
	\hgraphpage{couleur.png}
	\vspace{1cm}
\end{minipage}

\end{frame}

\begin{frame}[t]
\frametitle{Préparer un document : Pages}
\framesubtitle{Arrière-plan}

\begin{minipage}{0.49\textwidth}
	\begin{itemize}
		\item Onglet : \optword{Création}
		\item Groupe : \optword{Arrière-plan de la page}
		\item Un effet peut être : 
		\begin{itemize}
			\item Dégradé
			\item Texture
			\item Motif
			\item Image
		\end{itemize}
	\end{itemize}
\end{minipage}
\begin{minipage}{0.50\textwidth}
	\hgraphpage{remplissage.png}
\end{minipage}

\end{frame}



\begin{frame}
\frametitle{Préparer un document : Pages}
\framesubtitle{En-tête et pied de page}

\begin{minipage}{0.78\textwidth}
	\begin{itemize}
		\item Onglet : \optword{Insertion} 
		\item Groupe : \optword{En-tête et pied de page}
	\end{itemize}
\end{minipage}
\begin{minipage}{0.2\textwidth}
	\hgraphpage{entete-pied.png}
\end{minipage}

\vspace{6pt}
Si on double-clique sur l'entête, un onglet s'apparaitra
\begin{itemize}
	\item \optword{Première page différente} : en général, pour la page de garde 
	\item \optword{Pages paires et impaires différentes} : par exemple, pour afficher le nom du chapitre dans une page, et le nom de l'auteur dans la suivante
	\item \optword{Lier au précédent} : après un saut de section, on peut changer le style des entêtes et pieds de page en découchant cette option
\end{itemize}

\hgraphpage{entete-pied-barre.png}

\end{frame}


\subsection{Paragraphes}

\begin{frame}
\frametitle{Préparer un document : Paragraphes}
\framesubtitle{Paragraphes}

\begin{minipage}{0.38\textwidth}
	\begin{itemize}
		\item Onglet : \optword{Accueil} et \optword{Mise en page}
		\item Groupe : \optword{Paragraphe}
		\item Fonctionnalités :
		\begin{itemize}
			\item Alignement 
			\item Espacement vertical
			\item Interligne
			\item Retrait
			\item Retrait de 1ère ligne
			\item Listes
			\item Affichage des symboles non imprimables
		\end{itemize}
	\end{itemize}
\end{minipage}
\begin{minipage}{0.60\textwidth}
	\def\arraystretch{.5}
	\begin{tabular}{@{}ll}
		\hgraphpage[.48\textwidth]{accueil-paragraphe.png} &
		\hgraphpage[.48\textwidth]{misepage-paragraphe.png} \\
		\tiny Accueil & \tiny Mise en page \\
	\end{tabular}
%	\begin{center}
%		\vspace{-8pt}
		\hgraphpage[.62\textwidth]{paragraphe.png}
%	\end{center}
\end{minipage}

\end{frame}

\subsection{Mise en forme et styles}

\begin{frame}
\frametitle{Préparer un document : Mise en forme et styles}
\framesubtitle{Mise en forme}

\begin{minipage}{0.48\textwidth}
\begin{itemize}
	\item Onglet : \optword{Accueil}
	\item Groupe : \optword{Police}
	\item Cliquer sur la petite flèche pour avoir plus d'options
	\item Fonctionnalités :
	\begin{itemize}
		\item Police (famille et taille) 
		\item Style des caractères (Gras, italique, souligné, barré) 
		\item Couleur du texte
		\item Surbrillance du texte (marqueur)
		\item La casse (Majuscule, Minuscule)
		\item Effets du texte
	\end{itemize}
\end{itemize}
\end{minipage}
\begin{minipage}{0.50\textwidth}
	\hgraphpage{miseforme.png}
	
	\hgraphpage[0.8\textwidth]{police.png}
	%	\end{center}
\end{minipage}

\end{frame}

\begin{frame}
\frametitle{Préparer un document : Mise en forme et styles}
\framesubtitle{Casse et effets de texte}

\begin{minipage}{0.58\textwidth}
\begin{itemize}
	\item Onglet : \optword{Accueil}
	\item Groupe : \optword{Police}
	\item Sélectionner un texte
	\item On peut
	\begin{itemize}
		\item changer la casse 
		\item appliquer des effets sur ce texte
	\end{itemize}
\end{itemize}
\end{minipage}
\begin{minipage}{0.4\textwidth}
	\hgraphpage[.5\textwidth]{casse.png}
	
	\hgraphpage{effet-texte.png}
	%	\end{center}
\end{minipage}

\end{frame}


\begin{frame}
\frametitle{Préparer un document : Mise en forme et styles}
\framesubtitle{Styles}

\begin{minipage}{0.58\textwidth}
\begin{itemize}
	\item Onglet : \optword{Accueil}
	\item Groupe : \optword{Styles}
	\item L'intérêt : facilitent la mise en forme, texte uniforme, table de matière
	\item On peut
	\begin{itemize}
		\item appliquer un style sur un texte
		\item créer un nouveau style en se basant sur un texte formaté
		\item mettre à jour ou supprimer un style en appliquant un clic-droit sur ce style
	\end{itemize}
\end{itemize}
\end{minipage}
\begin{minipage}{0.4\textwidth}
	\hgraphpage{style.png}
	
	\vspace{12pt}
	
	\hgraphpage[.8\textwidth]{style-new.png}
\end{minipage}

\end{frame}

\subsection{Modification}%Recherche et remplacement

\begin{frame}[t]
\frametitle{Préparer un document : Modification}
\framesubtitle{Recherche}

\begin{minipage}{0.60\textwidth}
\begin{itemize}
	\item Onglet : \optword{Accueil}
	\item Groupe : \optword{\'Edition}
	\item lorsqu'on clique sur \optword{Rechercher}, une barre de navigation apparaitra
	\item On peut chercher :
	\begin{itemize}
		\item du texte
		\item des images
		\item des tableaux
	\end{itemize}
\end{itemize}
\end{minipage}
\begin{minipage}{0.38\textwidth}
	\begin{flushright}
		\hgraphpage[1.5cm]{edition.png}
	\end{flushright}

	\vspace{-12pt}
	\hgraphpage{rechercher.png}
\end{minipage}

\end{frame}

\begin{frame}[t]
\frametitle{Préparer un document : Modification}
\framesubtitle{Remplacement}

\begin{minipage}{0.38\textwidth}
\begin{itemize}
	\item Onglet : \optword{Accueil}
	\item Groupe : \optword{\'Edition}
	\item On peut remplacer un texte avec un autre, un par un ou toutes les occurrences
	\item D'autres options avancées peuvent être utilisées :
	\begin{itemize}
		\item La casse
		\item Mot entier
	\end{itemize}
\end{itemize}
\end{minipage}
\begin{minipage}{0.6\textwidth}	
	\begin{flushright}
		\hgraphpage[1.5cm]{edition.png}
	\end{flushright}

	\vspace{-12pt}
	\hgraphpage{remplacer.png}
\end{minipage}

\end{frame}

%===================================================================================
\section{Enrichir un document}
%===================================================================================

%\begin{frame}
%\frametitle{Enrichir un document}
%
%\end{frame}

\subsection{Tableaux}


\begin{frame}
\frametitle{Enrichir un document : Tableaux}
\framesubtitle{Création}

\begin{minipage}{0.70\textwidth}
\begin{itemize}
	\item Onglet : \optword{Insertion}
	\item Groupe : \optword{Tableaux}
	\item On peut choisir le nombre des lignes et des colonnes
	\item On peut convertir un texte en tableau :
	\begin{itemize}
		\item Les lignes du tableau sont les lignes du texte
		\item Les colonnes du tableau sont les textes séparés par des points virgules, des tabulations, ou une autre marque
	\end{itemize}
\end{itemize}
\end{minipage}
\begin{minipage}{0.28\textwidth}	
	\hgraphpage{tableau.png}
\end{minipage}

\end{frame}

\begin{frame}[t]
\frametitle{Enrichir un document : Tableaux}
\framesubtitle{Options : Disposition}

\hgraphpage{tableau-disposition.png}

\begin{itemize}
	\item Onglet : \optword{Outils de tableau : Disposition}
	\item Lorsqu'on sélectionne un tableau, cet onglet apparaitra
	\begin{itemize}
		\item \optword{Dessiner} : effacer des traits et ajouter d'autres
		\item \optword{Lignes et colonnes} : ajouter ou supprimer des lignes et des colonnes
		\item \optword{Fusionner} : fusionner les cellules sélectionnées
		\item \optword{Taille de la cellule} : fixer la longueur et la hauteur d'une cellule
		\item \optword{Alignement} : alignement horizontal et vertical du contenu d'une cellule, aussi sa orientation
		\item \optword{Données} : trier, convertir en texte et ajouter une formule
	\end{itemize}
\end{itemize}

\end{frame}

\begin{frame}[t]
\frametitle{Enrichir un document : Tableaux}
\framesubtitle{Options : Création}

\hgraphpage{tableau-creation.png}

\begin{itemize}
	\item Onglet : \optword{Outils de tableau : Création}
	\item Lorsqu'on sélectionne un tableau, cet onglet apparaitra
	\begin{itemize}
		\item \optword{Style de tableau} : appliquer des styles sur le tableau
		\item \optword{Bordures} : effacer les bordures, colorer, choisir le style de trait
	\end{itemize}
\end{itemize}

\end{frame}


\subsection{Illustrations}

\begin{frame}[t]
\frametitle{Enrichir un document : Illustrations}
\framesubtitle{Insérer une illustration}

\begin{minipage}{0.43\textwidth}
\begin{itemize}
	\item Onglet : \optword{Insertion}
	\item Groupe : \optword{Illustrations}
	\item On peut :
	\begin{itemize}
		\item ajouter des images,
		\item dessiner des formes
		\item insérer des graphiques
		\item insérer des formes SmartArt
	\end{itemize}
\end{itemize}
\end{minipage}
\begin{minipage}{0.28\textwidth}
	\hgraphpage{illustrations.png}
	
	\hgraphpage{illustrations-graphique.png}
	
	\hgraphpage{illustrations-smartart.png}
\end{minipage}
\begin{minipage}{0.27\textwidth}	
	\hgraphpage{illustrations-formes.png}
\end{minipage}

\end{frame}


\begin{frame}[t]
\frametitle{Enrichir un document : Illustrations}
\framesubtitle{Outils image}

\hgraphpage{illustration-image-outils.png}

\begin{minipage}{0.56\textwidth}
\begin{itemize}
	\item Onglet : \optword{Outils Image : Mise en forme}
	\item Lorsqu'on sélectionne une image, cet onglet apparaitra
	\begin{itemize}
		\item \optword{Ajuster} : réglage de l'image
		\item \optword{Style d'image} : bordure et forme
		\item \optword{Organiser} : position dans la page et position par rapport au texte (habillage)
		\item \optword{Taille} : longueur et largeur, rogner (supprimer des marges)
	\end{itemize}
\end{itemize}
\end{minipage}
\begin{minipage}{0.22\textwidth}
	\hgraphpage{illustrations-habillage.png}
\end{minipage}
\begin{minipage}{0.20\textwidth}	
	\hgraphpage{illustrations-position.png}
\end{minipage}

\end{frame}

\subsection{Liens et renvoi}

\begin{frame}[t]
\frametitle{Enrichir un document : Liens et renvoi}
\framesubtitle{Insérer un lien hypertexte}

\begin{minipage}{0.38\textwidth}
	\begin{itemize}
		\item Onglet : \optword{Insertion}
		\item Groupe : \optword{Liens}
		\item Cliquer sur \optword{Lien}
		\item Insérer le liens
		\item Insérer le texte à afficher
	\end{itemize}
\end{minipage}
\begin{minipage}{0.6\textwidth}	
	\begin{flushright}
		\hgraphpage[1cm]{liens.png}
	\end{flushright}
	
	\vspace{-12pt}
	\hgraphpage{liens-hypertexte.png}
\end{minipage}

\end{frame}

\begin{frame}[t]
\frametitle{Enrichir un document : Liens et renvoi}
\framesubtitle{Renvoi vers un emplacement dans le document}

\begin{minipage}{0.38\textwidth}
	\begin{itemize}
		\item Onglet : \optword{Insertion}
		\item Groupe : \optword{Liens}
		\item Cliquer sur l'emplacement de destination
		\item Cliquer sur \optword{Signet} et donner un nom
		\item Cliquer sur \optword{Lien}
		\item Choisir \optword{Signet}
		\item Insérer le texte à afficher
	\end{itemize}
\end{minipage}
\begin{minipage}{0.6\textwidth}	
	\begin{flushright}
		\begin{tabular}{@{}lll@{}}
		\hgraphpage[.40\textwidth, valign=t]{liens-signet.png} & 
		&
		\hgraphpage[1cm, valign=t]{liens.png} \\
	\end{tabular}
	\end{flushright}
	
	\vspace{-12pt}
	
	\hgraphpage{liens-renvoi.png}
\end{minipage}

\end{frame}

\subsection{Formules et symboles}

\begin{frame}
\frametitle{Enrichir un document : Formules et symboles}
\framesubtitle{Insérer une formule}

\hgraphpage{equation-barre.png}

\begin{minipage}{0.76\textwidth}
\begin{itemize}
	\item Onglet : \optword{Insertion}
	\item Groupe : \optword{Symboles}
	\item \'Equations prédéfinies ou nouvelles
	\item Un nouveau onglet apparaitra : \optword{Outils d'équation : Conception}
	\begin{itemize}
		\item Il existe des structures et des symboles pour formuler l'équation 
		\item On peut, aussi, introduire l'équation en utilisant la saisie manuelle
		\item Une nouvelle fonctionnalité de Word 2019 est les équations en utilisant \LaTeX
	\end{itemize}
\end{itemize}
\end{minipage}
\begin{minipage}{0.23\textwidth}
	\hgraphpage{equation.png}
		
	\hgraphpage{equation-manuelle.png}
\end{minipage}

\end{frame}

\begin{frame}
\frametitle{Enrichir un document : Formules et symboles}
\framesubtitle{Insérer des symboles}

\begin{minipage}{0.50\textwidth}
\begin{itemize}
	\item Onglet : \optword{Insertion}
	\item Groupe : \optword{Symboles}
	\item On peut choisir une symbole parmi celles les plus utilisées
	\item On peut naviguer la table des caractères
\end{itemize}
\end{minipage}
\begin{minipage}{0.49\textwidth}
	\hgraphpage[.3\textwidth]{symboles.png}
	
	\hgraphpage{symboles-table.png}
\end{minipage}

\end{frame}

%===================================================================================
\section{Références}
%===================================================================================

%\begin{frame}
%\frametitle{Références}
%
%\end{frame}

\subsection{Notes de bas de page}

\begin{frame}[t]
\frametitle{Références : Notes de bas de page}
\framesubtitle{Insérer une note de bas de page}

\begin{minipage}{0.69\textwidth}
\begin{itemize}
	\item Onglet : \optword{Références}
	\item Groupe : \optword{Notes de bas de page}
\end{itemize}
\end{minipage}
\begin{minipage}{0.30\textwidth}
	\hgraphpage{bas-page.png}
\end{minipage}

\begin{itemize}
	\item Sélectionner le mot concerné par la note
	\item Cliquer sur \optword{Insérer une note de bas de page}
	\item Rédiger la note
\end{itemize}

\end{frame}

\subsection{Tables (de matières et des illustrations)}

\begin{frame}
\frametitle{Références : Tables (de matières et des illustrations)}
\framesubtitle{Table de matières : Insérer une table de matières}

\begin{minipage}{0.59\textwidth}
	\begin{itemize}
		\item Onglet : \optword{Références}
		\item Groupe : \optword{Légendes}
		\item Mettre le curseur dans la position où on veut insérer la table
		\item Cliquer sur \optword{Insérer une table des illustrations}
		\item Choisir le type de la légende: Figure ou Tableau
		\item Choisir le style
	\end{itemize}
\end{minipage}
\begin{minipage}{0.40\textwidth}
	\hgraphpage{table-matiere.png}
\end{minipage}

\end{frame}

\begin{frame}
\frametitle{Références : Tables (de matières et des illustrations)}
\framesubtitle{Table de matières : Insérer une table de matières personnalisée}

\begin{minipage}{0.59\textwidth}
	\begin{itemize}
		\item On peut choisir une table de matière personnalisée
		\item Configurer le style de la table 
		\item Choisir les niveaux à afficher dans la table
		\item Dans \optword{Options}, on peut attribuer des niveaux aux différents styles. 
		Par exemple, si on a créé un style \expword{monTitre} on peut l'attribuer le niveau \expword{1} pour qu'il soit un titre de niveau 1.
	\end{itemize}
\end{minipage}
\begin{minipage}{0.40\textwidth}
	\hgraphpage{table-matiere-perso.png}
	
	\hgraphpage[.6\textwidth]{table-matiere-options.png}
\end{minipage}

\end{frame}

%\subsection{Table des illustrations}

\begin{frame}[t]
\frametitle{Références : Tables (de matières et des illustrations)}
\framesubtitle{Table des illustrations: Insérer des légendes}

\begin{minipage}{0.61\textwidth}
	\begin{itemize}
		\item Onglet : \optword{Références}
		\item Groupe : \optword{Légendes}
		\item Sélectionner l'illustration (tableau ou image)
		\item Cliquer sur \optword{Insérer une légende}
		\item Choisir le type de la légende: Figure ou Tableau
		\item Ajouter un titre à la légende
		\item Pour changer le style de numérotation, appuyer sur \optword{Numérotation}
	\end{itemize}
\end{minipage}
\begin{minipage}{0.38\textwidth}
	\hgraphpage[.6\textwidth]{legende-barre.png}
	
	\hgraphpage{legende.png}
	
	\hgraphpage[.8\textwidth]{legende-num.png}
\end{minipage}

\end{frame}

\begin{frame}[t]
\frametitle{Références : Tables (de matières et des illustrations)}
\framesubtitle{Table des illustrations : Insérer une table des illustrations}

\begin{minipage}{0.61\textwidth}
	\begin{itemize}
		\item Onglet : \optword{Références}
		\item Groupe : \optword{Légendes}
		\item Mettre le curseur dans la position où on veut insérer la table
		\item Cliquer sur \optword{Insérer une table des illustrations}
		\item Choisir le type de la légende: Figure ou Tableau
		\item Choisir le style
	\end{itemize}
\end{minipage}
\begin{minipage}{0.38\textwidth}
	\hgraphpage[.6\textwidth]{legende-barre.png}
	
	\hgraphpage{table-illustrations.png}
	
\end{minipage}

\end{frame}

\subsection{Index}

\begin{frame}[t]
\frametitle{Références : Index}
\framesubtitle{Insérer des entrées}

\begin{minipage}{0.69\textwidth}
\begin{itemize}
	\item Onglet : \optword{Références}
	\item Groupe : \optword{Index}
	\item Pour créer un index, il faut marquer des entrées 
	\item Sélectionner le mot à indexer 
	\item Cliquer sur \optword{Marquer entrée}
	\item On peut marquer ce mot, ou toutes ses occurrences dans le texte.
	\item Pour le renvoi, mettre le curseur de la  souris n'import où dans le document 
	\item Cliquer sur \optword{Marquer entrée} et choisir \optword{Renvoi}
\end{itemize}
\end{minipage}
\begin{minipage}{0.30\textwidth}
\begin{flushright}
	\hgraphpage[.7\textwidth]{index-barre.png}
\end{flushright}
\vspace{-6pt}
\hgraphpage{entree.png}

\end{minipage}

\end{frame}

\begin{frame}[t]
\frametitle{Références : Index}
\framesubtitle{Insérer un index}

\begin{minipage}{0.61\textwidth}
\begin{itemize}
	\item Onglet : \optword{Références}
	\item Groupe : \optword{Index}
	\item Après avoir marquer les entrées
	\item Cliquer sur la page où on veut insérer l'index
	\item Cliquer sur \optword{Insérer l'index}
	\item Configurer le style de l'index
\end{itemize}
\end{minipage}
\begin{minipage}{0.38\textwidth}
\begin{flushright}
	\hgraphpage[.6\textwidth]{index-barre.png}
\end{flushright}
\vspace{-6pt}
\hgraphpage{index.png}

\end{minipage}

\end{frame}

\subsection{Bibliographie}

\begin{frame}[t]
\frametitle{Références : Bibliographie}
\framesubtitle{Gestion des sources}

\begin{minipage}{0.48\textwidth}
	\begin{itemize}
		\item Onglet : \optword{Références}
		\item Groupe : \optword{Citations et Bibliographie}
		\item On peut ajouter, modifier ou supprimer une source  
		\item Une source peut être un livre, article de journal, rapport, etc.
	\end{itemize}
\end{minipage}
\begin{minipage}{0.50\textwidth}
	\hgraphpage[2cm]{biblio-barre.png}
	
	\hgraphpage{biblio-source.png}
	
	\hgraphpage[.8\textwidth]{biblio-source-ajouter.png}
	
\end{minipage}

\end{frame}

\begin{frame}[t]
\frametitle{Références : Bibliographie}
\framesubtitle{Insérer une citation}

\begin{minipage}{0.68\textwidth}
	\begin{itemize}
		\item Onglet : \optword{Références}
		\item Groupe : \optword{Citations et Bibliographie}
		\item Mettre le curseur à la fin de la phrase concernée par la citation
		\item Cliquer sur \optword{Insérer une citation}
		\item Choisir la source appropriée parmi la liste affichée
	\end{itemize}
\end{minipage}
\begin{minipage}{0.30\textwidth}
	\hgraphpage{biblio-citation.png}
	
%	\hgraphpage{biblio-source.png}
%	
%	\hgraphpage[.8\textwidth]{biblio-source-ajouter.png}
	
\end{minipage}

\end{frame}

\begin{frame}[t]
\frametitle{Références : Bibliographie}
\framesubtitle{Insérer la bibliographie}

\begin{minipage}{0.38\textwidth}
	\begin{itemize}
		\item Onglet : \optword{Références}
		\item Groupe : \optword{Citations et Bibliographie}
		\item Insérer la table bibliographique dans une page
		\item Modifier le style de la bibliographie
	\end{itemize}
\end{minipage}
\begin{minipage}{0.35\textwidth}
	\hgraphpage{biblio-biblio.png}
\end{minipage}
\begin{minipage}{0.25\textwidth}
	\hgraphpage{biblio-style.png}
	\vspace{1cm}
\end{minipage}

\end{frame}

%===================================================================================
\section{Révision et partage}
%===================================================================================

%\begin{frame}
%\frametitle{Révision et partage}
%
%\end{frame}

\subsection{Révision}

\begin{frame}[t]
\frametitle{Révision et partage : Révision}
\framesubtitle{Vérification de l'orthographe}

\begin{minipage}{0.74\textwidth}
	\begin{itemize}
		\item Onglet : \optword{Révision}
		\item Groupe : \optword{Vérification}
		\item On peut activer la vérification de la langue 
		\item Un volet de vérification d'orthographe s'apparaitra
		\item On peut ignorer ou choisir une correction des suggestions
	\end{itemize}
\end{minipage}
\begin{minipage}{0.25\textwidth}
	\begin{flushright}
		\hgraphpage[3cm]{revision-verification.png}
	\end{flushright}
	
	\vspace{-12pt}
	
	\hgraphpage{verification-orthographe.png}
	
\end{minipage}

\end{frame}

\begin{frame}[t]
\frametitle{Révision et partage : Révision}
\framesubtitle{Chercher les synonymes}

\begin{minipage}{0.64\textwidth}
	\begin{itemize}
		\item Onglet : \optword{Révision}
		\item Groupe : \optword{Vérification}
		\item On peut choisir le dictionnaire des synonymes  
		\item Un volet de synonymes s'apparaitra
		\item On peut chercher des synonymes à un mot et choisir des mots à insérer dans le texte 
	\end{itemize}
\end{minipage}
\begin{minipage}{0.35\textwidth}
	\begin{flushright}
		\hgraphpage[3cm]{revision-verification.png}
	\end{flushright}

	\vspace{-12pt}
	
	\hgraphpage{verification-synonymes.png}
	
\end{minipage}

\end{frame}

%\subsection{Commentaires et suivi}

\begin{frame}[t]
\frametitle{Révision et partage : Révision}
\framesubtitle{Commentaires et suivi}

\begin{minipage}{0.39\textwidth}
	\begin{itemize}
		\item Onglet : \optword{Révision}
		\item Groupes : \optword{Commentaires}, \optword{Suivi} et \optword{Modifications}
	\end{itemize}
\end{minipage}
\begin{minipage}{0.6\textwidth}
	\hgraphpage{revision-suivi.png}
\end{minipage}

\begin{minipage}{0.69\textwidth}
	\begin{itemize}
		\item Sélectionner un texte et ajouter un commentaire 
		\item Activer le suivi des modifications
		\item Afficher les marques de suivi simples ou toutes les marques
		\item Accepter et Refuser des modifications
	\end{itemize}
\end{minipage}
\begin{minipage}{0.30\textwidth}
	\hgraphpage[.4\textwidth]{revision-suivi-marques.png}
	
	\hgraphpage{revision-suivi-accepter.png}
	
	\hgraphpage{revision-suivi-refuser.png}
	
\end{minipage}

\end{frame}

%\subsection{Comparaison}

\begin{frame}
\frametitle{Révision et partage : Révision}
\framesubtitle{Comparer entre deux documents}

\begin{minipage}{0.41\textwidth}
	\begin{itemize}
		\item Onglet : \optword{Révision}
		\item Groupe : \optword{Comparer}
		\item On peut comparer entre deux documents : l'original et celui modifié 
	\end{itemize}
\end{minipage}
\begin{minipage}{0.58\textwidth}
%	\hgraphpage{revision-suivi.png}
	
	\hgraphpage{revision-comparer.png}
	
\end{minipage}

\end{frame}

\subsection{Partage}

%\begin{frame}
%\frametitle{Révision et partage}
%\framesubtitle{Partage}
%
%
%\end{frame}

\begin{frame}
\frametitle{Révision et partage : Partage}
\framesubtitle{Informations et Protection}

\begin{minipage}{0.41\textwidth}
	\begin{itemize}
		\item Onglet : \optword{Fichier}
		\item Menu : \optword{Informations}
		\item On peut modifier des informations du fichier, comme le titre, les mots clés, l'auteur, etc. 
		\item On peut protéger le document
		\item \optword{Restreindre la modification} peut être, aussi, accédée à partir de l'onglet \optword{Révision} le groupe \optword{Protéger}
	\end{itemize}
\end{minipage}
\begin{minipage}{0.58\textwidth}
	%	\hgraphpage{revision-suivi.png}
	\hgraphpage{partage-informations.png}
\end{minipage}

\end{frame}

\begin{frame}
\frametitle{Révision et partage : Partage}
\framesubtitle{Impression}

\begin{minipage}{0.61\textwidth}
	\begin{itemize}
		\item Onglet : \optword{Fichier}
		\item Menu : \optword{Imprimer}
		\item Clavier : \optword{Ctrl + P}
	\end{itemize}
\end{minipage}
\begin{minipage}{0.38\textwidth}
	%	\hgraphpage{revision-suivi.png}
	\hgraphpage{partage-imprimer.png}
\end{minipage}

\end{frame}

\begin{frame}
\frametitle{Révision et partage : Partage}
\framesubtitle{Sauvegarde}

\begin{minipage}{0.61\textwidth}
	\begin{itemize}
		\item Onglet : \optword{Fichier}
		\item Menu : \optword{Imprimer}
		\item Clavier : \optword{Ctrl + S}
		\item \optword{Enregistrer} et \optword{Enregistrer sous} font la même chose si le fichier est nouveau 
		\item On peut sauvegarder le document sous forme d'un modèle 
	\end{itemize}
\end{minipage}
\begin{minipage}{0.38\textwidth}
	%	\hgraphpage{revision-suivi.png}
	\hgraphpage{partage-enregistrer.png}
\end{minipage}

\end{frame}

\begin{frame}
\frametitle{Révision et partage : Partage}
\framesubtitle{Exporter et Partager}

\begin{minipage}{0.38\textwidth}
	\begin{itemize}
		\item Onglet : \optword{Fichier}
		\item Menu : \optword{Exporter} et \optword{Partager}
	\end{itemize}
\end{minipage}
\begin{minipage}{0.30\textwidth}
	\hgraphpage{partage-exporter.png}
\end{minipage}
\begin{minipage}{0.30\textwidth}
	\hgraphpage{partage-partager.png}
\end{minipage}

\end{frame}

\begin{frame}
\frametitle{Rédaction d'un document numérique : Microsoft Word}
\framesubtitle{Un peu d'humour}

\hgraphpage{word-humor.jpg}

\end{frame}

\insertbibliography{Bweb03}{*}


\end{document}

