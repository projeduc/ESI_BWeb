% !TEX TS-program = pdflatex
% !TeX program = pdflatex
% !TEX encoding = UTF-8
% !TEX spellcheck = fr

\documentclass[xcolor=table]{beamer}


%\usepackage{fullpage}
%\usepackage[left=2.8cm,right=2.2cm,top=2 cm,bottom=2 cm]{geometry}
\setbeamersize{text margin left=10pt,text margin right=10pt}
\usepackage{amsmath,amssymb} 
\usepackage[T1]{fontenc}
\usepackage[utf8]{inputenc}
\usepackage[english,french]{babel}
\usepackage{txfonts}
\usepackage[]{graphicx}
\usepackage{multirow}
\usepackage{hyperref}
\usepackage{colortbl}
\usepackage{listings}
\usepackage{wrapfig}
\usepackage{multicol}

\hypersetup{
	colorlinks,
	urlcolor = blue
}

%\renewcommand{\baselinestretch}{1.5}

\def\supit#1{\raisebox{0.8ex}{\small\it #1}\hspace{0.05em}}

\AtBeginSection{%
	\begin{frame}
		\sectionpage
	\end{frame}
}

\newcommand{\rottext}[2]{%
	\rotatebox{90}{%
	\begin{minipage}{#1}%
		\raggedleft#2%
	\end{minipage}%
	}%
}

\usepackage{longtable}
\usepackage{tabu}


\institute{ %
École  nationale Supérieure d'Informatique (ESI, ex. INI), Algérie
}
\author[ \textbf{\footnotesize  \insertframenumber/\inserttotalframenumber} \hspace*{\fill} ESI (2019-2020)] %
{ARIES Abdelkrime}
%\titlegraphic{\includegraphics[height=1cm]{../img/esi-logo.png}%\hspace*{4.75cm}~


\date{Année unniversitaire: 2019/2020} %\today

\usetheme{Warsaw} % Antibes Boadilla Warsaw

\beamertemplatenavigationsymbolsempty

%\setbeamertemplate{headline}{}

\definecolor{lightblue}{HTML}{D0D2FF}
\definecolor{lightyellow}{HTML}{FFFFAA}
\definecolor{darkblue}{HTML}{0000BB}
\definecolor{olivegreen}{HTML}{006600}
\definecolor{violet}{HTML}{6600CC}

\newcommand{\keyword}[1]{\textcolor{red}{\bfseries\itshape #1}}
\newcommand{\expword}[1]{\textcolor{olivegreen}{#1}}
\newcommand{\optword}[1]{\textcolor{violet}{\bfseries #1}}

\makeatletter
\newcommand\mysphere{%
	\parbox[t]{10pt}{\raisebox{0.2pt}{\beamer@usesphere{item projected}{bigsphere}}}}
\makeatother

%\let\oldtabular\tabular
%\let\endoldtabular\endtabular
%\renewenvironment{tabular}{\rowcolors{2}{white}{lightblue}\oldtabular\rowcolor{blue}}{\endoldtabular}


\NoAutoSpacing %french autospacing after ":"

\title[BWEB : 07- HTML] %
{Bureautique et Web \\Chapitre 07 : Développement Web\\ \slshape\small  HTML}  

\changegraphpath{../img/dev-web/}

\begin{document}

\begin{frame}
\frametitle{HTML}
\framesubtitle{Introduction}

\begin{itemize}
	\item HTML : HyperText Markup Language
	\item un langage de balises  
	\item il sert à structurer le contenu des pages web
\end{itemize}

\end{frame}

\begin{frame}
\frametitle{HTML}
\framesubtitle{Balise}

\begin{itemize}
	\item Une balise dans HTML s'écrit comme \expword{<nomBalise>}
	\item Un élément HTML se compose d'une balise ouvrante ( \expword{<nomBalise>}), d'un contenu textuel et d'une balise fermante (\expword{</nomBalise>})
	\item Il existe des balises auto-fermantes ; c-à-d, elles ne contiennent pas du contenu (pas de \expword{</nomBalise>})
	\item Les balises peuvent avoir des attributs \\\expword{<nomBalise attribut1 = "valeur1" >}
	\item HTML ne fait pas la différence entre majuscule et minuscule
\end{itemize}

\end{frame}


\begin{frame}
\frametitle{HTML}
\framesubtitle{Plan}

\begin{multicols}{2}
%	\small
\tableofcontents
\end{multicols}
\end{frame}

%===================================================================================
\section{Structure et éléments}
%===================================================================================

\subsection{Structure}

\begin{frame}[fragile]
\frametitle{HTML : Structure et éléments}
\framesubtitle{Structure}

\begin{minipage}{0.60\textwidth} 
\begin{itemize}
	\item \optword{Le doctype} : marqué par \keyword{<!DOCTYPE html>}. Il indique que le document est écrit avec du HTML standard.
	\item \optword{L'entête} : marquée par \keyword{<html>}. Elle fournit des informations générales (métadonnées) sur le document. Il ne peut y avoir qu'une seule entête par document
	\item \optword{Le corps} : marqué par \keyword{<body>}. Il représente le contenu principal du document HTML. Il ne peut y avoir qu'un seul corps par document
\end{itemize}
\end{minipage}
%
\begin{minipage}{0.38\textwidth}
\begin{block}{Structure d'un document HTML}
\lstset{escapeinside=**}
\scriptsize\bfseries
\begin{lstlisting}[language={html}]
<!DOCTYPE html>
<html>
  <head>
    <title>
      Exemple de HTML
    </title>
  </head>
  <body>
    Bonjour tout le monde!
  </body>
</html>
\end{lstlisting}
\end{block}
\end{minipage}

\end{frame}

\begin{frame}
\frametitle{HTML : Structure et éléments}
\framesubtitle{Structure : Les commentaires}

\begin{itemize}
	\item ils commencent par \keyword{<!- -} et terminent par \keyword{- ->}
	\item ils peuvent prendre plusieurs lignes
	\item leur intérêt : 
	\begin{itemize}
		\item commenter le code non utiliser
		\item ajouter des explications au code
	\end{itemize}
\end{itemize}

\end{frame}

\begin{frame}[fragile]
\frametitle{HTML : Structure et éléments}
\framesubtitle{Structure : L'entête}

\begin{minipage}{0.44\textwidth} 
\begin{itemize}
	\item \keyword{<title>} : le titre est affiché dans la barre de titre du navigateur ou dans l'onglet de la page.
	\item \keyword{<link>} : définit la relation entre le document courant et une ressource externe (on va voir ça dans CSS)
	
\end{itemize}
\end{minipage}
%
\begin{minipage}{0.55\textwidth}
\begin{exampleblock}{Exemple d'une entête}
\lstset{escapeinside=**}
\scriptsize\bfseries
\begin{lstlisting}[language={html}]
<head>
  <title>Titre</title>
  <!-- *définir l'icon du site* -->
  <link rel="icon" href="favicon.ico">
  <!-- *encodage des caractères* -->
  <meta charset="utf-8">
  <!-- *Rediriger la page après 3 secondes* -->
  <meta http-equiv="refresh" 
        content="3;url=http://esi.dz/">
</head>
\end{lstlisting}
\end{exampleblock}
\end{minipage}

\begin{itemize}
	\item \keyword{<meta>} : une donnée qui décrit une donnée. 
	Par exemple, \expword{l'auteur de la page}, \expword{son encodage}, etc.
	\item des scripts (\keyword{<script>}) et des styles (\keyword{<style>})
\end{itemize}

\end{frame}

\begin{frame}
\frametitle{HTML : Structure et éléments}
\framesubtitle{Structure : Le corps}

\begin{itemize}
	\item Le corps 
	\item 
	\item 
\end{itemize}

\end{frame}


\subsection{Formatage de texte}

\begin{frame}
\frametitle{HTML : Structure et éléments}
\framesubtitle{Formatage de texte : Balises de mise en forme}

\begin{itemize}
	\item \keyword{<i>} : un texte en italique
	\item \keyword{<b>} : un texte en gras
	\item \keyword{<em>} : un texte en emphase (important). En général, italique
	\item \keyword{<strong>} : un texte en emphase fort (très important). En général, gras
	\item \keyword{<p>} : un paragraphe
	\item \keyword{<h1>} jusqu'à \keyword{<h6>} : des titres de niveau 1 jusqu'à niveau 6
	\item \keyword{<pre>} : un texte préformaté ; le texte reste tel qu'il est, avec les espaces et les retours à la ligne
	\item \keyword{<sup>} : un exposant
	\item \keyword{<sub>} : une indice
	\item \textit{un exemple dans "formatage.html"}
\end{itemize}

\end{frame}

\begin{frame}[fragile]
\frametitle{HTML : Structure et éléments}
\framesubtitle{Formatage de texte : Citation}

\begin{minipage}{0.44\textwidth} 
	\begin{itemize}
		\item \keyword{<q>} : une citation courte. 
		La plupart des navigateurs modernes entoure le texte de cet élément avec des marques de citation.
		\item \keyword{<blockquote>} : une citation longue. 
		Le texte est généralement affiché avec une indentation.
	\end{itemize}
\end{minipage}
%
\begin{minipage}{0.55\textwidth}
\begin{exampleblock}{Exemple : formatage.html}
\lstset{escapeinside=**}
\scriptsize\bfseries
\begin{lstlisting}[language={html}]
Mahatma Gandhi a dit :
<q>
  Be the change that you wish to see 
  in the world.
</q>
Albert Einstein a comment*é* sur 
l'*ê*tre humain :
<blockquote>
  Two things are infinite: the universe 
  and human stupidity; 
  and I'm not sure about the universe.
</blockquote>
\end{lstlisting}
\end{exampleblock}
\end{minipage}

\end{frame}

\subsection{Listes}

\begin{frame}[fragile]
\frametitle{HTML : Structure et éléments}
\framesubtitle{Listes : Liste à puces}

\begin{minipage}{0.60\textwidth} 
	\begin{itemize}
		\item définie par la balise \keyword{<ul>} (unordered list)
		\item ses éléments sont définis par la balise \keyword{<li>}
		\item on peut créer une liste à l'intérieur d'une autre
	\end{itemize}
\end{minipage}
%
\begin{minipage}{0.38\textwidth}
\begin{exampleblock}{exemple: liste.html}
\lstset{escapeinside=**}
\scriptsize\bfseries
\begin{lstlisting}[language={html}]
<ul>
  <li>*élément 1*</li>
  <li>*élément 2* :
    <ul>
      <li>*élément 2-1*</li>
      <li>*élément 2-2*</li>
      <li>*élément 2-3*</li>
    </ul>
  </li>
  <li>*élément 3*</li>
</ul>
\end{lstlisting}
\end{exampleblock}
\end{minipage}

\end{frame}

\begin{frame}[fragile]
\frametitle{HTML : Structure et éléments}
\framesubtitle{Listes : Liste numérotée}

\begin{minipage}{0.60\textwidth} 
	\begin{itemize}
		\item définie par la balise \keyword{<ol>} (ordered list)
		\item ses éléments sont définis par la balise \keyword{<li>}
		\item on peut créer une liste à l'intérieur d'une autre
		\item on peut commencer la numérotation à partir d'un nombre en utilisant l'attribut \keyword{start}
	\end{itemize}
\end{minipage}
%
\begin{minipage}{0.38\textwidth}
\begin{exampleblock}{exemple: liste.html}
\lstset{escapeinside=**}
\scriptsize\bfseries
\begin{lstlisting}[language={html}]
<ol start="5">
  <li>*élément* 1</li>
  <li>*élément* 2</li>
  <li>*élément* 3</li>
</ol>
\end{lstlisting}
\end{exampleblock}
\end{minipage}

\end{frame}

\begin{frame}[fragile]
\frametitle{HTML : Structure et éléments}
\framesubtitle{Listes : Liste de définition}

\begin{minipage}{0.50\textwidth} 
	\begin{itemize}
		\item définie par la balise \keyword{<dl>} (definition list)
		\item dedans il y a une séquence de termes et descriptions
		\item un terme est défini par la balise \keyword{<dt>} (definition term)
		\item une description est définie par la balise \keyword{<dd>} (definition description)
	\end{itemize}
\end{minipage}
%
\begin{minipage}{0.49\textwidth}
\begin{exampleblock}{exemple: liste.html}
\lstset{escapeinside=**}
\scriptsize\bfseries
\begin{lstlisting}[language={html}]
<dl>
  <dt>HTML</dt>
  <dd>HyperText Markup Language</dd>

  <dt>CSS</dt>
  <dd>Cascading Style Sheets</dd>
</dl>
\end{lstlisting}
\end{exampleblock}
\end{minipage}

\end{frame}

\subsection{Tableaux}

\begin{frame}[fragile]
\frametitle{HTML : Structure et éléments}
\framesubtitle{Tableaux : Éléments d'un tableau}

\begin{minipage}{0.60\textwidth} 
	\begin{itemize}
		\item un tableau est défini par la balise \keyword{<table>}
		\item dans un tableau, il y a plusieurs lignes
		\item une ligne est définie par la balise \keyword{<tr>}
		\item chaque ligne contient des cellules
		\item une cellule est définie par la balise \keyword{<td>}
		\item dans la première ligne, on remplace \keyword{<td>} par \keyword{<th>} pour désigner qu'il s'agit des titres du tableau
	\end{itemize}
\end{minipage}
%
\begin{minipage}{0.39\textwidth}
\begin{exampleblock}{exemple: tableau.html}
\lstset{escapeinside=**}
\scriptsize\bfseries
\begin{lstlisting}[language={html}]
<table>
  <tr>
    <td>L1C1</td>
    <td>L1C2</td>
    <td>L1C3</td>
  </tr>
  <tr>
    <td>L2C1</td>
    <td>L2C2</td>
    <td>L2C3</td>
  </tr>
</table>
\end{lstlisting}
\end{exampleblock}
\end{minipage}

\end{frame}

\begin{frame}[fragile]
\frametitle{HTML : Structure et éléments}
\framesubtitle{Tableaux : Accessibilité}

\begin{minipage}{0.50\textwidth} 
	\begin{itemize}
		\item pour décrire un tableau, on utilise une légende : la balise \keyword{<caption>}
		\item on peut regrouper les lignes de l'entête d'un tableau dans une balise \keyword{<thead>}
		\item le corps du tableau peut être regroupé dans la balise \keyword{<tbody>}
		\item on peut spécifier le type du titre en utilisant l'attribut \keyword{scope} : un titre de colonnes (\keyword{scope="col"}) ou un titre des lignes (\keyword{scope="row"})
	\end{itemize}
\end{minipage}
%
\begin{minipage}{0.49\textwidth}
\begin{exampleblock}{exemple: tableau.html}
\lstset{escapeinside=**}
\tiny\bfseries\vspace{-6pt}
\begin{lstlisting}[language={html}]
<table>
  <caption>Notes des *étudiants*</caption>
  <thead>
    <tr>
      <th scope="col">Nom</th>
      <th scope="col">BWEB</th>
      <th scope="col">ALSDS</th>
    </tr>
  </thead>
  <tbody>
    <tr>
      <th scope="row">Etudiant1</th>
      <td>16</td>
      <td>14</td>
    </tr>
    <tr>
      <th scope="row">Etudiant2</th>
      <td>13</td>
      <td>17</td>
    </tr>
  </tbody>
</table>
\end{lstlisting}\vspace{-6pt}
\end{exampleblock}
\end{minipage}

\end{frame}


\begin{frame}[fragile]
\frametitle{HTML : Structure et éléments}
\framesubtitle{Tableaux : Fusionner les cellules}

\begin{minipage}{0.50\textwidth} 
	\begin{itemize}
		\item l'attribut \keyword{rowspan} : une cellule prend plusieurs lignes 
		\item l'attribut \keyword{colspan} : une cellule prend plusieurs colonnes
		\item lorsqu'une cellule prend plusieurs lignes (colonnes), on doit la compter comme si plusieurs cellules
		\item on peut omettre \keyword{rowspan="1"} et \keyword{colspan="1"}
	\end{itemize}
\end{minipage}
%
\begin{minipage}{0.49\textwidth}
\begin{exampleblock}{exemple: tableau.html}
\lstset{escapeinside=**}
\scriptsize\bfseries\vspace{-6pt}
\begin{lstlisting}[language={html}]
<table>
  <tr>
    <td colspan="2">L1C1</td>
    <!-- L1C2 est prise par L1C1 -->
    <td rowspan="2">L1C3</td>
  </tr>
  <tr>
    <td>L2C1</td>
    <td>L2C2</td>
    <!-- L2C3 est prise par L1C3 -->
  </tr>    
</table>
\end{lstlisting}\vspace{-6pt}
\end{exampleblock}
\hgraphpage[.5\textwidth]{html/table-fusionner.png}
\end{minipage}

\end{frame}

%===================================================================================
\section{Liens et multimédia}
%===================================================================================

\subsection{Liens et intégration}

\begin{frame}[fragile]
\frametitle{HTML : Liens et multimédia}
\framesubtitle{Liens et intégration : Lien hypertexte}

\begin{minipage}{0.50\textwidth} 
	\begin{itemize}
		\item Un lien hypertexte est défini par la balise \keyword{<a>} 
		\item la destination est spécifiée par l'attribut \keyword{href}
		\item pour ouvrir la page dans une nouvelle onglet, on utilise \keyword{target="\_blank"}
		\item pour spécifier que le lien est un émail, on utilise le mot clé \keyword{mailto:} avant l'adresse émail
	\end{itemize}
\end{minipage}
%
\begin{minipage}{0.49\textwidth}
\begin{exampleblock}{exemple: lien.html}
\lstset{escapeinside=**}
\scriptsize\bfseries\vspace{-6pt}
\begin{lstlisting}[language={html}]
<a href="liste.html">
  Voir les listes
</a>
<br>
<a href="https://www.google.com" 
                target="_blank">
  Aller vers Google
</a>
<br>
<a href="mailto:ab_aries@esi.dz">
  Envoyer un message
</a>
\end{lstlisting}\vspace{-6pt}
\end{exampleblock}
\end{minipage}

\end{frame}

\begin{frame}[fragile]
\frametitle{HTML : Liens et multimédia}
\framesubtitle{Liens et intégration : Lien vers un emplacement dans la même page}

\begin{minipage}{0.50\textwidth} 
	\begin{itemize}
		\item on doit marquer l'emplacement dans le fichier HTML
		\item on attribue un identifiant à un élément en utilisant l'attribut \keyword{id}
		\item pour référencer cet identifiant dans un lien, on le précède par \keyword{\#}
%		\item pour référencer un élément dans une autre page, on met l'adresse de la page suivi par \keyword{\#} suivi par l'identifiant de cet élément
	\end{itemize}
\end{minipage}
%
\begin{minipage}{0.49\textwidth}
\begin{exampleblock}{exemple: lien.html}
\lstset{escapeinside=**}
\scriptsize\bfseries\vspace{-6pt}
\begin{lstlisting}[language={html}]
<a href="#def">
  Aller vers la *définition*
</a> 
...
<h1 id="def">*Définition*</h1>
\end{lstlisting}\vspace{-6pt}
\end{exampleblock}
\end{minipage}

\begin{itemize}
	\item pour référencer un élément dans une autre page, on met l'adresse de la page suivi par \keyword{\#} suivi par l'identifiant de cet élément
\end{itemize}

\end{frame}

\begin{frame}[fragile]
\frametitle{HTML : Liens et multimédia}
\framesubtitle{Liens et intégration : Afficher une page dans une autre}

\begin{minipage}{0.50\textwidth} 
	\begin{itemize}
		\item Pour insérer une page dans une autre, on utilise la balise \keyword{<iframe>}
		\item l'attribut \keyword{width} spécifie la longueur 
		\item \keyword{height} spécifie l'hauteur 
		\item \keyword{src} spécifie l'URL de la page à insérer
	\end{itemize}
\end{minipage}
%
\begin{minipage}{0.49\textwidth}
\begin{exampleblock}{exemple: lien.html}
\lstset{escapeinside=**}
\scriptsize\bfseries\vspace{-6pt}
\begin{lstlisting}[language={html}]
<iframe 
      width="300" 
      height="200" 
      src="liste.html">
</iframe>	
\end{lstlisting}\vspace{-6pt}
\end{exampleblock}
\end{minipage}

\end{frame}

\subsection{Images et graphiques}

\begin{frame}[fragile]
\frametitle{HTML : Liens et multimédia}
\framesubtitle{Images et graphiques : Insérer une image}

\begin{minipage}{0.50\textwidth} 
	\begin{itemize}
		\item Pour insérer une image dans une page, on utilise la balise \keyword{<img>}
		\item c'est une balise auto-fermante
		\item l'attribut \keyword{src} spécifie l'URL de l'image
%		\item \keyword{width} spécifie la longueur 
%		\item \keyword{height} spécifie l'hauteur
%		\item \keyword{alt} spécifie le texte à afficher si l'image est introuvable 
%		\item \keyword{title} spécifie le texte qui s'affiche lorsqu'on met le curseur de la souri sur l'image
	\end{itemize}
\end{minipage}
%
\begin{minipage}{0.49\textwidth}
\begin{exampleblock}{exemple: image.html}
\lstset{escapeinside=**}
\scriptsize\bfseries\vspace{-6pt}
\begin{lstlisting}[language={html}]
<img src="img/orange.jpg"
     width="200px" height="100px"
     alt="Un oranger" 
     title="Oranger et nuages" >
\end{lstlisting}\vspace{-6pt}
\end{exampleblock}
\end{minipage}

\begin{itemize}
%	\item Pour insérer une image dans une page, on utilise la balise \keyword{<img>}
%	\item c'est une balise auto-fermante
%	\item l'attribut \keyword{src} spécifie l'URL de l'image
	\item \keyword{width} spécifie la longueur 
	\item \keyword{height} spécifie l'hauteur
	\item \keyword{alt} spécifie le texte à afficher si l'image est introuvable 
	\item \keyword{title} spécifie le texte qui s'affiche lorsqu'on met le curseur de la souri sur l'image
\end{itemize}

\end{frame} 

% <figure> et <figcaption>
\begin{frame}[fragile]
\frametitle{HTML : Liens et multimédia}
\framesubtitle{Images et graphiques : Figures}

\begin{minipage}{0.47\textwidth} 
	\begin{itemize}
		\item une figure est un région où on met des images ou des schémas 
		\item elle est définie par la balise \keyword{<figure>}
		\item elle peut contenir une légende, en utilisant la balise \keyword{<figcaption>}
	\end{itemize}
\end{minipage}
%
\begin{minipage}{0.52\textwidth}
\begin{exampleblock}{exemple: image.html}
\lstset{escapeinside=**}
\scriptsize\bfseries\vspace{-6pt}
\begin{lstlisting}[language={html}]
<figure>
  <img src="img/orange.jpg"
       width="200px" height="100px"
       alt="Un oranger" 
       title="Oranger et nuages" >
  <figcaption>
    Oranger et nuages
  </figcaption>
</figure>
\end{lstlisting}\vspace{-6pt}
\end{exampleblock}
\end{minipage}

\end{frame}

% <picture> et <source>
\begin{frame}[fragile]
\frametitle{HTML : Liens et multimédia}
\framesubtitle{Images et graphiques : Images adaptatives}

\begin{minipage}{0.50\textwidth} 
	\begin{itemize}
		\item On veut insérer une image sur la page selon la taille de l'écran
		\item une solution est d'utiliser la balise \keyword{<picture>}
		\item dedans, on met les différentes sources en utilisant la balise \keyword{<source>}
%		\item pour chaque source, on peut spécifier la taille préférée en utilisant l'attribut \keyword{<media>}
%		\item on doit ajouter une balise \keyword{<img>} au cas où toutes les sources ne sont pas satisfaites
	\end{itemize}
\end{minipage}
%
\begin{minipage}{0.49\textwidth}
\begin{exampleblock}{exemple: image.html}
\lstset{escapeinside=**}
\scriptsize\bfseries\vspace{-6pt}
\begin{lstlisting}[language={html}]
<picture>
  <source media="(max-width: 799px)" 
      srcset="img/orange-petite.jpg">
  <source media="(min-width: 800px)" 
      srcset="img/orange.jpg">
  <img src="img/orange.jpg" 
      alt="Un oranger">
</picture>
\end{lstlisting}\vspace{-6pt}
\end{exampleblock}
\end{minipage}

\begin{itemize}
%	\item On veut insérer une image sur la page selon la taille de l'écran
%	\item une solution est d'utiliser la balise \keyword{<picture>}
%	\item dedans, on met les différentes sources en utilisant la balise \keyword{<source>}
	\item pour chaque source, on peut spécifier la taille préférée en utilisant l'attribut \keyword{<media>}
	\item on doit ajouter une balise \keyword{<img>} au cas où toutes les sources ne sont pas satisfaites
\end{itemize}

\end{frame}

%<map> et <area>
%\begin{frame}[fragile]
%\frametitle{HTML : Liens et multimédia}
%\framesubtitle{Images et graphiques : Régions cliquables dans l'image}
%
%\begin{minipage}{0.50\textwidth} 
%	\begin{itemize}
%		\item On peut définir des régions cliquables dans une image en utilisant la balise \keyword{<map>}
%	\end{itemize}
%\end{minipage}
%%
%\begin{minipage}{0.49\textwidth}
%	\begin{exampleblock}{exemple: lien.html}
%		\lstset{escapeinside=**}
%		\scriptsize\bfseries\vspace{-6pt}
%		\begin{lstlisting}[language={html}]
%		
%		\end{lstlisting}\vspace{-6pt}
%	\end{exampleblock}
%\end{minipage}
%
%\end{frame}


\subsection{Vidéos et audio}

%Vidéo 
%Audio
\begin{frame}[fragile]
\frametitle{HTML : Liens et multimédia}
\framesubtitle{Vidéos et audios}

\begin{minipage}{0.50\textwidth} 
	\begin{itemize}
		\item Pour insérer une vidéo, on utilise la balise \keyword{<video>}
		\item l'attribut \keyword{controls} indique que la vidéo inclue les contrôles par défaut du navigateur
		\item pour indiquer l'URL de la vidéo, on peut utiliser l'attribut \keyword{src} ou l'élément \keyword{<source>}
		\item l'attribut \keyword{loop} indique que la vidéo  joue en boucle
	\end{itemize}
\end{minipage}
%
\begin{minipage}{0.49\textwidth}
\begin{exampleblock}{exemple: video\_audio.html}
\lstset{escapeinside=**}
\scriptsize\bfseries\vspace{-6pt}
\begin{lstlisting}[language={html}]
<video controls>
  <source src="maVideo.mp4" 
     type="video/mp4">
  <source src="maVideo.webm" 
     type="video/webm">
  <p>
    Votre navigateur ne prend pas 
    en charge les vid*é*os HTML5.
  </p>
</video>
\end{lstlisting}\vspace{-6pt}
\end{exampleblock}
\end{minipage}

\end{frame}

%===================================================================================
\section{Disposition}
%===================================================================================

\subsection{Type des éléments}

%Block
%Inline

\subsection{Conteneurs}

%<div>
%<span>


\subsection{Éléments sémantiques}

%<header> - Defines a header for a document or a section
%<nav> - Defines a container for navigation links
%<section> - Defines a section in a document
%<article> - Defines an independent self-contained article
%<aside> - Defines content aside from the content (like a sidebar)
%<footer> - Defines a footer for a document or a section
%<details> - Defines additional details
%<summary> - Defines a heading for the <details> element

%===================================================================================
\section{Formulaires}
%===================================================================================


\end{document}

