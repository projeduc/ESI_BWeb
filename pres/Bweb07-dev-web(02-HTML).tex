% !TEX TS-program = pdflatex
% !TeX program = pdflatex
% !TEX encoding = UTF-8
% !TEX spellcheck = fr

\documentclass[xcolor=table]{beamer}


%\usepackage{fullpage}
%\usepackage[left=2.8cm,right=2.2cm,top=2 cm,bottom=2 cm]{geometry}
\setbeamersize{text margin left=10pt,text margin right=10pt}
\usepackage{amsmath,amssymb} 
\usepackage[T1]{fontenc}
\usepackage[utf8]{inputenc}
\usepackage[english,french]{babel}
\usepackage{txfonts}
\usepackage[]{graphicx}
\usepackage{multirow}
\usepackage{hyperref}
\usepackage{colortbl}
\usepackage{listings}
\usepackage{wrapfig}
\usepackage{multicol}

\hypersetup{
	colorlinks,
	urlcolor = blue
}

%\renewcommand{\baselinestretch}{1.5}

\def\supit#1{\raisebox{0.8ex}{\small\it #1}\hspace{0.05em}}

\AtBeginSection{%
	\begin{frame}
		\sectionpage
	\end{frame}
}

\newcommand{\rottext}[2]{%
	\rotatebox{90}{%
	\begin{minipage}{#1}%
		\raggedleft#2%
	\end{minipage}%
	}%
}

\usepackage{longtable}
\usepackage{tabu}


\institute{ %
École  nationale Supérieure d'Informatique (ESI, ex. INI), Algérie
}
\author[ \textbf{\footnotesize  \insertframenumber/\inserttotalframenumber} \hspace*{\fill} ESI (2019-2020)] %
{ARIES Abdelkrime}
%\titlegraphic{\includegraphics[height=1cm]{../img/esi-logo.png}%\hspace*{4.75cm}~


\date{Année unniversitaire: 2019/2020} %\today

\usetheme{Warsaw} % Antibes Boadilla Warsaw

\beamertemplatenavigationsymbolsempty

%\setbeamertemplate{headline}{}

\definecolor{lightblue}{HTML}{D0D2FF}
\definecolor{lightyellow}{HTML}{FFFFAA}
\definecolor{darkblue}{HTML}{0000BB}
\definecolor{olivegreen}{HTML}{006600}
\definecolor{violet}{HTML}{6600CC}

\newcommand{\keyword}[1]{\textcolor{red}{\bfseries\itshape #1}}
\newcommand{\expword}[1]{\textcolor{olivegreen}{#1}}
\newcommand{\optword}[1]{\textcolor{violet}{\bfseries #1}}

\makeatletter
\newcommand\mysphere{%
	\parbox[t]{10pt}{\raisebox{0.2pt}{\beamer@usesphere{item projected}{bigsphere}}}}
\makeatother

%\let\oldtabular\tabular
%\let\endoldtabular\endtabular
%\renewenvironment{tabular}{\rowcolors{2}{white}{lightblue}\oldtabular\rowcolor{blue}}{\endoldtabular}


\NoAutoSpacing %french autospacing after ":"

\title[BWEB : 07- HTML] %
{Bureautique et Web \\Chapitre 07 : Développement Web\\ \slshape\small  HTML}  

\changegraphpath{../img/dev-web/}

\begin{document}

\begin{frame}
\frametitle{HTML}
\framesubtitle{Introduction}

\begin{itemize}
	\item HTML : HyperText Markup Language
	\item un langage de balises  
	\item il sert à structurer le contenu des pages web
\end{itemize}

\end{frame}

\begin{frame}
\frametitle{HTML}
\framesubtitle{Balise}

\begin{itemize}
	\item Une balise dans HTML s'écrit comme \expword{<nomBalise>}
	\item Un élément HTML se compose d'une balise ouvrante ( \expword{<nomBalise>}), d'un contenu textuel et d'une balise fermante (\expword{</nomBalise>})
	\item Il existe des balises auto-fermantes ; c-à-d, elles ne contiennent pas du contenu (pas de \expword{</nomBalise>})
	\item Les balises peuvent avoir des attributs \\\expword{<nomBalise attribut1 = "valeur1" >}
	\item HTML ne fait pas la différence entre majuscule et minuscule
\end{itemize}

\end{frame}


\begin{frame}
\frametitle{HTML}
\framesubtitle{Plan}

\begin{multicols}{2}
%	\small
\tableofcontents
\end{multicols}
\end{frame}

%===================================================================================
\section{Structure et éléments}
%===================================================================================

\subsection{Structure}

\begin{frame}[fragile]
\frametitle{HTML : Structure et éléments}
\framesubtitle{Structure}

\begin{minipage}{0.60\textwidth} 
\begin{itemize}
	\item \optword{Le doctype} : marqué par \keyword{<!DOCTYPE html>}. Il indique que le document est écrit avec du HTML standard.
	\item \optword{L'entête} : marquée par \keyword{<html>}. Elle fournit des informations générales (métadonnées) sur le document. Il ne peut y avoir qu'une seule entête par document
	\item \optword{Le corps} : marqué par \keyword{<body>}. Il représente le contenu principal du document HTML. Il ne peut y avoir qu'un seul corps par document
\end{itemize}
\end{minipage}
%
\begin{minipage}{0.38\textwidth}
\begin{block}{Structure d'un document HTML}
\lstset{escapeinside=**}
\scriptsize\bfseries
\begin{lstlisting}[language={html}]
<!DOCTYPE html>
<html>
  <head>
    <title>
      Exemple de HTML
    </title>
  </head>
  <body>
    Bonjour tout le monde!
  </body>
</html>
\end{lstlisting}
\end{block}
\end{minipage}

\end{frame}

\begin{frame}
\frametitle{HTML : Structure et éléments}
\framesubtitle{Structure : Les commentaires}

\begin{itemize}
	\item ils commencent par \keyword{<!- -} et terminent par \keyword{- ->}
	\item ils peuvent prendre plusieurs lignes
	\item leur intérêt : 
	\begin{itemize}
		\item commenter le code non utiliser
		\item ajouter des explications au code
	\end{itemize}
\end{itemize}

\end{frame}

\begin{frame}[fragile]
\frametitle{HTML : Structure et éléments}
\framesubtitle{Structure : L'entête}

\begin{minipage}{0.44\textwidth} 
\begin{itemize}
	\item \keyword{<title>} : le titre est affiché dans la barre de titre du navigateur ou dans l'onglet de la page.
	\item \keyword{<link>} : définit la relation entre le document courant et une ressource externe (on va voir ça dans CSS)
	
\end{itemize}
\end{minipage}
%
\begin{minipage}{0.55\textwidth}
\begin{exampleblock}{Exemple d'une entête}
\lstset{escapeinside=**}
\scriptsize\bfseries
\begin{lstlisting}[language={html}]
<head>
  <title>Titre</title>
  <!-- *définir l'icon du site* -->
  <link rel="icon" href="favicon.ico">
  <!-- *encodage des caractères* -->
  <meta charset="utf-8">
  <!-- *Rediriger la page après 3 secondes* -->
  <meta http-equiv="refresh" 
        content="3;url=http://esi.dz/">
</head>
\end{lstlisting}
\end{exampleblock}
\end{minipage}

\begin{itemize}
	\item \keyword{<meta>} : une donnée qui décrit une donnée. 
	Par exemple, \expword{l'auteur de la page}, \expword{son encodage}, etc.
	\item des scripts (\keyword{<script>}) et des styles (\keyword{<style>})
\end{itemize}

\end{frame}

%\begin{frame}
%\frametitle{HTML : Structure et éléments}
%\framesubtitle{Structure : Le corps}
%
%\begin{itemize}
%	\item Le corps 
%	\item 
%	\item 
%\end{itemize}
%
%\end{frame}


\subsection{Formatage de texte}

\begin{frame}
\frametitle{HTML : Structure et éléments}
\framesubtitle{Formatage de texte : Balises de mise en forme}

\begin{itemize}
	\item \keyword{<i>} : un texte en italique
	\item \keyword{<b>} : un texte en gras
	\item \keyword{<em>} : un texte en emphase (important). En général, italique
	\item \keyword{<strong>} : un texte en emphase fort (très important). En général, gras
	\item \keyword{<p>} : un paragraphe
	\item \keyword{<h1>} jusqu'à \keyword{<h6>} : des titres de niveau 1 jusqu'à niveau 6
	\item \keyword{<pre>} : un texte préformaté ; le texte reste tel qu'il est, avec les espaces et les retours à la ligne
	\item \keyword{<sup>} : un exposant
	\item \keyword{<sub>} : une indice
	\item \textit{un exemple dans "formatage.html"}
\end{itemize}

\end{frame}

\begin{frame}[fragile]
\frametitle{HTML : Structure et éléments}
\framesubtitle{Formatage de texte : Citation}

\begin{minipage}{0.44\textwidth} 
	\begin{itemize}
		\item \keyword{<q>} : une citation courte. 
		La plupart des navigateurs modernes entoure le texte de cet élément avec des marques de citation.
		\item \keyword{<blockquote>} : une citation longue. 
		Le texte est généralement affiché avec une indentation.
	\end{itemize}
\end{minipage}
%
\begin{minipage}{0.55\textwidth}
\begin{exampleblock}{Exemple : formatage.html}
\lstset{escapeinside=**}
\scriptsize\bfseries
\begin{lstlisting}[language={html}]
Mahatma Gandhi a dit :
<q>
  Be the change that you wish to see 
  in the world.
</q>
Albert Einstein a comment*é* sur 
l'*ê*tre humain :
<blockquote>
  Two things are infinite: the universe 
  and human stupidity; 
  and I'm not sure about the universe.
</blockquote>
\end{lstlisting}
\end{exampleblock}
\end{minipage}

\end{frame}

\subsection{Listes}

\begin{frame}[fragile]
\frametitle{HTML : Structure et éléments}
\framesubtitle{Listes : Liste à puces}

\begin{minipage}{0.60\textwidth} 
	\begin{itemize}
		\item définie par la balise \keyword{<ul>} (unordered list)
		\item ses éléments sont définis par la balise \keyword{<li>}
		\item on peut créer une liste à l'intérieur d'une autre
	\end{itemize}
\end{minipage}
%
\begin{minipage}{0.38\textwidth}
\begin{exampleblock}{exemple: liste.html}
\lstset{escapeinside=**}
\scriptsize\bfseries
\begin{lstlisting}[language={html}]
<ul>
  <li>*élément 1*</li>
  <li>*élément 2* :
    <ul>
      <li>*élément 2-1*</li>
      <li>*élément 2-2*</li>
      <li>*élément 2-3*</li>
    </ul>
  </li>
  <li>*élément 3*</li>
</ul>
\end{lstlisting}
\end{exampleblock}
\end{minipage}

\end{frame}

\begin{frame}[fragile]
\frametitle{HTML : Structure et éléments}
\framesubtitle{Listes : Liste numérotée}

\begin{minipage}{0.60\textwidth} 
	\begin{itemize}
		\item définie par la balise \keyword{<ol>} (ordered list)
		\item ses éléments sont définis par la balise \keyword{<li>}
		\item on peut créer une liste à l'intérieur d'une autre
		\item on peut commencer la numérotation à partir d'un nombre en utilisant l'attribut \keyword{start}
	\end{itemize}
\end{minipage}
%
\begin{minipage}{0.38\textwidth}
\begin{exampleblock}{exemple: liste.html}
\lstset{escapeinside=**}
\scriptsize\bfseries
\begin{lstlisting}[language={html}]
<ol start="5">
  <li>*élément* 1</li>
  <li>*élément* 2</li>
  <li>*élément* 3</li>
</ol>
\end{lstlisting}
\end{exampleblock}
\end{minipage}

\end{frame}

\begin{frame}[fragile]
\frametitle{HTML : Structure et éléments}
\framesubtitle{Listes : Liste de définition}

\begin{minipage}{0.50\textwidth} 
	\begin{itemize}
		\item définie par la balise \keyword{<dl>} (definition list)
		\item dedans il y a une séquence de termes et descriptions
		\item un terme est défini par la balise \keyword{<dt>} (definition term)
		\item une description est définie par la balise \keyword{<dd>} (definition description)
	\end{itemize}
\end{minipage}
%
\begin{minipage}{0.49\textwidth}
\begin{exampleblock}{exemple: liste.html}
\lstset{escapeinside=**}
\scriptsize\bfseries
\begin{lstlisting}[language={html}]
<dl>
  <dt>HTML</dt>
  <dd>HyperText Markup Language</dd>

  <dt>CSS</dt>
  <dd>Cascading Style Sheets</dd>
</dl>
\end{lstlisting}
\end{exampleblock}
\end{minipage}

\end{frame}

\subsection{Tableaux}

\begin{frame}[fragile]
\frametitle{HTML : Structure et éléments}
\framesubtitle{Tableaux : Éléments d'un tableau}

\begin{minipage}{0.60\textwidth} 
	\begin{itemize}
		\item un tableau est défini par la balise \keyword{<table>}
		\item dans un tableau, il y a plusieurs lignes
		\item une ligne est définie par la balise \keyword{<tr>}
		\item chaque ligne contient des cellules
		\item une cellule est définie par la balise \keyword{<td>}
		\item dans la première ligne, on remplace \keyword{<td>} par \keyword{<th>} pour désigner qu'il s'agit des titres du tableau
	\end{itemize}
\end{minipage}
%
\begin{minipage}{0.39\textwidth}
\begin{exampleblock}{exemple: tableau.html}
\lstset{escapeinside=**}
\scriptsize\bfseries
\begin{lstlisting}[language={html}]
<table>
  <tr>
    <td>L1C1</td>
    <td>L1C2</td>
    <td>L1C3</td>
  </tr>
  <tr>
    <td>L2C1</td>
    <td>L2C2</td>
    <td>L2C3</td>
  </tr>
</table>
\end{lstlisting}
\end{exampleblock}
\end{minipage}

\end{frame}

\begin{frame}[fragile]
\frametitle{HTML : Structure et éléments}
\framesubtitle{Tableaux : Accessibilité}

\begin{minipage}{0.50\textwidth} 
	\begin{itemize}
		\item pour décrire un tableau, on utilise une légende : la balise \keyword{<caption>}
		\item on peut regrouper les lignes de l'entête d'un tableau dans une balise \keyword{<thead>}
		\item le corps du tableau peut être regroupé dans la balise \keyword{<tbody>}
		\item on peut spécifier le type du titre en utilisant l'attribut \keyword{scope} : un titre de colonnes (\keyword{scope="col"}) ou un titre des lignes (\keyword{scope="row"})
	\end{itemize}
\end{minipage}
%
\begin{minipage}{0.49\textwidth}
\begin{exampleblock}{exemple: tableau.html}
\lstset{escapeinside=**}
\tiny\bfseries\vspace{-6pt}
\begin{lstlisting}[language={html}]
<table>
  <caption>Notes des *étudiants*</caption>
  <thead>
    <tr>
      <th scope="col">Nom</th>
      <th scope="col">BWEB</th>
      <th scope="col">ALSDS</th>
    </tr>
  </thead>
  <tbody>
    <tr>
      <th scope="row">Etudiant1</th>
      <td>16</td>
      <td>14</td>
    </tr>
    <tr>
      <th scope="row">Etudiant2</th>
      <td>13</td>
      <td>17</td>
    </tr>
  </tbody>
</table>
\end{lstlisting}\vspace{-6pt}
\end{exampleblock}
\end{minipage}

\end{frame}


\begin{frame}[fragile]
\frametitle{HTML : Structure et éléments}
\framesubtitle{Tableaux : Fusionner les cellules}

\begin{minipage}{0.50\textwidth} 
	\begin{itemize}
		\item l'attribut \keyword{rowspan} : une cellule prend plusieurs lignes 
		\item l'attribut \keyword{colspan} : une cellule prend plusieurs colonnes
		\item lorsqu'une cellule prend plusieurs lignes (colonnes), on doit la compter comme si plusieurs cellules
		\item on peut omettre \keyword{rowspan="1"} et \keyword{colspan="1"}
	\end{itemize}
\end{minipage}
%
\begin{minipage}{0.49\textwidth}
\begin{exampleblock}{exemple: tableau.html}
\lstset{escapeinside=**}
\scriptsize\bfseries\vspace{-6pt}
\begin{lstlisting}[language={html}]
<table>
  <tr>
    <td colspan="2">L1C1</td>
    <!-- L1C2 est prise par L1C1 -->
    <td rowspan="2">L1C3</td>
  </tr>
  <tr>
    <td>L2C1</td>
    <td>L2C2</td>
    <!-- L2C3 est prise par L1C3 -->
  </tr>    
</table>
\end{lstlisting}\vspace{-6pt}
\end{exampleblock}
\hgraphpage[.5\textwidth]{table-fusionner.png}
\end{minipage}

\end{frame}

%===================================================================================
\section{Liens et multimédia}
%===================================================================================

\subsection{Liens et intégration}

\begin{frame}[fragile]
\frametitle{HTML : Liens et multimédia}
\framesubtitle{Liens et intégration : Lien hypertexte}

\begin{minipage}{0.50\textwidth} 
	\begin{itemize}
		\item Un lien hypertexte est défini par la balise \keyword{<a>} 
		\item la destination est spécifiée par l'attribut \keyword{href}
		\item pour ouvrir la page dans une nouvelle onglet, on utilise \keyword{target="\_blank"}
		\item pour spécifier que le lien est un émail, on utilise le mot clé \keyword{mailto:} avant l'adresse émail
	\end{itemize}
\end{minipage}
%
\begin{minipage}{0.49\textwidth}
\begin{exampleblock}{exemple: lien.html}
\lstset{escapeinside=**}
\scriptsize\bfseries\vspace{-6pt}
\begin{lstlisting}[language={html}]
<a href="liste.html">
  Voir les listes
</a>
<br>
<a href="https://www.google.com" 
                target="_blank">
  Aller vers Google
</a>
<br>
<a href="mailto:ab_aries@esi.dz">
  Envoyer un message
</a>
\end{lstlisting}\vspace{-6pt}
\end{exampleblock}
\end{minipage}

\end{frame}

\begin{frame}[fragile]
\frametitle{HTML : Liens et multimédia}
\framesubtitle{Liens et intégration : Lien vers un emplacement dans la même page}

\begin{minipage}{0.50\textwidth} 
	\begin{itemize}
		\item on doit marquer l'emplacement dans le fichier HTML
		\item on attribue un identifiant à un élément en utilisant l'attribut \keyword{id}
		\item pour référencer cet identifiant dans un lien, on le précède par \keyword{\#}
%		\item pour référencer un élément dans une autre page, on met l'adresse de la page suivi par \keyword{\#} suivi par l'identifiant de cet élément
	\end{itemize}
\end{minipage}
%
\begin{minipage}{0.49\textwidth}
\begin{exampleblock}{exemple: lien.html}
\lstset{escapeinside=**}
\scriptsize\bfseries\vspace{-6pt}
\begin{lstlisting}[language={html}]
<a href="#def">
  Aller vers la *définition*
</a> 
...
<h1 id="def">*Définition*</h1>
\end{lstlisting}\vspace{-6pt}
\end{exampleblock}
\end{minipage}

\begin{itemize}
	\item pour référencer un élément dans une autre page, on met l'adresse de la page suivi par \keyword{\#} suivi par l'identifiant de cet élément
\end{itemize}

\end{frame}

\begin{frame}[fragile]
\frametitle{HTML : Liens et multimédia}
\framesubtitle{Liens et intégration : Afficher une page dans une autre}

\begin{minipage}{0.50\textwidth} 
	\begin{itemize}
		\item Pour insérer une page dans une autre, on utilise la balise \keyword{<iframe>}
		\item l'attribut \keyword{width} spécifie la longueur 
		\item \keyword{height} spécifie l'hauteur 
		\item \keyword{src} spécifie l'URL de la page à insérer
	\end{itemize}
\end{minipage}
%
\begin{minipage}{0.49\textwidth}
\begin{exampleblock}{exemple: lien.html}
\lstset{escapeinside=**}
\scriptsize\bfseries\vspace{-6pt}
\begin{lstlisting}[language={html}]
<iframe 
      width="300" 
      height="200" 
      src="liste.html">
</iframe>	
\end{lstlisting}\vspace{-6pt}
\end{exampleblock}
\end{minipage}

\end{frame}

\subsection{Images et graphiques}

\begin{frame}[fragile]
\frametitle{HTML : Liens et multimédia}
\framesubtitle{Images et graphiques : Insérer une image}

\begin{minipage}{0.50\textwidth} 
	\begin{itemize}
		\item Pour insérer une image dans une page, on utilise la balise \keyword{<img>}
		\item c'est une balise auto-fermante
		\item l'attribut \keyword{src} spécifie l'URL de l'image
%		\item \keyword{width} spécifie la longueur 
%		\item \keyword{height} spécifie l'hauteur
%		\item \keyword{alt} spécifie le texte à afficher si l'image est introuvable 
%		\item \keyword{title} spécifie le texte qui s'affiche lorsqu'on met le curseur de la souri sur l'image
	\end{itemize}
\end{minipage}
%
\begin{minipage}{0.49\textwidth}
\begin{exampleblock}{exemple: image.html}
\lstset{escapeinside=**}
\scriptsize\bfseries\vspace{-6pt}
\begin{lstlisting}[language={html}]
<img src="assets/orange.jpg"
     width="200px" height="100px"
     alt="Un oranger" 
     title="Oranger et nuages" >
\end{lstlisting}\vspace{-6pt}
\end{exampleblock}
\end{minipage}

\begin{itemize}
%	\item Pour insérer une image dans une page, on utilise la balise \keyword{<img>}
%	\item c'est une balise auto-fermante
%	\item l'attribut \keyword{src} spécifie l'URL de l'image
	\item \keyword{width} spécifie la longueur 
	\item \keyword{height} spécifie l'hauteur
	\item \keyword{alt} spécifie le texte à afficher si l'image est introuvable 
	\item \keyword{title} spécifie le texte qui s'affiche lorsqu'on met le curseur de la souri sur l'image
\end{itemize}

\end{frame} 

% <figure> et <figcaption>
\begin{frame}[fragile]
\frametitle{HTML : Liens et multimédia}
\framesubtitle{Images et graphiques : Figures}

\begin{minipage}{0.47\textwidth} 
	\begin{itemize}
		\item une figure est un région où on met des images ou des schémas 
		\item elle est définie par la balise \keyword{<figure>}
		\item elle peut contenir une légende, en utilisant la balise \keyword{<figcaption>}
	\end{itemize}
\end{minipage}
%
\begin{minipage}{0.52\textwidth}
\begin{exampleblock}{exemple: image.html}
\lstset{escapeinside=**}
\scriptsize\bfseries\vspace{-6pt}
\begin{lstlisting}[language={html}]
<figure>
  <img src="assets/orange.jpg"
       width="200px" height="100px"
       alt="Un oranger" 
       title="Oranger et nuages" >
  <figcaption>
    Oranger et nuages
  </figcaption>
</figure>
\end{lstlisting}\vspace{-6pt}
\end{exampleblock}
\end{minipage}

\end{frame}

% <picture> et <source>
\begin{frame}[fragile]
\frametitle{HTML : Liens et multimédia}
\framesubtitle{Images et graphiques : Images adaptatives}

\begin{minipage}{0.50\textwidth} 
	\begin{itemize}
		\item On veut insérer une image sur la page selon la taille de l'écran
		\item une solution est d'utiliser la balise \keyword{<picture>}
		\item dedans, on met les différentes sources en utilisant la balise \keyword{<source>}
%		\item pour chaque source, on peut spécifier la taille préférée en utilisant l'attribut \keyword{<media>}
%		\item on doit ajouter une balise \keyword{<img>} au cas où toutes les sources ne sont pas satisfaites
	\end{itemize}
\end{minipage}
%
\begin{minipage}{0.49\textwidth}
\begin{exampleblock}{exemple: image.html}
\lstset{escapeinside=**}
\scriptsize\bfseries\vspace{-6pt}
\begin{lstlisting}[language={html}]
<picture>
  <source media="(max-width: 799px)" 
      srcset="assets/orange-petite.jpg">
  <source media="(min-width: 800px)" 
      srcset="assets/orange.jpg">
  <img src="assets/orange.jpg" 
      alt="Un oranger">
</picture>
\end{lstlisting}\vspace{-6pt}
\end{exampleblock}
\end{minipage}

\begin{itemize}
%	\item On veut insérer une image sur la page selon la taille de l'écran
%	\item une solution est d'utiliser la balise \keyword{<picture>}
%	\item dedans, on met les différentes sources en utilisant la balise \keyword{<source>}
	\item pour chaque source, on peut spécifier la taille préférée en utilisant l'attribut \keyword{<media>}
	\item on doit ajouter une balise \keyword{<img>} au cas où toutes les sources ne sont pas satisfaites
\end{itemize}

\end{frame}

%<map> et <area>
%\begin{frame}[fragile]
%\frametitle{HTML : Liens et multimédia}
%\framesubtitle{Images et graphiques : Régions cliquables dans l'image}
%
%\begin{minipage}{0.50\textwidth} 
%	\begin{itemize}
%		\item On peut définir des régions cliquables dans une image en utilisant la balise \keyword{<map>}
%	\end{itemize}
%\end{minipage}
%%
%\begin{minipage}{0.49\textwidth}
%	\begin{exampleblock}{exemple: lien.html}
%		\lstset{escapeinside=**}
%		\scriptsize\bfseries\vspace{-6pt}
%		\begin{lstlisting}[language={html}]
%		
%		\end{lstlisting}\vspace{-6pt}
%	\end{exampleblock}
%\end{minipage}
%
%\end{frame}


\subsection{Vidéo et audio}

%Vidéo 
%Audio
\begin{frame}[fragile]
\frametitle{HTML : Liens et multimédia}
\framesubtitle{Vidéo et audio : Vidéo}

\begin{minipage}{0.50\textwidth} 
	\begin{itemize}
		\item Pour insérer une vidéo, on utilise la balise \keyword{<video>}
		\item l'attribut \keyword{controls} indique que la vidéo inclue les contrôles par défaut du navigateur
		\item pour indiquer l'URL de la vidéo, on peut utiliser l'attribut \keyword{src} ou l'élément \keyword{<source>}
		\item l'attribut \keyword{loop} indique que la vidéo  joue en boucle
	\end{itemize}
\end{minipage}
%
\begin{minipage}{0.49\textwidth}
\begin{exampleblock}{exemple: video\_audio.html}
\lstset{escapeinside=**}
\scriptsize\bfseries\vspace{-6pt}
\begin{lstlisting}[language={html}]
<video controls>
  <source src="assets/chat.mp4" 
     type="video/mp4">
  <p>
    Votre navigateur ne prend pas 
    en charge les vid*é*os HTML5.
  </p>
</video>
\end{lstlisting}\vspace{-6pt}
\end{exampleblock}
\end{minipage}

\end{frame}

\begin{frame}[fragile]
\frametitle{HTML : Liens et multimédia}
\framesubtitle{Vidéo et audio : Audio}

\begin{minipage}{0.50\textwidth} 
	\begin{itemize}
		\item Pour insérer un contenu audio, on utilise la balise \keyword{<audio>}
		\item pour indiquer l'URL du contenu, on peut utiliser l'attribut \keyword{src} ou l'élément \keyword{<source>}
	\end{itemize}
\end{minipage}
%
\begin{minipage}{0.49\textwidth}
\begin{exampleblock}{exemple: video\_audio.html}
\lstset{escapeinside=**}
\scriptsize\bfseries\vspace{-6pt}
\begin{lstlisting}[language={html}]
<audio controls loop 
    src="assets/Night-bird-call.mp3">
  Votre navigateur ne prend pas en 
  charge les *éléments* audio
</audio>
\end{lstlisting}\vspace{-6pt}
\end{exampleblock}
\end{minipage}

\end{frame}

%===================================================================================
\section{Disposition}
%===================================================================================

\subsection{Type des éléments}

%Block
%Inline
\begin{frame}[fragile]
\frametitle{HTML : Disposition}
\framesubtitle{Type des éléments}

\begin{itemize}
	\item \optword{élément en ligne} [\textit{inline element}]
	\begin{itemize}
		\item occupe l'espace délimités par ses balises ;  prend uniquement la largeur qui lui est nécessaire 
		\item ne commence pas sur une nouvelle ligne
		\item par exemple, \keyword{<a>}, \keyword{<b>}, \keyword{<i>}, \keyword{<img>}, \keyword{<q>}, \keyword{<sub>}, \keyword{<sup>}, etc.
	\end{itemize}
	\item \optword{élément de bloc} [\textit{block element}]
	\begin{itemize}
		\item prend toute la largeur disponible 
		\item commence toujours sur une nouvelle ligne
		\item par exemple, \keyword{<dl>}, \keyword{<figure>}, \keyword{<h1>}, \keyword{<ol>}, \keyword{<p>}, \keyword{<pre>}, \keyword{<table>}, \keyword{<ul>}, etc.
	\end{itemize}
\end{itemize}

\end{frame}

\subsection{Conteneurs génériques}

%<div>
%<span>
\begin{frame}[fragile]
\frametitle{HTML : Disposition}
\framesubtitle{Conteneurs génériques}

\begin{itemize}
	\item \keyword{<span>}
	\begin{itemize}
		\item conteneur générique en ligne
		\item peut être utilisé pour grouper des éléments
		\item BUT : mettre en forme plusieurs éléments en ligne à la fois
		\item BUT : attribuer la même valeur d'un attribut à pluusieurs éléments. 
		Pr exemple, l'attribut \keyword{lang}.
	\end{itemize}
	\item \keyword{<div>}
	\begin{itemize}
		\item conteneur générique en bloc
		\item peut être utilisé pour grouper des éléments
		\item BUT : même que \keyword{<span>}
		\item BUT : organiser le contenu
	\end{itemize}
\end{itemize}

\end{frame}

\subsection{Éléments sémantiques}

\begin{frame}[fragile]
\frametitle{HTML : Disposition}
\framesubtitle{Éléments sémantiques}

\begin{itemize}
	\item Des \keyword{<div>} destinés pour une fonctionnalité spécifique
	\item \keyword{<header>} : L'entête de la page ou d'une section. Généralement une grande bande placée en travers au haut de la page avec un titre ou un logo.
	\item \keyword{<nav>} : Barre de navigation (menu). Elle fait le lien vers les principales parties du site.
	\item \keyword{<main>} : Contenu principal. Il contient des sections et des articles.
	\item \keyword{<section>} : Une section générique d'un document ou d'une application. 
	\item \keyword{<article>} : Un type spécifique de \keyword{<section>}. Représente du contenu autonome dans un document, une page, une application, ou un site.
	\item \keyword{<footer>} : Pied de page. Une bande au bas de la page qui contient généralement, en petits caractères, des avis de droit d'auteur ou des coordonnées de contact.
\end{itemize}

\end{frame}


%===================================================================================
\section{Formulaires}
%===================================================================================

%\begin{frame}[fragile]
%\frametitle{HTML : Formulaires}
%
%\begin{itemize}
%	\item 
%\end{itemize}
%
%\end{frame}

\subsection{Structure générale}

\begin{frame}[fragile]
\frametitle{HTML : Formulaires}
\framesubtitle{Structure générale}

\begin{minipage}{0.44\textwidth} 
	\begin{itemize}
		\item Un formulaire est défini par un élément \keyword{<form>}
		\item L'attribut \keyword{<action>} définit l'emplacement (une URL) où doivent être envoyées les données collectées par le formulaire
		\item L'attribut \keyword{method} définit la méthode \keyword{HTTP} utilisée pour envoyer les données
		\item Il contient des champs ; en générale, des éléments \keyword{<input>}
	\end{itemize}
\end{minipage}
%
\begin{minipage}{0.55\textwidth}
\begin{exampleblock}{exemple: formulaire.html}
\lstset{escapeinside=**}
\scriptsize\bfseries\vspace{-6pt}
\begin{lstlisting}[language={html}]
<form 
    action="https://www.google.com/search" 
    method="get">
  <input type="text" name="q" >
  <input type="submit">
</form>
\end{lstlisting}\vspace{-6pt}
\end{exampleblock}
\end{minipage}

\end{frame}

\subsection{Envoi des données}

\begin{frame}[fragile]
\frametitle{HTML : Formulaires}
\framesubtitle{Envoi des données}

\begin{itemize}
	\item L'attribut \keyword{action} spécifie la destination des données
	\item Si cet attribut n'ai pas spécifié, la page courante sera considérée comme destination
	\item L'attribut \keyword{method} définit la méthode \keyword{HTTP} utilisée pour envoyer les données
	\item Il existe deux méthodes 
	\begin{itemize}
		\item \keyword{get} c'est la valeur par défaut (au cas où cette attribut n'ai pas défini). 
		Les informations sont envoyées dans l'URL comme suite: 
		l'adresse de destination suivie par \keyword{?} suivie par \expword{nom\_champ1=valeur1\&nom\_champ2=valeur2}
		\item \keyword{post} les données sont envoyées dans le corps de la requête HTTP
	\end{itemize}
\end{itemize}

\end{frame}

\subsection{Éléments d'un formulaire}

\begin{frame}[fragile]
\frametitle{HTML : Formulaires}
\framesubtitle{Éléments d'un formulaire}

\begin{itemize}
	\item \keyword{<input>}
	\begin{itemize}
		\item Balise auto-fermante 
		\item \keyword{type} : Le type du champ de saisie
		\item \keyword{required} : L'utilisateur doit nécessairement saisir une valeur afin de pouvoir envoyer le formulaire
		\item \keyword{value} : La valeur du champ
	\end{itemize}
	\item \keyword{<textarea>}
	\begin{itemize}
		\item Utilisé pour contenir un paragraphe
	\end{itemize}
	\item \keyword{<select>} et \keyword{<option>}
	\begin{itemize}
		\item Utilisés pour créer des listes déroulantes
	\end{itemize}
	\item \keyword{<label>}
	\begin{itemize}
		\item Utilisés pour créer des listes déroulantes
	\end{itemize}
\end{itemize}

\end{frame}

\begin{frame}[fragile]
\frametitle{HTML : Formulaires}
\framesubtitle{Éléments d'un formulaire : les étiquettes}

\begin{minipage}{0.64\textwidth} 
	\begin{itemize}
		\item l'intérêt d'utiliser une étiquette est d'associer un texte à un champs ; on peut cliquer sur le libellé pour passer le focus voire activer le champ. 
		\item une étiquette est associée à ce champ en utilisant l'attribut \keyword{for}
		\item la valeur \keyword{for} de l'étiquette est celle de \keyword{id} du champ
%		\item on peut également créer un lien implicite en imbriquant l'élément \keyword{<input>} directement au sein d'un élément \keyword{<label>}.
	\end{itemize}
\end{minipage}
%
\begin{minipage}{0.35\textwidth}
\begin{exampleblock}{exemple: formulaire.html}
\lstset{escapeinside=**}
\scriptsize\bfseries\vspace{-6pt}
\begin{lstlisting}[language={html}]
<label for="nom">
  Nom : 
</label>
<input type="text" 
    name="n" id="nom">
<label>
  Pr*é*nom : 
  <input type="text" 
        name="p">
</label>
\end{lstlisting}\vspace{-6pt}
\end{exampleblock}
\end{minipage}
\begin{itemize}
%	\item l'intérêt d'utiliser une étiquette est d'associer un texte à un champs ; on peut cliquer sur le libellé pour passer le focus voire activer le champ. 
%	\item une étiquette est associée à ce champ en utilisant l'attribut \keyword{for}
%	\item la valeur \keyword{for} de l'étiquette est celle de \keyword{id} du champ
	\item on peut également créer un lien implicite en imbriquant l'élément \keyword{<input>} directement au sein d'un élément \keyword{<label>}.
\end{itemize}

\end{frame}

\begin{frame}[fragile]
\frametitle{HTML : Formulaires}
\framesubtitle{Éléments d'un formulaire : les boutons d'envoi et de réinitialisation}

\begin{minipage}{0.59\textwidth} 
	\begin{itemize}
		\item des éléments de type \keyword{<input>}
		\item \keyword{<input type="submit">} : Un bouton qui nous permet d'envoyer le formulaire vers la destination
		\item \keyword{<input type="reset">} : Un bouton qui nous permet de réinitialiser les champs du formulaire
	\end{itemize}
\end{minipage}
%
\begin{minipage}{0.4\textwidth}
\begin{exampleblock}{exemple: formulaire.html}
\lstset{escapeinside=**}
\scriptsize\bfseries\vspace{-6pt}
\begin{lstlisting}[language={html}]
<form action="serveur">
  <label for="nom">
    Nom : 
  </label>
  <input type="text" 
        name="nom" id="nom">
  <br>
  <input type="submit" 
        value="envoyer">
  <input type="reset" 
        value="r*é*initialiser">
</form>
\end{lstlisting}\vspace{-6pt}
\end{exampleblock}
\end{minipage}

\end{frame}


\begin{frame}[fragile]
\frametitle{HTML : Formulaires}
\framesubtitle{Éléments d'un formulaire : champ de texte sur une seule ligne}

\begin{minipage}{0.59\textwidth} 
	\begin{itemize}
		\item \keyword{<input type="text">}
		\item \keyword{maxlength} : Le nombre de caractères maximal qui peut être écrit dans ce champ.
	\end{itemize}
\end{minipage}
%
\begin{minipage}{0.4\textwidth}
\begin{exampleblock}{exemple: formulaire.html}
\lstset{escapeinside=**}
\scriptsize\bfseries\vspace{-6pt}
\begin{lstlisting}[language={html}]
<input type="text" 
  name="nom" id="nom"
  minlength="2" 
  size="20"
  placeholder="Votre nom">
\end{lstlisting}\vspace{-6pt}
\end{exampleblock}
\end{minipage}

\begin{itemize}
	\item \keyword{minlength} : Le nombre de caractères minimal qui peut être écrit dans ce champ.
	\item \keyword{placeholder} : Une valeur d'exemple qui sera affichée lorsqu'aucune valeur n'est saisie.
	\item \keyword{size} : Le nombre de caractères affichés par le champ.
\end{itemize}

\end{frame}

\begin{frame}[fragile]
\frametitle{HTML : Formulaires}
\framesubtitle{Éléments d'un formulaire : champ de texte sur plusieurs lignes}

\begin{minipage}{0.59\textwidth} 
	\begin{itemize}
		\item \keyword{<textarea></textarea>}
		\item la valeur par défaut est le contenu de cette élément
		\item les attributs \keyword{maxlength}, \keyword{minlength}, \keyword{placeholder}
	\end{itemize}
\end{minipage}
%
\begin{minipage}{0.4\textwidth}
\begin{exampleblock}{exemple: formulaire.html}
\lstset{escapeinside=**}
\scriptsize\bfseries\vspace{-6pt}
\begin{lstlisting}[language={html}]
<textarea name="msg"
  size="20" rows="4"
  placeholder="Votre opinion">
\end{lstlisting}\vspace{-6pt}
\end{exampleblock}
\end{minipage}

\begin{itemize}
	\item \keyword{cols}: La largeur visible du contrôle de saisie, exprimée en largeur moyenne de caractères (par défaut : 20).
	\item \keyword{rows} : Le nombre de lignes de texte visibles.
\end{itemize}

\end{frame}

\begin{frame}[fragile]
\frametitle{HTML : Formulaires}
\framesubtitle{Éléments d'un formulaire : mot de passe}

\begin{itemize}
	\item \keyword{<input type="password">}
	\item permettent à l'utilisateur de saisir un mot de passe sans que celui-ci ne soit lisible à l'écran. 
	\item les attributs \keyword{maxlength}, \keyword{minlength}, \keyword{placeholder}, \keyword{size} 
\end{itemize}

\begin{exampleblock}{exemple: formulaire.html}
\lstset{escapeinside=**}
\scriptsize\bfseries\vspace{-6pt}
\begin{lstlisting}[language={html}]
<label for="mdp">Mot de passe : </label>
<input type="password" name="pass" id="mdp" >
\end{lstlisting}\vspace{-6pt}
\end{exampleblock}

\end{frame}

\begin{frame}[fragile]
\frametitle{HTML : Formulaires}
\framesubtitle{Éléments d'un formulaire : données cachées}

\begin{itemize}
	\item \keyword{<input type="hidden">}
	\item permettent aux développeurs web d'inclure des données qui ne peuvent pas être vues ou modifiées lorsque le formulaire est envoyé. 
	\item envoyer un identifiant d'une commande ou un jeton de sécurité unique  
\end{itemize}

\begin{exampleblock}{exemple d'une donnée cachée}
\lstset{escapeinside=**}
\scriptsize\bfseries\vspace{-6pt}
\begin{lstlisting}[language={html}]
<input id="prodId" name="prodId" type="hidden" value="xm234jq">
\end{lstlisting}\vspace{-6pt}
\end{exampleblock}

\end{frame}

\begin{frame}[fragile]
\frametitle{HTML : Formulaires}
\framesubtitle{Éléments d'un formulaire : joindre un fichier}

\begin{itemize}
	\item \keyword{<input type="file">}
	\item permettent à un utilisateur de sélectionner un ou plusieurs fichiers depuis leur appareil et de les \textit{uploader} vers un serveur 
	\item \keyword{accept} : la liste des types de fichiers à accepter. Exemple, \expword{.pdf, .doc, image/*}
	\item \keyword{multiple} : un attribut booléen qui, lorsqu'il est présent, indique que plusieurs fichiers peuvent être sélectionnés
	\item \keyword{capture} : La source à utiliser pour capturer des images ou des vidéos. 
	\keyword{capture="user"} : la caméra qui fait face à l'utilisateur. 
	\keyword{capture="environment"} : la caméra qui est tournée vers l'extérieur. 
\end{itemize}

\begin{exampleblock}{exemple de jointure de fichiers}
\lstset{escapeinside=**}
\scriptsize\bfseries\vspace{-6pt}
\begin{lstlisting}[language={html}]
<input type="file" name="devoir" accept=".pdf, .doc, .docx" multiple>
\end{lstlisting}\vspace{-6pt}
\end{exampleblock}

\end{frame}

\begin{frame}[fragile]
\frametitle{HTML : Formulaires}
\framesubtitle{Éléments d'un formulaire : les boutons radio}

\begin{itemize}
	\item \keyword{<input type="radio">}
	\item les boutons du même groupe doivent avoir la même valeur \keyword{name}
	\item \keyword{checked} : Un attribut booléen qui indique si le bouton radio est l'élément sélectionné du groupe.
\end{itemize}

\begin{exampleblock}{exemple: formulaire.html}
\lstset{escapeinside=**}
\scriptsize\bfseries\vspace{-6pt}
\begin{lstlisting}[language={html}]
<input type="radio" name="sex" id="sm" value="M" >
<label for="sm">Masculin</label>
<input type="radio" name="sex" id="sf" value="F" >
<label for="sf">F*é*minin</label>
\end{lstlisting}\vspace{-6pt}
\end{exampleblock}

\end{frame}

\begin{frame}[fragile]
\frametitle{HTML : Formulaires}
\framesubtitle{Éléments d'un formulaire : les boites à cocher}

\begin{itemize}
	\item \keyword{<input type="checkbox">}
	\item les boutons du même groupe peuvent avoir la même valeur \keyword{name} ou des valeurs différentes
	\item \keyword{checked} peut être spécifié à plusieurs champs
\end{itemize}

\begin{exampleblock}{exemple: formulaire.html}
\lstset{escapeinside=**}
\scriptsize\bfseries\vspace{-6pt}
\begin{lstlisting}[language={html}]
Quelles sont vos couleurs *préférées* ? <br>
<input type="checkbox" name="colpref" value="red" id="r">
<label for="r">Rouge</label><br>
<input type="checkbox" name="colpref" value="blue" id="b">
<label for="b">Bleu</label><br>
<input type="checkbox" name="colpref" value="green" id="v">
<label for="v">Vert</label>
\end{lstlisting}\vspace{-6pt}
\end{exampleblock}

\end{frame}

\begin{frame}[fragile]
\frametitle{HTML : Formulaires}
\framesubtitle{Éléments d'un formulaire : liste déroulante}

\begin{itemize}
	\item \keyword{<select>} défini par \keyword{name}
	\item dedans, un ensemble de \keyword{<option>} et leurs valeurs \keyword{value}
	\item on peut regrouper un ensemble d'options en utilisant \keyword{<optgroup>}
\end{itemize}

\begin{exampleblock}{exemple d'une liste déroulante}
\lstset{escapeinside=**}
\scriptsize\bfseries\vspace{-6pt}
\begin{lstlisting}[language={html}]
<select name="pays">
  <option disabled selected>Choisir</option>
  <optgroup label="Afrique">
    <option value="dz">Alg*é*rie</option>
    <option value="ma">Maroc</option>
    <option value="tn">Tunisie</option>
  </optgroup>
  <optgroup label="Europe">
    <option value="fr">France</option>
  </optgroup>
</select>
\end{lstlisting}\vspace{-6pt}
\end{exampleblock}

\end{frame}

\begin{frame}[fragile]
\frametitle{HTML : Formulaires}
\framesubtitle{Éléments d'un formulaire : liste déroulante (Humour)}

\hgraphpage{select-humour.jpg}

\end{frame}

\subsection{Éléments HTML5}

\begin{frame}[fragile]
\frametitle{HTML : Formulaires}
\framesubtitle{Éléments HTML5 : couleur et recherche}

\begin{itemize}
	\item \keyword{<input type="color">} : sélectionner une couleur
	\begin{itemize}
		\item sa valeur est stockée sous forme \expword{\#RRGGBB}
	\end{itemize}
	\item \keyword{<input type="search">}
	\begin{itemize}
		\item comme les zones de texte
		\item les navigateurs modernes proposent souvent une auto-complétion basée sur les termes de recherche déjà utilisés sur le site
	\end{itemize}
\end{itemize}

\begin{exampleblock}{exemple de la couleur et de recherche}
\lstset{escapeinside=**}
\scriptsize\bfseries\vspace{-6pt}
\begin{lstlisting}[language={html}]
<label for="c">Couleur : </label>
<input type="color" name="c" id="c" value="#ffffff">
<br>
<label for="s">Rechercher : </label>
<input type="search" name="q" id="s">

\end{lstlisting}\vspace{-6pt}
\end{exampleblock}

\end{frame}

\begin{frame}[fragile]
\frametitle{HTML : Formulaires}
\framesubtitle{Éléments HTML5 : date et heur}

\begin{itemize}
	\item \keyword{<input type="date">} : sélectionner une date 
	\begin{itemize}
		\item \keyword{max} : la valeur maximale, \keyword{min} : la valeur minimale 
	\end{itemize}
	\item \keyword{<input type="time">} : sélectionner l'heur
	\begin{itemize}
		\item \keyword{max}, \keyword{min}
	\end{itemize}
\end{itemize}

\begin{exampleblock}{exemple de date et d'heur}
\lstset{escapeinside=**}
\scriptsize\bfseries\vspace{-6pt}
\begin{lstlisting}[language={html}]
Rendez-vous <br>
<label for="d">Date</label>
<input type="date" name="d" id="d" min="2020-05-31" ><br>
<label for="h">Heur</label>
<input type="time" name="h" id="h" min="8:30" max="15:30" >
\end{lstlisting}\vspace{-6pt}
\end{exampleblock}

\end{frame}

\begin{frame}[fragile]
\frametitle{HTML : Formulaires}
\framesubtitle{Éléments HTML5 : nombres}

\begin{itemize}
	\item \keyword{<input type="number">} :  saisir des nombres
	\begin{itemize}
		\item \keyword{max}, \keyword{min}, \keyword{placeholder}
		\item \keyword{step} : Le pas à utiliser pour incrémenter la valeur à l'aide du contrôle fourni par le navigateur
	\end{itemize}
	\item \keyword{<input type="range">} : nombre compris entre deux bornes
	\begin{itemize}
		\item \keyword{max}, \keyword{min}, \keyword{step}
	\end{itemize}
	\item \keyword{<input type="tel">} : numéro de téléphone
	\begin{itemize}
		\item \keyword{maxlength}, \keyword{minlength}, \keyword{placeholder}, \keyword{size}
	\end{itemize}
\end{itemize}

\begin{exampleblock}{exemple de nombres}
\lstset{escapeinside=**}
\scriptsize\bfseries\vspace{-6pt}
\begin{lstlisting}[language={html}]
<label for="age">Age</label>
<input type="number" name="age" id="age" min="18" ><br>
<label for="t">*Théléphone*</label>
<input type="tel" name="t" id="t"><br>
<label for="note">Note</label>
<input type="range" name="age" id="age" min="0" max="20" step="0.25">
\end{lstlisting}\vspace{-6pt}
\end{exampleblock}

\end{frame}

\begin{frame}[fragile]
\frametitle{HTML : Formulaires}
\framesubtitle{Éléments HTML5 : URL}

\begin{itemize}
	\item \keyword{<input type="url">} :  saisir un lien
	\begin{itemize}
		\item \keyword{maxlength}, \keyword{minlength}, \keyword{placeholder}, \keyword{size}
	\end{itemize}
	\item \keyword{<input type="email">} : saisir un émail
	\begin{itemize}
		\item \keyword{maxlength}, \keyword{minlength}, \keyword{placeholder}, \keyword{size}
	\end{itemize}
\end{itemize}

\begin{exampleblock}{exemple des URL}
\lstset{escapeinside=**}
\scriptsize\bfseries\vspace{-6pt}
\begin{lstlisting}[language={html}]
<label for="site">Site</label>
<input type="url" name="site" id="site" value="https://" ><br>
<label for="mail">Email</label>
<input type="email" name="mail" id="mail">
\end{lstlisting}\vspace{-6pt}
\end{exampleblock}

\end{frame}

\begin{frame}
\frametitle{HTML}
\framesubtitle{Un peu d'humour}

\begin{center}
	\vgraphpage{html-humour.jpg}
\end{center}

\end{frame}

\insertbibliography{Bweb07}{*}

\end{document}

