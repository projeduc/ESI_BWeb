% !TEX TS-program = pdflatex
% !TeX program = pdflatex
% !TEX encoding = UTF-8
% !TEX spellcheck = fr

\documentclass[xcolor=table]{beamer}


%\usepackage{fullpage}
%\usepackage[left=2.8cm,right=2.2cm,top=2 cm,bottom=2 cm]{geometry}
\setbeamersize{text margin left=10pt,text margin right=10pt}
\usepackage{amsmath,amssymb} 
\usepackage[T1]{fontenc}
\usepackage[utf8]{inputenc}
\usepackage[english,french]{babel}
\usepackage{txfonts}
\usepackage[]{graphicx}
\usepackage{multirow}
\usepackage{hyperref}
\usepackage{colortbl}
\usepackage{listings}
\usepackage{wrapfig}
\usepackage{multicol}

\hypersetup{
	colorlinks,
	urlcolor = blue
}

%\renewcommand{\baselinestretch}{1.5}

\def\supit#1{\raisebox{0.8ex}{\small\it #1}\hspace{0.05em}}

\AtBeginSection{%
	\begin{frame}
		\sectionpage
	\end{frame}
}

\newcommand{\rottext}[2]{%
	\rotatebox{90}{%
	\begin{minipage}{#1}%
		\raggedleft#2%
	\end{minipage}%
	}%
}

\usepackage{longtable}
\usepackage{tabu}


\institute{ %
École  nationale Supérieure d'Informatique (ESI, ex. INI), Algérie
}
\author[ \textbf{\footnotesize  \insertframenumber/\inserttotalframenumber} \hspace*{\fill} ESI (2019-2020)] %
{ARIES Abdelkrime}
%\titlegraphic{\includegraphics[height=1cm]{../img/esi-logo.png}%\hspace*{4.75cm}~


\date{Année unniversitaire: 2019/2020} %\today

\usetheme{Warsaw} % Antibes Boadilla Warsaw

\beamertemplatenavigationsymbolsempty

%\setbeamertemplate{headline}{}

\definecolor{lightblue}{HTML}{D0D2FF}
\definecolor{lightyellow}{HTML}{FFFFAA}
\definecolor{darkblue}{HTML}{0000BB}
\definecolor{olivegreen}{HTML}{006600}
\definecolor{violet}{HTML}{6600CC}

\newcommand{\keyword}[1]{\textcolor{red}{\bfseries\itshape #1}}
\newcommand{\expword}[1]{\textcolor{olivegreen}{#1}}
\newcommand{\optword}[1]{\textcolor{violet}{\bfseries #1}}

\makeatletter
\newcommand\mysphere{%
	\parbox[t]{10pt}{\raisebox{0.2pt}{\beamer@usesphere{item projected}{bigsphere}}}}
\makeatother

%\let\oldtabular\tabular
%\let\endoldtabular\endtabular
%\renewenvironment{tabular}{\rowcolors{2}{white}{lightblue}\oldtabular\rowcolor{blue}}{\endoldtabular}


\NoAutoSpacing %french autospacing after ":"

\title[BWEB: 03- Diffusion et collecte] %
{Bureautique et Web \\Chapitre 04: Diffusion et collecte d'informations}  

\changegraphpath{../img/Bweb04-diffusion-collecte/}

\begin{document}

%\begin{frame}
%\frametitle{Diffusion et collecte d'informations}
%\framesubtitle{Introduction: Motivation}
%
%
%\end{frame}

%===================================================================================
\section{Publipostage}
%===================================================================================

\begin{frame}
\frametitle{Publipostage}
\framesubtitle{Motivation}

\begin{itemize}
	\item Imaginez que vous avez réalisé un projet et vous voulez envoyer des messages à des testeurs
	\item Supposons qu'il y a trois types de testeurs, pour : interface, fonctionnement et sécurité 
	\item Vous voulez envoyer un message personnalisé à chacun d'eux (Car, les messages personnalisés ont un taux de réponse plus élevé que les messages généralisés) 
	\item La première réflexion est de rédiger un message par personne 
	\item Le problème est que cette approche consome du temps; or les messages sont presque identiques
	\item La solution est d'utiliser le publipostage
\end{itemize}

\end{frame}

\begin{frame}
\frametitle{Publipostage}
\framesubtitle{Définition}

\begin{definition}
	Le publipostage est une technique de marketing direct qui consiste à envoyer, par voie postale ou électronique, des informations ou un prospectus publicitaire afin de promouvoir un produit, un service, une marque, etc.
\end{definition}

Pour appliquer le publipostage : 
\begin{itemize}
	\item une base de données contenant les informations des destinataires. Par exemple, \expword{nom}, \expword{prénom}, \expword{email}, etc.
	\item un document type appelé "masque de publipostage". Il contient du texte et les champs de remplacements	
\end{itemize}

\end{frame}

\begin{frame}
\frametitle{Publipostage}
\framesubtitle{Microsoft Word}
 
\begin{itemize}
	\item Onglet : \optword{Publipostage}
	\item Pour activer cet onglet : 
	\begin{enumerate}
		\item aller vers l'onglet \optword{Fichier}
		\item puis le menu \optword{Options}
		\item choisir la rubrique \optword{Personnaliser le ruban}
		\item dans la partie droite de la fenêtre ensuite, dans la colonne de droite qui affiche la liste des onglets du ruban, cocher la case devant Publipostage
	\end{enumerate}
\end{itemize}


\hgraphpage{publi-barre.png}

\end{frame}

\subsection{Les données}

\begin{frame}
\frametitle{Publipostage : Les données}
\framesubtitle{Création d'une source de données}

\begin{minipage}{0.40\textwidth}
	\begin{itemize}
		\item utiliser Excel pour créer une table de contactes
		\item nommer les colonnes dans la première ligne
		\item ces noms seront utilisés comme champs de fusion
	\end{itemize}
\end{minipage}
\begin{minipage}{0.59\textwidth}
	\hgraphpage{publi-source.png}
\end{minipage}

\end{frame}

\begin{frame}
\frametitle{Publipostage : Les données}
\framesubtitle{Importation de la source de données}

\begin{minipage}{0.40\textwidth}
	\begin{itemize}
%		\item Onglet : \optword{Publipostage}
		\item Groupe : \optword{Démarrer la fusion et le publipostage}
%		\item Option : \optword{Sélection de destinataires}
		\item on peut créer une nouvelle liste
		\item ou importer la liste Excel qu'on a créé
		\item utiliser la première ligne pour les noms des champs
		\item sinon, les noms des champs seront : F1, F2, F3, etc.
	\end{itemize}
\end{minipage}
\begin{minipage}{0.59\textwidth}
	\hgraphpage[.5\textwidth]{publi-source-creation.png}
	
	\hgraphpage{publi-source-importer.png}
\end{minipage}

\end{frame}

\begin{frame}
\frametitle{Publipostage : Les données}
\framesubtitle{Opérations sur la source de données}

\begin{minipage}{0.53\textwidth}
	\begin{itemize}
%		\item Onglet : \optword{Publipostage}
		\item Groupe : \optword{Démarrer la fusion et le publipostage}
		\item Option : \optword{Modifier la liste de destinataires}
		\item on peut trier et filtrer les données
	\end{itemize}
\end{minipage}
\begin{minipage}{0.45\textwidth}
	\hgraphpage{publi-source-operation.png}
\end{minipage}

\hgraphpage[.49\textwidth]{publi-source-filtrer.png}
\hgraphpage[.49\textwidth]{publi-source-trier.png}

\end{frame}

\subsection{Masque de publipostage}

\begin{frame}
\frametitle{Publipostage : Masque de publipostage}
\framesubtitle{Création du masque}

\begin{minipage}{0.74\textwidth}
	\begin{itemize}
%		\item Onglet : \optword{Publipostage}
		\item Groupe : \optword{Démarrer la fusion et le publipostage}
%		\item Option : \optword{Démarrer la fusion et le publipostage}
	\end{itemize}
\end{minipage}
\begin{minipage}{0.25\textwidth}
	\hgraphpage{publi-masque-type.png}
\end{minipage}

\begin{itemize}
	\item \optword{lettre} : des formules d'appel personnalisées. Chaque lettre est imprimée sur une page séparée. 
	\item \optword{email} : on doit fournir adresses email des destinataires. Les emails seront envoyés à partir de Word.
	\item \optword{enveloppe} : avec l'adresse destinataire et expéditeur
	\item \optword{étiquette} : plusieurs lettres sur la même page
	\item \optword{répertoire} : lister tous les contactes dans la même page
\end{itemize}

\end{frame}

\begin{frame}
\frametitle{Publipostage : Masque de publipostage}
\framesubtitle{Enveloppes}

\begin{minipage}{0.53\textwidth}
	\begin{itemize}
%		\item Onglet : \optword{Publipostage}
%		\item Groupe : \optword{Démarrer la fusion et le publipostage}
%		\item Option : \optword{Démarrer la fusion et le publipostage}
		\item on peut choisir la taille de l'enveloppe 
		\item on peut définir les marges 
		\item l'adresse du destinataire est en haut de la page
		\item l'adresse de l'expéditeur (l'adresse de retour) est en bas de la page
	\end{itemize}
\end{minipage}
\begin{minipage}{0.45\textwidth}
	\hgraphpage{publi-masque-enveloppe.png}
\end{minipage}

\end{frame}

\begin{frame}
\frametitle{Publipostage : Masque de publipostage}
\framesubtitle{Étiquettes}

\begin{minipage}{0.63\textwidth}
	\begin{itemize}
%		\item Onglet : \optword{Publipostage}
%		\item Groupe : \optword{Démarrer la fusion et le publipostage}
%		\item Option : \optword{Démarrer la fusion et le publipostage}
		\item on peut choisir le nombre des étiquettes par page
		\item modifier seulement la première étiquette 
		\item pour recopier son contenu sur les autres étiquettes, cliquer sur \optword{Mettre à jour les étiquettes}
	\end{itemize}
\end{minipage}
\begin{minipage}{0.36\textwidth}
	\hgraphpage{publi-masque-etiquette1.png}
	
	\hgraphpage{publi-masque-etiquette2.png}
\end{minipage}

\end{frame}

\begin{frame}
\frametitle{Publipostage : Masque de publipostage}
\framesubtitle{Champs de fusion}

\begin{minipage}{0.63\textwidth}
	\begin{itemize}
%		\item Onglet : \optword{Publipostage}
		\item Groupe : \optword{Champs d'écriture et d'insertion}
%		\item Option : \optword{Insérer un champs de diffusion}
		\item Les titres de la première ligne de Excel seront les noms des champs 
		\item Dans le texte, on peut appliquer des styles sur ces champs
	\end{itemize}
\end{minipage}
\begin{minipage}{0.36\textwidth}
	\hgraphpage[.5\textwidth]{publi-masque-champs.png}
	
	\hgraphpage{publi-masque-champs2.png}
\end{minipage}

\end{frame}

\begin{frame}
\frametitle{Publipostage : Masque de publipostage}
\framesubtitle{Bloc d'adresse et formule d'appel}

\begin{minipage}{0.69\textwidth}
	\begin{itemize}
%		\item Onglet : \optword{Publipostage}
%		\item Groupe : \optword{Champs d'écriture et d'insertion}
		\item Il faut faire correspondre les champs 
		\item on peut insérer des blocs d'adresse et des formules d'appel prédéfinis
	\end{itemize}

	\hgraphpage[.59\textwidth]{publi-masque-adresse.png}
	\hgraphpage[.39\textwidth]{publi-masque-appel.png}
\end{minipage}
\begin{minipage}{0.30\textwidth}
	\hgraphpage{publi-bloc.png}
	
%	\null\vfill\null
	~\vfill
	
	\hgraphpage{publi-masque-corresp.png}
\end{minipage}

\end{frame}

\begin{frame}
\frametitle{Publipostage : Masque de publipostage}
\framesubtitle{Règles (1/3)}

\begin{minipage}{0.63\textwidth}
	\begin{itemize}
		%		\item Onglet : \optword{Publipostage}
		%		\item Groupe : \optword{Champs d'écriture et d'insertion}
		\item \optword{Demander} : ajouter du texte lors de fusion
		\begin{itemize}
			\item créer un signet avec l'invite et le texte par défaut
			\item dans le masque, mettre le curseur dans l'emplacement où on veut insérer le texte
			\item \optword{CTRL + F9} ensuite écrire le nom du signet entre les accolades
		\end{itemize}
		\item \optword{Remplir} : comme \optword{Demander} mais sans l'utilisation des signets. Donc, on peut insérer le texte dans un seul emplacement dans le masque
		\item \optword{N\textdegree Enregistr. de fusion} : afficher la position numérotée du contact dans la liste
	\end{itemize}
\end{minipage}
\begin{minipage}{0.36\textwidth}
	\hgraphpage[.5\textwidth]{publi-masque-regles.png}
	
	\hgraphpage{publi-masque-demander.png}
	
	\hgraphpage[.65\textwidth]{publi-masque-remplir.png}
\end{minipage}

\end{frame}

\begin{frame}
\frametitle{Publipostage : Masque de publipostage}
\framesubtitle{Règles (2/3)}

\begin{minipage}{0.63\textwidth}
	\begin{itemize}
		\item \optword{Si ... Alors ... Sinon} : insérer un texte selon une condition
		\item \optword{N\textdegree séquence de fusion} : afficher l'ordre de fusion
		\item \optword{Suivant} : sauter vers le contact suivant après cette règle
		\item \optword{Suivant si} : sauter vers le contact suivant après cette règle si la condition est satisfaite
		\item \optword{Définir signet} : définir un champs avec une valeur qu'on puisse changer avant la fusion
		\item \optword{Sauter l'enregistrement si} : comme \optword{Suivant si}, mais la fusion courante sera annulée avant de passer à la suivante
	\end{itemize}
\end{minipage}
\begin{minipage}{0.36\textwidth}
	
	\hgraphpage{publi-masque-regles-si.png}
	
	\hgraphpage[.6\textwidth]{publi-masque-regles-suivantsi.png}
	
	\hgraphpage[.6\textwidth]{publi-masque-regles-sautersi.png}
\end{minipage}

\end{frame}

\subsection{Aperçu et Fusion}

\begin{frame}
\frametitle{Publipostage: Aperçu et Fusion}
\framesubtitle{Aperçu}

\begin{minipage}{0.74\textwidth}
	\begin{itemize}
%		\item Onglet : \optword{Publipostage}
		\item Groupe : \optword{Aperçu des résultats}
		\item afficher la lettre telle qu'elle sera envoyée au premier destinataire de la liste. Les flèches situées à droite de ce bouton permettent de faire défiler les destinataire.
	\end{itemize}
\end{minipage}
\begin{minipage}{0.25\textwidth}
	\hgraphpage{publi-apercu.png}
\end{minipage}

\vfill

\hgraphpage[.3\textwidth]{publi-apercu1.png}\hfill
\hgraphpage[.3\textwidth]{publi-apercu2.png}

\end{frame}

\begin{frame}
\frametitle{Publipostage: Aperçu et Fusion}
\framesubtitle{Fusion}

\begin{minipage}{0.69\textwidth}
	\begin{itemize}
%		\item Onglet : \optword{Publipostage}
		\item Groupe : \optword{Terminer \& fusionner}
		\item on peut générer un nouveau document avec les données fusionnées
		\item on peut fusionner directement vers l'imprimante
		\item on peut envoyer les lettres fusionnées par email
	\end{itemize}
\end{minipage}
\begin{minipage}{0.30\textwidth}
	\hgraphpage{publi-fusion.png}
	
	%	\hgraphpage{publi-apercu2.png}
\end{minipage}


\end{frame}

\begin{frame}
\frametitle{Publipostage}
\framesubtitle{Un peu d'humeur}
\begin{center}
	\vgraphpage{mailmerge-humour.png}
\end{center}
\end{frame}


%===================================================================================
\section{Formulaires}
%===================================================================================

\begin{frame}
\frametitle{Formulaires}

\begin{minipage}{0.50\textwidth}
	D'après Larousse, un formulaire est 
	{
	\setlength{\textwidth}{.9\textwidth}%
	\begin{definition}
		Imprimé sur lequel figure une série de questions administratives auxquelles l'intéressé doit répondre ; questionnaire.
	\end{definition}
    }

	Un formulaire électronique contient:
	\begin{itemize}
		\item du texte explicatif et des étiquettes
		\item des champs à saisir 
	\end{itemize}
\end{minipage}
\begin{minipage}{0.49\textwidth}
	\hgraphpage{forms.png}
\end{minipage}

\end{frame}

\begin{frame}
\frametitle{Formulaires}

\begin{itemize}
	\item supprimer les informations non nécessaires. Par exemple, le champs \expword{age} si le champs \expword{date de naissance} existe
	\item toujours utiliser des étiquettes pour indiquer le sens des champs
	\item l'insertion des étiquettes au dessus des champs améliore la lisibilité
	\item éviter l'insertion d'une question à coté de l'autre
	\item regrouper les champs en sections 
	\item les tailles des champs doivent refléter la taille de leurs contenus. Par exemple, le champs \expword{code postal} doit être moins long que le champs \expword{adresse}
\end{itemize}

\end{frame}

\subsection{Les champs}

%\begin{frame}
%\frametitle{Formulaires}
%\framesubtitle{Les champs}
%
%\begin{itemize}
%	\item champs de texte 
%	\item zone de texte
%	\item bouton radio
%	\item case à couché 
%	\item liste déroulante  
%	\item sélecteur de date 
%\end{itemize}
%
%\end{frame}

\begin{frame}
\frametitle{Formulaires: Les champs}
\framesubtitle{champs texte}

\begin{minipage}{0.69\textwidth}
	\begin{itemize}
		\item pour saisir du texte
		\item sur une seule ligne 
		\item exemple: \expword{nom}, \expword{prénom}, \expword{adresse}, etc.
	\end{itemize}
\end{minipage}
\begin{minipage}{0.3\textwidth}
	\hgraphpage{input.png}
\end{minipage}

\end{frame}

\begin{frame}
\frametitle{Formulaires: Les champs}
\framesubtitle{zone de texte}

\begin{minipage}{0.50\textwidth}
	\begin{itemize}
		\item pour saisir du texte
		\item sur plusieurs lignes
		\item exemple: \expword{message}, \expword{commentaire}, etc.
	\end{itemize}
\end{minipage}
\begin{minipage}{0.49\textwidth}
	\hgraphpage{zonetexte.png}
\end{minipage}

\end{frame}


\begin{frame}
\frametitle{Formulaires: Les champs}
\framesubtitle{Bouton radio}

\begin{minipage}{0.59\textwidth}
	\begin{itemize}
		\item pour choisir une seule option parmi plusieurs
%		\item sur plusieurs lignes
%		\item exemple: \expword{message}, \expword{commentaire}, etc.
	\end{itemize}
\end{minipage}
\begin{minipage}{0.40\textwidth}
	\hgraphpage{radiobutton.png}
\end{minipage}

\begin{itemize}
	\item les boutons qui ont le même sens doivent avoir le même nom ou le même groupe
	\item de préférence, ils sont alignés verticalement 
	\item on les utilise : 
	\begin{itemize}
		\item lorsqu'il y a moins de 5 options 
		\item quand on veut comparer les options. Par exemple, \expword{la liste des prix}
		\item quand la visibilité et la rapidité de réponse sont prioritaires
	\end{itemize}
\end{itemize}

\end{frame}

\begin{frame}
\frametitle{Formulaires: Les champs}
\framesubtitle{Case à couché}

\begin{minipage}{0.59\textwidth}
	\begin{itemize}
		\item pour choisir plusieurs options
	\end{itemize}
\end{minipage}
\begin{minipage}{0.40\textwidth}
	\hgraphpage{checkbox.png}
\end{minipage}

\begin{itemize}
	\item de préférence, elles sont alignées verticalement 
	\item on les utilise : 
	\begin{itemize}
		\item lorsqu'il y a moins de 5 options 
		\item quand la visibilité et la rapidité de réponse sont prioritaires
		\item lorsqu'on a une question avec réponse booléenne (oui/non)
	\end{itemize}
\end{itemize}

\end{frame}


\begin{frame}
\frametitle{Formulaires: Les champs}
\framesubtitle{Liste déroulante}

\begin{minipage}{0.69\textwidth}
	\begin{itemize}
		\item pour choisir une seule option parmi plusieurs
		\item on les utilise : 
		\begin{itemize}
			\item lorsqu'il y a plus de 5 options 
			\item quand la première option est celle par défaut
		\end{itemize}
		\item des fois, la liste déroulante avec plusieurs options est trop gênante. C'est mieux d'utiliser un champs de texte avec la auto-complétion. Par exemple, \expword{choix du pays}.
	\end{itemize}
\end{minipage}
\begin{minipage}{0.30\textwidth}
	\hgraphpage{combobox.png}
\end{minipage}

\end{frame}

\begin{frame}
\frametitle{Formulaires: Les champs}
\framesubtitle{Sélecteur de date}

\begin{minipage}{0.50\textwidth}
	\begin{itemize}
		\item pour choisir une date
		\item exemples, \expword{date de naissance}, \expword{date de départ}, etc.
		\item par défaut, la date est celle d'aujourd'hui (du système)
		\item on peut fixer une date par défaut et des dates limites
	\end{itemize}
\end{minipage}
\begin{minipage}{0.49\textwidth}
	\hgraphpage{date.png}
\end{minipage}


\end{frame}

\subsection{Formulaires Word}

\begin{frame}
\frametitle{Formulaires}
\framesubtitle{Formulaires Word}

%\hgraphpage{publi-barre.png}

\begin{itemize}
	\item 
\end{itemize}


\end{frame}

\begin{frame}
\frametitle{Formulaires}
\framesubtitle{Formulaires Word}

\end{frame}

\subsection{Formulaires Google}

\begin{frame}
\frametitle{Formulaires}
\framesubtitle{Formulaires Google}

\end{frame}

\begin{frame}
\frametitle{Formulaires}
\framesubtitle{Formulaires Google}

\end{frame}


\begin{frame}
\frametitle{Formulaires}
\framesubtitle{Un peu d'humeur}
\begin{center}
	\vgraphpage{forms-humour.jpg}
\end{center}
\end{frame}


\insertbibliography{Bweb04}{*}


\end{document}

