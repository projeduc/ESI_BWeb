% !TEX TS-program = pdflatex
% !TeX program = pdflatex
% !TEX encoding = UTF-8
% !TEX spellcheck = fr

\documentclass[xcolor=table]{beamer}


%\usepackage{fullpage}
%\usepackage[left=2.8cm,right=2.2cm,top=2 cm,bottom=2 cm]{geometry}
\setbeamersize{text margin left=10pt,text margin right=10pt}
\usepackage{amsmath,amssymb} 
\usepackage[T1]{fontenc}
\usepackage[utf8]{inputenc}
\usepackage[english,french]{babel}
\usepackage{txfonts}
\usepackage[]{graphicx}
\usepackage{multirow}
\usepackage{hyperref}
\usepackage{colortbl}
\usepackage{listings}
\usepackage{wrapfig}
\usepackage{multicol}

\hypersetup{
	colorlinks,
	urlcolor = blue
}

%\renewcommand{\baselinestretch}{1.5}

\def\supit#1{\raisebox{0.8ex}{\small\it #1}\hspace{0.05em}}

\AtBeginSection{%
	\begin{frame}
		\sectionpage
	\end{frame}
}

\newcommand{\rottext}[2]{%
	\rotatebox{90}{%
	\begin{minipage}{#1}%
		\raggedleft#2%
	\end{minipage}%
	}%
}

\usepackage{longtable}
\usepackage{tabu}


\institute{ %
École  nationale Supérieure d'Informatique (ESI, ex. INI), Algérie
}
\author[ \textbf{\footnotesize  \insertframenumber/\inserttotalframenumber} \hspace*{\fill} ESI (2019-2020)] %
{ARIES Abdelkrime}
%\titlegraphic{\includegraphics[height=1cm]{../img/esi-logo.png}%\hspace*{4.75cm}~


\date{Année unniversitaire: 2019/2020} %\today

\usetheme{Warsaw} % Antibes Boadilla Warsaw

\beamertemplatenavigationsymbolsempty

%\setbeamertemplate{headline}{}

\definecolor{lightblue}{HTML}{D0D2FF}
\definecolor{lightyellow}{HTML}{FFFFAA}
\definecolor{darkblue}{HTML}{0000BB}
\definecolor{olivegreen}{HTML}{006600}
\definecolor{violet}{HTML}{6600CC}

\newcommand{\keyword}[1]{\textcolor{red}{\bfseries\itshape #1}}
\newcommand{\expword}[1]{\textcolor{olivegreen}{#1}}
\newcommand{\optword}[1]{\textcolor{violet}{\bfseries #1}}

\makeatletter
\newcommand\mysphere{%
	\parbox[t]{10pt}{\raisebox{0.2pt}{\beamer@usesphere{item projected}{bigsphere}}}}
\makeatother

%\let\oldtabular\tabular
%\let\endoldtabular\endtabular
%\renewenvironment{tabular}{\rowcolors{2}{white}{lightblue}\oldtabular\rowcolor{blue}}{\endoldtabular}


\NoAutoSpacing %french autospacing after ":"

\title[BWEB : 03- Rédaction (LibreOffice Writer)] %
{Bureautique et Web \\Chapitre 03 : Rédaction d'un document numérique \\ \slshape\small  LibreOffice Writer}  

\changegraphpath{../img/Bweb03-redaction/writer/}

\begin{document}

\begin{frame}
\frametitle{Rédaction d'un document numérique}
\framesubtitle{LibreOffice Writer}

\begin{itemize}
	\item WYSIWYG : What You See Is What You Get
	\item fait parti de la suite bureautique : Libre office
	\item open source et gratuit
	\item disponible sur plusieurs systèmes d'exploitation (Windows, Linux, macOS)
\end{itemize}

\end{frame}

\begin{frame}
\frametitle{Rédaction d'un document numérique}
\framesubtitle{LibreOffice Writer}

\begin{center}
	\vgraphpage{libreoffice-writer-preview1.png}
\end{center}

\end{frame}

\begin{frame}
\frametitle{Rédaction d'un document numérique}
\framesubtitle{LibreOffice Writer : Onglets}
\vspace{-.7em}
Affichage --> Interface utilisateur --> Onglets

\begin{center}
	\vgraphpage[.8\textheight]{libreoffice-writer-preview2.png}
\end{center}

\end{frame}

\begin{frame}
\frametitle{Rédaction d'un document numérique}
\framesubtitle{LibreOffice Writer : Plan}

\begin{multicols}{2}
%	\small
	\tableofcontents
\end{multicols}
\end{frame}

%===================================================================================
\section{Préparer un document}
%===================================================================================

%\begin{frame}
%\frametitle{Préparer un document}
%
%\end{frame}

\subsection{Pages}

\begin{frame}[t]
\frametitle{Préparer un document : Pages}
\framesubtitle{Taille, Orientation, Marges}

\begin{minipage}{0.38\textwidth}
	\begin{itemize}
		\item Menu : \optword{Format} 
		\item Sous menu : \optword{Page}
	\end{itemize}
\end{minipage}
\begin{minipage}{0.6\textwidth}
	\hgraphpage{page.png}
\end{minipage}

\end{frame}

\begin{frame}[t]
\frametitle{Préparer un document : Pages}
\framesubtitle{Colonnes (Page)}

\begin{minipage}{0.38\textwidth}
	\begin{itemize}
		\item Menu : \optword{Format} 
		\item Sous menu : \optword{Colonnes}
		\item pour définir des pages avec plusieurs colonnes
	\end{itemize}
\end{minipage}
\begin{minipage}{0.6\textwidth}
	\hgraphpage{colonnes-page.png}
\end{minipage}

\end{frame}

\begin{frame}[t]
\frametitle{Préparer un document : Pages}
\framesubtitle{Colonnes (Sélection)}

\begin{minipage}{0.38\textwidth}
	\begin{itemize}
		\item Sélectionner le texte à mettre en colonnes
		\item Menu : \optword{Format} 
		\item Sous menu : \optword{Colonnes}
		\item pour définir des pages avec plusieurs colonnes
	\end{itemize}
\end{minipage}
\begin{minipage}{0.6\textwidth}
	\hgraphpage{colonnes-selection.png}
\end{minipage}

\end{frame}

\begin{frame}[t]
\frametitle{Préparer un document : Pages}
\framesubtitle{Saute de page et section}

\begin{minipage}{0.44\textwidth}
	Pour insérer un saut de page : 
	\begin{itemize}
		\item Menu : \optword{Insertion} 
		\item Sous menu : \optword{Saut de page}
	\end{itemize}

	Pour insérer un saut de page avec changement de style des pages : 
	\begin{itemize}
		\item Menu : \optword{Insertion} 
		\item Sous menu : \optword{Autres sauts}
		\item Sous sous menu : \optword{Saut manuel}
	\end{itemize}
\end{minipage}
\begin{minipage}{0.55\textwidth}
	\hgraphpage{sautes.png}

	\vspace{6pt}
	\hgraphpage[.6\textwidth]{saute-manuel.png}
\end{minipage}

\end{frame}

\begin{frame}[t]
\frametitle{Préparer un document : Pages}
\framesubtitle{Page de titre}

\begin{minipage}{0.54\textwidth}
	\begin{itemize}
		\item Menu : \optword{Format} 
		\item Sous menu : \optword{Page de titre}
	\end{itemize}
\end{minipage}
\begin{minipage}{0.45\textwidth}
	\hgraphpage{pagetitre.png}
\end{minipage}

\end{frame}

\begin{frame}[t]
\frametitle{Préparer un document : Pages}
\framesubtitle{Bordures}

\begin{minipage}{0.43\textwidth}
	\begin{itemize}
		\item Menu : \optword{Format} 
		\item Sous menu : \optword{Page}
		\item Onglet : \optword{Bordures}
	\end{itemize}
\end{minipage}
\begin{minipage}{0.55\textwidth}
	\hgraphpage{bordures.png}
\end{minipage}

\end{frame}

\begin{frame}[t]
\frametitle{Préparer un document : Pages}
\framesubtitle{Filigrane}

\begin{minipage}{0.43\textwidth}
	\begin{itemize}
		\item Menu : \optword{Format} 
		\item Sous menu : \optword{Filigrane}
	\end{itemize}
\end{minipage}
\begin{minipage}{0.55\textwidth}
	\hgraphpage{filigrane.png}
\end{minipage}

\end{frame}

\begin{frame}[t]
\frametitle{Préparer un document : Pages}
\framesubtitle{Arrière-plan}

\begin{minipage}{0.43\textwidth}
	\begin{itemize}
		\item Menu : \optword{Format} 
		\item Sous menu : \optword{Page}
		\item Onglet : \optword{Arrière plan}
		\item Un arrière-plan peut être : 
		\begin{itemize}
			\item Couleur unique
			\item Dégradé
			\item Bitmap (comme Texture de Word)
			\item Motif
			\item Hachure (comme Motif)
		\end{itemize}
	\end{itemize}
\end{minipage}
\begin{minipage}{0.55\textwidth}
	\hgraphpage{arriereplan.png}
\end{minipage}

\end{frame}


\begin{frame}
\frametitle{Préparer un document : Pages}
\framesubtitle{En-tête et pied de page}

\begin{itemize}
	\item Menu : \optword{Insertion} 
	\item Sous menu : \optword{En-tête et pied de page}
	\item Ensuite, on choisit l'entête ou le pied de page avec le style
\end{itemize}

On peut ajouter des champs : 
\begin{itemize}
	\item Menu : \optword{Insertion} 
	\item Sous menu : \optword{Champ}
	\item Numéro de page, nombre de pages, la date, etc. 
\end{itemize}

Pour afficher le contenu des champs : 
\begin{itemize}
	\item Menu : \optword{Affichage} 
	\item Désactiver le Sous menu : \optword{Nom des champs} 
\end{itemize}

\end{frame}


\subsection{Paragraphes}

\begin{frame}
\frametitle{Préparer un document : Paragraphes}
\framesubtitle{Accès rapide}

\begin{minipage}{0.64\textwidth}
	\begin{itemize}
		\item Il existe des options sur la barre d'outils
		\item Aussi, sur le volet latéral (à droite de la fenêtre)
		\begin{itemize}
			\item Retraits et espacement 
			\item Alignement
			\item Insertion des listes
			\item Interligne
		\end{itemize}
	\end{itemize}

	\hgraphpage[.6\textwidth]{paragraphe1.png}

\end{minipage}
\begin{minipage}{0.35\textwidth}
	
	\hgraphpage{volet-propriete.png}
\end{minipage}

\end{frame}

\begin{frame}
\frametitle{Préparer un document : Paragraphes}
\framesubtitle{Options avancées}

\begin{minipage}{0.39\textwidth}
\begin{itemize}
	\item Menu : \optword{Format} 
	\item Sous menu : \optword{Paragraphe}
	\item Plusieurs onglets : 
	\begin{itemize}
		\item Retraits et espacement 
		\item Alignement
		\item d'autres onglets pour l'arrière-plan du paragraphe, etc.
	\end{itemize}
\end{itemize}
\end{minipage}
\begin{minipage}{0.60\textwidth}
	\hgraphpage{paragraphe.png}
\end{minipage}

\end{frame}

\subsection{Mise en forme et styles}

\begin{frame}
\frametitle{Préparer un document : Mise en forme et styles}
\framesubtitle{Mise en forme : Accès rapide}

\begin{minipage}{0.64\textwidth}
\begin{itemize}
	\item Il existe des options sur la barre d'outils
	\item Aussi, sur le volet latéral (à droite de la fenêtre)
	\begin{itemize}
		\item Police
		\item Couleur 
		\item Espacement de caractère
	\end{itemize}
\end{itemize}

\hfill\null

\hgraphpage{miseforme.png}

\end{minipage}
\begin{minipage}{0.35\textwidth}
	\hgraphpage{volet-propriete.png}
%	\end{center}
\end{minipage}

\end{frame}

\begin{frame}
\frametitle{Préparer un document : Mise en forme et styles}
\framesubtitle{Mise en forme : Options avancées}

\begin{minipage}{0.39\textwidth}
\begin{itemize}
	\item Menu : \optword{Format} 
	\item Sous menu : \optword{Caractères}
	\item Plusieurs onglets : 
	\begin{itemize}
		\item Police
		\item Effets de caractère
		\item d'autres onglets pour la position, etc.
	\end{itemize}
\end{itemize}
\end{minipage}
\begin{minipage}{0.60\textwidth}
	
	\hgraphpage{caractere.png}
	%	\end{center}
\end{minipage}

\end{frame}


\begin{frame}
\frametitle{Préparer un document : Mise en forme et styles}
\framesubtitle{Styles}

\begin{minipage}{0.69\textwidth}
	Trois emplacements pour gérer les styles : 
	\begin{itemize}
		\item Barre d'outils
		\item Menu \optword{Styles} 
		\item Volet latéral (à droite de la fenêtre)
	\end{itemize}

	\vspace{12pt}

	Il existe des styles pour : 
	\begin{itemize}
		\item les paragraphes
		\item les caractères 
		\item les pages
		\item etc.
	\end{itemize}
\end{minipage}
\begin{minipage}{0.30\textwidth}
	\hgraphpage{styles.png}
	
	\vspace{6pt} 
	
	\hgraphpage{volet-styles.png}
\end{minipage}

\end{frame}

\subsection{Modification}%Recherche et remplacement

\begin{frame}[t]
\frametitle{Préparer un document : Modification}
\framesubtitle{Recherche}

%\begin{minipage}{0.60\textwidth}
	Pour afficher la barre de recherche en bas de la fenêtre :
	\begin{itemize}
		\item Menu : \optword{Affichage}
		\item Sous menu : \optword{Barres d'outils}
		\item sélectionner \optword{Rechercher}
	\end{itemize}
%\end{minipage}
%\begin{minipage}{0.38\textwidth}
%	
%\end{minipage}

\hgraphpage{rechercher.png}

\end{frame}

\begin{frame}[t]
\frametitle{Préparer un document : Modification}
\framesubtitle{Remplacement}

\begin{minipage}{0.38\textwidth}
	\begin{itemize}
		\item Menu : \optword{Edition}
		\item Sous menu : \optword{Rechercher \& remplacer}
		\item Clavier : \optword{Ctrl + H}
	\end{itemize}
\end{minipage}
\begin{minipage}{0.6\textwidth}	
	\hgraphpage{remplacer.png}
\end{minipage}

\end{frame}

%===================================================================================
\section{Enrichir un document}
%===================================================================================

%\begin{frame}
%\frametitle{Enrichir un document}
%
%\end{frame}

\subsection{Tableaux}

\begin{frame}
\frametitle{Enrichir un document : Tableaux}
\framesubtitle{Création}

\begin{minipage}{0.38\textwidth}
	\begin{itemize}
		\item Soit dans la barre d'outils
		\item ou
		\item Menu : \optword{Tableau}
		\item Sous menu : \optword{Insérer un tableau}
	\end{itemize}
\end{minipage}
\begin{minipage}{0.20\textwidth}	
	\hgraphpage{tableau1.png}
\end{minipage}
\begin{minipage}{0.40\textwidth}	
	\hgraphpage{tableau2.png}
\end{minipage}

\end{frame}

\begin{frame}
\frametitle{Enrichir un document : Tableaux}
\framesubtitle{Convertir un texte en tableau}

\begin{minipage}{0.59\textwidth}
	\begin{itemize}
		\item Sélectionner le texte
		\item Menu : \optword{Tableau}
		\item Sous menu : \optword{Convertir}
		\item Sous sous menu : \optword{Texte en tableau}
	\end{itemize}
\end{minipage}
\begin{minipage}{0.40\textwidth}	
	\hgraphpage{tableau-convertir.png}
\end{minipage}

\end{frame}

\begin{frame}[t]
\frametitle{Enrichir un document : Tableaux}
\framesubtitle{Options}

\hgraphpage{tableau-options.png}

\begin{itemize}
	\item Lorsqu'on clique sur le tableau, une barre s'apparaitra
	\item Les mêmes options peuvent être accédées à partir du menu \optword{Tableau}
	\begin{itemize}
		\item Insérer des lignes et des colonnes
		\item Supprimer les lignes, les colonnes ou le tableau
		\item Fusionner les cellules
		\item Taille des cellules
		\item Couleurs et bordures
		\item Formule
	\end{itemize}
\end{itemize}

\end{frame}

\subsection{Illustrations}

\begin{frame}[t]
\frametitle{Enrichir un document : Illustrations}
\framesubtitle{Insérer une illustration}

\begin{minipage}{0.64\textwidth}
	\begin{itemize}
		\item La barre d'outils (deux ensembles d'outils), par ordre : 
		\begin{itemize}
			\item ajouter une image,
			\item ajouter un diagramme,
			\item ajouter ue zone de texte,
			\item dessiner des lignes,
			\item dessiner des formes,
			\item afficher les fonctions de dessin 
		\end{itemize}
		\item Pour ajouter du fontwork, aller vers le menu \optword{Insertion}, le sous-menu \optword{Fontwork}
	\end{itemize}
\end{minipage}
\begin{minipage}{0.35\textwidth}
	\hgraphpage[.48\textwidth]{illustrations.png}
	\hgraphpage[.48\textwidth]{illustrations2.png}
%	\hgraphpage{illustrations-graphique.png}
	
	\hgraphpage{fontwork.png}
\end{minipage}

\hgraphpage{illustrations-dessin.png}

\end{frame}

\begin{frame}[t]
\frametitle{Enrichir un document : Illustrations}
\framesubtitle{Les outils de l'image}

\hgraphpage{image-outils.png}

\begin{itemize}
	\item Lorsqu'on clique sur une image, une barre des outils image s'apparaitra
	\item On peut : 
	\begin{itemize}
		\item définir l'habillage de l'image (sa position par rapport au texte)
		\item aligner l'image par rapport à la page
		\item définir la position de l'image par rapport d'autres images
		\item appliquer un cadre sur l'image
		\item appliquer des effets sur l'image (Noir et blanc, etc.) 
		\item pivoter l'image
	\end{itemize}
\end{itemize}

\end{frame}

\subsection{Liens et renvoi}

\begin{frame}[t]
\frametitle{Enrichir un document : Liens et renvoi}
\framesubtitle{Insérer un lien hypertexte}

\begin{minipage}{0.38\textwidth}
	\begin{itemize}
		\item Menu : \optword{Insertion}
		\item Sous menu : \optword{Hyperlien}
		\item Onglet : \optword{Internet}
		\item Insérer le lien
		\item Insérer le texte à afficher
	\end{itemize}
\end{minipage}
\begin{minipage}{0.6\textwidth}	
	\hgraphpage{liens-hypertexte.png}
\end{minipage}

\end{frame}


\begin{frame}[t]
\frametitle{Enrichir un document : Liens et renvoi}
\framesubtitle{Renvoi vers un emplacement dans le document}

\begin{minipage}{0.5\textwidth}
	
	\begin{itemize}
		\item Sélectionner le texte de destination 
		\item Menu : \optword{Insertion}
		\item Sous menu : \optword{Repère de texte}
		\item Donner un nom à la destination
		\item Cliquer sur un autre emplacement
		\item Sous menu : \optword{Hyperlien}
		\item Onglet : \optword{Document}
		\item Dans \optword{Cible}, cliquer sur l'icône à droite pour avoir les cibles
		\item Choisir \optword{Repère de texte}
		\item On peut, aussi, cibler des titres, etc.
	\end{itemize}
\end{minipage}
\begin{minipage}{0.49\textwidth}	
	
	\hgraphpage[.4\textwidth]{renvoi-marquer.png}
	
		\begin{tabular}{@{}l@{}l@{}}
			\hgraphpage[.8\textwidth, valign=t]{renvoi-emplacement.png} & 
			\hgraphpage[.18\textwidth, valign=t]{renvoi-cible.png} \\
		\end{tabular}
	
\end{minipage}

\end{frame}

\subsection{Formules et symboles}

\begin{frame}
\frametitle{Enrichir un document : Formules et symboles}
\framesubtitle{Insérer une formule}

\begin{minipage}{0.3\textwidth}
	\begin{itemize}
		\item Menu : \optword{Insertion}
		\item Sous menu : \optword{Objet}
		\item Sous sous menu : \optword{Formule}
	\end{itemize}
\end{minipage}
\begin{minipage}{0.69\textwidth}
	\hgraphpage{equation.png}
\end{minipage}

\end{frame}

\begin{frame}
\frametitle{Enrichir un document : Formules et symboles}
\framesubtitle{Insérer des symboles}

\begin{minipage}{0.50\textwidth}
	\begin{itemize}
		\item Soit dans la barre d'outils
		\item Ou
		\item Menu : \optword{Insertion}
		\item Sous menu : \optword{Caractères spéciaux}
		\item On peut choisir une symbole parmi celles les plus utilisées
		\item On peut naviguer la table des caractères
	\end{itemize}
\end{minipage}
\begin{minipage}{0.49\textwidth}
	\hgraphpage[.3\textwidth]{symboles.png}
	
	\hgraphpage{symboles-table.png}
\end{minipage}

\end{frame}

%===================================================================================
\section{Références}
%===================================================================================

%\begin{frame}
%\frametitle{Références}
%
%\end{frame}

\subsection{Notes de bas de page}

\begin{frame}[t]
\frametitle{Références : Notes de bas de page}
\framesubtitle{Insérer une note de bas de page}

\begin{minipage}{0.89\textwidth}
	\begin{itemize}
		\item Sélectionner le mot concerné par la note
		\item Menu : \optword{Insertion}
		\item Sous menu : \optword{Notes de bas de page / de fin}
		\item Sous sous menu : \optword{Notes de bas de page}
		\item Ou
		\item A partir de la barre d'outils
		\item Rédiger la note
	\end{itemize}
\end{minipage}
\begin{minipage}{0.10\textwidth}
	\hgraphpage{bas-page.png}
\end{minipage}

\end{frame}

\subsection{Tables (de matières et des illustrations)}

\begin{frame}
\frametitle{Références : Tables (de matières et des illustrations)}
\framesubtitle{Table de matières : Insérer une table de matières}

\begin{itemize}
	\item Mettre le curseur dans la position où on veut insérer la table
	\item Menu : \optword{Insertion}
	\item Sous menu : \optword{Table des matières et index}
\end{itemize}
\begin{minipage}{0.39\textwidth}
	\begin{itemize}
		%	\item Mettre le curseur dans la position où on veut insérer la table
		%	\item Menu : \optword{Insertion}
		%	\item Sous menu : \optword{Table des matières et index}
		\item Sous sous menu : \optword{Table des matières, index ou bibliographie}
		\item Choisir le type : \optword{Table des matières}
		\item On peut créer une table de matière pour chaque chapitre
		\item Ecrire le titre de la table
	\end{itemize}
\end{minipage}
\begin{minipage}{0.60\textwidth}
	\hgraphpage{table-matiere.png}
\end{minipage}

\end{frame}

\begin{frame}[t]
\frametitle{Références : Tables (de matières et des illustrations)}
\framesubtitle{Table des illustrations: Insérer des légendes}

\begin{minipage}{0.66\textwidth}
\begin{itemize}
	\item Sélectionner l'illustration (tableau ou image)
	\item Menu : \optword{Insertion}
	\item Sous menu : \optword{Légende}
	\item Choisir le type de la légende: Figure ou Tableau
	\item Ajouter un titre à la légende
	\item Pour changer le style de numérotation (par chapitre), appuyer sur \optword{Options}
\end{itemize}
\end{minipage}
\begin{minipage}{0.33\textwidth}

\hgraphpage{legende.png}

\hgraphpage[.80\textwidth]{legende-num.png}
\end{minipage}

\end{frame}

\begin{frame}[t]
\frametitle{Références : Tables (de matières et des illustrations)}
\framesubtitle{Table des illustrations : Insérer une table des illustrations}

\begin{itemize}
	\item Mettre le curseur dans la position où on veut insérer la table
	\item Menu : \optword{Insertion}
	\item Sous menu : \optword{Table des matières et index}
\end{itemize}
\begin{minipage}{0.39\textwidth}
\begin{itemize}
%	\item Mettre le curseur dans la position où on veut insérer la table
%	\item Menu : \optword{Insertion}
%	\item Sous menu : \optword{Table des matières et index}
	\item Sous sous menu : \optword{Table des matières, index ou bibliographie}
	\item Choisir le type : \optword{Index des figures} ou \optword{Index des tableaux}
	\item Ecrire le titre de la table
\end{itemize}
\end{minipage}
\begin{minipage}{0.60\textwidth}
	\hgraphpage{table-illustrations.png}
\end{minipage}

\end{frame}

\subsection{Index}

\begin{frame}[t]
\frametitle{Références : Index}
\framesubtitle{Insérer des entrées}

\begin{minipage}{0.69\textwidth}
	\begin{itemize}
		\item Sélectionner le mot à indexer 
		\item Menu : \optword{Insertion}
		\item Sous menu : \optword{Table des matières et index}
		\item Sous sous menu : \optword{Entrée d'index}
		\item On peut marquer ce mot, ou toutes ses occurrences dans le texte.
%		\item Pour le renvoi, mettre le curseur de la  souris n'import où dans le document 
%		\item Cliquer sur \optword{Marquer entrée} et choisir \optword{Renvoi}
	\end{itemize}
\end{minipage}
\begin{minipage}{0.30\textwidth}
	\hgraphpage{entree.png}
	
\end{minipage}

\end{frame}

\begin{frame}[t]
\frametitle{Références : Index}
\framesubtitle{Insérer un index}

\begin{itemize}
	\item Mettre le curseur dans la position où on veut insérer la table
	\item Menu : \optword{Insertion}
	\item Sous menu : \optword{Table des matières et index}
\end{itemize}
\begin{minipage}{0.39\textwidth}
	\begin{itemize}
		%	\item Mettre le curseur dans la position où on veut insérer la table
		%	\item Menu : \optword{Insertion}
		%	\item Sous menu : \optword{Table des matières et index}
		\item Sous sous menu : \optword{Table des matières, index ou bibliographie}
		\item Choisir le type : \optword{Index lexical} 
	\end{itemize}
\end{minipage}
\begin{minipage}{0.60\textwidth}
	\hgraphpage{index.png}
\end{minipage}

\end{frame}

\subsection{Bibliographie}

\begin{frame}[t]
\frametitle{Références : Bibliographie}
\framesubtitle{Gestion des sources}

\begin{minipage}{0.29\textwidth}
	\begin{itemize}
		\item Menu : \optword{Outils}
		\item Sous menu : \optword{Base de données bibliographique}
		\item On peut ajouter, modifier ou supprimer une source  
		\item Une source peut être un livre, article de journal, rapport, etc.
		\item Définir un abrégé pour la source
	\end{itemize}
\end{minipage}
\begin{minipage}{0.70\textwidth}	
	\hgraphpage{biblio-source.png}	
\end{minipage}

\end{frame}

\begin{frame}[t]
\frametitle{Références : Bibliographie}
\framesubtitle{Insérer une citation}

\begin{minipage}{0.68\textwidth}
\begin{itemize}
	\item Mettre le curseur à la fin de la phrase concernée par la citation
	\item Menu : \optword{Insertion}
	\item Sous menu : \optword{Table des matières et index}
	\item Sous sous menu : \optword{Entrée de bibliographie}
	\item Choisir la source appropriée en utilisant son abrégé
\end{itemize}
\end{minipage}
\begin{minipage}{0.30\textwidth}
	\hgraphpage{biblio-citation.png}
\end{minipage}

\end{frame}

\begin{frame}[t]
\frametitle{Références : Bibliographie}
\framesubtitle{Insérer la bibliographie}

\begin{itemize}
	\item Mettre le curseur dans la position où on veut insérer la table
	\item Menu : \optword{Insertion}
	\item Sous menu : \optword{Table des matières et index}
\end{itemize}
\begin{minipage}{0.39\textwidth}
	\begin{itemize}
		%	\item Mettre le curseur dans la position où on veut insérer la table
		%	\item Menu : \optword{Insertion}
		%	\item Sous menu : \optword{Table des matières et index}
		\item Sous sous menu : \optword{Table des matières, index ou bibliographie}
		\item Choisir le type : \optword{Bibliographie}
	\end{itemize}
\end{minipage}
\begin{minipage}{0.60\textwidth}
	\hgraphpage{biblio-biblio.png}
\end{minipage}

\end{frame}

%===================================================================================
\section{Révision et partage}
%===================================================================================

%\begin{frame}
%\frametitle{Révision et partage}
%
%\end{frame}

\subsection{Révision}

\begin{frame}[t]
\frametitle{Révision et partage : Révision}
\framesubtitle{Vérification de l'orthographe}

\begin{minipage}{0.49\textwidth}
	\begin{itemize}
		\item Menu : \optword{Outils}
		\item Sous menu : \optword{Orthographe}
		\item On peut sélectionner la langue 
		\item On peut ignorer ou choisir une correction des suggestions
	\end{itemize}
\end{minipage}
\begin{minipage}{0.50\textwidth}
	\hgraphpage{verification-orthographe.png}
\end{minipage}

\end{frame}

\begin{frame}[t]
\frametitle{Révision et partage : Révision}
\framesubtitle{Chercher les synonymes}

\begin{minipage}{0.49\textwidth}
	\begin{itemize}
		\item Menu : \optword{Outils}
		\item Sous menu : \optword{Dictionnaire des synonymes}
		\item On peut sélectionner la langue 
		\item On peut choisir une des suggestions à insérer dans le texte
	\end{itemize}
\end{minipage}
\begin{minipage}{0.50\textwidth}
	\hgraphpage{verification-synonymes.png}
\end{minipage}

\end{frame}

%\subsection{Commentaires et suivi}

\begin{frame}[t]
\frametitle{Révision et partage : Révision}
\framesubtitle{Commentaires et suivi}

Pour ajouter des commentaires : 

\begin{itemize}
	\item Sélectionner le texte concerné par le commentaire
	\item Menu : \optword{Insertion}
	\item Sous menu : \optword{Commentaire}
\end{itemize} 

Pour suivre les modifications :

\begin{itemize}
	\item Menu : \optword{\'Edition}
	\item Sous menu : \optword{Suivi des modifications}
	\item Sous sous menu : \optword{Enregistrer} pour enregistrer les modifications
	\item d'autres options pour accepter ou refuser les modifications, comparer avec un autre document, etc.
\end{itemize}


\end{frame}

%\subsection{Comparaison}



\subsection{Partage}

%\begin{frame}
%\frametitle{Révision et partage}
%\framesubtitle{Partage}
%
%
%\end{frame}

\begin{frame}
\frametitle{Révision et partage : Partage}
\framesubtitle{Informations et Protection}

\begin{itemize}
	\item Menu : \optword{Fichier}
	\item Sous menu : \optword{Propriétés}
	\item On peut modifier des informations du fichier, comme le titre, les mots clés, l'auteur, etc. 
	\item On peut protéger le document
\end{itemize}

\hgraphpage[.40\textwidth]{partage-informations.png}
\hgraphpage[.40\textwidth]{partage-protection.png}

\end{frame}

\begin{frame}
\frametitle{Révision et partage : Partage}
\framesubtitle{Impression}

\begin{minipage}{0.30\textwidth}
\begin{itemize}
\item Menu : \optword{Fichier}
\item Sous menu : \optword{Imprimer}
\item Clavier : \optword{Ctrl + P}
\end{itemize}
\end{minipage}
\begin{minipage}{0.69\textwidth}
%	\hgraphpage{revision-suivi.png}
\hgraphpage{partage-imprimer.png}
\end{minipage}

\end{frame}

\begin{frame}
\frametitle{Révision et partage : Partage}
\framesubtitle{Sauvegarde}

\begin{itemize}
\item Menu : \optword{Fichier}
\item Sous menu : \optword{Imprimer}
\item Clavier : \optword{Ctrl + S}
\item \optword{Enregistrer} et \optword{Enregistrer sous} font la même chose si le fichier est nouveau 
\item On peut sauvegarder le document sous forme d'un modèle 
\end{itemize}


\end{frame}

\begin{frame}
\frametitle{Révision et partage : Partage}
\framesubtitle{Exporter et Partager}

Pour exporter le fichier :
\begin{itemize}
	\item Menu : \optword{Fichier}
	\item Sous menu : \optword{Exporter}
	\item Les formats : HTML, PDF, EPUB, Image 
\end{itemize}

Pour envoyer le fichier :
\begin{itemize}
	\item Menu : \optword{Fichier}
	\item Sous menu : \optword{Envoyer}
\end{itemize}

\end{frame}

\begin{frame}
\frametitle{Rédaction d'un document numérique : Microsoft Word}
\framesubtitle{Un peu d'humour}

\begin{center}
	\vgraphpage{writer-humour.jpg}
\end{center}

\end{frame}

\insertbibliography{Bweb03}{*}

\end{document}

