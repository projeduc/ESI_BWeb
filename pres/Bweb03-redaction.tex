% !TEX TS-program = pdflatex
% !TeX program = pdflatex
% !TEX encoding = UTF-8
% !TEX spellcheck = fr

\documentclass[xcolor=table]{beamer}


%\usepackage{fullpage}
%\usepackage[left=2.8cm,right=2.2cm,top=2 cm,bottom=2 cm]{geometry}
\setbeamersize{text margin left=10pt,text margin right=10pt}
\usepackage{amsmath,amssymb} 
\usepackage[T1]{fontenc}
\usepackage[utf8]{inputenc}
\usepackage[english,french]{babel}
\usepackage{txfonts}
\usepackage[]{graphicx}
\usepackage{multirow}
\usepackage{hyperref}
\usepackage{colortbl}
\usepackage{listings}
\usepackage{wrapfig}
\usepackage{multicol}

\hypersetup{
	colorlinks,
	urlcolor = blue
}

%\renewcommand{\baselinestretch}{1.5}

\def\supit#1{\raisebox{0.8ex}{\small\it #1}\hspace{0.05em}}

\AtBeginSection{%
	\begin{frame}
		\sectionpage
	\end{frame}
}

\newcommand{\rottext}[2]{%
	\rotatebox{90}{%
	\begin{minipage}{#1}%
		\raggedleft#2%
	\end{minipage}%
	}%
}

\usepackage{longtable}
\usepackage{tabu}


\institute{ %
École  nationale Supérieure d'Informatique (ESI, ex. INI), Algérie
}
\author[ \textbf{\footnotesize  \insertframenumber/\inserttotalframenumber} \hspace*{\fill} ESI (2019-2020)] %
{ARIES Abdelkrime}
%\titlegraphic{\includegraphics[height=1cm]{../img/esi-logo.png}%\hspace*{4.75cm}~


\date{Année unniversitaire: 2019/2020} %\today

\usetheme{Warsaw} % Antibes Boadilla Warsaw

\beamertemplatenavigationsymbolsempty

%\setbeamertemplate{headline}{}

\definecolor{lightblue}{HTML}{D0D2FF}
\definecolor{lightyellow}{HTML}{FFFFAA}
\definecolor{darkblue}{HTML}{0000BB}
\definecolor{olivegreen}{HTML}{006600}
\definecolor{violet}{HTML}{6600CC}

\newcommand{\keyword}[1]{\textcolor{red}{\bfseries\itshape #1}}
\newcommand{\expword}[1]{\textcolor{olivegreen}{#1}}
\newcommand{\optword}[1]{\textcolor{violet}{\bfseries #1}}

\makeatletter
\newcommand\mysphere{%
	\parbox[t]{10pt}{\raisebox{0.2pt}{\beamer@usesphere{item projected}{bigsphere}}}}
\makeatother

%\let\oldtabular\tabular
%\let\endoldtabular\endtabular
%\renewenvironment{tabular}{\rowcolors{2}{white}{lightblue}\oldtabular\rowcolor{blue}}{\endoldtabular}


\NoAutoSpacing %french autospacing after ":"

\title[BWEB: 03- Rédaction] %
{Bureautique et Web \\Chapitre 03: Rédaction d'un document numérique}  

\begin{document}
	
	\newcolumntype{L}[2]{%
		>{\vbox to #2\bgroup\vfill\flushleft}%
		p{#1}%
		<{\egroup}} 

\begin{frame}[plain]
	\maketitle
\end{frame}

%===================================================================================
\section{Structure d'un rapport}
%===================================================================================

\begin{frame}
\frametitle{Structure d'un rapport}

\begin{itemize}
	\item pages préliminaires
	\begin{itemize}
		\item couverture
		\item remerciements
		\item résumé
		\item table des matières
		\item listes diverses (figures, tableaux, abréviations, symboles, algorithmes)
	\end{itemize}

	\item corps du texte
	\begin{itemize}
		\item introduction
		\item développement (chapitres)
		\item conclusion
	\end{itemize}

	\item pages complémentaires
	\begin{itemize}
		\item annexes
		\item glossaire
		\item index
		\item bibliographie
	\end{itemize}

\end{itemize}

\end{frame}

\subsection{Pages préliminaires}

\begin{frame}
\frametitle{Structure d'un rapport}
\framesubtitle{Pages préliminaires}

\end{frame}


\begin{frame}
\frametitle{Structure d'un rapport: Pages préliminaires}
\framesubtitle{Couverture}

\begin{minipage}{0.52\textwidth}
	\begin{itemize}
		\item Nom de l'université 
		\item Année universitaire
		\item Sujet
		\item Nom de l'étudiant
		\item Nom de l'encadrant 
		\item Nom du laboratoire ou de l'entreprise
		\item Logos (université, entreprise, laboratoire)
	\end{itemize}
\end{minipage}
\begin{minipage}{0.42\textwidth}
	\includegraphics[width=\textwidth]{..//img/Bweb03-redaction/couverture.png}
\end{minipage}

\end{frame}

\begin{frame}
\frametitle{Structure d'un rapport: Pages préliminaires}
\framesubtitle{Remerciements}

\begin{minipage}{0.52\textwidth}
	\begin{itemize}
		\item Promoteur (Organisme d'accueil)
		\item Encadreur (L'université)
		\item Personnes ayant contribué au travail (aide financière, révision, etc.) en mentionnant la nature de l'aide.
		\item Famille pour soutien
		\item ...
		\item Elles doit être:
		\begin{itemize}
			\item \optword{simple}: pas d'exagération
			\item \optword{professionnel}: moins de familiarité
		\end{itemize}
	\end{itemize}
\end{minipage}
\begin{minipage}{0.42\textwidth}
	\includegraphics[width=\textwidth]{..//img/Bweb03-redaction/remerciements.png}
\end{minipage}

\end{frame}

\begin{frame}
\frametitle{Structure d'un rapport: Pages préliminaires}
\framesubtitle{Résumé}

\begin{minipage}{0.52\textwidth}
	\begin{itemize}
		\item Mini-version du rapport
		\item Il doit être:
		\begin{itemize}
			\item \optword{Court}: généralement, ne dépasse pas une page
			\item \optword{Suffisant}: il fournit l'essentiel du rapport
			\item \optword{Clair}: on peut comprendre le travail avec un minimum d'expertise
			\item \optword{Simple}: des paragraphes (sans illustrations, etc.)
		\end{itemize}
		\item Comme le résumé décrit un travail terminé, il est généralement écrit au passé
	\end{itemize}
\end{minipage}
\begin{minipage}{0.42\textwidth}
	\includegraphics[width=\textwidth]{..//img/Bweb03-redaction/resume.png}
\end{minipage}

\end{frame}

\begin{frame}
\frametitle{Structure d'un rapport: Pages préliminaires}
\framesubtitle{Table des matières}

\begin{minipage}{0.52\textwidth}
	\begin{itemize}
		\item Aperçu de la structure du rapport
		\item Liste des divisions avec leurs numéros de pages 
		\item Il ne faut pas dépasser 3 niveaux de titres
		\item Si elle est placée en fin d'ouvrage
		\begin{itemize}
			\item Il est utile de fournir un sommaire
			\item Aperçu sur les parties et les chapitres
			\item Sans numéros de pages
		\end{itemize}
	\end{itemize}
\end{minipage}
\begin{minipage}{0.42\textwidth}
	\includegraphics[width=\textwidth]{..//img/Bweb03-redaction/sommaire.png}
\end{minipage}

\end{frame}


\subsection{Corps du texte}

\begin{frame}
\frametitle{Structure d'un document}
\framesubtitle{Corps du texte}

\end{frame}

\begin{frame}
\frametitle{Structure d'un rapport: Corps du texte}
\framesubtitle{Introduction}

\begin{itemize}
	\item Contexte du travail 
	\begin{itemize}
		\item ...
		\item ...
	\end{itemize}

	\item Problématique 
	\begin{itemize}
		\item ...
		\item ...
	\end{itemize}

	\item Objectifs 
	\begin{itemize}
		\item ....
		\item ....
	\end{itemize}

	\item Plan 
	\begin{itemize}
		\item ....
		\item ....
	\end{itemize}

\end{itemize}

\end{frame}

\begin{frame}
\frametitle{Structure d'un document: Corps du texte}
\framesubtitle{Développement}

\begin{itemize}
	\item \optword{partie}: 
	\begin{itemize}
		\item regroupe des chapitres
		\item exemple: \expword{étude bibliographique}, \expword{contribution}
	\end{itemize}

	\item \optword{chapitre}:
	\begin{itemize}
		\item 
		\item 
	\end{itemize}

	\item \optword{section}: 
	\begin{itemize}
		\item section, sous-section et subdivision (3 niveaux doivent être suffisants)
		\item 
	\end{itemize}

	\item \optword{paragraphe}: 
	\begin{itemize}
		\item regroupe les phrases de la même idée
		\item 
	\end{itemize}

	\item \optword{phrase}:
	\begin{itemize}
		\item 
		\item 
	\end{itemize}
\end{itemize}

\end{frame}


\begin{frame}
\frametitle{Structure d'un document: Corps du texte}
\framesubtitle{Conclusion}

\end{frame}

\subsection{Fin du rapport}

\begin{frame}
\frametitle{Structure d'un document}
\framesubtitle{Annexes}

\end{frame}


\begin{frame}
\frametitle{Structure d'un rapport: Annexes}
\framesubtitle{Bibliographie}

\end{frame}



\begin{frame}
\frametitle{Références}

\tiny

%\begin{itemize}
%	\item Felix Naumann (2011). 
%	"\textit{Chapter 2 – Architecture}". 
%	Search Engines. 
%	Universität Potsdam, Allemagne. Présentation.
%	\url{https://hpi.de/fileadmin/user_upload/fachgebiete/naumann/folien/SS11/Search_Engines/SearchEngines_02_Architecture.pdf}
%	
%	\item Raghu Ramakrishnan (date-indéfinie).
%	"\textit{Chapter 27: Information Retrieval and XML Data Management (Web search engines}". 
%	Database Management Systems. 
%	University of Wisconsin-Madison, Etats unis. Présentation.
%	\url{http://pages.cs.wisc.edu/~dbbook/openAccess/thirdEdition/slides/slides3ed-english/Ch27c_ir3-websearch-95.pdf}
%	
%	\item \url{https://support.google.com/websearch/answer/134479?hl=fr}
	
%\end{itemize}

\end{frame}

\nocite{*}
%\subsection{Bibliography}
%\frame[allowframebreaks]%
%{\frametitle{Références}
%\tiny
%\bibliography{Bweb01}
%\bibliographystyle{plain} 
%}

%\begin{multicols*}{2}[\usebeamertemplate*{frametitle}\frametitle{Références}]%
%	\tiny
%	\bibliography{Bweb03}
%	\bibliographystyle{acm}
%\end{multicols*}


\end{document}

