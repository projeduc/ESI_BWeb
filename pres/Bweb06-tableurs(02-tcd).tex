% !TEX TS-program = pdflatex
% !TeX program = pdflatex
% !TEX encoding = UTF-8
% !TEX spellcheck = fr

\documentclass[xcolor=table, usenames,dvipsnames]{beamer}


%\usepackage{fullpage}
%\usepackage[left=2.8cm,right=2.2cm,top=2 cm,bottom=2 cm]{geometry}
\setbeamersize{text margin left=10pt,text margin right=10pt}
\usepackage{amsmath,amssymb} 
\usepackage[T1]{fontenc}
\usepackage[utf8]{inputenc}
\usepackage[english,french]{babel}
\usepackage{txfonts}
\usepackage[]{graphicx}
\usepackage{multirow}
\usepackage{hyperref}
\usepackage{colortbl}
\usepackage{listings}
\usepackage{wrapfig}
\usepackage{multicol}

\hypersetup{
	colorlinks,
	urlcolor = blue
}

%\renewcommand{\baselinestretch}{1.5}

\def\supit#1{\raisebox{0.8ex}{\small\it #1}\hspace{0.05em}}

\AtBeginSection{%
	\begin{frame}
		\sectionpage
	\end{frame}
}

\newcommand{\rottext}[2]{%
	\rotatebox{90}{%
	\begin{minipage}{#1}%
		\raggedleft#2%
	\end{minipage}%
	}%
}

\usepackage{longtable}
\usepackage{tabu}


\institute{ %
École  nationale Supérieure d'Informatique (ESI, ex. INI), Algérie
}
\author[ \textbf{\footnotesize  \insertframenumber/\inserttotalframenumber} \hspace*{\fill} ESI (2019-2020)] %
{ARIES Abdelkrime}
%\titlegraphic{\includegraphics[height=1cm]{../img/esi-logo.png}%\hspace*{4.75cm}~


\date{Année unniversitaire: 2019/2020} %\today

\usetheme{Warsaw} % Antibes Boadilla Warsaw

\beamertemplatenavigationsymbolsempty

%\setbeamertemplate{headline}{}

\definecolor{lightblue}{HTML}{D0D2FF}
\definecolor{lightyellow}{HTML}{FFFFAA}
\definecolor{darkblue}{HTML}{0000BB}
\definecolor{olivegreen}{HTML}{006600}
\definecolor{violet}{HTML}{6600CC}

\newcommand{\keyword}[1]{\textcolor{red}{\bfseries\itshape #1}}
\newcommand{\expword}[1]{\textcolor{olivegreen}{#1}}
\newcommand{\optword}[1]{\textcolor{violet}{\bfseries #1}}

\makeatletter
\newcommand\mysphere{%
	\parbox[t]{10pt}{\raisebox{0.2pt}{\beamer@usesphere{item projected}{bigsphere}}}}
\makeatother

%\let\oldtabular\tabular
%\let\endoldtabular\endtabular
%\renewenvironment{tabular}{\rowcolors{2}{white}{lightblue}\oldtabular\rowcolor{blue}}{\endoldtabular}


\NoAutoSpacing %french autospacing after ":"
%\usepackage{calligra}

\title[BWEB : 06- Tableaux croisés dynamiques] %
{Bureautique et Web \\Chapitre 06 : Tableurs \\ \slshape\small  Tableaux croisés dynamiques}  

\changegraphpath{../img/tableurs/}

\begin{document}

\begin{frame}
\frametitle{Tableaux croisés dynamiques}
\framesubtitle{Définition}

D'après le dictionnaire \href{http://dictionnaire.sensagent.leparisien.fr/Tableau\%20crois\%C3\%A9\%20dynamique/fr-fr/}{Sensagent Le Parisien} :

\begin{definition}
	\textbf{Un tableau croisé dynamique} (en anglais \textbf{\textit{pivot table}}), en abgrégé « TCD », est une fonctionnalité de certains tableurs qui permet de générer une synthèse d'une table de données brutes. Le « TCD » permet de regrouper des données selon une ou plusieurs de ses propres catégories (colonnes ou champs) et faire les opérations nécessaires entre les montants correspondants (sommes, moyennes, comptages, etc.). 
\end{definition}

\end{frame}
	
\begin{frame}
\frametitle{Tableaux croisés dynamiques}
\framesubtitle{Les avantages}
\begin{itemize}
	\item Utiliser les mêmes données pour faire des statistiques sous différents angles ou perspectives
	\item Résumer rapidement de longues listes de données
	\item Calculer des informations récapitulatives sans écrire de formules ni copier de cellules
	\item Répondre aux questions spécifiques concernant les données. Par exemple, \expword{Quel est le nombre des travailleurs d'une entreprise par fonction ?}
\end{itemize}
\end{frame}

\begin{frame}
\frametitle{Tableaux croisés dynamiques}
\framesubtitle{Plan}

\begin{multicols}{2}
	%	\small
	\tableofcontents
\end{multicols}
\end{frame}

%===================================================================================
\section{Création}
%===================================================================================

\subsection{Les données}

\begin{frame}
\frametitle{TCD : Création}
\framesubtitle{Les données}

\begin{minipage}{0.49\textwidth}
	\begin{itemize}
		\item Les données sont représentées par une table 
		\item Les lignes représentent  les enregistrements. Par exemple, \expword{liste des étudiants}
		\item Les colonnes représentent les champs(caractéristiques). Par exemple, \expword{nom}, \expword{note}, etc.
		\item La première ligne doit contenir les titres des colonnes (champs)
	\end{itemize}
\end{minipage}
%
\begin{minipage}{0.5\textwidth} 
	\hgraphpage{tcd-donnees.png} 
\end{minipage}

\end{frame}

\subsection{Insertion d'un Tableau croisé dynamique}

\begin{frame}
\frametitle{TCD : Création}
\framesubtitle{Insertion d'un tableau croisé dynamique}

\begin{minipage}{0.69\textwidth}
	\begin{itemize}
		\item Onglet : \optword{Insertion}
		\item Groupe : \optword{Tableaux}
		\item Option : \optword{Tableaux}
		\item C'est mieux de créer le TCD dans une nouvelle feuille 
	\end{itemize}
\end{minipage}
%
\begin{minipage}{0.3\textwidth} 
	\hgraphpage{tcd-barre.png}  
\end{minipage}

\begin{flushright}
	\hgraphpage[.8\textwidth]{tcd-creer.png}
\end{flushright}

\end{frame}

\subsection{Choisir les zones}

\begin{frame}
\frametitle{TCD : Création}
\framesubtitle{Choisir les zones}

\begin{minipage}{0.39\textwidth}
	\begin{itemize}
		\item \optword{LIGNES} : afficher les différentes valeurs d'un champ dans les lignes
		\item \optword{VALEURS} : faire des calculs sur les valeurs d'un champ
		\item \optword{COLONNES} : afficher les différentes valeurs d'un champ dans les colonnes
		\item \optword{FILTRES} : filtrer les enregistrements selon les valeurs d'un champ
	\end{itemize}
\end{minipage}
%
\begin{minipage}{0.6\textwidth} 
	\hgraphpage{tcd-zones.png}  
\end{minipage}

\end{frame}

\begin{frame}
\frametitle{TCD : Création}
\framesubtitle{Choisir les zones : lignes}

\begin{minipage}{0.39\textwidth}
	\begin{itemize}
		\item Afficher les différentes valeurs d'un champ dans les lignes
		\item Exemple, \expword{Afficher les niveaux des étudiants}
		\item Glisser le champ \expword{Niveau} vers la zone \optword{LIGNES}
		\item Pour changer le titre des lignes : double clic sur lui
	\end{itemize}
\end{minipage}
%
\begin{minipage}{0.6\textwidth} 
	\hgraphpage{tcd-ligne.png}  
	
	\hgraphpage[.3\textwidth]{tcd-ligne-renom.png} 
\end{minipage}

\end{frame}

\begin{frame}
\frametitle{TCD : Création}
\framesubtitle{Choisir les zones : valeurs}

\begin{minipage}{0.39\textwidth}
	\begin{itemize}
		\item Faire des calculs sur les valeurs d'un champ selon les valeurs d'une ligne
		\item Exemple, \expword{Calculer le nombre des étudiants dans chaque niveaux d'étude}
		\item Glisser le champ \expword{Niveau} vers la zone \optword{VALEURS}
%		\item Cela créera une colonne \expword{Nombre de Niveau} qui calcule le nombre des enregistrements pour chaque valeur du champ \expword{Niveau}
	\end{itemize}
\end{minipage}
%
\begin{minipage}{0.6\textwidth} 
	\hgraphpage{tcd-valeur.png}  
\end{minipage}
\begin{itemize}
	%		\item Faire des calculs sur les valeurs d'un champ selon les valeurs d'une ligne
	%		\item Exemple, \expword{Calculer le nombre des étudiants dans chaque niveaux d'étude}
	%		\item Glisser le champ \expword{Niveau} vers la zone \optword{VALEURS}
	\item Cela créera une colonne \expword{Nombre de Niveau} qui calcule le nombre des enregistrements pour chaque valeur du champ \expword{Niveau}
\end{itemize}

\end{frame}

\begin{frame}
\frametitle{TCD : Création}
\framesubtitle{Choisir les zones : valeurs (2)}

\begin{itemize}
	\item On peut choisir d'autres fonctions à appliquer sur les valeurs
	\item Exemple, \expword{Calculer la moyenne des notes de chaque niveau d'étude}
	\item Cliquer sur la flèche sur le champs situé dans la zone \optword{VALEURS}
	\item Choisir \optword{Paramètres des champs de valeurs ...}
\end{itemize}

\hgraphpage[.38\textwidth]{tcd-fonc.png}
\hgraphpage[.59\textwidth]{tcd-fonc3.png}


\end{frame}

\begin{frame}
\frametitle{TCD : Création}
\framesubtitle{Choisir les zones : colonnes}

\begin{minipage}{0.39\textwidth}
	\begin{itemize}
		\item Afficher les différentes valeurs d'un champ dans les colonnes
		\item L'intérêt est de détaillé les calculs
	\end{itemize}
\end{minipage}
%
\begin{minipage}{0.6\textwidth} 
	\hgraphpage{tcd-colonne.png}  
\end{minipage}
\begin{itemize}
	\item Exemple, \expword{Calculer le nombre des étudiants dans chaque niveaux d'étude avec le détail de Wilaya}
	\item Glisser le champ \expword{Adresse} vers la zone \optword{COLONNES}
	\item Cela créera autant de colonnes que le nombre des Wilayas dans nos données
\end{itemize}

\end{frame}

\begin{frame}
\frametitle{TCD : Création}
\framesubtitle{Choisir les zones : filtres}

\begin{minipage}{0.39\textwidth}
	\begin{itemize}
		\item Filtrer les enregistrements selon les valeurs d'un champ
		\item Exemple, \expword{Calculer le nombre des étudiants dans chaque niveaux d'étude avec la possibilité de sélectionner les Wilayas de ces étudiants}
	\end{itemize}
\end{minipage}
%
\begin{minipage}{0.6\textwidth} 
	\hgraphpage{tcd-filtre.png}  
\end{minipage}

\begin{minipage}{0.79\textwidth}
	\begin{itemize}
		\item Glisser le champ \expword{Adresse} vers la zone \optword{FILTRES}
		\item Cela créera une liste déroulante de haut du TCD
		\item On peut cocher quelques Wilayas 
	\end{itemize}
\end{minipage}
%
\begin{minipage}{0.2\textwidth} 
	\hgraphpage{tcd-filtre2.png}  
\end{minipage}


\end{frame}

%===================================================================================
\section{Outils}
%===================================================================================

\begin{frame}
\frametitle{TCD}
\framesubtitle{Outils}

\begin{itemize}
	\item Lorsqu'on clique sur le TCD, on aura deux onglets supplémentaires 
	\item \optword{Analyse} : des outils pour manipuler le tableau 
\end{itemize}

\hgraphpage{tcd-barre-analyse.png} 

\begin{itemize}
	\item \optword{Création} : des outils de style et de disposition
\end{itemize}

\hgraphpage{tcd-barre-creation.png} 



\end{frame}

\subsection{Affichage des totaux}

\begin{frame}
\frametitle{TCD : Outils}
\framesubtitle{Affichage des totaux}

\begin{minipage}{0.59\textwidth}
	\begin{itemize}
		\item Onglet : \optword{Création}
		\item Groupe : \optword{Disposition}
		\item Option : \optword{Totaux généraux}
		\item On peut afficher ou cacher les totaux des lignes et des colonnes 
	\end{itemize}
\end{minipage}
%
\begin{minipage}{0.4\textwidth} 
	\hgraphpage{tcd-disposition1.png} 
	\hgraphpage[.6\textwidth]{tcd-disposition2.png}
	\hgraphpage{tcd-disposition3.png}
\end{minipage}

\end{frame}


\subsection{Créer des groupes}

\begin{frame}
\frametitle{TCD : Outils}
\framesubtitle{Créer des groupes}

\begin{minipage}{0.49\textwidth}
	\begin{itemize}
		\item Onglet : \optword{Analyse}
		\item Groupe : \optword{Groupe}
		\item Option : \optword{Grouper la sélection}
		\item Sélectionner les lignes (en utilisant \keyword{CTRL} et clic)
		\item Créer le groupe
		\item Cliquer sur le nom du groupe et modifier son nom dans la barre de formule
	\end{itemize}
\end{minipage}
%
\begin{minipage}{0.5\textwidth} 
	\hgraphpage[.45\textwidth]{tcd-groupe1.png} 
	\hgraphpage[.35\textwidth]{tcd-groupe2.png}
	\hgraphpage{tcd-groupe3.png}
\end{minipage}

\end{frame}


\subsection{Graphiques croisés dynamiques}

\begin{frame}
\frametitle{TCD : Outils}
\framesubtitle{Graphiques croisés dynamiques}

\begin{minipage}{0.34\textwidth}
	\begin{itemize}
		\item Onglet : \optword{Analyse}
		\item Groupe : \optword{Outils}
		\item Option : \optword{Graphique croisé dynamique}
		\item Si on change le tableau, le graphique changera
	\end{itemize}
\end{minipage}
%
\begin{minipage}{0.65\textwidth} 
	\hgraphpage{tcd-graphique1.png} 
	
	\hgraphpage{tcd-graphique2.png} 
\end{minipage}

\end{frame}

\begin{frame}
\frametitle{TCD}
\framesubtitle{Un peu d'humour}

\begin{center}
	\vgraphpage{tcd-humour.jpg}
\end{center}

\end{frame}


\insertbibliography{Bweb06}{*}


\end{document}

