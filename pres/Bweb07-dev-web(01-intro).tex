% !TEX TS-program = pdflatex
% !TeX program = pdflatex
% !TEX encoding = UTF-8
% !TEX spellcheck = fr

\documentclass[xcolor=table]{beamer}


%\usepackage{fullpage}
%\usepackage[left=2.8cm,right=2.2cm,top=2 cm,bottom=2 cm]{geometry}
\setbeamersize{text margin left=10pt,text margin right=10pt}
\usepackage{amsmath,amssymb} 
\usepackage[T1]{fontenc}
\usepackage[utf8]{inputenc}
\usepackage[english,french]{babel}
\usepackage{txfonts}
\usepackage[]{graphicx}
\usepackage{multirow}
\usepackage{hyperref}
\usepackage{colortbl}
\usepackage{listings}
\usepackage{wrapfig}
\usepackage{multicol}

\hypersetup{
	colorlinks,
	urlcolor = blue
}

%\renewcommand{\baselinestretch}{1.5}

\def\supit#1{\raisebox{0.8ex}{\small\it #1}\hspace{0.05em}}

\AtBeginSection{%
	\begin{frame}
		\sectionpage
	\end{frame}
}

\newcommand{\rottext}[2]{%
	\rotatebox{90}{%
	\begin{minipage}{#1}%
		\raggedleft#2%
	\end{minipage}%
	}%
}

\usepackage{longtable}
\usepackage{tabu}


\institute{ %
École  nationale Supérieure d'Informatique (ESI, ex. INI), Algérie
}
\author[ \textbf{\footnotesize  \insertframenumber/\inserttotalframenumber} \hspace*{\fill} ESI (2019-2020)] %
{ARIES Abdelkrime}
%\titlegraphic{\includegraphics[height=1cm]{../img/esi-logo.png}%\hspace*{4.75cm}~


\date{Année unniversitaire: 2019/2020} %\today

\usetheme{Warsaw} % Antibes Boadilla Warsaw

\beamertemplatenavigationsymbolsempty

%\setbeamertemplate{headline}{}

\definecolor{lightblue}{HTML}{D0D2FF}
\definecolor{lightyellow}{HTML}{FFFFAA}
\definecolor{darkblue}{HTML}{0000BB}
\definecolor{olivegreen}{HTML}{006600}
\definecolor{violet}{HTML}{6600CC}

\newcommand{\keyword}[1]{\textcolor{red}{\bfseries\itshape #1}}
\newcommand{\expword}[1]{\textcolor{olivegreen}{#1}}
\newcommand{\optword}[1]{\textcolor{violet}{\bfseries #1}}

\makeatletter
\newcommand\mysphere{%
	\parbox[t]{10pt}{\raisebox{0.2pt}{\beamer@usesphere{item projected}{bigsphere}}}}
\makeatother

%\let\oldtabular\tabular
%\let\endoldtabular\endtabular
%\renewenvironment{tabular}{\rowcolors{2}{white}{lightblue}\oldtabular\rowcolor{blue}}{\endoldtabular}


\NoAutoSpacing %french autospacing after ":"

\title[BWEB : 08- Web (HTML)] %
{Bureautique et Web \\Chapitre 07 : Développement Web\\ \slshape\small  Introduction}  

\changegraphpath{../img/dev-web/}

\begin{document}
	
\section{Généralités}

%\begin{frame}
%\frametitle{Développement Web}
%\framesubtitle{Généralités}
%
%
%\end{frame}

\subsection{Site web}

\begin{frame}
\frametitle{Développement Web : Généralités}
\framesubtitle{Site web : Définitions}

\begin{definition}[Site web]
	Un ensemble de pages web et de ressources reliées par des hyperliens.
	Un site web est défini et accessible par une adresse web.
	
	Un hyperlien est un lien associé à un élément d'un document hypertexte, qui pointe vers un autre élément textuel ou multimédia [Larousse]
\end{definition}

\begin{definition}[Page web]
	Elle est constitué d'un document texte écrit en Hypertext Markup Language (\keyword{HTML}) pour la structure de base, d'images numériques, de feuilles de style en cascade (\keyword{CSS}) pour la mise en page, et de JavaScript pour des fonctionnalités plus avancées. 
\end{definition}

\end{frame}

\begin{frame}
\frametitle{Développement Web : Généralités}
\framesubtitle{Site web : Fonctionnement}

\begin{enumerate}
	\item le navigateur demande au \keyword{DNS} l'adresse réelle du serveur contenant le site web.
	\item le navigateur envoie une requête \keyword{HTTP} au serveur pour lui demander d'envoyer une copie du site web au client. 
	\item si le serveur accepte la requête émise par le client, le serveur envoie un message « 200 OK » au client qui signifie : « Pas de problème, tu peux consulter ce site web, le voici ». 
	\item Ensuite le serveur commence à envoyer les fichiers du site web au navigateur sous forme d'une série de petits morceaux nommés "paquet".
	\item le navigateur assemble les différents morceaux pour recomposer le site web en entier puis l'affiche sur l'écran.
\end{enumerate}

\textbf{Remarque} : Veuillez consulter le chapitre \optword{Environnement de travail}

\end{frame}

\subsection{Environnement serveur-client}

\begin{frame}
\frametitle{Développement Web : Généralités}
\framesubtitle{Environnement serveur-client : Statique ou dynamique}

\begin{definition}[Page web statique]
	Souvent, c'est un document \keyword{HTML}, qui peut contenir ou faire appel au \keyword{CSS} et \keyword{Javascript}.
	Son contenu ne varie pas ; le serveur envoi cette page telle qu'elle est, sans modifications.
\end{definition}

\begin{definition}[Page web dynamique]
	C'est une page web générée par un serveur à la demande. 
	Son contenu peut donc varier selon la requête du client et selon les informations stockées dans le serveur. 
\end{definition}

\end{frame}

\begin{frame}
\frametitle{Développement Web : Généralités}
\framesubtitle{Environnement serveur-client : Développement}

\begin{itemize}
	\item Dans le côté client 
	\begin{itemize}
		\item Les pages web sont exécutées sur un navigateur web 
		\item Les navigateurs peuvent comprendre \keyword{HTML}, \keyword{HTML} et \keyword{Javascript}
		\item Un développeur web spécialisé dans le côté client s'appelle \keyword{développeur front-end}
	\end{itemize}
	\item Dans le côté serveur
	\begin{itemize}
		\item Les pages web sont générées sur le serveur en se basant sur les requêtes du client et/ou les ressources stockées sur le serveur.
		\item Les applications qui génèrent ces pages peuvent être implémentées en utilisant une variété de langages de programmation.
		\item Un développeur web spécialisé dans le côté serveur s'appelle \keyword{développeur back-end}
	\end{itemize}
	\item Un développeur web spécialisé dans les deux côtés s'appelle \keyword{développeur full-stack}
\end{itemize}


\end{frame}


\subsection{Développer un site web}

\begin{frame}
\frametitle{Développement Web : Généralités}
\framesubtitle{Développer un site web : Conception et préparation}

\begin{itemize}
	\item Définir les principaux objectifs et public cible. 
	Par exemple, \expword{Un portfolio dont l'objectif est de présenter l'expérience d'une personne}
	\item Planifier le site : définir les différentes pages. 
	Par exemple, dans un site d'un hôtel, on peut penser aux pages : \expword{Présentation},  \expword{Les services}, \expword{Réservation},  \expword{Contacter nous}, etc.
	\item Définir la structure des pages et les liens entre elles. 
	Par exemple, \expword{toutes les pages doivent avoir une entête avec le logo de l'hôtel, un menu à gauche et le contenu à droite}
	\item Préparer le contenu et chercher les images. 
	Il faut penser à utiliser des images de petites tailles. 
	Par exemple, \expword{une page web avec plusieurs images de 4MO sera lente lors du chargement sur un réseaux internet faible}
\end{itemize}

\end{frame}

\begin{frame}
\frametitle{Développement Web : Généralités}
\framesubtitle{Développer un site : Réalisation}

\begin{itemize}
	\item Choisir les outils à utiliser (langages de programmation, frameworks, bibliothèques, éditeurs web, etc.) 
	\item Implémenter la solution 
	\item Si les pages sont statiques, tester directement sur un navigateur
	\item S'il y a des traitements sur le côté serveur, installer un serveur \keyword{HTTP} sur votre machine pour les tester
\end{itemize}

\end{frame}

\begin{frame}
\frametitle{Développement Web : Généralités}
\framesubtitle{Développer un site : Hébergement}

\begin{itemize}
	\item Penser à un nom du domaine non pris déjà. 
	Par exemple, \expword{monsite.com} 
	\item Chercher et acheter ce nom du domaine. Par exemple, en utilisant \url{https://www.name.com/domain/search} 
	\item Choisir un hébergeur selon vos besoins (les langages de programmation, les outils proposés, les prix, etc.)
	\item Vous pouvez envoyer vos pages web via une interface graphique ou d'autres moyens comme \keyword{FTP} (en utilisant \expword{filezilla} par exemple)
	\item Il existe des hébergeurs gratuits où vous pouvez tester votre site. 
	Par exemple, \url{https://www.000webhost.com/}, \url{https://sites.google.com/}, \url{https://www.freehostia.com/}, \url{https://wordpress.com/}, \url{https://www.pythonanywhere.com/}
\end{itemize}

\end{frame}

\section{Front-End}

\begin{frame}
\frametitle{Développement Web}
\framesubtitle{Front-End}

\begin{itemize}
	\item Cross-browser : le site doit supporter le maximum des navigateurs web
	\item Site web réactif (\textit{responsive web design}) : affichage confortable du contenu selon la taille de l'écran
	\item Accessibilité du web : accès aux contenus par les personnes handicapées (déficients visuels, sourds, malentendants, etc.) 
	\item Interface utilisateur
	\item La performance du site web : la rapidité de téléchargement et d'affichage du contenu
\end{itemize}

\end{frame}

\begin{frame}
\frametitle{Développement Web}
\framesubtitle{Front-End}

\hgraphpage{frontend.pdf}

\end{frame}

\subsection{Les bases}

\begin{frame}[fragile]
\frametitle{Développement Web : Front-End}
\framesubtitle{Les bases : HTML}

\begin{tabular}{p{.05\textwidth}p{.85\textwidth}}
	\hgraphpage[0.05\textwidth, valign=t]{HTML5-logo.png} & 
	\mysphere\  \keyword{HTML} signifie « HyperText Markup Language »
	
	\mysphere\  traduit par « Langage de balises pour l'hypertexte »
\end{tabular}
\begin{minipage}{0.49\textwidth}
\begin{itemize}
%	\item \keyword{HTML} signifie « HyperText Markup Language »
%	\item traduit par « langage de balises pour l'hypertexte »
	\item utilisé afin de créer et de représenter le contenu d'une page web et sa structure
	\item sa dernière version est HTML5 (2008)
\end{itemize}
\end{minipage}
%
\begin{minipage}{0.5\textwidth}
\begin{exampleblock}{Exemple d'un code HTML}
\scriptsize\bfseries
\lstset{escapeinside=**}\novocalize
\begin{lstlisting}
<!DOCTYPE HTML>
<html>
  <head>
    <title>Titre de la *fenêtre*</title>
  </head>
  <body>
    <h1>Exemple d'un titre</h1>
    <p>Exemple d'un paragraphe</p>
    <p>Un *deuxième* paragraphe</p>
  </body>
</html>
\end{lstlisting}
\end{exampleblock}
\end{minipage}

\end{frame}

\begin{frame}[fragile]
\frametitle{Développement Web : Front-End}
\framesubtitle{Les bases : CSS}

\begin{tabular}{p{.05\textwidth}p{.85\textwidth}}
	\hgraphpage[0.05\textwidth, valign=t]{CSS3-logo.png} & 
	\mysphere\  \keyword{CSS} signifie « Cascading Style Sheets »
	
	\mysphere\  traduit par « Feuilles de style en cascade »
\end{tabular}
\begin{minipage}{0.59\textwidth}
	\begin{itemize}
		\item utilisé afin de décrire la présentation d'un document écrit en HTML
		\item \keyword{CSS} décrit la façon dont les éléments doivent être affichés à l'écran, sur du papier, en vocalisation, ou sur d'autres supports.
		\item sa dernière évolution est CSS3
	\end{itemize}
\end{minipage}
%
\begin{minipage}{0.4\textwidth}
\begin{exampleblock}{Exemple d'un code CSS}
\scriptsize\bfseries
\begin{lstlisting}
h1 {
  color: red; 
  font-size: 20pt;
  font-style: italic;
}
\end{lstlisting}
\end{exampleblock}
\end{minipage}

\end{frame}

\begin{frame}[fragile]
\frametitle{Développement Web : Front-End}
\framesubtitle{Les bases : Javascript}

\begin{tabular}{p{.07\textwidth}p{.85\textwidth}}
	\hgraphpage[0.07\textwidth, valign=t]{Javascript-badge.png} & 
	\mysphere\  \keyword{Javascript} est un langage de programmation
	
	\mysphere\  principalement employé dans les pages web
\end{tabular}
\begin{minipage}{0.49\textwidth}
	\begin{itemize}
		\item utilisé afin de permettre des pages web interactives
		\item \keyword{Javascript} ajoute de l'interactivité aux pages web. 
		Par exemple, \expword{Si on clique sur un bouton, une fonction peut faire la somme de deux nombres et écrire le résultat dans le premier élément de type "p"}
%		\item sa dernière version est ES6 (ECMAScript 2015)
%		\item la version en cours de rédaction, actuellement c'est ECMAScript 2020
	\end{itemize}
\end{minipage}
%
\begin{minipage}{0.5\textwidth}
\begin{exampleblock}{Exemple d'un code Javascript}
\tiny\bfseries
\begin{lstlisting}
function somme (a, b) {
  let s = a + b;
  let p = document.getElementsByTagName("p")[0];
  p.innerHTML = "la somme est " + s;
}
\end{lstlisting}
\end{exampleblock}
\end{minipage}
	\begin{itemize}
%	\item utilisé afin de permettre des pages web interactives
%	\item \keyword{Javascript} ajoute de l'interactivité aux pages web. 
%	Par exemple, \expword{Si on clique sur un bouton, une fonction peut faire la somme de deux nombres et écrire le résultat dans le premier élément de type "p"}
	\item sa dernière version est ES6 (ECMAScript 2015)
	\item la version en cours de rédaction, actuellement c'est ECMAScript 2020
\end{itemize}

\end{frame}

\subsection{Les frameworks front-end}

\begin{frame}
\frametitle{Développement Web : Front-End}
\framesubtitle{Les frameworks front-end}

\end{frame}


\subsection{Les préprocesseurs}

\begin{frame}
\frametitle{Développement Web : Front-End}
\framesubtitle{Les préprocesseurs}

\end{frame}


\subsection{Générateurs des sites statiques}

\begin{frame}
\frametitle{Développement Web : Front-End}
\framesubtitle{Générateurs des sites statiques}

\end{frame}


\section{Back-End}

\begin{frame}
\frametitle{Développement Web}
\framesubtitle{Back-End}

\begin{itemize}
	\item Gestion des bases de données : stocker et de retrouver les informations dans le serveur
	\item Extensibilité (scalabilité) : le site (serveur) doit pouvoir maintenir ses fonctionnalités et ses performances en cas des demandes d'un grand nombre de clients
	\item Haute disponibilité du site
	\item Sécurité : les données personnelles des clients doivent être sécurisées
	\item Sauvegarde (\textit{backup}) : les données doivent être dupliquées pour éviter leur perte
\end{itemize}

\end{frame}

\begin{frame}
\frametitle{Développement Web}
\framesubtitle{Back-End}

\hgraphpage{backend.pdf}

\end{frame}

\subsection{Les langages de programmation}

\begin{frame}
\frametitle{Développement Web : Back-End}
\framesubtitle{Les langages de programmation}

\end{frame}

\subsection{Les frameworks back-end}

\begin{frame}
\frametitle{Développement Web : Front-End}
\framesubtitle{Les frameworks back-end}

\end{frame}

\subsection{Serveurs et plateformes}

\begin{frame}
\frametitle{Développement Web : Front-End}
\framesubtitle{Serveurs et plateformes}

\end{frame}

\subsection{CMS}

\begin{frame}
\frametitle{Développement Web : Front-End}
\framesubtitle{CMS}

\end{frame}

\section{Outils}

\subsection{Éditeurs Web}

\begin{frame}
\frametitle{Développement Web (Introduction) : Outils}
\framesubtitle{Éditeurs Web : Éditeurs de texte}

\def\arraystretch{0}

\begin{tabular}{p{.2\textwidth}cp{.6\textwidth}}%p{.3\textwidth}
	
	\hline
	
	\vgraphpage[.8cm]{aptana-logo.png} &
	& 
	\url{http://www.aptana.com/}\\
	
	\hline
	
	\vgraphpage[.8cm]{atom-logo.png} &
	& 
	\url{https://atom.io/}\\
	
	\hline
	
	\vgraphpage[.8cm]{bluefish-logo.png} &
	& 
	\url{https://sourceforge.net/projects/bluefish/}\\
	
	\hline
	
	\vgraphpage[.8cm]{brackets-logo.png} &
	& 
	\url{http://brackets.io/}\\
	
	\hline
	
	\vgraphpage[.8cm]{visual-studio-code-logo.png} &
	& 
	\url{https://code.visualstudio.com/}\\
	
	\hline
	
	
\end{tabular}

\end{frame}

\begin{frame}
\frametitle{Développement Web : Outils}
\frametitle{Éditeurs Web : Éditeurs WYSIWYG}

\def\arraystretch{0}

\begin{tabular}{p{.2\textwidth}cp{.7\textwidth}}%p{.3\textwidth}
	
	\hline
	
	\vgraphpage[.8cm]{bluegriffon-logo.png} &
	& 
	\url{http://bluegriffon.org/}\\
	
	\hline
	
	\vgraphpage[.8cm]{google-web-designer-logo.png} &
	& 
	\url{https://webdesigner.withgoogle.com/} \\
	
	\hline
	
	\vgraphpage[.8cm]{open-element-logo.png} &
	& 
	\url{https://www.openelement.com/}\\
	
	\hline
	
	\vgraphpage[.8cm]{dreamweaver-logo.png} &
	& 
	\url{https://www.adobe.com/products/dreamweaver.html}\\
	
	\hline
	
\end{tabular}

\end{frame}

\begin{frame}
\frametitle{Développement Web}
\framesubtitle{Un peu d'humour}

\begin{center}
	\vgraphpage{webdev-humour.jpg}
	\vgraphpage{webdev-humour2.jpg}
\end{center}

\end{frame}

\insertbibliography{Bweb07}{*}


\end{document}

