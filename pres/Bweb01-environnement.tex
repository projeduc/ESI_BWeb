% !TEX TS-program = pdflatex
% !TeX program = pdflatex
% !TEX encoding = UTF-8
% !TEX spellcheck = fr

\documentclass[xcolor=table]{beamer}


%\usepackage{fullpage}
%\usepackage[left=2.8cm,right=2.2cm,top=2 cm,bottom=2 cm]{geometry}
\setbeamersize{text margin left=10pt,text margin right=10pt}
\usepackage{amsmath,amssymb} 
\usepackage[T1]{fontenc}
\usepackage[utf8]{inputenc}
\usepackage[english,french]{babel}
\usepackage{txfonts}
\usepackage[]{graphicx}
\usepackage{multirow}
\usepackage{hyperref}
\usepackage{colortbl}
\usepackage{listings}
\usepackage{wrapfig}
\usepackage{multicol}

\hypersetup{
	colorlinks,
	urlcolor = blue
}

%\renewcommand{\baselinestretch}{1.5}

\def\supit#1{\raisebox{0.8ex}{\small\it #1}\hspace{0.05em}}

\AtBeginSection{%
	\begin{frame}
		\sectionpage
	\end{frame}
}

\newcommand{\rottext}[2]{%
	\rotatebox{90}{%
	\begin{minipage}{#1}%
		\raggedleft#2%
	\end{minipage}%
	}%
}

\usepackage{longtable}
\usepackage{tabu}


\institute{ %
École  nationale Supérieure d'Informatique (ESI, ex. INI), Algérie
}
\author[ \textbf{\footnotesize  \insertframenumber/\inserttotalframenumber} \hspace*{\fill} ESI (2019-2020)] %
{ARIES Abdelkrime}
%\titlegraphic{\includegraphics[height=1cm]{../img/esi-logo.png}%\hspace*{4.75cm}~


\date{Année unniversitaire: 2019/2020} %\today

\usetheme{Warsaw} % Antibes Boadilla Warsaw

\beamertemplatenavigationsymbolsempty

%\setbeamertemplate{headline}{}

\definecolor{lightblue}{HTML}{D0D2FF}
\definecolor{lightyellow}{HTML}{FFFFAA}
\definecolor{darkblue}{HTML}{0000BB}
\definecolor{olivegreen}{HTML}{006600}
\definecolor{violet}{HTML}{6600CC}

\newcommand{\keyword}[1]{\textcolor{red}{\bfseries\itshape #1}}
\newcommand{\expword}[1]{\textcolor{olivegreen}{#1}}
\newcommand{\optword}[1]{\textcolor{violet}{\bfseries #1}}

\makeatletter
\newcommand\mysphere{%
	\parbox[t]{10pt}{\raisebox{0.2pt}{\beamer@usesphere{item projected}{bigsphere}}}}
\makeatother

%\let\oldtabular\tabular
%\let\endoldtabular\endtabular
%\renewenvironment{tabular}{\rowcolors{2}{white}{lightblue}\oldtabular\rowcolor{blue}}{\endoldtabular}


\NoAutoSpacing %french autospacing after ":"

\title[BWEB: 01- Environnement] %
{Bureautique et Web \\Chapitre 01: Environnement de travail} 

\changegraphpath{../img/Bweb01-environnement/}

\begin{document}

\section{Introduction aux environnements de travail}

\begin{frame}
\frametitle{Introduction aux environnements de travail}

Un environnement de travail doit avoir:
%\begin{itemize}
%	\item un poste de travail: l'ordinateur et ses périphériques
%	\item des applications
%	\item espaces de stockage
%\end{itemize}

\begin{center}
	\begin{tabular}{p{.3\textwidth}p{.3\textwidth}p{.3\textwidth}}
	un poste de travail & des applications & espace de stockage \\
	\vgraphpage[.25\textheight]{poste-travail.png} &
	\vgraphpage[.25\textheight]{applications.jpg} &
	\vgraphpage[.25\textheight]{stockage.jpg} \\
	
	&&\\
	
	matériel (Processeur, RAM, périphériques, etc.) et logiciel (système d'exploitation et pilotes) &
	logiciels applicatifs & 
	disque dur, stockage en ligne, etc. \\
	\end{tabular}
\end{center}

\end{frame}

\begin{frame}
\frametitle{Introduction aux environnements de travail}
\framesubtitle{Système d'exploitation}

\begin{itemize}
\item Le système d'exploitation est une couche logiciel entre le matériel et les applications installées sur la machine.

\item Il sert à cacher la complexité du matériel à l'utilisateur.

\item Parmi ses fonctions: 
\begin{itemize}
	\item gérer les périphériques (écran, imprimante, clavier, souris, etc.) via des pilotes
	\item exécuter les applications et définir comment chacune doit accéder aux ressources (mémoire, processeur, périphériques)
	\item fournir une interface utilisateur (ligne de commande ou interface graphique)
\end{itemize}
\end{itemize}

\end{frame}

\begin{frame}
\frametitle{Introduction aux environnements de travail}
\framesubtitle{Types d'environnements de travail}

Selon l'emplacement des données et des applications, un environnement de travail peut être: 

\rowcolors{2}{lightblue}{lightyellow}

\begin{tabular}{p{.08\textwidth}p{.41\textwidth}p{.41\textwidth}}
%	\hline\hline
	\rowcolor{darkblue}
	& \textcolor{white}{Local} & \textcolor{white}{Distant} \\
%	\hline\hline
	
	Gain &
	+ contrôle et accès personnel 
	
	+ accès sans internet

	 & 
	+ maintenance et mise à jours faciles des applications et ressources
	
	+ partage et travaille en équipe
	
	+ accessible partout (internet et navigateur)
	 \\
	
%	\hline
	Perte &
	- installation et configuration des applications
	
	- perte des données
	&
	- nécessité d'accès internet
	
	- données personnelles chez une autre entité 
	\\
%	\hline\hline
\end{tabular}

\end{frame}

\section{Environnement de travail local}

\begin{frame}
\frametitle{Environnement de travail local}

\begin{itemize}
	\item Applications et données à la porté de l'utilisateur 
	\item Gérées par un système d'exploitation (SE) de bureau ou mobile 
	\item En BWeb, on s'intéresse par l'aspect utilisation des SE. 
	\item L'aspect technique (composants) sera entamé dans d'autres modules. 
	\item Les points suivants serons présentés:
	\begin{itemize}
		\item L'interface graphique et facilité d'utilisation 
		\item Gestion des utilisateurs
		\item Gestion des fichiers
		\item Gestion des applications
	\end{itemize}
	
\end{itemize}

\end{frame}


\begin{frame}
\frametitle{Environnement de travail local}
\framesubtitle{Système d'exploitation (Bureau)}

\begin{tabular}{ll}
	\vgraphpage[.25\textheight]{windows.png} 
	Windows (Microsoft) &
	\vgraphpage[.25\textheight]{mac-finder.png} 
	Mac OS (Apple) \\
	
	\vgraphpage[.25\textheight]{tux.png} 
	Linux &
	\vgraphpage[.25\textheight]{chrome.png} 
	Chrome OS (Google) \\
	
	\vgraphpage[.25\textheight]{freebsd.png} 
	Free BSD &
	\\
	
\end{tabular}

\end{frame}


\begin{frame}
\frametitle{Environnement de travail local}
\framesubtitle{Système d'exploitation (Mobile)}

\begin{center}
	\footnotesize\bfseries
	\begin{tabular}{llllll}
	\rottext{.35\textheight}{Android 10 (Google)} &
	\vgraphpage[.4\textheight]{AndroidQ.png} & 
	
	\rottext{.35\textheight}{iOS 13 (Apple)} &
	\vgraphpage[.4\textheight]{iOS13.png} &
	
	\rottext{.35\textheight}{Windows 10 Mobile (Microsoft)} &
	\vgraphpage[.4\textheight]{Windows10Mobile.png} \\
	
	\rottext{.35\textheight}{Tizen 2.2 (Tizen Association)} &
	\vgraphpage[.4\textheight]{Tizen.png} & 
	
	\rottext{.35\textheight}{Ubuntu Touch (Canonical)} &
	\vgraphpage[.4\textheight]{UbuntuTouch.png} &
	
	\rottext{.35\textheight}{Plasma Phone (KDE et Blue systems)} &
	\vgraphpage[.4\textheight]{plasmaphone.png} \\
\end{tabular}
\end{center}

\end{frame}

\begin{frame}
\frametitle{Environnement de travail local}
\framesubtitle{Système d'exploitation (Windows vs. Linux)}

%D'après \url{https://gs.statcounter.com/os-market-share/desktop/worldwide/#monthly-201901-201911}
% https://www.w3schools.com/browsers/browsers_os.asp
\rowcolors{2}{lightblue}{lightyellow}
	
\begin{tabular}{p{.20\textwidth}p{.20\textwidth}p{.5\textwidth}}
%	\hline\hline
	\rowcolor{darkblue}
	& \textcolor{white}{Windows} & \textcolor{white}{Linux} \\
%	\hline\hline
	
	Popularité & > 70\% & < 5\% \\
%	\hline
	
	Type usagers & tous & avoir un peu d'expertise \\
%	\hline
	
	Cout & payant & gratuit (la plupart de distributions) \\
%	\hline
	
	GUI & standard & plusieurs choix \\
%	\hline
	
	Applications & beaucoup & peu\\
%	\hline
	
	Support & bon (Microsoft) & selon la distribution (communauté) \\
%	\hline
	
	Attaques & régulières & rares\\
%	\hline
	
	Installation & autonome & répertoires avec outil de paquetage \\
%	\hline
	
	Personnalisation & limitée & personnalisable \\
%	\hline\hline
\end{tabular}

\end{frame}


\begin{frame}
\frametitle{Environnement de travail local}
\framesubtitle{Système d'exploitation: un peu d'humeur}

\begin{center}
	\vgraphpage{os-humour.jpg}
\end{center}

\end{frame}

\subsection{Windows}

\begin{frame}
\frametitle{Environnement de travail local}
\framesubtitle{Windows}

On utilise Windows pour ces raisons:
\begin{itemize}
	\item Installation facile des applications
	\item Disponibilité des applications 
	\item Support: beaucoup d'utilisateurs implique plusieurs gens peuvent aider lors d'un problème
\end{itemize}

\end{frame}

\begin{frame}
\frametitle{Environnement de travail local: Windows}
\framesubtitle{Environnements de bureau (Luna: XP)}

\begin{center}
	\vgraphpage{xp.png}
\end{center}

\end{frame}

\begin{frame}
\frametitle{Environnement de travail local: Windows}
\framesubtitle{Environnements de bureau (Aero: Windows 7)}

\begin{center}
	\vgraphpage{win7.png}
\end{center}

\end{frame}

\begin{frame}
\frametitle{Environnement de travail local: Windows}
\framesubtitle{Environnements de bureau (Fluent: Windows 10)}

\begin{center}
	\vgraphpage{win10.png}
\end{center}

\end{frame}

\begin{frame}
\frametitle{Environnement de travail local: Windows}
\framesubtitle{Configuration (Windows 7)}

\begin{center}
	\vgraphpage{win7-config.png}
\end{center}

\end{frame}

\begin{frame}
\frametitle{Environnement de travail local: Windows}
\framesubtitle{Gestion de fichiers (Windows 7)}

\begin{center}
	\vgraphpage{win7-files.png}
\end{center}

\end{frame}

\begin{frame}
\frametitle{Environnement de travail local: Windows}
\framesubtitle{Gestion des utilisateurs (Windows 7)}

\begin{center}
	\vgraphpage{win7-users.png}
\end{center}

\end{frame}

\subsection{Linux}

\begin{frame}
\frametitle{Environnement de travail local}
\framesubtitle{Linux}

On utilise Linux pour ces raisons:
\begin{itemize}
	\item Personnalisation: on peut personnaliser Linux selon nous besoins
	\item Cout: la plupart des distributions sont gratuites
	\item Sécurité: la plupart des logiciels sont open sources. Donc, on peut savoir ce qu'un logiciel fait.
\end{itemize}

\end{frame}

\begin{frame}
\frametitle{Environnement de travail local: Linux}
\framesubtitle{Distributions}

Quelques distributions:

\begin{tabular}{llll}
	\hgraphpage[.2\textwidth]{mint-logo.png} & 
	\hgraphpage[.2\textwidth]{ubuntu-logo.png} &
	\hgraphpage[.2\textwidth]{opensuse-logo.png} &
	\hgraphpage[.2\textwidth]{arch-logo.png} \\
	
	\hgraphpage[.2\textwidth]{gentoo-logo.png} & 
	\hgraphpage[.2\textwidth]{slackware-logo.png} &
	\hgraphpage[.2\textwidth]{debian-logo.png} & 
	\hgraphpage[.2\textwidth]{fedora-logo.png} \\
	
\end{tabular}

\end{frame}

\begin{frame}
\frametitle{Environnement de travail local: Linux}
\framesubtitle{Environnements de bureau: KDE Plasma}

\begin{center}
	\vgraphpage{kde.png}
\end{center}

\end{frame}

\begin{frame}
\frametitle{Environnement de travail local: Linux}
\framesubtitle{Environnements de bureau: GNOME3}

\begin{center}
	\vgraphpage{gnome.png}
\end{center}

\end{frame}

\begin{frame}
\frametitle{Environnement de travail local: Linux}
\framesubtitle{Environnements de bureau: LXDE}

\begin{center}
	\vgraphpage{lxde.png}
\end{center}

\end{frame}

\begin{frame}
\frametitle{Environnement de travail local: Linux}
\framesubtitle{Environnements de bureau: MATE}

\begin{center}
	\vgraphpage{mate.png}
\end{center}

\end{frame}


\begin{frame}
\frametitle{Environnement de travail local: Linux}
\framesubtitle{Environnements de bureau: Cinnamon}

\begin{center}
	\vgraphpage{cinnamon.png}
\end{center}

\end{frame}

\begin{frame}
\frametitle{Environnement de travail local: Linux}
\framesubtitle{Environnements de bureau: XFCE}

\begin{center}
	\vgraphpage{xfce.png}
\end{center}

\end{frame}

\begin{frame}
\frametitle{Environnement de travail local: Linux}
\framesubtitle{Configuration (KDE)}

\begin{center}
	\vgraphpage{kde-config.png}
\end{center}

\end{frame}

\begin{frame}
\frametitle{Environnement de travail local: Linux}
\framesubtitle{Gestion de fichiers (KDE)}

\begin{center}
	\vgraphpage{kde-files.png}
\end{center}

\end{frame}

\begin{frame}
\frametitle{Environnement de travail local: Linux}
\framesubtitle{Gestion des utilisateurs (KDE)}

\begin{center}
	\vgraphpage{kde-users.png}
\end{center}

\end{frame}

\subsection{Machine virtuelle}

\begin{frame}
\frametitle{Environnement de travail local}
\framesubtitle{Machine virtuelle}

On utilise une machine virtuelle pour ces raisons:
\begin{itemize}
	\item Expérimenter avec d'autre systèmes d'exploitation (SE)
	\item Utiliser des logiciels non existants sur notre SE principal
	\item Sécurité: si une machine virtuelle est compromise, on la supprime et remplace
	\item Réutilisation: on peut installer plusieurs applications sur une machine virtuelle pour la diffuser sur plusieurs utilisateurs 
\end{itemize}

\end{frame}

\begin{frame}
\frametitle{Environnement de travail local: Machine virtuelle}
\framesubtitle{VirtualBox}

\begin{center}
	\vgraphpage{vbox.png}
\end{center}

\end{frame}

\begin{frame}
\frametitle{Environnement de travail local: Machine virtuelle}
\framesubtitle{VirtualBox: créer une nouvelle machine}

\begin{itemize}
	\item Dans la Bios, activer les options de la virtualisation (VT et AMD-V)
	\item Créer une nouvelle machine: donner un nom, choisir l'emplacement, le type et la version. 
	\item Spécifier la taille de RAM pour la machine. Il faut laisser de la RAM pour votre système principale 
	\item Spécifier la taille du disque dur virtuel et son nom. Il faut avoir de l'espace sur votre disque
	\item Vous pouvez spécifier les périphériques dans les paramètres de la machine 
	\item Lors de la première ouverture, on vous demande d'insérer un CD virtuel. Dans ce cas, c'est un fichier (.iso). 
\end{itemize}

\end{frame}


\section{Environnement de travail distant}

\begin{frame}
\frametitle{Environnement de travail distant}

\begin{itemize}
	\item Les données et/ou les applications sont accessibles via un réseau
	\item Le service d'hébergement est fourni par des serveurs
	\item On doit avoir une machine (ordinateur ou mobile) accéder ces services
	\item En général, on utilise un navigateur web ou des applications dédiées pour ce faire
\end{itemize}

D'après Larousse, un serveur informatique est:
\begin{definition}
	Ensemble matériel et logiciel, branché sur un réseau télématique et mettant à la disposition des utilisateurs de ce réseau des banques de données ou de programmes spécialisées ; organisme qui assure ce service.
\end{definition}

\end{frame}

\subsection{Accès à l'environnement de travail distant}

\begin{frame}
\frametitle{Environnement de travail distant}
\framesubtitle{Accès à l'environnement de travail distant}

Selon l'emplacement du serveur, le réseau peut être:
\begin{itemize}
	\item local (LAN: Local Area Network): au seine d'une même entité (entreprise, administration, etc.)
	\item internet (international network): un réseau mondial
\end{itemize}

Pour se connecter à un réseau:

\begin{tabular}{ccc}
	\vgraphpage[2cm]{ethernet.jpg} &
	\vgraphpage[2cm]{wifi.png} &
	\vgraphpage[2cm]{4g.jpeg} \\
	
	Cable ethernet &
	WI-FI &
	4G \\
\end{tabular}

\end{frame}

\begin{frame}
\frametitle{Environnement de travail distant: Accès}
\framesubtitle{Adressage (IP)}

\begin{itemize}
	\item \keyword{IP}: Internet Protocol (anglais)
	\item utilisée pour identifier une machine sur un réseau
	\item attribuée d'une manière permanente (ex. les entreprises) ou provisoire (ex. modem de votre maison). 
	\item deux versions:  version 4 et 6. 
	\item version 4: notée avec quatre nombres compris entre \keyword{0} et \keyword{255}, séparés par des points
	\item exemple: \expword{192.168.1.9}
	\item deux parties: 
	\begin{itemize}
		\item gauche et fixe (ne se change pas), représentant le sous réseau 
		\item droite et variable, représentant le hôte
	\end{itemize}
\end{itemize}


\end{frame}

\begin{frame}
\frametitle{Environnement de travail distant: Accès}
\framesubtitle{Adressage (Masque)}

\begin{itemize}
	\item représenté comme l'adresse IP (sur 4 nombres)
	\item utilisé pour spécifier la partie sous-réseau et la partie hôte
	\item une version simple du masque: là où il y a un \keyword{255} (sous réseau), là où il y a \keyword{0} (hôte)
	\item exemple: \expword{255.255.255.0}; Les trois premiers nombres dans l'adresse IP représentent le sous réseau, le dernier représente l'hôte 
	\item une version plus détaillée sera entamée dans le module réseau.
\end{itemize}

Par exemple: 
\begin{itemize}
	\item On veut créer un sous réseau pour la bibliothèque de l'école. 
	\item On définit l'adresse IP de la bibliothèque comme \expword{10.0.7.1}
	\item On définit le masque de ce sous-réseau comme \expword{255.255.255.0}
	\item Lorsqu'un étudiant se connecte à ce sous-réseau, il/elle aura une adresse IP entre \expword{10.0.7.2} et \expword{10.0.7.254} 
	\item La partie \expword{10.0.7} ne se change pas 
\end{itemize}

\end{frame}

\begin{frame}
\frametitle{Environnement de travail distant: Accès}
\framesubtitle{Adressage (MAC)}

\begin{itemize}
	\item \keyword{MAC}: Media Access Control (anglais)
	\item un identifiant physique de la carte réseau 
	\item unique pour chaque carte réseau  
	\item représentée par 6 nombres hexadécimaux, compris entre 00 et FF, séparés par des double points
	\item exemple: \expword{CA:B2:34:A4:DC:57}
	\item deux parties (3 nombres chacune): 
	\begin{itemize}
		\item \keyword{OUI} (Organizationally Unique Identifier): réservée aux fabricants. 
		Liste des fabriquants: \url{http://standards-oui.ieee.org/oui.txt}
		\item \keyword{NIC} (Network Interface Controller): numéro de série des cartes réseaux d'un même fabriquant
	\end{itemize}
\end{itemize}

\end{frame}

\begin{frame}
\frametitle{Environnement de travail distant: Accès}
\framesubtitle{Adressage (URL)}

\begin{itemize}
	\item \keyword{URL}: Uniform Resource Locator (anglais)
	\item identifier une ressource sur le web en utilisant une chaine de caractères
	\item unique pour chaque carte réseau  
	\item composée de plusieurs parties; une version simple: 
	\begin{itemize}
		\item protocole, suivie par \keyword{://}
		\item nom de domaine, indiquant l'adresse du serveur
		\item chemin absolu par rapport au serveur 
		\item le fichier 
	\end{itemize}
\end{itemize}

\hgraphpage{url.pdf}

\end{frame}

\begin{frame}
\frametitle{Environnement de travail distant: Accès}
\framesubtitle{Adressage: un peu d'humeur}

\begin{center}
	\vgraphpage{adress-humour.jpg}
\end{center}


\end{frame}

%\begin{frame}
%\frametitle{Environnement de travail distant: Accès}
%\framesubtitle{Adressage: Exercice}
%
%
%
%
%\end{frame}


\begin{frame}
\frametitle{Environnement de travail distant: Accès}
\framesubtitle{Le web}

Abréviation de \keyword{World Wide Web}, définit par Larousse comme:
\begin{definition}
	Système hypermédia permettant d'accéder aux ressources du réseau Internet.
\end{definition}

\begin{itemize}
	\item un site web se compose de plusieurs pages web
	\item une page web est un fichier texte écrit en langage HTML et accompagné par d'autres ressources comme les images. 
	\item une page web peut se référer à une autre en utilisant des \keyword{liens hypertextes} 
\end{itemize}

\end{frame}


\begin{frame}
\frametitle{Environnement de travail distant: Accès}
\framesubtitle{Protocoles}

\begin{itemize}
	\item \optword{HTTP (Hypertext Transfer Protocol)}: transfert hypertexte pour naviguer sur le web  (HTTPS pour la
	version sécurisée)
	\item \optword{FTP (File Transfer Protocol)}: transfert des fichiers
	\item \optword{SMTP (Simple Mail Transport Protocol)}: envoi et réception du courrier électronique
	\item \optword{DNS (Domain Name System)}: transformation d'un \keyword{URL} vers une adresse \keyword{IP}
	\item \optword{DHCP (Dynamic Host Configuration protocol)}: affectation des adresses \keyword{IP} à des clients (hôtes)
\end{itemize}

\end{frame}


\begin{frame}
\frametitle{Environnement de travail distant: Accès}
\framesubtitle{Serveurs}

\rowcolors{2}{lightblue}{lightyellow}
	
\begin{tabular}{p{.10\textwidth}p{.45\textwidth}p{.35\textwidth}}
%	\hline\hline
	\rowcolor{darkblue}
	\textcolor{white}{Serveur} & \textcolor{white}{Utilisation} & \textcolor{white}{Exemple} \\
	
%	\hline\hline
	Web & accès au web (HTTP, HTTPS) & 
	\vgraphpage[.8cm]{apache.png} Apache, 
	
	\vgraphpage[.8cm]{iis.png} IIS (Windows) \\
%	\hline
	
	FTP & transfert des fichiers & 
	\vgraphpage[.8cm]{filezilla.png} FileZilla (Windows), 
	VsFTPd (Unix-like) \\
	
%	\hline
	SMTP & gestion de messagerie & 
	\vgraphpage[.8cm]{exim.png} exim \\
	
%	\hline
	DNS & avoir un IP à partir d'une URL & \\
	
%	\hline
	DHCP & fournir des adresse IP & \\
	
%	\hline
	Proxy & cache, filtrage, log & 
	\vgraphpage[.8cm]{squid.jpg} Squid, Apache, IIS \\
	
%	\hline\hline
\end{tabular}



\end{frame}

\begin{frame}
\frametitle{Environnement de travail distant: Accès}
\framesubtitle{Serveurs (DNS)}

\begin{minipage}{0.65\textwidth}
	\hgraphpage{dns.png}
\end{minipage}
%
\begin{minipage}{0.30\textwidth}
	DNS de Google
	\begin{itemize}
		\item 8.8.8.8
		\item 8.8.4.4
	\end{itemize}

	OpenDNS
	\begin{itemize}
		\item 208.67.222.222
		\item 208.67.220.220
	\end{itemize}

\end{minipage}

\end{frame}

\begin{frame}
\frametitle{Environnement de travail distant: Accès}
\framesubtitle{Exemple d'accès au Web}

\hgraphpage{access-web.pdf}

\end{frame}


%\begin{frame}
%\frametitle{Environnement de travail distant: Accès}
%\framesubtitle{Exercice: créer un serveur HTTP sous Windows avec réseau adhoc}
%
%
%
%
%\end{frame}


\subsection{Le cloud (le nuage)}

\begin{frame}
\frametitle{Environnement de travail distant}
\framesubtitle{Le cloud}

Le cloud est définit selon le NIST (National Institute of Standards and Technology)\footnote{La traduction en Français est fournie par \url{http://www.culture-informatique.net/cest-quoi-le-cloud/}}
\begin{definition}
	Le cloud computing est un modèle qui permet un accès omniprésent, pratique et à la demande à un réseau partagé et à un ensemble de ressources informatiques configurables (comme par exemple : des réseaux, des serveurs, du stockage, des applications et des services) qui peuvent être provisionnées et libérées avec un minimum d’administration ...
\end{definition}

\end{frame}

\begin{frame}
\frametitle{Environnement de travail distant: Le cloud}
\framesubtitle{Caractéristiques}

\begin{itemize}
	\item Un service à la demande
	\item Un accès aux ressources par le réseau
	\item Mise en commun des ressources
	\item Flexibilité des ressources
	\item Un service mesuré
\end{itemize}


\end{frame}


\begin{frame}
\frametitle{Environnement de travail distant: Le cloud}
\framesubtitle{Types}

\hgraphpage{cloud-types.png}

\end{frame}

\begin{frame}
\frametitle{Environnement de travail distant: Le cloud}
\framesubtitle{Exemples}

\rowcolors{2}{lightblue}{lightyellow}
\begin{tabular}{p{.27\textwidth}p{.27\textwidth}p{.27\textwidth}}
%	\hline\hline
	\rowcolor{darkblue}
	\textcolor{white}{IaaS}  & \textcolor{white}{PaaS} & \textcolor{white}{SaaS} \\
%	\hline
	
	* \href{http://azure.microsoft.com/}{Microsoft Azure}
	
	* \href{https://aws.amazon.com}{Amazon Web Services}
	
	* \href{https://cloud.google.com/compute/}{Google Compute Engine}
	
	&
	
	* \href{https://www.heroku.com}{Heroku}
	
	* \href{https://www.ibm.com/cloud-computing/bluemix/}{IBM Bluemix}
	
	* \href{https://www.openshift.org/}{OpenShift de RedHat}
	
	&
	
	* \href{https://www.google.com/drive/}{Google Drive}
	
	* \href{https://www.dropbox.com/}{DropBox}
	
	* \href{https://onedrive.live.com/}{Microsoft OneDrive}
	
	* \href{http://one.ubuntu.com/}{Ubuntu One}
	 
	* \href{https://docs.google.com/}{Google Docs}
	
	* \href{https://microsoftonline.com}{Microsoft Office}
	\\
%	\hline\hline
\end{tabular}


\end{frame}

\begin{frame}
\frametitle{Environnement de travail distant: Le cloud}
\framesubtitle{Un peu d'humour}

\begin{center}
	\vgraphpage{cloud-humour.png}
\end{center}

\end{frame}

\subsection{Google Drive}

\begin{frame}
\frametitle{Environnement de travail distant}
\framesubtitle{Google Drive}

\begin{itemize}
	\item un service de stockage et de partage de fichiers dans le cloud
	\item lancé par Google
	\item stockage gratuit de 15G (pour avoir plus, il faut payer)
	\item intègre une suite bureautique (Google Docs, Sheets, Slides, etc.)
	\item URL: \url{https://drive.google.com/}
\end{itemize}

\end{frame}


\begin{frame}
\frametitle{Environnement de travail distant: Google Drive}
\framesubtitle{Vue d'ensemble}

\begin{center}
	\hgraphpage{drive.png}
\end{center}

\end{frame}

\begin{frame}
\frametitle{Environnement de travail distant: Google Drive}
\framesubtitle{Créer ou importer}

\begin{center}
	\vgraphpage{drive-new.png}
\end{center}

\end{frame}

\begin{frame}
\frametitle{Environnement de travail distant: Google Drive}
\framesubtitle{Paramètres d'un fichier (dossier)}

\begin{center}
	\vgraphpage{drive-options1.png}
	\vgraphpage{drive-options2.png}
\end{center}

\end{frame}

\begin{frame}
\frametitle{Environnement de travail distant: Google Drive}
\framesubtitle{Gestion des versions (fichiers no supportés par la suite Google)}

\begin{center}
	\vgraphpage{drive-version.png}
\end{center}

\end{frame}

\begin{frame}
\frametitle{Environnement de travail distant: Google Drive}
\framesubtitle{Gestion des versions (fichiers supportés par la suite Google)}

\begin{center}
	\hgraphpage[.45\textwidth]{drive-versionB1.png}
	\vline
	\hgraphpage[.45\textwidth]{drive-versionB2.png}
\end{center}

\end{frame}

\begin{frame}
\frametitle{Environnement de travail distant: Google Drive}
\framesubtitle{Partage de fichiers (dossiers)}

\begin{center}
	\hgraphpage[.45\textwidth]{drive-share1.png}
	\hgraphpage[.45\textwidth]{drive-share2.png}
\end{center}

\end{frame}

\insertbibliography{Bweb01}{*}

\end{document}

