% !TEX TS-program = pdflatex
% !TeX program = pdflatex
% !TEX encoding = UTF-8
% !TEX spellcheck = fr

\documentclass[xcolor=table]{beamer}


%\usepackage{fullpage}
%\usepackage[left=2.8cm,right=2.2cm,top=2 cm,bottom=2 cm]{geometry}
\setbeamersize{text margin left=10pt,text margin right=10pt}
\usepackage{amsmath,amssymb} 
\usepackage[T1]{fontenc}
\usepackage[utf8]{inputenc}
\usepackage[english,french]{babel}
\usepackage{txfonts}
\usepackage[]{graphicx}
\usepackage{multirow}
\usepackage{hyperref}
\usepackage{colortbl}
\usepackage{listings}
\usepackage{wrapfig}
\usepackage{multicol}

\hypersetup{
	colorlinks,
	urlcolor = blue
}

%\renewcommand{\baselinestretch}{1.5}

\def\supit#1{\raisebox{0.8ex}{\small\it #1}\hspace{0.05em}}

\AtBeginSection{%
	\begin{frame}
		\sectionpage
	\end{frame}
}

\newcommand{\rottext}[2]{%
	\rotatebox{90}{%
	\begin{minipage}{#1}%
		\raggedleft#2%
	\end{minipage}%
	}%
}

\usepackage{longtable}
\usepackage{tabu}


\institute{ %
École  nationale Supérieure d'Informatique (ESI, ex. INI), Algérie
}
\author[ \textbf{\footnotesize  \insertframenumber/\inserttotalframenumber} \hspace*{\fill} ESI (2019-2020)] %
{ARIES Abdelkrime}
%\titlegraphic{\includegraphics[height=1cm]{../img/esi-logo.png}%\hspace*{4.75cm}~


\date{Année unniversitaire: 2019/2020} %\today

\usetheme{Warsaw} % Antibes Boadilla Warsaw

\beamertemplatenavigationsymbolsempty

%\setbeamertemplate{headline}{}

\definecolor{lightblue}{HTML}{D0D2FF}
\definecolor{lightyellow}{HTML}{FFFFAA}
\definecolor{darkblue}{HTML}{0000BB}
\definecolor{olivegreen}{HTML}{006600}
\definecolor{violet}{HTML}{6600CC}

\newcommand{\keyword}[1]{\textcolor{red}{\bfseries\itshape #1}}
\newcommand{\expword}[1]{\textcolor{olivegreen}{#1}}
\newcommand{\optword}[1]{\textcolor{violet}{\bfseries #1}}

\makeatletter
\newcommand\mysphere{%
	\parbox[t]{10pt}{\raisebox{0.2pt}{\beamer@usesphere{item projected}{bigsphere}}}}
\makeatother

%\let\oldtabular\tabular
%\let\endoldtabular\endtabular
%\renewenvironment{tabular}{\rowcolors{2}{white}{lightblue}\oldtabular\rowcolor{blue}}{\endoldtabular}


\NoAutoSpacing %french autospacing after ":"
\usepackage{lstautogobble}

\title[BWEB: 03- Rédaction (\LaTeX)] %
{Bureautique et Web \\Chapitre 03: Rédaction d'un document numérique \\ \LaTeX}  

\changegraphpath{..//img/Bweb03-redaction/}

%\lstset{
%	basicstyle=\scriptsize\ttfamily,language=[LaTeX]Tex,breaklines=true,
%	breakautoindent=true,breakindent=2ex,
%}

\begin{document}
	
	\newcolumntype{L}[2]{%
		>{\vbox to #2\bgroup\vfill\flushleft}%
		p{#1}%
		<{\egroup}} 

\begin{frame}[plain]
	\maketitle
\end{frame}

\begin{frame}
\frametitle{Rédaction d'un document numérique}
\framesubtitle{\LaTeX}

\end{frame}

\begin{frame}
\frametitle{Rédaction d'un document numérique}
\framesubtitle{\LaTeX : Plan}

\begin{multicols}{2}
%	\small
\tableofcontents
\end{multicols}
\end{frame}

%===================================================================================
\section{Préparer un document}
%===================================================================================

\begin{frame}[fragile]
\frametitle{Préparer un document}
Un document \LaTeX se compose de ces parties :
\begin{itemize}
	\item préambule, contenant la la classe du document et des commandes
	\item contenu entre \keyword{\textbackslash begin\{document\}} et \keyword{\textbackslash end\{document\}}
\end{itemize}

\begin{minipage}{0.60\textwidth} 
	Parmi les classes disponibles
	\begin{itemize}
		\item book
		\item memoir
		\item report 
		\item article
		\item letter 
		\item beamer
	\end{itemize}
\end{minipage}
%
\begin{minipage}{0.38\textwidth}
\begin{exampleblock}{exemple: hello.tex}
\scriptsize\bfseries
\begin{lstlisting}[language={[LaTeX]TeX}]
\documentclass{book}
% espace pour les packages
\begin{document}
    % le contenu
    Hello World!
\end{document}
\end{lstlisting}
\end{exampleblock}
\end{minipage}

\end{frame}

\subsection{Sections}

\begin{frame}
\frametitle{Préparer un document}
\framesubtitle{Sections}

\rowcolors{2}{lightblue}{lightyellow}

\begin{tabular}{p{.3\textwidth}cp{.6\textwidth}}
	%	\hline\hline
	\rowcolor{darkblue}
	\textcolor{white}{Section} && \textcolor{white}{Fonction} \\
	%	\hline\hline
	\textbackslash part\{...\} && \\
	\textbackslash chapter\{...\} && \\
	\textbackslash section\{...\} && \\
	\textbackslash subsection\{...\} && \\
	\textbackslash subsubsection\{...\} && \\
\end{tabular}

\end{frame}

%\begin{frame}[fragile]
%\frametitle{Préparer un document: Sections}
%\framesubtitle{Parties et Chapitres}
%
%\begin{exampleblock}{exemple: titre.tex}
%\scriptsize\bfseries
%\begin{lstlisting}
%\documentclass{book}
%
%\title{Introduction LaTeX}
%\author{Abdelkrime Aries}
%\date{2020}
%
%\begin{document}
%
%    \maketitle
%	\chapter*{Introduction}
%	mot mot mot mot
%
%    \chapter{Un chapite numerote}
%    mot mot mot mot
%    
%    \section{une section}
%	
%    \chapter{Un chapite numerote}
%    mot mot mot mot
%    
%    
%
%\end{document}
%\end{lstlisting}
%\end{exampleblock}
%
%\end{frame}

\subsection{Pages}

\begin{frame}
\frametitle{Préparer un document}
\framesubtitle{Pages}

\end{frame}

\begin{frame}[fragile]
\frametitle{Préparer un document: Pages}
\framesubtitle{Titre (Page de garde)}

\begin{exampleblock}{exemple: titre.tex}
	\scriptsize\bfseries
	\begin{lstlisting}
	\documentclass{article}
	
	\title{Introduction LaTeX}
	\author{Abdelkrime Aries}
	\date{2020}
	
	\begin{document}
	
		\maketitle
	
		Hello World!
	
	\end{document}
	\end{lstlisting}
\end{exampleblock}

\end{frame}

\begin{frame}[fragile]
\frametitle{Préparer un document: Pages}
\framesubtitle{Taille}

\begin{exampleblock}{exemple: la taille de la page}
	\scriptsize\bfseries
	\begin{lstlisting}
	\documentclass[a4paper, 12pt]{article}
	\end{lstlisting}
\end{exampleblock}

\end{frame}

%\begin{frame}
%\frametitle{Préparer un document: Pages}
%\framesubtitle{Orientation}
%
%\end{frame}

\begin{frame}[fragile]
\frametitle{Préparer un document: Pages}
\framesubtitle{Marges}

\begin{exampleblock}{exemple: les marges de la page}
\scriptsize\bfseries
\begin{lstlisting}
\documentclass[a4paper, 12pt]{article}
\usepackage[left=2.8cm,right=2.2cm,top=2 cm,bottom=2 cm]{geometry}
\end{lstlisting}
\end{exampleblock}

\end{frame}

\begin{frame}
\frametitle{Préparer un document: Pages}
\framesubtitle{Colonnes}

\end{frame}

\begin{frame}[fragile]
\frametitle{Préparer un document: Pages}
\framesubtitle{Saute de page}

\begin{exampleblock}{exemple: nouvpage.tex}
	\scriptsize\bfseries
	\begin{lstlisting}
	\documentclass{article}
	\begin{document}
	Avant la nouvelle page
	\newpage
	Apres la nouvelle page
	\end{document}
	\end{lstlisting}
\end{exampleblock}

\end{frame}

%\begin{frame}
%\frametitle{Préparer un document: Pages}
%\framesubtitle{Bordures}
%
%\end{frame}

%\begin{frame}
%\frametitle{Préparer un document: Pages}
%\framesubtitle{Filigrane}
%
%\end{frame}

\begin{frame}[fragile]
\frametitle{Préparer un document: Pages}
\framesubtitle{Arrière-plan}

\begin{exampleblock}{exemple: nouvpage.tex}
	\scriptsize\bfseries
	\begin{lstlisting}
	\documentclass{article}
	\usepackage{xcolor}
	
	\begin{document}
	
	\pagecolor{yellow}
	
	Hello World!
	
	\end{document}
	\end{lstlisting}
\end{exampleblock}

\end{frame}

\begin{frame}
\frametitle{Préparer un document: Pages}
\framesubtitle{En-tête et pied de page}



\end{frame}


\subsection{Paragraphes}

\begin{frame}
\frametitle{Préparer un document}
\framesubtitle{Paragraphes}

\end{frame}

\begin{frame}
\frametitle{Préparer un document: Paragraphes}
\framesubtitle{Alignement}

\end{frame}

\begin{frame}
\frametitle{Préparer un document: Paragraphes}
\framesubtitle{Espacement vertical}

\end{frame}

\begin{frame}
\frametitle{Préparer un document: Paragraphes}
\framesubtitle{Interligne}

\end{frame}

\begin{frame}
\frametitle{Préparer un document: Paragraphes}
\framesubtitle{Retrait}

\end{frame}

\begin{frame}
\frametitle{Préparer un document: Paragraphes}
\framesubtitle{Retrait de 1ère ligne}

\end{frame}

\begin{frame}
\frametitle{Préparer un document: Paragraphes}
\framesubtitle{Orientation du texte}

\end{frame}

\begin{frame}
\frametitle{Préparer un document: Paragraphes}
\framesubtitle{Listes}

\end{frame}


\subsection{Mise en forme et styles}

\begin{frame}
\frametitle{Préparer un document}
\framesubtitle{Mise en forme et styles}

\end{frame}

\begin{frame}
\frametitle{Préparer un document: Mise en forme et styles}
\framesubtitle{Police}

\end{frame}

\begin{frame}
\frametitle{Préparer un document: Mise en forme et styles}
\framesubtitle{Couleur}

\end{frame}

\begin{frame}
\frametitle{Préparer un document: Mise en forme et styles}
\framesubtitle{Casse}

\end{frame}

\begin{frame}
\frametitle{Préparer un document: Mise en forme et styles}
\framesubtitle{Style des caractères}

\end{frame}

\begin{frame}
\frametitle{Préparer un document: Mise en forme et styles}
\framesubtitle{Effets de texte}

\end{frame}


%===================================================================================
\section{Enrichir un document}
%===================================================================================

\begin{frame}
\frametitle{Enrichir un document}

\end{frame}

\subsection{Tableaux}

\begin{frame}
\frametitle{Enrichir un document}
\framesubtitle{Tableaux}

\end{frame}

\begin{frame}
\frametitle{Enrichir un document: Tableaux}
\framesubtitle{Formatage: Bordures}

\end{frame}


\subsection{Images et dessins}

\begin{frame}
\frametitle{Enrichir un document}
\framesubtitle{Images}

\end{frame}

\begin{frame}
\frametitle{Enrichir un document: Images}
\framesubtitle{Insérer une image}

\end{frame}

\begin{frame}
\frametitle{Enrichir un document: Images et dessins}
\framesubtitle{Modifier une image}

\end{frame}

\begin{frame}
\frametitle{Enrichir un document: Images}
\framesubtitle{Habillage}

\end{frame}

\begin{frame}
\frametitle{Enrichir un document: Images}
\framesubtitle{Positionnement}

\end{frame}

\subsection{Liens et renvoi}

\begin{frame}
\frametitle{Enrichir un document}
\framesubtitle{Liens et renvoi}

\end{frame}

\begin{frame}
\frametitle{Enrichir un document: Liens et renvoi}
\framesubtitle{Insérer un lien hypertexte}

\end{frame}

\begin{frame}
\frametitle{Enrichir un document: Liens et renvoi}
\framesubtitle{Renvoi vers un emplacement dans le document}

\end{frame}

\subsection{Formules et symboles}

\begin{frame}
\frametitle{Enrichir un document}
\framesubtitle{Formules et symboles}

\end{frame}

\begin{frame}
\frametitle{Enrichir un document: Formules et symboles}
\framesubtitle{Insérer une formule}

\end{frame}

\begin{frame}
\frametitle{Enrichir un document: Formules et symboles}
\framesubtitle{Insérer des symboles}

\end{frame}

%===================================================================================
\section{Références}
%===================================================================================

\begin{frame}
\frametitle{Références}

\end{frame}

\subsection{Notes de bas de page}

\begin{frame}
\frametitle{Références}
\framesubtitle{Notes de bas de page}

\end{frame}

\subsection{Table de matières}

\begin{frame}
\frametitle{Références}
\framesubtitle{Table de matières}

\end{frame}

\subsection{Légendes}

\begin{frame}
\frametitle{Références}
\framesubtitle{Légendes}

\end{frame}

\subsection{Index}

\begin{frame}
\frametitle{Références}
\framesubtitle{Index}

\end{frame}

%%===================================================================================
%\section{Révision et partage}
%%===================================================================================
%
%\begin{frame}
%\frametitle{Révision et partage}
%
%\end{frame}
%
%\subsection{Vérification}
%
%\begin{frame}
%\frametitle{Révision et partage}
%\framesubtitle{Vérification}
%
%\end{frame}
%
%\subsection{Commentaires et suivi}
%
%\begin{frame}
%\frametitle{Révision et partage}
%\framesubtitle{Commentaires et suivi}
%
%\end{frame}
%
%\subsection{Comparaison}
%
%\begin{frame}
%\frametitle{Révision et partage}
%\framesubtitle{Comparaison}
%
%\end{frame}
%
%\subsection{Partage}
%
%\begin{frame}
%\frametitle{Révision et partage}
%\framesubtitle{Partage}
%
%
%\end{frame}
%
%\begin{frame}
%\frametitle{Révision et partage: Partage}
%\framesubtitle{Protection}
%
%\end{frame}
%
%\begin{frame}
%\frametitle{Révision et partage: Partage}
%\framesubtitle{Impression}
%
%\end{frame}
%
%\begin{frame}
%\frametitle{Révision et partage: Partage}
%\framesubtitle{Sauvegarde}
%
%\end{frame}
%
%\begin{frame}
%\frametitle{Révision et partage: Partage}
%\framesubtitle{Sauvegarde sur cloud}
%
%
%\end{frame}

\nocite{*}
%\subsection{Bibliography}
%\frame[allowframebreaks]%
%{\frametitle{Références}
%\tiny
%\bibliography{Bweb01}
%\bibliographystyle{plain} 
%}

\begin{multicols*}{2}[\usebeamertemplate*{frametitle}\frametitle{Références}]%
	\tiny
	\bibliography{Bweb03}
	\bibliographystyle{acm}
\end{multicols*}


\end{document}

