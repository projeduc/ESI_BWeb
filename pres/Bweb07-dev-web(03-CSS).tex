% !TEX TS-program = pdflatex
% !TeX program = pdflatex
% !TEX encoding = UTF-8
% !TEX spellcheck = fr

\documentclass[xcolor=table]{beamer}


%\usepackage{fullpage}
%\usepackage[left=2.8cm,right=2.2cm,top=2 cm,bottom=2 cm]{geometry}
\setbeamersize{text margin left=10pt,text margin right=10pt}
\usepackage{amsmath,amssymb} 
\usepackage[T1]{fontenc}
\usepackage[utf8]{inputenc}
\usepackage[english,french]{babel}
\usepackage{txfonts}
\usepackage[]{graphicx}
\usepackage{multirow}
\usepackage{hyperref}
\usepackage{colortbl}
\usepackage{listings}
\usepackage{wrapfig}
\usepackage{multicol}

\hypersetup{
	colorlinks,
	urlcolor = blue
}

%\renewcommand{\baselinestretch}{1.5}

\def\supit#1{\raisebox{0.8ex}{\small\it #1}\hspace{0.05em}}

\AtBeginSection{%
	\begin{frame}
		\sectionpage
	\end{frame}
}

\newcommand{\rottext}[2]{%
	\rotatebox{90}{%
	\begin{minipage}{#1}%
		\raggedleft#2%
	\end{minipage}%
	}%
}

\usepackage{longtable}
\usepackage{tabu}


\institute{ %
École  nationale Supérieure d'Informatique (ESI, ex. INI), Algérie
}
\author[ \textbf{\footnotesize  \insertframenumber/\inserttotalframenumber} \hspace*{\fill} ESI (2019-2020)] %
{ARIES Abdelkrime}
%\titlegraphic{\includegraphics[height=1cm]{../img/esi-logo.png}%\hspace*{4.75cm}~


\date{Année unniversitaire: 2019/2020} %\today

\usetheme{Warsaw} % Antibes Boadilla Warsaw

\beamertemplatenavigationsymbolsempty

%\setbeamertemplate{headline}{}

\definecolor{lightblue}{HTML}{D0D2FF}
\definecolor{lightyellow}{HTML}{FFFFAA}
\definecolor{darkblue}{HTML}{0000BB}
\definecolor{olivegreen}{HTML}{006600}
\definecolor{violet}{HTML}{6600CC}

\newcommand{\keyword}[1]{\textcolor{red}{\bfseries\itshape #1}}
\newcommand{\expword}[1]{\textcolor{olivegreen}{#1}}
\newcommand{\optword}[1]{\textcolor{violet}{\bfseries #1}}

\makeatletter
\newcommand\mysphere{%
	\parbox[t]{10pt}{\raisebox{0.2pt}{\beamer@usesphere{item projected}{bigsphere}}}}
\makeatother

%\let\oldtabular\tabular
%\let\endoldtabular\endtabular
%\renewenvironment{tabular}{\rowcolors{2}{white}{lightblue}\oldtabular\rowcolor{blue}}{\endoldtabular}


\NoAutoSpacing %french autospacing after ":"

\title[BWEB : 07- CSS] %
{Bureautique et Web \\Chapitre 07 : Développement Web\\ \slshape\small  CSS}  

\changegraphpath{../img/dev-web/}

\begin{document}
	
\begin{frame}
\frametitle{CSS}
\framesubtitle{Introduction}

\begin{itemize}
	\item CSS : Cascading Style Sheets
	\item un langage de feuille de style 
	\item il sert à décrire la présentation d'une page web
\end{itemize}

\end{frame}

\begin{frame}
\frametitle{CSS}
\framesubtitle{Fonctionnement}

\begin{figure}
	\centering
	\hgraphpage{rendering.pdf}
	\caption{Application de CSS \cite{mdn}}
\end{figure}
\begin{itemize}
	\item \keyword{DOM} : Document Object Model
\end{itemize}

\end{frame}


\begin{frame}
\frametitle{CSS}
\framesubtitle{Plan}

\begin{multicols}{2}
%	\small
\tableofcontents
\end{multicols}
\end{frame}

%===================================================================================
\section{Les règles et la cascade}
%===================================================================================

\subsection{Les règles et les valeurs}

\begin{frame}[fragile]
\frametitle{CSS : Les règles et la cascade}
\framesubtitle{Les règles et les valeurs : Les règles}

\begin{minipage}{0.60\textwidth} 
\begin{itemize}
	\item Une règle se compose d'un sélecteur et d'une déclaration (propriété/valeur)
	\item \optword{Sélecteur} : permet de sélectionner les éléments sur lesquels appliquer le style souhaité
	\item \optword{Propriété} : différentes façons dont on peut mettre en forme un élément HTML
	\item \optword{Valeur} : permet de choisir une mise en forme parmi d'autres pour une propriété donnée
\end{itemize}
\end{minipage}
%
\begin{minipage}{0.38\textwidth}
\begin{exampleblock}{Structure d'une règle CSS}
\lstset{escapeinside=**}
\scriptsize\bfseries
\begin{lstlisting}[language={html}]
*sélecteur* {
  *proriété1*: valeur1;
  *proriété2*: valeur2;
}
\end{lstlisting}
\end{exampleblock}
\begin{figure}
	\centering
	\hgraphpage[.6\textwidth]{CSS-regle.pdf}
	\caption{Structure d'une règle CSS \cite{mdn}}
\end{figure}
\end{minipage}

\end{frame}

\begin{frame}[fragile]
\frametitle{CSS : Les règles et la cascade}
\framesubtitle{Les règles et les valeurs : Unités absolues de longueur}

\begin{minipage}{0.60\textwidth} 
	\begin{itemize}
		\item Une règle se compose d'un sélecteur et d'une déclaration (propriété/valeur)
		\item \optword{Sélecteur} : permet de sélectionner les éléments sur lesquels appliquer le style souhaité
		\item \optword{Propriété} : différentes façons dont on peut mettre en forme un élément HTML
		\item \optword{Valeur} : permet de choisir une mise en forme parmi d'autres pour une propriété donnée
	\end{itemize}
\end{minipage}
%
\begin{minipage}{0.38\textwidth}
	\begin{exampleblock}{Structure d'une règle CSS}
		\lstset{escapeinside=**}
		\scriptsize\bfseries
		\begin{lstlisting}[language={html}]
		*sélecteur* {
		*proriété1*: valeur1;
		*proriété2*: valeur2;
		}
		\end{lstlisting}
	\end{exampleblock}
	\begin{figure}
		\centering
		\hgraphpage[.6\textwidth]{CSS-regle.pdf}
		\caption{Structure d'une règle CSS \cite{mdn}}
	\end{figure}
\end{minipage}

\end{frame}

\subsection{Application de CSS}

\begin{frame}[fragile]
\frametitle{CSS : Les règles et la cascade}
\framesubtitle{Application de CSS}

\end{frame}

\begin{frame}[fragile]
\frametitle{CSS : Les règles et la cascade}
\framesubtitle{Application de CSS : l'attribut "style"}

\begin{minipage}{0.60\textwidth} 
	\begin{itemize}
		\item 
	\end{itemize}
\end{minipage}
%
\begin{minipage}{0.38\textwidth}
	\begin{block}{Structure d'un document HTML}
		\lstset{escapeinside=**}
		\scriptsize\bfseries
		\begin{lstlisting}[language={html}]
		h1 {
		color: red;
		}
		\end{lstlisting}
	\end{block}
\end{minipage}
\end{frame}

\begin{frame}[fragile]
\frametitle{CSS : Les règles et la cascade}
\framesubtitle{Application de CSS : la balise "style"}

\begin{minipage}{0.60\textwidth} 
	\begin{itemize}
		\item 
	\end{itemize}
\end{minipage}
%
\begin{minipage}{0.38\textwidth}
\begin{exampleblock}{Structure d'une règle CSS}
\lstset{escapeinside=**}
\scriptsize\bfseries
\begin{lstlisting}[language={html}]
*sélecteur* {
*proriété1*: valeur1;
*proriété2*: valeur2;
}
\end{lstlisting}
\end{exampleblock}
\end{minipage}
\end{frame}

\begin{frame}[fragile]
\frametitle{CSS : Les règles et la cascade}
\framesubtitle{Application de CSS : Style externe}

\begin{minipage}{0.60\textwidth} 
	\begin{itemize}
		\item 
	\end{itemize}
\end{minipage}
%
\begin{minipage}{0.38\textwidth}
	\begin{block}{Structure d'un document HTML}
		\lstset{escapeinside=**}
		\scriptsize\bfseries
		\begin{lstlisting}[language={html}]
		h1 {
		color: red;
		}
		\end{lstlisting}
	\end{block}
\end{minipage}
\end{frame}

\subsection{Les sélecteurs}

\begin{frame}
\frametitle{CSS : Les règles et la cascade}
\framesubtitle{Les sélecteurs}

\begin{itemize}
	\item Les sélecteurs simples
	\begin{itemize}
		\item \optword{Sélecteur universel}
		\item \optword{Sélecteurs de type}
		\item \optword{Sélecteurs de classe}
		\item \optword{Sélecteurs d'identifiant}
		\item \optword{Sélecteurs d'attribut}
	\end{itemize}
	\item Les combinateurs
	\begin{itemize}
		\item \optword{Sélecteurs d'éléments descendants}
		\item \optword{Sélecteurs d'éléments fils}
		\item \optword{Sélecteurs de voisin}
		\item \optword{Sélecteurs de voisin direct}
	\end{itemize}
	\item \optword{Les pseudo-classes}
	\item \optword{Les pseudo-éléments}
\end{itemize}

\end{frame}

\begin{frame}[fragile]
\frametitle{CSS : Les règles et la cascade}
\framesubtitle{Les sélecteurs : Les sélecteurs simples}

\begin{minipage}{0.60\textwidth} 
	\begin{itemize}
		\item \optword{universel} sélectionne tous les éléments en utilisant le mot clé \keyword{*}
		\item \optword{de type} l'élément est ciblé par so type
		\item \optword{de classe} l'élément est ciblé par la valeur de son attribut \keyword{class} en précédant cette valeur par \keyword{.}
		\item \optword{d'identifiant} l'élément est ciblé par la valeur de son attribut \keyword{id} en précédant cette valeur par \keyword{\#}
		\item \optword{d'attribut} l'élément est ciblé par la valeur d'un attribut donné en utilisant la syntaxe \expword{[attr operateur valeur]}
	\end{itemize}
\end{minipage}
%
\begin{minipage}{0.38\textwidth}
\begin{block}{selecteur\_simple.html}
%\lstset{escapeinside=**}
\scriptsize\bfseries
\begin{lstlisting}[language={html}]
* {font-size: 12pt;}
p {color: magenta;}
.p {color: green;}
#p {color: blue;}
input[type=text] {color: red;}
\end{lstlisting}
\end{block}
\end{minipage}
\end{frame}


\begin{frame}[fragile]
\frametitle{CSS : Les règles et la cascade}
\framesubtitle{Les sélecteurs : Les combinateurs}

\begin{minipage}{0.60\textwidth} 
	\begin{itemize}
		\item \optword{d'éléments descendants} on sépare l'élément et son descendant par un espace
		\item \optword{d'éléments fils} on sépare l'élément et son fils par \keyword{>}
		\item \optword{de voisin} on sépare l'élément et son voisin par \keyword{\textasciitilde}
		\item \optword{de voisin direct} on sépare l'élément et son voisin par \keyword{+}
	\end{itemize}
\end{minipage}
%
\begin{minipage}{0.38\textwidth}
\begin{block}{combinateur.html}
\lstset{escapeinside=**}
\scriptsize\bfseries
\begin{lstlisting}[language={html}]
#id1 p {color:blue;}
#id2 > p {color:red;}
h1 ~ p {color:green;}
h2 + p {color:orange;}
\end{lstlisting}
\end{block}
\end{minipage}
\end{frame}

\begin{frame}[fragile]
\frametitle{CSS : Les règles et la cascade}
\framesubtitle{Les sélecteurs : Les pseudo-classes}

\begin{minipage}{0.60\textwidth} 
	\begin{itemize}
		\item l'élément est ciblé par un état spécifique
		\item \keyword{:hover} la souris est positionnée sur l'élément
		\item \keyword{:visited} un lien visité
		\item \keyword{:link} un lien non visité
		\item \keyword{:active} un élément activé par l'utilisateur.
		\item \keyword{:checked} un bouton radio, case à couché ou une option sélectionnée
		\item \keyword{:disabled} un élément désactivé
		\item \keyword{:enabled} un élément activé
		\item \keyword{:required} un champs obligatoire
	\end{itemize}
\end{minipage}
%
\begin{minipage}{0.38\textwidth}
\begin{block}{pseudo-classes1.html}
%\lstset{escapeinside=**}
\scriptsize\bfseries
\begin{lstlisting}[language={html}]
a:link {color: blue;}
a:visited {color: purple;}
a:hover {color: yellow;}
a:active {color: red;}
input:checked {color:blue;}
input:disabled {color:grey;}
input:required {color:red;}
\end{lstlisting}
\end{block}
\end{minipage}
\end{frame}

\begin{frame}[fragile]
\frametitle{CSS : Les règles et la cascade}
\framesubtitle{Les sélecteurs : Les pseudo-classes (2)}

\begin{minipage}{0.60\textwidth} 
	\begin{itemize}
		\item \keyword{:not(s)} un élément qui n'est pas un élément sélectionné par le sélecteur \expword{s}
		\item \keyword{:nth-child(an+b)} un élément qui possède an+b-1 éléments voisins (au même niveau) avant lui 
		\item \keyword{:nth-of-type(an+b)} des éléments d'un type donné, en fonction de leur position au sein d'un groupe de frères et sœurs.
		\item \keyword{:out-of-range} un élément \keyword{<input>} lorsque sa valeur est en dehors de l'intervalle autorisé par les attributs \keyword{min} et \keyword{max}.
	\end{itemize}
\end{minipage}
%
\begin{minipage}{0.38\textwidth}
\begin{block}{pseudo-classes2.html}
%\lstset{escapeinside=**}
\scriptsize\bfseries\vspace{-6pt}
\begin{lstlisting}[language={html}]
*:not(h1){font-style: italic;}
tr:nth-child(2n) {
  background: lightblue;
}
tr:nth-child(2n+1) {
  background: lightgreen;
}
#id1 p:nth-child(2n+1) {
  color: red;
}
#id2 p:nth-of-type(2n+1) {
  color: blue;
}
input:out-of-range {
  background: red;
}
\end{lstlisting}\vspace{-6pt}
\end{block}
\end{minipage}
\end{frame}

\begin{frame}[fragile]
\frametitle{CSS : Les règles et la cascade}
\framesubtitle{Les sélecteurs : Les pseudo-éléments}

\begin{minipage}{0.60\textwidth} 
	\begin{itemize}
		\item Mettre en forme certaines parties de l'élément ciblé par la règle.
		\item \keyword{::first-line} la première ligne d'un élément
		\item \keyword{::first-letter} la première lettre d'un élément
		\item \keyword{::selection} la portion sélectionnée d'un élément
		\item \keyword{::before} insérer une chose avant le contenu d'un élément
		\item \keyword{::after} insérer une chose après le contenu d'un élément
	\end{itemize}
\end{minipage}
%
\begin{minipage}{0.38\textwidth}
\begin{block}{pseudo-element.html}
\lstset{escapeinside=**}
\scriptsize\bfseries
\begin{lstlisting}[language={html}]
p#id1::first-letter {
  color: blue; }
p#id2::first-line {
  color: red; }
p#id3::selection {
  color: green; }
p#id4::before {
  content: "avant ";
  color: orange;
}
p#id4::after {
  content: " *après*";
  color: magenta;
}
\end{lstlisting}
\end{block}
\end{minipage}
\end{frame}

\subsection{Cascade et héritage}

\begin{frame}
\frametitle{CSS : Les règles et la cascade}
\framesubtitle{Cascade et héritage}

\begin{itemize}
	\item 
\end{itemize}

\end{frame}

\begin{frame}[fragile]
\frametitle{CSS : Les règles et la cascade}
\framesubtitle{Cascade et héritage : L'héritage}

\begin{minipage}{0.60\textwidth}
\begin{itemize}
	\item 
\end{itemize}
\end{minipage}
%
\begin{minipage}{0.38\textwidth}
\begin{block}{Structure d'un document HTML}
	\lstset{escapeinside=**}
	\scriptsize\bfseries
	\begin{lstlisting}[language={html}]
	h1 {
	color: red;
	}
	\end{lstlisting}
\end{block}
\end{minipage}
\end{frame}

\begin{frame}[fragile]
\frametitle{CSS : Les règles et la cascade}
\framesubtitle{Cascade et héritage : Cascade}

\begin{minipage}{0.60\textwidth}
	\begin{itemize}
		\item 
	\end{itemize}
\end{minipage}
%
\begin{minipage}{0.38\textwidth}
	\begin{block}{Structure d'un document HTML}
		\lstset{escapeinside=**}
		\scriptsize\bfseries
		\begin{lstlisting}[language={html}]
		h1 {
		color: red;
		}
		\end{lstlisting}
	\end{block}
\end{minipage}
\end{frame}

\begin{frame}[fragile]
\frametitle{CSS : Les règles et la cascade}
\framesubtitle{Cascade et héritage : Spécificité}

\begin{minipage}{0.60\textwidth}
	\begin{itemize}
		\item 
	\end{itemize}
\end{minipage}
%
\begin{minipage}{0.38\textwidth}
	\begin{block}{Structure d'un document HTML}
		\lstset{escapeinside=**}
		\scriptsize\bfseries
		\begin{lstlisting}[language={html}]
		h1 {
		color: red;
		}
		\end{lstlisting}
	\end{block}
\end{minipage}
\end{frame}

%===================================================================================
\section{Mise en page}
%===================================================================================

\subsection{Le modèle en boite}

\begin{frame}[fragile]
\frametitle{CSS : Mise en page}
\framesubtitle{Le modèle en boite}

\begin{minipage}{0.60\textwidth}
	\begin{itemize}
		\item 
	\end{itemize}
\end{minipage}
%
\begin{minipage}{0.38\textwidth}
	\begin{block}{Structure d'un document HTML}
		\lstset{escapeinside=**}
		\scriptsize\bfseries
		\begin{lstlisting}[language={html}]
		h1 {
		color: red;
		}
		\end{lstlisting}
	\end{block}
\end{minipage}
\end{frame}


\subsection{Les flotteurs}

\begin{frame}[fragile]
\frametitle{CSS : Mise en page}
\framesubtitle{Les flotteurs}

\begin{minipage}{0.60\textwidth}
	\begin{itemize}
		\item 
	\end{itemize}
\end{minipage}
%
\begin{minipage}{0.38\textwidth}
	\begin{block}{Structure d'un document HTML}
		\lstset{escapeinside=**}
		\scriptsize\bfseries
		\begin{lstlisting}[language={html}]
		h1 {
		color: red;
		}
		\end{lstlisting}
	\end{block}
\end{minipage}
\end{frame}


\subsection{La position}

\begin{frame}[fragile]
\frametitle{CSS : Mise en page}
\framesubtitle{La position}

\begin{minipage}{0.60\textwidth}
	\begin{itemize}
		\item 
	\end{itemize}
\end{minipage}
%
\begin{minipage}{0.38\textwidth}
	\begin{block}{Structure d'un document HTML}
		\lstset{escapeinside=**}
		\scriptsize\bfseries
		\begin{lstlisting}[language={html}]
		h1 {
		color: red;
		}
		\end{lstlisting}
	\end{block}
\end{minipage}
\end{frame}


\subsection{Réactivité (responsiveness)}

\begin{frame}[fragile]
\frametitle{CSS : Mise en page}
\framesubtitle{Réactivité (responsiveness)}

\begin{minipage}{0.60\textwidth}
	\begin{itemize}
		\item 
	\end{itemize}
\end{minipage}
%
\begin{minipage}{0.38\textwidth}
	\begin{block}{Structure d'un document HTML}
		\lstset{escapeinside=**}
		\scriptsize\bfseries
		\begin{lstlisting}[language={html}]
		h1 {
		color: red;
		}
		\end{lstlisting}
	\end{block}
\end{minipage}
\end{frame}

%===================================================================================
\section{Propriétés de texte}
%===================================================================================

\subsection{Le texte}

\begin{frame}[fragile]
\frametitle{CSS : Propriétés de texte}
\framesubtitle{Le texte}

\begin{minipage}{0.60\textwidth}
	\begin{itemize}
		\item \keyword{text-transform}
		\item \keyword{text-decoration}
		\item \keyword{text-shadow}
		\item \keyword{text-align}
		\item \keyword{text-indent}
		\item \keyword{text-orientation}
		\item \keyword{direction}
		\item \keyword{line-height}
	\end{itemize}
\end{minipage}
%
\begin{minipage}{0.38\textwidth}
	\begin{block}{Structure d'un document HTML}
		\lstset{escapeinside=**}
		\scriptsize\bfseries
		\begin{lstlisting}[language={html}]
		h1 {
		color: red;
		}
		\end{lstlisting}
	\end{block}
\end{minipage}
\end{frame}

\subsection{La police}

\begin{frame}[fragile]
\frametitle{CSS : Propriétés de texte}
\framesubtitle{La police}

\begin{minipage}{0.60\textwidth}
	\begin{itemize}
		\item \keyword{font-family}
		\item \keyword{font-size}
		\item \keyword{font-style}
		\item \keyword{font-weight}
	\end{itemize}
\end{minipage}
%
\begin{minipage}{0.38\textwidth}
	\begin{block}{Structure d'un document HTML}
		\lstset{escapeinside=**}
		\scriptsize\bfseries
		\begin{lstlisting}[language={html}]
		h1 {
		color: red;
		}
		\end{lstlisting}
	\end{block}
\end{minipage}
\end{frame}

\begin{frame}[fragile]
\frametitle{CSS : Propriétés de texte}
\framesubtitle{La police : définir une famille de police}

\begin{minipage}{0.60\textwidth}
	\begin{itemize}
		\item \keyword{@font-face}
	\end{itemize}
\end{minipage}
%
\begin{minipage}{0.38\textwidth}
\begin{block}{Structure d'un document HTML}
\lstset{escapeinside=**}
\scriptsize\bfseries
\begin{lstlisting}[language={html}]
@font-face {
  font-family: "maPolice";
  src: url("unePolice.ttf");
}
\end{lstlisting}
\end{block}
\end{minipage}
\end{frame}

\subsection{Mots et lettres}

\begin{frame}[fragile]
\frametitle{CSS : Propriétés de texte}
\framesubtitle{Mots et lettres}

\begin{minipage}{0.60\textwidth}
	\begin{itemize}
		\item \keyword{word-spacing}
		\item \keyword{word-break}
		\item \keyword{letter-spacing}
	\end{itemize}
\end{minipage}
%
\begin{minipage}{0.38\textwidth}
	\begin{block}{Structure d'un document HTML}
		\lstset{escapeinside=**}
		\scriptsize\bfseries
		\begin{lstlisting}[language={html}]
		h1 {
		color: red;
		}
		\end{lstlisting}
	\end{block}
\end{minipage}
\end{frame}

\subsection{Listes}

\begin{frame}[fragile]
\frametitle{CSS : Propriétés de texte}
\framesubtitle{Listes}

\begin{minipage}{0.60\textwidth}
	\begin{itemize}
		\item \keyword{list-style-type}
		\item \keyword{list-style-position}
		\item \keyword{list-style-image}
	\end{itemize}
\end{minipage}
%
\begin{minipage}{0.38\textwidth}
	\begin{block}{Structure d'un document HTML}
		\lstset{escapeinside=**}
		\scriptsize\bfseries
		\begin{lstlisting}[language={html}]
		h1 {
		color: red;
		}
		\end{lstlisting}
	\end{block}
\end{minipage}
\end{frame}

%===================================================================================
\section{Aspect et animations}
%===================================================================================

\subsection{Arrière-plan}

\begin{frame}[fragile]
\frametitle{CSS : Aspect et animations}
\framesubtitle{Arrière-plan}

\begin{minipage}{0.60\textwidth}
	\begin{itemize}
		\item \keyword{background-color}
		\item \keyword{background-image}
		\item \keyword{background-repeat}
		\item \keyword{background-attachment}
	\end{itemize}
\end{minipage}
%
\begin{minipage}{0.38\textwidth}
	\begin{block}{Structure d'un document HTML}
		\lstset{escapeinside=**}
		\scriptsize\bfseries
		\begin{lstlisting}[language={html}]
		h1 {
		color: red;
		}
		\end{lstlisting}
	\end{block}
\end{minipage}
\end{frame}

\begin{frame}[fragile]
\frametitle{CSS : Aspect et animations}
\framesubtitle{Arrière-plan : Arrière-plan dégradé}

\begin{minipage}{0.60\textwidth}
	\begin{itemize}
		\item 
	\end{itemize}
\end{minipage}
%
\begin{minipage}{0.38\textwidth}
	\begin{block}{Structure d'un document HTML}
		\lstset{escapeinside=**}
		\scriptsize\bfseries
		\begin{lstlisting}[language={html}]
		h1 {
		color: red;
		}
		\end{lstlisting}
	\end{block}
\end{minipage}
\end{frame}

\subsection{Transformations}

\begin{frame}[fragile]
\frametitle{CSS : Aspect et animations}
\framesubtitle{Transformations}

\begin{minipage}{0.60\textwidth}
	\begin{itemize}
		\item 
	\end{itemize}
\end{minipage}
%
\begin{minipage}{0.38\textwidth}
	\begin{block}{Structure d'un document HTML}
		\lstset{escapeinside=**}
		\scriptsize\bfseries
		\begin{lstlisting}[language={html}]
		h1 {
		color: red;
		}
		\end{lstlisting}
	\end{block}
\end{minipage}
\end{frame}

\subsection{Animations}

\begin{frame}[fragile]
\frametitle{CSS : Aspect et animations}
\framesubtitle{Animations}

\begin{minipage}{0.60\textwidth}
	\begin{itemize}
		\item 
	\end{itemize}
\end{minipage}
%
\begin{minipage}{0.38\textwidth}
	\begin{block}{Structure d'un document HTML}
		\lstset{escapeinside=**}
		\scriptsize\bfseries
		\begin{lstlisting}[language={html}]
		h1 {
		color: red;
		}
		\end{lstlisting}
	\end{block}
\end{minipage}
\end{frame}


\begin{frame}
\frametitle{CSS}
\framesubtitle{Un peu d'humour}

\begin{center}
	\vgraphpage{css-humour.jpg}
\end{center}

\end{frame}

\insertbibliography{Bweb07}{*}

\end{document}

