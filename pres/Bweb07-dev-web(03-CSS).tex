% !TEX TS-program = pdflatex
% !TeX program = pdflatex
% !TEX encoding = UTF-8
% !TEX spellcheck = fr

\documentclass[xcolor=table]{beamer}


%\usepackage{fullpage}
%\usepackage[left=2.8cm,right=2.2cm,top=2 cm,bottom=2 cm]{geometry}
\setbeamersize{text margin left=10pt,text margin right=10pt}
\usepackage{amsmath,amssymb} 
\usepackage[T1]{fontenc}
\usepackage[utf8]{inputenc}
\usepackage[english,french]{babel}
\usepackage{txfonts}
\usepackage[]{graphicx}
\usepackage{multirow}
\usepackage{hyperref}
\usepackage{colortbl}
\usepackage{listings}
\usepackage{wrapfig}
\usepackage{multicol}

\hypersetup{
	colorlinks,
	urlcolor = blue
}

%\renewcommand{\baselinestretch}{1.5}

\def\supit#1{\raisebox{0.8ex}{\small\it #1}\hspace{0.05em}}

\AtBeginSection{%
	\begin{frame}
		\sectionpage
	\end{frame}
}

\newcommand{\rottext}[2]{%
	\rotatebox{90}{%
	\begin{minipage}{#1}%
		\raggedleft#2%
	\end{minipage}%
	}%
}

\usepackage{longtable}
\usepackage{tabu}


\institute{ %
École  nationale Supérieure d'Informatique (ESI, ex. INI), Algérie
}
\author[ \textbf{\footnotesize  \insertframenumber/\inserttotalframenumber} \hspace*{\fill} ESI (2019-2020)] %
{ARIES Abdelkrime}
%\titlegraphic{\includegraphics[height=1cm]{../img/esi-logo.png}%\hspace*{4.75cm}~


\date{Année unniversitaire: 2019/2020} %\today

\usetheme{Warsaw} % Antibes Boadilla Warsaw

\beamertemplatenavigationsymbolsempty

%\setbeamertemplate{headline}{}

\definecolor{lightblue}{HTML}{D0D2FF}
\definecolor{lightyellow}{HTML}{FFFFAA}
\definecolor{darkblue}{HTML}{0000BB}
\definecolor{olivegreen}{HTML}{006600}
\definecolor{violet}{HTML}{6600CC}

\newcommand{\keyword}[1]{\textcolor{red}{\bfseries\itshape #1}}
\newcommand{\expword}[1]{\textcolor{olivegreen}{#1}}
\newcommand{\optword}[1]{\textcolor{violet}{\bfseries #1}}

\makeatletter
\newcommand\mysphere{%
	\parbox[t]{10pt}{\raisebox{0.2pt}{\beamer@usesphere{item projected}{bigsphere}}}}
\makeatother

%\let\oldtabular\tabular
%\let\endoldtabular\endtabular
%\renewenvironment{tabular}{\rowcolors{2}{white}{lightblue}\oldtabular\rowcolor{blue}}{\endoldtabular}


\NoAutoSpacing %french autospacing after ":"

\title[BWEB : 07- CSS] %
{Bureautique et Web \\Chapitre 07 : Développement Web\\ \slshape\small  CSS}  

\changegraphpath{../img/dev-web/}

\begin{document}
	
\begin{frame}
\frametitle{CSS}
\framesubtitle{Introduction}

\begin{itemize}
	\item CSS : Cascading Style Sheets
	\item un langage de feuille de style 
	\item il sert à décrire la présentation d'une page web
\end{itemize}

\end{frame}

\begin{frame}
\frametitle{CSS}
\framesubtitle{Fonctionnement}

\begin{figure}
	\centering
	\hgraphpage{rendering.pdf}
	\caption{Application de CSS \cite{mdn}}
\end{figure}
\begin{itemize}
	\item \keyword{DOM} : Document Object Model
\end{itemize}

\end{frame}


\begin{frame}
\frametitle{CSS}
\framesubtitle{Plan}

\begin{multicols}{2}
%	\small
\tableofcontents
\end{multicols}
\end{frame}

%===================================================================================
\section{Les règles et la cascade}
%===================================================================================

\subsection{Les règles et les valeurs}

\begin{frame}[fragile]
\frametitle{CSS : Les règles et la cascade}
\framesubtitle{Les règles et les valeurs : Les règles}

\begin{minipage}{0.60\textwidth} 
\begin{itemize}
	\item Une règle se compose d'un sélecteur et d'une déclaration (propriété/valeur)
	\item \optword{Sélecteur} : permet de sélectionner les éléments sur lesquels appliquer le style souhaité
	\item \optword{Propriété} : différentes façons dont on peut mettre en forme un élément HTML
	\item \optword{Valeur} : permet de choisir une mise en forme parmi d'autres pour une propriété donnée
\end{itemize}
\end{minipage}
%
\begin{minipage}{0.38\textwidth}
\begin{exampleblock}{Structure d'une règle CSS}
\lstset{escapeinside=**}
\scriptsize\bfseries
\begin{lstlisting}[language={html}]
*sélecteur* {
  *proriété1*: valeur1;
  *proriété2*: valeur2;
}
\end{lstlisting}
\end{exampleblock}
\begin{figure}
	\centering
	\hgraphpage[.6\textwidth]{CSS-regle.pdf}
	\caption{Structure d'une règle CSS \cite{mdn}}
\end{figure}
\end{minipage}

\end{frame}

\begin{frame}[fragile]
\frametitle{CSS : Les règles et la cascade}
\framesubtitle{Les règles et les valeurs : Unités absolues}

\scriptsize\bfseries
\rowcolors{2}{lightblue}{lightyellow}
\begin{minipage}{0.50\textwidth} 
\begin{center}
	\begin{tabular}{lll}
		\rowcolor{darkblue}
		\color{white}Unité & \color{white}Nom & \color{white}Description \\
		cm & Centimètres & 1cm = 96px/2.54\\
		mm & Millimètres & 1mm = 1/10 cm\\
		Q & 1/4 millimètres & 1Q = 1/40 cm\\
		in & Pouces & 1in = 2.54cm = 96px\\
		pc & Picas & 1pc = 1/6 in\\
		pt & Points & 1pt = 1/72 in\\
		px & Pixels & 1px = 1/96 in\\
	\end{tabular}
\end{center}
\end{minipage}
%
\begin{minipage}{0.49\textwidth}
	\begin{center}
		\begin{tabular}{llp{.3\textwidth}}
			\rowcolor{darkblue}
			
			\color{white}Unité & \color{white}Nom & \textcolor{white}{Description} \\
			s & Secondes &  \\
			ms & Millisecondes & Un millième de seconde.\\
		\end{tabular}
	\end{center}
	\begin{center}
		\begin{tabular}{llp{.4\textwidth}}
			\rowcolor{darkblue}
			\color{white}Unité & \color{white}Nom & \color{white}Description \\
			Hz & Hertz & Nombre de fois par seconde.\\
			kHz & Kilohertz & 1000 Hertz.\\
		\end{tabular}
	\end{center}
\end{minipage}

%\scriptsize\bfseries
%\rowcolors{2}{lightblue}{lightyellow}
\begin{center}
	\begin{tabular}{lll}
		\rowcolor{darkblue}
		\color{white}Unité & \color{white}Nom & \color{white}Description \\
		deg & Degrés & Un cercle se divise en 360 degrés égaux. \\
		grad & Grades/Gradians & Un cercle se compose de 400 grades. \\
		rad & Radians & Un cercle se compose de $2\pi$ radians. \\
		turn & Tours & Un cercle se compose d'un tour. \\
	\end{tabular}
\end{center}

\end{frame}

\begin{frame}[fragile]
\frametitle{CSS : Les règles et la cascade}
\framesubtitle{Les règles et les valeurs : Unités relatives}

\scriptsize\bfseries
\rowcolors{2}{lightblue}{lightyellow}

\begin{center}
	\begin{tabular}{p{.1\textwidth}p{.8\textwidth}}
		\rowcolor{darkblue}
		\color{white}Unité & \color{white}Sens\\
		em & * Dans le cas de \keyword{font-size} : Taille de police du parent 
		
		* Dans d'autres cas comme \keyword{width} : Taille de police  de l'élément lui-même\\
		ex & La hauteur du caractère minuscule "x"\\
		ch & La largeur du caractère "0"\\
		rem & Taille de police de l'élément racine\\
		lh & Hauteur de ligne de l'élément\\
		vw & 1 \% de la largeur de la fenêtre\\
		vh & 1 \% de la longueur de la fenêtre\\
		vmin & 1 \% de la dimension minimale de la fenêtre\\
		vmax & 1 \% de la dimension maximale de la fenêtre\\
		\% & Pourcentage par rapport à l'élément parent\\
	\end{tabular}
\end{center}

\end{frame}

\subsection{Application de CSS}

\begin{frame}[fragile]
\frametitle{CSS : Les règles et la cascade}
\framesubtitle{Application de CSS}

\begin{itemize}
	\item Styles en ligne
	\begin{itemize}
		\item définir un style dans la balise de l'élément
		\item en utilisant l'attribut \keyword{style} de l'élément
	\end{itemize}
	\item Feuille de style interne
	\begin{itemize}
		\item définir une feuille de style dans une page HTML
		\item en utilisant la balise \keyword{<style>} dans \keyword{<head>}
	\end{itemize}
	\item Feuille de style externe
	\begin{itemize}
		\item définir une feuille de style séparée
		\item en utilisant la balise \keyword{<link>} dans \keyword{<head>}
	\end{itemize}
\end{itemize}

\end{frame}

\begin{frame}[fragile]
\frametitle{CSS : Les règles et la cascade}
\framesubtitle{Application de CSS : Style en ligne}

\begin{minipage}{0.60\textwidth} 
	\begin{itemize}
		\item En utilisant l'attribut \keyword{style} de la balise
		\item Mise en forme d'un seul élément
		\item Si on veut utiliser la même mise en forme pour plusieurs éléments, on doit la recopier
		\item Exemple, \expword{Si on veut utiliser la même couleur pour tous les éléments de type <p>, on doit recopier le style dans chaque éléments}
	\end{itemize}
\end{minipage}
%
\begin{minipage}{0.38\textwidth}
\begin{block}{Style en ligne}
\lstset{escapeinside=**}
\scriptsize\bfseries
\begin{lstlisting}[language={html}]
<html>
  <head>
    <title>en ligne</title>
  </head>
  <body>
    <p style="color:red;">
      Paragraphe en rouge
    </p>
  </body>
<html>
\end{lstlisting}
\end{block}
\end{minipage}
\end{frame}

\begin{frame}[fragile]
\frametitle{CSS : Les règles et la cascade}
\framesubtitle{Application de CSS : Feuille de style interne}

\begin{minipage}{0.60\textwidth} 
	\begin{itemize}
		\item En utilisant la balise \keyword{<style>} dans l'entête
		\item \keyword{type="text/css"} pour dire au navigateur qu'il s'agit de CSS
		\item Mise en forme des éléments d'une page HTML
		\item Si on veut utiliser la même feuille pour plusieurs pages, on doit la recopier
		\item Exemple, \expword{Si on veut utiliser la même couleur pour tous les éléments de type <p>, on doit recopier le style dans chaque éléments}
	\end{itemize}
\end{minipage}
%
\begin{minipage}{0.38\textwidth}
\begin{block}{Style en ligne}
\lstset{escapeinside=**}
\scriptsize\bfseries
\begin{lstlisting}[language={html}]
<html>
  <head>
    <title>interne</title>
    <style type="text/css">
      p {color: red;}
    </style>
  </head>
  <body>
    <p>Paragraphe1</p>
    <div>DIV1</div>
    <p>Paragraphe2</p>
    <p>Paragraphe3</p>
  </body>
<html>
\end{lstlisting}
\end{block}
\end{minipage}
\end{frame}

\begin{frame}[fragile]
\frametitle{CSS : Les règles et la cascade}
\framesubtitle{Application de CSS : Feuille de style externe}

\begin{minipage}{0.60\textwidth} 
	\begin{itemize}
		\item En utilisant la balise \keyword{<style>} dans l'entête
		\item \keyword{type="text/css"} peut être omis
		\item \keyword{rel="stylesheet"} pour dire au navigateur qu'il s'agit d'une feuille de style
		\item \keyword{href} l'emplacement de la feuille
		\item Mise en forme des éléments de plusieurs pages 
	\end{itemize}
\begin{minipage}{0.95\textwidth} 
\begin{block}{style.css}
	\lstset{escapeinside=**}
	\scriptsize\bfseries\vspace{-6pt}
\begin{lstlisting}[language={CSS}]
p {
  color: red;
}
\end{lstlisting}\vspace{-6pt}
\end{block}
\end{minipage}
\end{minipage}
%
\begin{minipage}{0.38\textwidth}
\begin{block}{Lier la page avec une feuille de style externe}
\lstset{escapeinside=**}
\scriptsize\bfseries
\begin{lstlisting}[language={html}]
<html>
  <head>
    <title>externe</title>
    <link rel="stylesheet" 
         href="style.css">
  </head>
  <body>
    <p>Paragraphe1</p>
    <div>DIV1</div>
    <p>Paragraphe2</p>
  </body>
<html>
\end{lstlisting}
\end{block}
\end{minipage}

\end{frame}

\subsection{Les sélecteurs}

\begin{frame}
\frametitle{CSS : Les règles et la cascade}
\framesubtitle{Les sélecteurs}

\begin{itemize}
	\item Les sélecteurs simples
	\begin{itemize}
		\item \optword{Sélecteur universel}
		\item \optword{Sélecteurs de type}
		\item \optword{Sélecteurs de classe}
		\item \optword{Sélecteurs d'identifiant}
		\item \optword{Sélecteurs d'attribut}
	\end{itemize}
	\item Les combinateurs
	\begin{itemize}
		\item \optword{Sélecteurs d'éléments descendants}
		\item \optword{Sélecteurs d'éléments fils}
		\item \optword{Sélecteurs de voisin}
		\item \optword{Sélecteurs de voisin direct}
	\end{itemize}
	\item \optword{Les pseudo-classes}
	\item \optword{Les pseudo-éléments}
\end{itemize}

\end{frame}

\begin{frame}[fragile]
\frametitle{CSS : Les règles et la cascade}
\framesubtitle{Les sélecteurs : Les sélecteurs simples}

\begin{minipage}{0.60\textwidth} 
	\begin{itemize}
		\item \optword{universel} sélectionne tous les éléments en utilisant le mot clé \keyword{*}
		\item \optword{de type} l'élément est ciblé par so type
		\item \optword{de classe} l'élément est ciblé par la valeur de son attribut \keyword{class} en précédant cette valeur par \keyword{.}
		\item \optword{d'identifiant} l'élément est ciblé par la valeur de son attribut \keyword{id} en précédant cette valeur par \keyword{\#}
		\item \optword{d'attribut} l'élément est ciblé par la valeur d'un attribut donné en utilisant la syntaxe \expword{[attr operateur valeur]}
	\end{itemize}
\end{minipage}
%
\begin{minipage}{0.38\textwidth}
\begin{block}{selecteur\_simple.html}
%\lstset{escapeinside=**}
\scriptsize\bfseries
\begin{lstlisting}[language={CSS}]
* {font-size: 12pt;}
p {color: magenta;}
.p {color: green;}
#p {color: blue;}
input[type=text] {
  color: red;
}
\end{lstlisting}
\end{block}
\end{minipage}
\end{frame}


\begin{frame}[fragile]
\frametitle{CSS : Les règles et la cascade}
\framesubtitle{Les sélecteurs : Les combinateurs}

\begin{minipage}{0.60\textwidth} 
	\begin{itemize}
		\item \optword{d'éléments descendants} on sépare l'élément et son descendant par un espace
		\item \optword{d'éléments fils} on sépare l'élément et son fils par \keyword{>}
		\item \optword{de voisin} on sépare l'élément et son voisin par \keyword{\textasciitilde}
		\item \optword{de voisin direct} on sépare l'élément et son voisin par \keyword{+}
	\end{itemize}
\end{minipage}
%
\begin{minipage}{0.38\textwidth}
\begin{block}{combinateur.html}
\lstset{escapeinside=**}
\scriptsize\bfseries
\begin{lstlisting}[language={CSS}]
#id1 p {color:blue;}
#id2 > p {color:red;}
h1 ~ p {color:green;}
h2 + p {color:orange;}
\end{lstlisting}
\end{block}
\end{minipage}
\end{frame}

\begin{frame}[fragile]{Code}
\frametitle{CSS : Les règles et la cascade}
\framesubtitle{Les sélecteurs : Les pseudo-classes}

\begin{minipage}{0.60\textwidth} 
	\begin{itemize}
		\item l'élément est ciblé par un état spécifique
		\item \keyword{:hover} la souris est positionnée sur l'élément
		\item \keyword{:visited} un lien visité
		\item \keyword{:link} un lien non visité
		\item \keyword{:active} un élément activé par l'utilisateur.
		\item \keyword{:checked} un bouton radio, case à couché ou une option sélectionnée
		\item \keyword{:disabled} un élément désactivé
		\item \keyword{:enabled} un élément activé
		\item \keyword{:required} un champs obligatoire
	\end{itemize}
\end{minipage}
%
\begin{minipage}{0.38\textwidth}
\begin{block}{pseudo-classes1.html}
%\lstset{escapeinside=**}
\scriptsize\bfseries
\begin{lstlisting}[language={CSS}]
a:link {color: blue;}
a:visited {color: purple;}
a:hover {color: yellow;}
a:active {color: red;}
input:checked {color:blue;}
input:disabled {color:grey;}
input:required {color:red;}
\end{lstlisting}
\end{block}
\end{minipage}

\end{frame}

\begin{frame}[fragile]
\frametitle{CSS : Les règles et la cascade}
\framesubtitle{Les sélecteurs : Les pseudo-classes (2)}

\begin{minipage}{0.60\textwidth} 
	\begin{itemize}
		\item \keyword{:not(s)} un élément qui n'est pas un élément sélectionné par le sélecteur \expword{s}
		\item \keyword{:nth-child(an+b)} un élément qui possède an+b-1 éléments voisins (au même niveau) avant lui 
		\item \keyword{:nth-of-type(an+b)} des éléments d'un type donné, en fonction de leur position au sein d'un groupe de frères et sœurs.
		\item \keyword{:out-of-range} un élément \keyword{<input>} lorsque sa valeur est en dehors de l'intervalle autorisé par les attributs \keyword{min} et \keyword{max}.
	\end{itemize}
\end{minipage}
%
\begin{minipage}{0.38\textwidth}
\begin{block}{pseudo-classes2.html}
%\lstset{escapeinside=**}
\scriptsize\bfseries\vspace{-6pt}
\begin{lstlisting}[language={html}]
*:not(h1){font-style: italic;}
tr:nth-child(2n) {
  background: lightblue;
}
tr:nth-child(2n+1) {
  background: lightgreen;
}
#id1 p:nth-child(2n+1) {
  color: red;
}
#id2 p:nth-of-type(2n+1) {
  color: blue;
}
input:out-of-range {
  background: red;
}
\end{lstlisting}\vspace{-6pt}
\end{block}
\end{minipage}
\end{frame}

\begin{frame}[fragile]
\frametitle{CSS : Les règles et la cascade}
\framesubtitle{Les sélecteurs : Les pseudo-éléments}

\begin{minipage}{0.60\textwidth} 
	\begin{itemize}
		\item Mettre en forme certaines parties de l'élément ciblé par la règle.
		\item \keyword{::first-line} la première ligne d'un élément
		\item \keyword{::first-letter} la première lettre d'un élément
		\item \keyword{::selection} la portion sélectionnée d'un élément
		\item \keyword{::before} insérer une chose avant le contenu d'un élément
		\item \keyword{::after} insérer une chose après le contenu d'un élément
	\end{itemize}
\end{minipage}
%
\begin{minipage}{0.38\textwidth}
\begin{block}{pseudo-element.html}
\lstset{escapeinside=**}
\scriptsize\bfseries
\begin{lstlisting}[language={html}]
p#id1::first-letter {
  color: blue; }
p#id2::first-line {
  color: red; }
p#id3::selection {
  color: green; }
p#id4::before {
  content: "avant ";
  color: orange;
}
p#id4::after {
  content: " *après*";
  color: magenta;
}
\end{lstlisting}
\end{block}
\end{minipage}
\end{frame}

\subsection{Cascade et héritage}

%\begin{frame}
%\frametitle{CSS : Les règles et la cascade}
%\framesubtitle{Cascade et héritage}
%
%\begin{itemize}
%	\item 
%\end{itemize}
%
%\end{frame}

\begin{frame}[fragile]
\frametitle{CSS : Les règles et la cascade}
\framesubtitle{Cascade et héritage : L'héritage}

\begin{minipage}{0.60\textwidth}
\begin{itemize}
	\item Si un élément contient un autre élément, ce dernier hérite les propriétés e son parent
	\item Sauf si la propriété a été redéfinie
	\item Par exemple, \expword{Si on définit la couleur pour l'élément <body>, tous les éléments dedans vont hériter cette propriété}
\end{itemize}
\end{minipage}
%
\begin{minipage}{0.38\textwidth}
\begin{block}{Héritage dans CSS}
\lstset{escapeinside=**}
\scriptsize\bfseries
\begin{lstlisting}[language={HTML5}]
<html>
  <head>
    <title>interne</title>
    <style type="text/css">
      body {color: red;}
    </style>
  </head>
  <body>
    <p>Paragraphe1</p>
    <div>DIV1</div>
    <p>Paragraphe2</p>
    <p>Paragraphe3</p>
  </body>
<html>
\end{lstlisting}
\end{block}
\end{minipage}
\end{frame}

\begin{frame}[fragile]
\frametitle{CSS : Les règles et la cascade}
\framesubtitle{Cascade et héritage : Cascade}

\begin{minipage}{0.60\textwidth}
	\begin{itemize}
		\item L'origine de la feuille
		\begin{itemize}
			\item \optword{Agent utilisateur} : chaque navigateur fournit une feuille de style par défaut
			\item \optword{Auteur} : celui qui a créé le site web 
			\item \optword{Utilisateur} : celui qui utilise le site web 
		\end{itemize}
		\item L'algorithme de la cascade
		\begin{enumerate}
			\item Filtrer les règles des différentes sources pour garder celles applicables sur un élément
			\item Trier les règles en se basant sur leurs origines et importances.
			\item Appliquer les règles les plus importantes 
		\end{enumerate}
	\end{itemize}
\end{minipage}
%
\begin{minipage}{0.38\textwidth}
\begin{center}
	\scriptsize\bfseries
	\rowcolors{2}{lightblue}{lightyellow}
	\begin{tabular}{ll}
		\rowcolor{darkblue}
		\color{white}Origine & \color{white}Importance\\
		Agent utilisateur & normale\\
		Utilisateur & normale\\
		Auteur & normale\\
		Auteur & !important\\
		Utilisateur & !important\\
		Agent utilisateur & !important\\
	\end{tabular}
\end{center}
\begin{exampleblock}{Exemple d'une règle importante}
\lstset{escapeinside=**}
\tiny\bfseries
\begin{lstlisting}[language={HTML5}]
p {
  color: red;
  font-size: 14pt !important;
}
\end{lstlisting}
\end{exampleblock}
\end{minipage}
\end{frame}


\begin{frame}[fragile]
\frametitle{CSS : Les règles et la cascade}
\framesubtitle{Cascade et héritage : Spécificité}

\begin{minipage}{0.60\textwidth}
	\begin{itemize}
		\item On calcule la spécificité en se basant sur le sélecteur
		\item une règle en ligne : spécificité \optword{1000} 
		\item le nombre des IDs dans le sélecteur multiplié par \optword{100}
		\item le nombre des classes, les sélecteurs d'attributs et pseudo-classes dans le sélecteur multiplié par \optword{10}
		\item le nombre des éléments et pseudo-éléments dans le sélecteur
		\item on fait la somme
	\end{itemize}
\end{minipage}
%
\begin{minipage}{0.38\textwidth}
\begin{exampleblock}{Exemple de spécificité}
\lstset{escapeinside=!!}
\scriptsize\bfseries
\begin{lstlisting}[language={html}]
p a {}/*!spécificité! = 002*/
p .a {}/*!spécificité! = 011*/
p #a {}/*!spécificité! = 101*/
/*!spécificité! = 112*/
div#a p.c {}
/*!spécificité! = 011*/
input[type=text] {}
/*!spécificité! = 012*/
a:hover img {}
\end{lstlisting}
\end{exampleblock}
\end{minipage}
\end{frame}

%===================================================================================
\section{Mise en page}
%===================================================================================

\subsection{Le modèle en boîte}

\begin{frame}[fragile]
\frametitle{CSS : Mise en page}
\framesubtitle{Le modèle en boîte : Rendu et disposition}

\begin{minipage}{0.60\textwidth}
	\begin{itemize}
		\item On définit le rendu et la disposition des éléments en utilisant la propriété \keyword{display}
		\item \optword{none} : l'élément est caché
		\item \optword{block} : l'élément est défini comme bloc
		\item \optword{inline} : l'élément est défini comme en ligne
		\item \optword{inline-block} : l'élément est défini comme en ligne, mais on peut définir la hauteur et la largeur
	\end{itemize}
\end{minipage}
%
\begin{minipage}{0.38\textwidth}
\begin{block}{display.html}
\lstset{escapeinside=**}
\scriptsize\bfseries
\begin{lstlisting}[language={CSS}]
p {
  background: red;
  height: 50px;
}
#p1 {display:none;}
#p2 {display:block;}
#p3 {display:inline;}
#p4 {display:inline-block;}
\end{lstlisting}
\end{block}
\end{minipage}
\end{frame}

\begin{frame}[fragile]
\frametitle{CSS : Mise en page}
\framesubtitle{Le modèle en boîte : Rendu et disposition}

\begin{minipage}{0.60\textwidth}
	\begin{itemize}
		\item \keyword{display: flex;}
		\item les éléments d'un conteneur flexible peuvent être disposés dans n'importe quelle direction
		\item l'attribut \keyword{flex-direction} définit la direction des éléments fils
		\begin{itemize}
			\item \optword{row} : la direction du texte
			\item \optword{row-inverse} : l'inverse de la direction du texte
			\item \optword{column} : de haut en bas
			\item \optword{column-inverse} : de bas en haut
		\end{itemize}
	\item \keyword{flex-wrap: wrap;} : les éléments peuvent être affichés sur plusieurs lignes 
	\end{itemize}
\end{minipage}
%
\begin{minipage}{0.38\textwidth}
\begin{block}{display.html}
\lstset{escapeinside=**}
\tiny\bfseries
\begin{lstlisting}[language={CSS}]
div {
  display: flex;
  flex-wrap: wrap;
}
.c {width: 30%;}
#div5 {
  flex-direction:row;}
#div6 {
  flex-direction:row-reverse;}
#div7 {
  flex-direction:column;}
#div8 {
  flex-direction:column-reverse;}
\end{lstlisting}
\end{block}
\end{minipage}
\end{frame}

\begin{frame}[fragile]
\frametitle{CSS : Mise en page}
\framesubtitle{Le modèle en boîte : composition d'une boîte}

\begin{minipage}{0.60\textwidth}
	\begin{itemize}
		\item \keyword{margin} : la marge entre l'élément et ses voisins
		\item \keyword{border} : la bordure de l'élément 
		\item \keyword{paddin} : la marge entre la bordure et le contenu
		\item \keyword{width} : la largeur du contenu
		\item \keyword{height} : la hauteur du contenu
		\item pour spécifier la direction des marges (internes ou externes) ou la bordure, on ajoute \keyword{-top} (haut), \keyword{-left} (gauche), \keyword{-right} (droit) ou \keyword{-bottom} (bas)
	\end{itemize}
\end{minipage}
%
\begin{minipage}{0.38\textwidth}
	\hgraphpage{box-model.pdf}
	\begin{block}{box.html}
		\lstset{escapeinside=**}
		\tiny\bfseries
\begin{lstlisting}[language={html}]
#d1 {
  background: red;
  width: 200px; height: 100px;
  padding: 5px;
  border-top: 5px solid green;
  border-bottom: 5px dashed green;
  margin-bottom: 20px;
}
#d1 p {background: blue;}
\end{lstlisting}
	\end{block}
\end{minipage}
\end{frame}

\subsection{Les flotteurs et débordement}

\begin{frame}[fragile]
\frametitle{CSS : Mise en page}
\framesubtitle{Les flotteurs et débordement : Les flotteurs}

\begin{minipage}{0.60\textwidth}
	\begin{itemize}
		\item créer des mises en page de sites web avec plusieurs colonnes d'informations flottantes les unes à côté des autres
		\item éléments de type bloc
		\item l'attribut \keyword{float} avec la valeur \optword{left} (à gauche) ou \optword{right} (à droite)
		\item pour terminer le flottement, utiliser l'attribut \keyword{clear} avec la valeur \optword{both} (les deux types de flottement), \optword{left} (à gauche) ou \optword{right} (à droite)
	\end{itemize}
\end{minipage}
%
\begin{minipage}{0.38\textwidth}
	\begin{block}{float.html}
		\lstset{escapeinside=**}
		\scriptsize\bfseries
\begin{lstlisting}[language={html}]
.c1 {background: red;}
.c2 {background: blue;}
.c2::after {
  content: "";
  clear: both;
  display: block;
}
.c3 {background: green;}
#f {
  width: 200px;
  height: 200px;
  float: left;
  background: cyan;
}
\end{lstlisting}
	\end{block}
\end{minipage}
\end{frame}

\begin{frame}[fragile]
\frametitle{CSS : Mise en page}
\framesubtitle{Les flotteurs et débordement : Le débordement}

\begin{minipage}{0.60\textwidth}
	\begin{itemize}
		\item Dans un élément en bloc, si le contenu dépasse la taille de l'élément, on peut spécifier comment le gérer
		\item on utilise l'attribut \keyword{overflow} avec la valeur
		\begin{itemize}
			\item \optword{visible} le contenu n'est pas rogné (par défaut)
			\item \optword{hidden} le contenu est rogné
			\item \optword{scroll} ajouter des barres de défilement
			\item \optword{auto} comportement de l'agent utilisateur
		\end{itemize}
		\item On peut spécifier le dépassement vertical en utilisant l'attribut \keyword{overflow-y}
		\item On peut spécifier le dépassement horizontal en utilisant l'attribut \keyword{overflow-x}
	\end{itemize}
\end{minipage}
%
\begin{minipage}{0.38\textwidth}
	\begin{block}{overflow.html}
		\lstset{escapeinside=**}
		\tiny\bfseries
\begin{lstlisting}[language={html}]
div {
  width: 150px; height: 150px;
}

#d1 {
  background: yellow;
  overflow: scroll; }

#d2 {
  background: red;
  overflow-x: scroll;
  overflow-y: hidden; }

#d3 {
  background: blue;
  overflow: hidden; }

#d4 {
  background: orange;
  overflow: visible; }
\end{lstlisting}
	\end{block}
\end{minipage}
\end{frame}

\subsection{La position}

\begin{frame}[fragile]
\frametitle{CSS : Mise en page}
\framesubtitle{La position}

\begin{minipage}{0.60\textwidth}
	\begin{itemize}
		\item 
	\end{itemize}
\end{minipage}
%
\begin{minipage}{0.38\textwidth}
	\begin{block}{Structure d'un document HTML}
		\lstset{escapeinside=**}
		\scriptsize\bfseries
\begin{lstlisting}[language={html}]

\end{lstlisting}
	\end{block}
\end{minipage}
\end{frame}


\subsection{Réactivité (responsiveness)}

\begin{frame}[fragile]
\frametitle{CSS : Mise en page}
\framesubtitle{Réactivité (responsiveness)}

\begin{minipage}{0.60\textwidth}
	\begin{itemize}
		\item 
	\end{itemize}
\end{minipage}
%
\begin{minipage}{0.38\textwidth}
	\begin{block}{Structure d'un document HTML}
		\lstset{escapeinside=**}
		\scriptsize\bfseries
		\begin{lstlisting}[language={html}]
		h1 {
		color: red;
		}
		\end{lstlisting}
	\end{block}
\end{minipage}
\end{frame}

%===================================================================================
\section{Propriétés de texte}
%===================================================================================

\subsection{Le texte}

\begin{frame}[fragile]
\frametitle{CSS : Propriétés de texte}
\framesubtitle{Le texte}

\begin{minipage}{0.60\textwidth}
	\begin{itemize}
		\item \keyword{text-transform}
		\item \keyword{text-decoration}
		\item \keyword{text-shadow}
		\item \keyword{text-align}
		\item \keyword{text-indent}
		\item \keyword{text-orientation}
		\item \keyword{direction}
		\item \keyword{line-height}
	\end{itemize}
\end{minipage}
%
\begin{minipage}{0.38\textwidth}
	\begin{block}{Structure d'un document HTML}
		\lstset{escapeinside=**}
		\scriptsize\bfseries
		\begin{lstlisting}[language={html}]
		h1 {
		color: red;
		}
		\end{lstlisting}
	\end{block}
\end{minipage}
\end{frame}

\subsection{La police}

\begin{frame}[fragile]
\frametitle{CSS : Propriétés de texte}
\framesubtitle{La police}

\begin{minipage}{0.60\textwidth}
	\begin{itemize}
		\item \keyword{font-family}
		\item \keyword{font-size}
		\item \keyword{font-style}
		\item \keyword{font-weight}
	\end{itemize}
\end{minipage}
%
\begin{minipage}{0.38\textwidth}
	\begin{block}{Structure d'un document HTML}
		\lstset{escapeinside=**}
		\scriptsize\bfseries
		\begin{lstlisting}[language={html}]
		h1 {
		color: red;
		}
		\end{lstlisting}
	\end{block}
\end{minipage}
\end{frame}

\begin{frame}[fragile]
\frametitle{CSS : Propriétés de texte}
\framesubtitle{La police : définir une famille de police}

\begin{minipage}{0.60\textwidth}
	\begin{itemize}
		\item \keyword{@font-face}
	\end{itemize}
\end{minipage}
%
\begin{minipage}{0.38\textwidth}
\begin{block}{Structure d'un document HTML}
\lstset{escapeinside=**}
\scriptsize\bfseries
\begin{lstlisting}[language={html}]
@font-face {
  font-family: "maPolice";
  src: url("unePolice.ttf");
}
\end{lstlisting}
\end{block}
\end{minipage}
\end{frame}

\subsection{Mots et lettres}

\begin{frame}[fragile]
\frametitle{CSS : Propriétés de texte}
\framesubtitle{Mots et lettres}

\begin{minipage}{0.60\textwidth}
	\begin{itemize}
		\item \keyword{word-spacing}
		\item \keyword{word-break}
		\item \keyword{letter-spacing}
	\end{itemize}
\end{minipage}
%
\begin{minipage}{0.38\textwidth}
	\begin{block}{Structure d'un document HTML}
		\lstset{escapeinside=**}
		\scriptsize\bfseries
		\begin{lstlisting}[language={html}]
		h1 {
		color: red;
		}
		\end{lstlisting}
	\end{block}
\end{minipage}
\end{frame}

\subsection{Listes}

\begin{frame}[fragile]
\frametitle{CSS : Propriétés de texte}
\framesubtitle{Listes}

\begin{minipage}{0.60\textwidth}
	\begin{itemize}
		\item \keyword{list-style-type}
		\item \keyword{list-style-position}
		\item \keyword{list-style-image}
	\end{itemize}
\end{minipage}
%
\begin{minipage}{0.38\textwidth}
	\begin{block}{Structure d'un document HTML}
		\lstset{escapeinside=**}
		\scriptsize\bfseries
		\begin{lstlisting}[language={html}]
		h1 {
		color: red;
		}
		\end{lstlisting}
	\end{block}
\end{minipage}
\end{frame}

%===================================================================================
\section{Aspect et animations}
%===================================================================================

\subsection{Arrière-plan}

\begin{frame}[fragile]
\frametitle{CSS : Aspect et animations}
\framesubtitle{Arrière-plan}

\begin{minipage}{0.60\textwidth}
	\begin{itemize}
		\item \keyword{background-color}
		\item \keyword{background-image}
		\item \keyword{background-repeat}
		\item \keyword{background-attachment}
	\end{itemize}
\end{minipage}
%
\begin{minipage}{0.38\textwidth}
	\begin{block}{Structure d'un document HTML}
		\lstset{escapeinside=**}
		\scriptsize\bfseries
		\begin{lstlisting}[language={html}]
		h1 {
		color: red;
		}
		\end{lstlisting}
	\end{block}
\end{minipage}
\end{frame}

\begin{frame}[fragile]
\frametitle{CSS : Aspect et animations}
\framesubtitle{Arrière-plan : Arrière-plan dégradé}

\begin{minipage}{0.60\textwidth}
	\begin{itemize}
		\item 
	\end{itemize}
\end{minipage}
%
\begin{minipage}{0.38\textwidth}
	\begin{block}{Structure d'un document HTML}
		\lstset{escapeinside=**}
		\scriptsize\bfseries
		\begin{lstlisting}[language={html}]
		h1 {
		color: red;
		}
		\end{lstlisting}
	\end{block}
\end{minipage}
\end{frame}

\subsection{Transformations}

\begin{frame}[fragile]
\frametitle{CSS : Aspect et animations}
\framesubtitle{Transformations}

\begin{minipage}{0.60\textwidth}
	\begin{itemize}
		\item 
	\end{itemize}
\end{minipage}
%
\begin{minipage}{0.38\textwidth}
	\begin{block}{Structure d'un document HTML}
		\lstset{escapeinside=**}
		\scriptsize\bfseries
		\begin{lstlisting}[language={html}]
		h1 {
		color: red;
		}
		\end{lstlisting}
	\end{block}
\end{minipage}
\end{frame}

\subsection{Animations}

\begin{frame}[fragile]
\frametitle{CSS : Aspect et animations}
\framesubtitle{Animations}

\begin{minipage}{0.60\textwidth}
	\begin{itemize}
		\item 
	\end{itemize}
\end{minipage}
%
\begin{minipage}{0.38\textwidth}
	\begin{block}{Structure d'un document HTML}
		\lstset{escapeinside=**}
		\scriptsize\bfseries
		\begin{lstlisting}[language={html}]
		h1 {
		color: red;
		}
		\end{lstlisting}
	\end{block}
\end{minipage}
\end{frame}


\begin{frame}
\frametitle{CSS}
\framesubtitle{Un peu d'humour}

\begin{center}
	\vgraphpage{css-humour.jpg}
\end{center}

\end{frame}

\insertbibliography{Bweb07}{*}

\end{document}

