% !TEX TS-program = pdflatex
% !TeX program = pdflatex
% !TEX encoding = UTF-8
% !TEX spellcheck = fr

\documentclass[xcolor=table]{beamer}


%\usepackage{fullpage}
%\usepackage[left=2.8cm,right=2.2cm,top=2 cm,bottom=2 cm]{geometry}
\setbeamersize{text margin left=10pt,text margin right=10pt}
\usepackage{amsmath,amssymb} 
\usepackage[T1]{fontenc}
\usepackage[utf8]{inputenc}
\usepackage[english,french]{babel}
\usepackage{txfonts}
\usepackage[]{graphicx}
\usepackage{multirow}
\usepackage{hyperref}
\usepackage{colortbl}
\usepackage{listings}
\usepackage{wrapfig}
\usepackage{multicol}

\hypersetup{
	colorlinks,
	urlcolor = blue
}

%\renewcommand{\baselinestretch}{1.5}

\def\supit#1{\raisebox{0.8ex}{\small\it #1}\hspace{0.05em}}

\AtBeginSection{%
	\begin{frame}
		\sectionpage
	\end{frame}
}

\newcommand{\rottext}[2]{%
	\rotatebox{90}{%
	\begin{minipage}{#1}%
		\raggedleft#2%
	\end{minipage}%
	}%
}

\usepackage{longtable}
\usepackage{tabu}


\institute{ %
École  nationale Supérieure d'Informatique (ESI, ex. INI), Algérie
}
\author[ \textbf{\footnotesize  \insertframenumber/\inserttotalframenumber} \hspace*{\fill} ESI (2019-2020)] %
{ARIES Abdelkrime}
%\titlegraphic{\includegraphics[height=1cm]{../img/esi-logo.png}%\hspace*{4.75cm}~


\date{Année unniversitaire: 2019/2020} %\today

\usetheme{Warsaw} % Antibes Boadilla Warsaw

\beamertemplatenavigationsymbolsempty

%\setbeamertemplate{headline}{}

\definecolor{lightblue}{HTML}{D0D2FF}
\definecolor{lightyellow}{HTML}{FFFFAA}
\definecolor{darkblue}{HTML}{0000BB}
\definecolor{olivegreen}{HTML}{006600}
\definecolor{violet}{HTML}{6600CC}

\newcommand{\keyword}[1]{\textcolor{red}{\bfseries\itshape #1}}
\newcommand{\expword}[1]{\textcolor{olivegreen}{#1}}
\newcommand{\optword}[1]{\textcolor{violet}{\bfseries #1}}

\makeatletter
\newcommand\mysphere{%
	\parbox[t]{10pt}{\raisebox{0.2pt}{\beamer@usesphere{item projected}{bigsphere}}}}
\makeatother

%\let\oldtabular\tabular
%\let\endoldtabular\endtabular
%\renewenvironment{tabular}{\rowcolors{2}{white}{lightblue}\oldtabular\rowcolor{blue}}{\endoldtabular}


\NoAutoSpacing %french autospacing after ":"

\title[BWEB: 03- Rédaction (Introduction)] %
{Bureautique et Web \\Chapitre 03: Rédaction d'un document numérique\\ \slshape\small  Introduction}  

\changegraphpath{../img/Bweb03-redaction/intro/}

\begin{document}

\begin{frame}
\frametitle{Rédaction d'un document numérique}
\framesubtitle{Introduction: Motivation}

\begin{itemize}
	\item La préparation d'un rapport prend du temps surtout dans l'étape d'édition 
	\item Un rapport bien structuré est plus lisible et réutilisable 
	\item Donc, il faut :
	\begin{itemize}
		\item comprendre la structure d'un rapport 
		\item maitriser un (ou plusieurs) outil de rédaction des documents
	\end{itemize}
\end{itemize}

\end{frame}

\begin{frame}
\frametitle{Rédaction d'un document numérique}
\framesubtitle{Introduction: Un peu d'humeur}

\begin{center}
	\vgraphpage{editing-humour.jpg}
\end{center}

\end{frame}

\begin{frame}
\frametitle{Rédaction d'un document numérique}
\framesubtitle{Introduction : Plan}

\begin{multicols}{2}
	%	\small
	\tableofcontents
\end{multicols}
\end{frame}

%===================================================================================
\section{Structure d'un rapport}
%===================================================================================

\begin{frame}
\frametitle{Structure d'un rapport}

\begin{itemize}
	\item pages préliminaires
	\begin{itemize}
		\item couverture
		\item remerciements
		\item résumé
		\item table des matières
		\item listes diverses (figures, tableaux, abréviations, symboles, algorithmes)
	\end{itemize}

	\item corps du texte
	\begin{itemize}
		\item introduction
		\item développement (chapitres)
		\item conclusion
	\end{itemize}

	\item pages complémentaires
	\begin{itemize}
		\item annexes et appendices
		\item glossaire
		\item index
		\item bibliographie
	\end{itemize}

\end{itemize}

\end{frame}

\subsection{Pages préliminaires}

\begin{frame}
\frametitle{Structure d'un rapport}
\framesubtitle{Pages préliminaires}

\begin{itemize}
	\item Pages qui précèdent le développement du rapport
	\item Toutes les pages sont numérotées, sauf la page de garde
	\item Les numéros de pages sont en chiffres romains (I, II, III, IV, V, etc.) 
\end{itemize}

\end{frame}


\begin{frame}
\frametitle{Structure d'un rapport: Pages préliminaires}
\framesubtitle{Couverture}

\begin{minipage}{0.60\textwidth}
	\begin{itemize}
		\item Nom de l'université 
		\item Année universitaire
		\item Sujet
		\item Nom de l'étudiant
		\item Nom de l'encadrant 
		\item Nom du laboratoire ou de l'entreprise
		\item Logos (université, entreprise, laboratoire)
	\end{itemize}
\end{minipage}
\begin{minipage}{0.38\textwidth}
	\hgraphpage[\textwidth,frame]{couverture.png}
\end{minipage}

\end{frame}

\begin{frame}
\frametitle{Structure d'un rapport: Pages préliminaires}
\framesubtitle{Remerciements}

\begin{minipage}{0.60\textwidth}
	\begin{itemize}
		\item Promoteur (Organisme d'accueil)
		\item Encadreur (L'université)
		\item Personnes ayant contribué au travail (aide financière, révision, etc.) en mentionnant la nature de l'aide.
		\item Famille pour soutien
		\item ...
		\item Elles doivent être:
		\begin{itemize}
			\item \optword{simples}: pas d'exagération
			\item \optword{professionnelles}: moins de familiarité
		\end{itemize}
	\end{itemize}
\end{minipage}
\begin{minipage}{0.38\textwidth}
	\hgraphpage[\textwidth,frame]{remerciements.png}
\end{minipage}

\end{frame}

\begin{frame}
\frametitle{Structure d'un rapport: Pages préliminaires}
\framesubtitle{Résumé}

\begin{minipage}{0.60\textwidth}
	\begin{itemize}
		\item Mini-version du rapport
		\item Il doit être:
		\begin{itemize}
			\item \optword{Court}: généralement, ne dépasse pas une page
			\item \optword{Suffisant}: il fournit l'essentiel du rapport
			\item \optword{Clair}: on peut comprendre le travail avec un minimum d'expertise
			\item \optword{Simple}: des paragraphes (sans illustrations, etc.)
		\end{itemize}
		\item Comme le résumé décrit un travail terminé, il est généralement écrit au passé
	\end{itemize}
\end{minipage}
\begin{minipage}{0.38\textwidth}
	\hgraphpage[\textwidth,frame]{resume.png}
\end{minipage}

\end{frame}

\begin{frame}
\frametitle{Structure d'un rapport: Pages préliminaires}
\framesubtitle{Table des matières}

\begin{minipage}{0.60\textwidth}
	\begin{itemize}
		\item Aperçu de la structure du rapport
		\item Liste des divisions avec leurs numéros de pages 
		\item Il ne faut pas dépasser 3 niveaux de titres
		\item Si elle est placée en fin d'ouvrage
		\begin{itemize}
			\item Il est utile de fournir un sommaire
			\item Aperçu sur les parties et les chapitres
			\item Sans numéros de pages
		\end{itemize}
	\end{itemize}
\end{minipage}
\begin{minipage}{0.38\textwidth}
	\hgraphpage[\textwidth,frame]{sommaire.png}
\end{minipage}

\end{frame}

\begin{frame}
\frametitle{Structure d'un rapport: Pages préliminaires}
\framesubtitle{Liste des illustrations (tableaux et figures)}

\begin{minipage}{0.60\textwidth}
	\begin{itemize}
		\item Une liste des tableaux et une autre pour les figures
		\item Utile pour chercher les illustrations dans le document
		\item Formée des légendes des illustrations et leurs pages
		\item La légende se forme du numéro d'apparition (soit dans le rapport entier ou dans chaque chapitre) et le titre de l'illustration.
	\end{itemize}
\end{minipage}
\begin{minipage}{0.38\textwidth}
	\hgraphpage[\textwidth,frame]{table-lst.png}
\end{minipage}

\end{frame}

\begin{frame}
\frametitle{Structure d'un rapport: Pages préliminaires}
\framesubtitle{Liste des abréviations}

\begin{minipage}{0.60\textwidth}
	\begin{itemize}
		\item Abréviation: forme réduite d'un mot
		\item Exemple: \expword{ESI}
		\item Facilite la lecture du document 
		\item Liste des termes et leurs définitions
		\item Ordre alphabétique des termes
	\end{itemize}
\end{minipage}
\begin{minipage}{0.38\textwidth}
	\hgraphpage[\textwidth,frame]{abbrv-lst.png}
\end{minipage}

\end{frame}

\subsection{Corps du texte}

\begin{frame}
\frametitle{Structure d'un document}
\framesubtitle{Corps du texte}

\begin{itemize}
	\item Se compose de: 
	\begin{itemize}
		\item Introduction
		\item Développement  
		\item Conclusion 
	\end{itemize}
	\item Les numéros de pages sont en chiffres arabes (1, 2, 3, etc.) 
	\item La numérotation se commence par 1 
\end{itemize}

\end{frame}

\begin{frame}
\frametitle{Structure d'un rapport: Corps du texte}
\framesubtitle{Introduction}

\begin{minipage}{0.60\textwidth}
	\begin{itemize}
		\item Contexte du travail:
		\begin{itemize}
			\item Présentation du sujet
			\item Intérêt du sujet
			\item Historique du sujet
			\item Relation avec la littérature ou travaux existants
		\end{itemize}
		
		\item Problématique 
		\begin{itemize}
			\item Motivation du travail
			\item Limites du sujet
		\end{itemize}
		
		\item Objectifs 
		\begin{itemize}
			\item Principaux apports du travail
		\end{itemize}
		
		\item Plan 
		\begin{itemize}
			\item Annonce et justification des parties
		\end{itemize}
		
	\end{itemize}
\end{minipage}
\begin{minipage}{0.38\textwidth}
	\hgraphpage[\textwidth,frame]{intro.png}
\end{minipage}

\end{frame}

\begin{frame}
\frametitle{Structure d'un document: Corps du texte}
\framesubtitle{Développement}

\begin{itemize}
	\item \optword{partie}: 
	\begin{itemize}
		\item regroupe des chapitres (ex. \expword{étude bibliographique}, \expword{contribution}) 
	\end{itemize}

	\item \optword{chapitre}:
	\begin{itemize}
		\item regroupe des sections (introduction, développement, conclusion)
		\item on continue la numérotation même dans une nouvelle partie
	\end{itemize}

	\item \optword{section}: 
	\begin{itemize}
		\item section, sous-section et subdivision (3 niveaux doivent être suffisants)
	\end{itemize}

	\item \optword{paragraphe}: 
	\begin{itemize}
		\item regroupe les phrases discutants la même idée
		\item ne doit pas être trop long ou trop court
	\end{itemize}

	\item \optword{phrase}:
	\begin{itemize}
		\item dans un rapport scientifique, elle doit être simple
		\item ne doit pas être longue (dans ce cas il faut la diviser)
	\end{itemize}
\end{itemize}

\end{frame}

\begin{frame}
\frametitle{Structure d'un document: Corps du texte}
\framesubtitle{Développement: Pied et entête des pages}

\begin{minipage}{0.60\textwidth}
	\begin{itemize}
		\item des zones dans les marges haute et basse de chaque page 
		%	\item différents éléments utiles : le numéro de page, la date, un logo de société, le titre du document, le nom du fichier ou le nom de l'auteur.
		\item généralement, la première page d'un chapitre ne contient pas ces deux zones 
		\item \optword{Pied de page}: 
		\begin{itemize}
			\item utilisé généralement pour inclure les numéros de pages
		\end{itemize}
		
		\item \optword{entête de page}: 
		\begin{itemize}
			\item utilisée généralement pour inclure des titres
			\item pages impaires : titre du chapitre
			\item pages paires : titre de la section courante
		\end{itemize}
	\end{itemize}
\end{minipage}
\begin{minipage}{0.38\textwidth}
	\hgraphpage[\textwidth,frame]{dev.png}
\end{minipage}

\end{frame}


\begin{frame}
\frametitle{Structure d'un document: Corps du texte}
\framesubtitle{Développement: Notes de bas de page}

\begin{minipage}{0.60\textwidth}
	\begin{itemize}
		\item deux utilités :
		
		\begin{itemize}
			\item C'est une forme de citation bibliographique
			\item C'est un moyen de fournir plus d'informations ou des précisions
		\end{itemize}
	
		\item Dans le deuxième cas : 
		
		\begin{itemize}
			\item Définition d'une expression
			\item Lien vers un site web
		\end{itemize}
	\end{itemize}
\end{minipage}
\begin{minipage}{0.38\textwidth}
	\hgraphpage[\textwidth,frame]{dev.png}
\end{minipage}

\end{frame}

\begin{frame}
\frametitle{Structure d'un document: Corps du texte}
\framesubtitle{Développement : Illustrations}

\begin{minipage}{0.60\textwidth}
	\begin{itemize}
		\item les figures et les tableaux
		\item généralement, centrées 
		\item doivent avoir des légendes 
		\item une légende est le numéro de l'apparition dans le document ou dans le chapitre suivi par le titre
		\item les légendes sont en bas de l'illustration (il y a des ouvrages où les légendes sont en haut des tableaux)
		
	\end{itemize}
\end{minipage}
\begin{minipage}{0.38\textwidth}
	\hgraphpage[\textwidth,frame]{dev.png}
\end{minipage}

\end{frame}

\begin{frame}
\frametitle{Structure d'un document : Corps du texte}
\framesubtitle{Conclusion}

\begin{minipage}{0.60\textwidth}
	\begin{itemize}
		\item Un bref retour sur la problématique, les objectifs et les hypothèses de départ
		\item Une présentation des différentes parties du rapport
		\item Un résumé des solutions proposées avec analyse des résultats obtenues
		\item Un état des limites de son travail 
		\item Des perspectives: suggérer des améliorations de son travail
	\end{itemize}
\end{minipage}
\begin{minipage}{0.38\textwidth}
	\hgraphpage[\textwidth,frame]{conclusion.png}
\end{minipage}

\end{frame}

\subsection{Fin du rapport}

\begin{frame}
\frametitle{Structure d'un document}
\framesubtitle{Pages complémentaires}

\begin{itemize}
	\item Pages qui viennent après le développement du rapport
	\item On continue la numérotation des pages comme le développement
\end{itemize}

\end{frame}

\begin{frame}
\frametitle{Structure d'un rapport : Pages complémentaires}
\framesubtitle{Annexe a Appendice}

\begin{minipage}{0.60\textwidth}
	
	\begin{itemize}
		\item Comme un chapitre numéroté par (A1, A2, etc.) ou (A, B, etc.)
%		\item Cité dans le document. Par exemple: \expword{(pour plus d'informations, voir l'annexe A)}
		\item \optword{Une annexe} : un complément d'information (ex. : dessins, plans, schémas complexes, calculs très techniques, etc.) jugé nécessaire à la compréhension du rapport.
		\item \optword{Une appendice} : un supplément d'information jugé non essentiel à la compréhension du rapport mais qui possède quand même un certain intérêt.
	\end{itemize}
\end{minipage}
\begin{minipage}{0.38\textwidth}
	\hgraphpage[\textwidth,frame]{annex.png}
\end{minipage}

\end{frame}

\begin{frame}
\frametitle{Structure d'un rapport : Pages complémentaires}
\framesubtitle{Glossaire}

\begin{minipage}{0.60\textwidth}
	\begin{itemize}
		\item aider un lecteur à comprendre des termes dans le rapport
		\item un dictionnaire des termes utilisés dans le rapport 
		\item chaque terme est accompagné par sa définition 
		\item ordre alphabétique des termes 
	\end{itemize}
\end{minipage}
\begin{minipage}{0.38\textwidth}
	\hgraphpage[\textwidth,frame]{glossaire.png}
\end{minipage}

\end{frame}

\begin{frame}
\frametitle{Structure d'un rapport : Pages complémentaires}
\framesubtitle{Index}

\begin{minipage}{0.60\textwidth}
	\begin{itemize}
		\item aider le lecteur à trouver des mots ou des expressions importantes dans le document
		\item les mots communs, les noms propres, les mots clés, etc. 
		\item par exemple: les mots clés d'un langage de programmation 
		\item si on veut indexer un mot qui n'existe pas dans le document mais qui a un mot relative, on utilise une référence croisée (renvoi) 
		\item exemple \expword{writeln ...... \itshape voir write}
	\end{itemize}
\end{minipage}
\begin{minipage}{0.38\textwidth}
	\hgraphpage[\textwidth,frame]{index.png}
\end{minipage}

\end{frame}


\begin{frame}
\frametitle{Structure d'un rapport: Pages complémentaires}
\framesubtitle{Bibliographie}

\begin{minipage}{0.60\textwidth}
	\begin{itemize}
		\item permettre au lecteur de se documenter sur les travaux antérieurs
		\item liste structurée de références d'ouvrages ou d'autres documents cités dans le rapport
		\item contient des informations comme: les noms des auteurs, le titre de l'ouvrage, la date de publication, etc.
	\end{itemize}
\end{minipage}
\begin{minipage}{0.38\textwidth}
	\hgraphpage[\textwidth,frame]{bibliographie.png}
\end{minipage}

\end{frame}

\begin{frame}
\frametitle{Structure d'un rapport}
\framesubtitle{Un peu d'humeur}

\begin{center}
	\vgraphpage{structure-humour.png}
\end{center}

\end{frame}


%===================================================================================
\section{Outils de rédaction}
%===================================================================================

\begin{frame}
\frametitle{Outils de rédaction}

Pour rédiger un document riche (pas seulement du texte), il existe plusieurs outils:
%
\begin{itemize}
	\item Logiciels de traitement de texte : en général ils appartiennent à des suites bureautiques (traitement du texte, tableurs, présentateurs, etc.). Ils sont des outils WYSIWYG (What You See Is What You Get).
	\item \LaTeX : c'est comme si programmer son document. On écrit des commandes \LaTeX dans un fichier texte et on compile ce dernier vers un format de documents (en général PDF). 
	\item Les outils de rédaction en ligne: ce sont des logiciels comme services (SaaS). Il existe plusieurs outils gratuits avec un espace de stockage limité qu'on puisse étendre en payant.
\end{itemize}
\end{frame}

\subsection{Logiciels de traitement de texte}

\begin{frame}
\frametitle{Outils de rédaction}
\framesubtitle{Logiciels de traitement de texte}


\def\arraystretch{.5}

\begin{tabular}{p{.3\textwidth}cp{.5\textwidth}}%p{.3\textwidth}
	
	\hline
	
	\vgraphpage[.8cm, valign=t]{freeoffice-logo.png} &
	\vgraphpage[.8cm, valign=t]{freeoffice-textmaker-logo.png} &
	FreeOffice TextMaker  
	
	\url{https://www.freeoffice.com/}  \\
	\hline
	
	\vgraphpage[.8cm, valign=t]{libreoffice-logo.png} &
	\vgraphpage[.8cm, valign=t]{libreoffice-writer-logo.png} & 
	LibreOffice Writer 
	
	\url{https://www.libreoffice.org/}  \\
	\hline
	
	\vgraphpage[.8cm, valign=t]{msoffice-logo.png} &
	\vgraphpage[.8cm, valign=t]{msoffice-word-logo.png} & 
	Microsoft office Word 
	
	\url{https://www.office.com/}  \\
	\hline
	
	\vgraphpage[.7cm, valign=t]{onlyoffice-logo.png} & &
%	\graphintable{.8cm}{onlyoffice-documents-logo.png} & 
	OnlyOffice Documents 
	
	\url{https://www.onlyoffice.com/}  \\
	\hline
	
	\vgraphpage[.8cm, valign=t]{wps-logo.png} & 
	\vgraphpage[.8cm, valign=t]{wps-writer-logo.png} & 
	WPS Writer
	
	\url{https://www.wps.com/}  \\
	\hline
	
	
\end{tabular}
 
\end{frame}

\subsection{\LaTeX}

\begin{frame}
\frametitle{Outils de rédaction}
\framesubtitle{\LaTeX}

\def\arraystretch{.5}

Des distributions \LaTeX :

\begin{tabular}{p{.1\textwidth}cp{.7\textwidth}}%p{.3\textwidth}
	
	\hline
	
	\vgraphpage[.8cm, valign=t]{texlive-logo.png} &
	&
	TexLive  
	
	\url{https://www.tug.org/texlive/}
	
	Ubuntu: \expword{apt install texlive-full}\\
	\hline
	
	\vgraphpage[.8cm, valign=t]{miktex-logo.png} &
	& 
	MiKTex 
	
	\url{https://miktex.org/download}  \\
	\hline
	
\end{tabular}

\vspace{\fill}

Des éditeurs \LaTeX :

\begin{tabular}{p{.1\textwidth}cp{.7\textwidth}}%p{.3\textwidth}
	
	\hline
	
	\vgraphpage[.8cm, valign=t]{texstudio-logo.png} &
	&
	TexStudio  
	
	\url{http://www.texstudio.org/}  \\
	\hline
	
	\vgraphpage[.8cm, valign=t]{lyx-logo.png} &
	& 
	LyX (WYSIWYM)
	
	\url{https://www.lyx.org/}  \\
	\hline
	
\end{tabular}

\end{frame}


\subsection{Rédaction en ligne}


\begin{frame}
\frametitle{Outils de rédaction}
\framesubtitle{Rédaction en ligne}

\def\arraystretch{.5}

\begin{tabular}{p{.1\textwidth}cp{.7\textwidth}}%p{.3\textwidth}
	
	\hline
	
	\vgraphpage[.8cm, valign=t]{google-docs-logo.png} &
	&
	Google Docs  
	
	\url{http://docs.google.com/document/}  \\
	\hline
	
	\vgraphpage[.8cm, valign=t]{msoffice-word-logo.png} &
	&
	Microsoft Office Word 
	
	\url{https://www.office.com/launch/word}  \\
	\hline
	
	\vgraphpage[.8cm, valign=t]{icloud-pages-logo.png} &
	&
	iCloud Pages 
	
	\url{https://www.icloud.com/pages/}  \\
	\hline
	
	\vgraphpage[.8cm, valign=t]{overleaf-logo.png} &
	&
	Overleaf (\LaTeX) 
	
	\url{https://www.overleaf.com/}  \\
	\hline
	
\end{tabular}

\end{frame}

\begin{frame}
\frametitle{Outils de rédaction: Rédaction en ligne}
\framesubtitle{Google Docs Document}

\begin{center}
	\vgraphpage{google-docs-preview.png}
\end{center}

\end{frame}

\begin{frame}
\frametitle{Outils de rédaction: Rédaction en ligne}
\framesubtitle{Microsoft Office Word}

\begin{center}
	\vgraphpage{msoffice-word-preview.png}
\end{center}

\end{frame}

\begin{frame}
\frametitle{Outils de rédaction: Rédaction en ligne}
\framesubtitle{iCloud Pages}

\begin{center}
	\vgraphpage{icloud-pages-preview.png}
\end{center}

\end{frame}

\begin{frame}
\frametitle{Outils de rédaction: Rédaction en ligne}
\framesubtitle{Overleaf}

\begin{center}
	\vgraphpage{overleaf-preview.png}
\end{center}

\end{frame}

\subsection{Raccourcis clavier}

\begin{frame}
\frametitle{Outils de rédaction}
\framesubtitle{Raccourcis clavier}

\begin{itemize}
	\item \optword{Ctrl + O}: Ouvrir un document
	\item \optword{Ctrl + N}: Créer un nouveau
	\item \optword{Ctrl + S}: Enregistrer un document
	\item \optword{Ctrl + P}: Imprimer un document 
	\item \optword{Ctrl + W}: Fermer un document
	\item \optword{F1}: Aidez-moi
\end{itemize}

\end{frame}

\begin{frame}
\frametitle{Outils de rédaction: Raccourcis clavier}
\framesubtitle{Édition}

\begin{itemize}
	\item \optword{Ctrl + A}: Sélectionner tout
	\item \optword{Ctrl + C}: Copier
	\item \optword{Suppr (Del)}: Supprimer
	\item \optword{Ctrl + X}: Couper (copier et supprimer)
	\item \optword{Ctrl + V}: Coller
	\item \optword{Ctrl + Z}: Annuler
	\item \optword{Ctrl + Y}: Répéter (l'inverse de Annuler)
\end{itemize}

\end{frame}

\begin{frame}
\frametitle{Outils de rédaction: Raccourcis clavier}
\framesubtitle{Mise en forme}

\begin{itemize}
	\item \optword{Ctrl + G}: Gras
	\item \optword{Ctrl + I}: Italique
	\item \optword{Ctrl + U}: Souligner
	\item \optword{Ctrl + L}: Alignez à gauche
	\item \optword{Ctrl + C}: Centrer
	\item \optword{Ctrl + R}: Aligner à droite
	\item \optword{Ctrl + J}: Justifier 
\end{itemize}

\end{frame}

\begin{frame}
\frametitle{Outils de rédaction: Raccourcis clavier}
\framesubtitle{Navigation}

\begin{itemize}
	\item \optword{Page suivante}: Monter un écran
	\item \optword{Page précédente}: Descendre un écran
	\item \optword{Accueil}: Début de ligne
	\item \optword{Fin}: Fin de ligne
	\item \optword{Ctrl + Accueil}: Début de document
	\item \optword{Ctrl + Fin}: Fin du document 
\end{itemize}

\end{frame}

\begin{frame}
\frametitle{Outils de rédaction}
\framesubtitle{Un peu d'humeur}

\begin{center}
	\vgraphpage{tools-humour.png}
\end{center}

\end{frame}

\insertbibliography{Bweb03}{*}


\end{document}

