% !TEX TS-program = pdflatex
% !TeX program = pdflatex
% !TEX encoding = UTF-8
% !TEX spellcheck = fr

\documentclass{beamer}


%\usepackage{fullpage}
%\usepackage[left=2.8cm,right=2.2cm,top=2 cm,bottom=2 cm]{geometry}
\setbeamersize{text margin left=10pt,text margin right=10pt}
\usepackage{amsmath,amssymb} 
\usepackage[T1]{fontenc}
\usepackage[utf8]{inputenc}
\usepackage[english,french]{babel}
\usepackage{txfonts}
\usepackage[]{graphicx}
\usepackage{multirow}
\usepackage{hyperref}
\usepackage{colortbl}
\usepackage{listings}
\usepackage{wrapfig}
\usepackage{multicol}

\hypersetup{
	colorlinks,
	urlcolor = blue
}

%\renewcommand{\baselinestretch}{1.5}

\def\supit#1{\raisebox{0.8ex}{\small\it #1}\hspace{0.05em}}

\AtBeginSection{%
	\begin{frame}
		\sectionpage
	\end{frame}
}

\newcommand{\rottext}[2]{%
	\rotatebox{90}{%
	\begin{minipage}{#1}%
		\raggedleft#2%
	\end{minipage}%
	}%
}

\usepackage{longtable}
\usepackage{tabu}


\institute{ %
École  nationale Supérieure d'Informatique (ESI, ex. INI), Algérie
}
\author[ \textbf{\footnotesize  \insertframenumber/\inserttotalframenumber} \hspace*{\fill} ESI (2019-2020)] %
{ARIES Abdelkrime}
%\titlegraphic{\includegraphics[height=1cm]{../img/esi-logo.png}%\hspace*{4.75cm}~


\date{Année unniversitaire: 2019/2020} %\today

\usetheme{Warsaw} % Antibes Boadilla Warsaw

\beamertemplatenavigationsymbolsempty

%\setbeamertemplate{headline}{}

\definecolor{lightblue}{HTML}{D0D2FF}
\definecolor{lightyellow}{HTML}{FFFFAA}
\definecolor{darkblue}{HTML}{0000BB}
\definecolor{olivegreen}{HTML}{006600}
\definecolor{violet}{HTML}{6600CC}

\newcommand{\keyword}[1]{\textcolor{red}{\bfseries\itshape #1}}
\newcommand{\expword}[1]{\textcolor{olivegreen}{#1}}
\newcommand{\optword}[1]{\textcolor{violet}{\bfseries #1}}

\makeatletter
\newcommand\mysphere{%
	\parbox[t]{10pt}{\raisebox{0.2pt}{\beamer@usesphere{item projected}{bigsphere}}}}
\makeatother

%\let\oldtabular\tabular
%\let\endoldtabular\endtabular
%\renewenvironment{tabular}{\rowcolors{2}{white}{lightblue}\oldtabular\rowcolor{blue}}{\endoldtabular}


\NoAutoSpacing %french autospacing after ":"

\title[BWEB: Introduction] %
{Bureautique et Web \\Introduction} 

\begin{document}

\begin{frame}
\frametitle{Pourquoi ce module?}

Au long de leur cursus, les étudiants doivent:
\begin{itemize}
	\item préparer et présenter pleinement de rapports.
	\item manipuler des données tabulaires.
	\item collaborer sur des projets en ligne
	\item savoir comment se documenter
	\item avoir une idée sur la programmation web
\end{itemize}

\end{frame}


\begin{frame}
	\frametitle{Objectifs}
	En passant ce module, les étudiants doivent pouvoir:
	\begin{itemize}
		\item utiliser les outils de la bureautique (Word, Powerpoint, Excel, etc.)
		\item gérer leurs messageries.
		\item collaborer en utilisant des services cloud (Drive, etc.)
		\item chercher dans le web et gérer ses documents
		\item créer un site web statique
	\end{itemize}

\end{frame}


\begin{frame}
\frametitle{Présentation du module}

\begin{itemize}
	\item Un séance de 2h par semaine
	\item Cours/TD/TP dans la même séance  
	\item 13 séances au total
\end{itemize}

Notation:
\begin{itemize}
	\item Crédit: 1
	\item un contrôle continu (CC), un contrôle intermédiaire (CI) et un contrôle final (CF)
	\item La note du module (N) sera: $ N = 0.2 CC + 0.4 CI + 0.4 CF$
\end{itemize}

\end{frame}

\begin{frame}
\frametitle{Contenu du module}

\begin{enumerate}
	\item Environnement de travail (1 semaine)
	\item Recherche d'information et messagerie (1 semaine)
	\item Rédaction d'un document numérique (3 semaines)
%	\item Documentation (1 semaine)
	\item Diffusion et collecte d'informations (1 semaine)
	\item Présentations (1 semaine)
	\item Tableurs (2 semaines) 
	\item Développement web (4 semaines)
\end{enumerate}

\end{frame}


\end{document}

