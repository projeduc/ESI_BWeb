\documentclass{beamer}
%\usepackage{fullpage}
%\usepackage[left=2.8cm,right=2.2cm,top=2 cm,bottom=2 cm]{geometry}
\setbeamersize{text margin left=10pt,text margin right=10pt}
\usepackage{amsmath,amssymb} 
\usepackage[T1]{fontenc}
\usepackage[utf8]{inputenc}
\usepackage[english,french]{babel}
\usepackage{txfonts}
\usepackage[]{graphicx}
\usepackage{multirow}
\usepackage{hyperref}

\hypersetup{
	colorlinks,
	urlcolor = blue
}

%\renewcommand{\baselinestretch}{1.5}

\def\supit#1{\raisebox{0.8ex}{\small\it #1}\hspace{0.05em}}

\title[BWEB: Introduction] %
{Bureautique et Web \\Introduction} 
\institute{ %
École  nationale Supérieure d'Informatique (ESI, ex. INI), Algérie
}
\author[ \textbf{\footnotesize  \insertframenumber/\inserttotalframenumber} \hspace*{1cm} ESI - ARIES Abdelkrime (2019-2020)] %
{ARIES Abdelkrime}
%\titlegraphic{\includegraphics[height=1cm]{../img/esi-logo.png}%\hspace*{4.75cm}~


\date{Année unniversitaire: 2019/2020} %\today

\usetheme{Warsaw} % Antibes Boadilla Warsaw

\beamertemplatenavigationsymbolsempty

%\setbeamertemplate{headline}{}


\begin{document}

\begin{frame}[plain]
	\maketitle
\end{frame}

\begin{frame}
\frametitle{Pourquoi ce module?}

Au long de leur cursus, les étudiants doivent:
\begin{itemize}
	\item préparer et présenter pleinement de rapports.
	\item manipuler des données tabulaires.
	\item collaborer sur des projets en ligne
	\item savoir comment se documenter
	\item avoir une idée sur la programmation web
\end{itemize}

\end{frame}


\begin{frame}
	\frametitle{Objectifs}
	En passant ce module, les étudiants doivent pouvoir:
	\begin{itemize}
		\item utiliser les outils de la bureautique (Word, Powerpoint, Excel, etc.)
		\item gérer sa messagerie.
		\item collaborer en utilisant des services cloud (Drive, etc.)
		\item chercher dans le web et gérer ses documents
		\item créer un site web statique
	\end{itemize}

\end{frame}


\begin{frame}
\frametitle{Présentation du module}

2h/semaine 

30h total 

Cours/TD/TP inclus 

Crédit: 1

\end{frame}

%\subsection{Bibliography}
\frame[allowframebreaks]%
{\frametitle{Bibliography}
\tiny
\bibliography{biblio}
\bibliographystyle{apalike} 
}


\end{document}

