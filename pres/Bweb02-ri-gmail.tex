% !TEX TS-program = pdflatex
% !TeX program = pdflatex
% !TEX encoding = UTF-8
% !TEX spellcheck = fr

\documentclass{beamer}
%\usepackage{fullpage}
%\usepackage[left=2.8cm,right=2.2cm,top=2 cm,bottom=2 cm]{geometry}
\setbeamersize{text margin left=10pt,text margin right=10pt}
\usepackage{amsmath,amssymb} 
\usepackage[T1]{fontenc}
\usepackage[utf8]{inputenc}
\usepackage[english,french]{babel}
\usepackage{txfonts}
\usepackage[]{graphicx}
\usepackage{multirow}
\usepackage{hyperref}
\usepackage{listings}


\hypersetup{
	colorlinks,
	urlcolor = blue
}

%\renewcommand{\baselinestretch}{1.5}

\def\supit#1{\raisebox{0.8ex}{\small\it #1}\hspace{0.05em}}

\AtBeginSection{%
	\begin{frame}
		\sectionpage
	\end{frame}
}

\newcommand{\rottext}[2]{%
	\rotatebox{90}{%
	\begin{minipage}{#1}%
		\raggedleft#2%
	\end{minipage}%
	}%
}

\usepackage{longtable}
\usepackage{tabu}


\title[BWEB: 01- RI et Gmail] %
{Bureautique et Web \\Chapitre 02: Recherche d'information et messagerie} 
\institute{ %
École  nationale Supérieure d'Informatique (ESI, ex. INI), Algérie
}
\author[ \textbf{\footnotesize  \insertframenumber/\inserttotalframenumber} \hspace*{\fill} ESI (2019-2020)] %
{ARIES Abdelkrime}
%\titlegraphic{\includegraphics[height=1cm]{../img/esi-logo.png}%\hspace*{4.75cm}~


\date{Année unniversitaire: 2019/2020} %\today

\usetheme{Warsaw} % Antibes Boadilla Warsaw

\beamertemplatenavigationsymbolsempty

%\setbeamertemplate{headline}{}


\begin{document}

\begin{frame}[plain]
	\maketitle
\end{frame}

\section{Recherche d'information}

\begin{frame}
\frametitle{Recherche d'information}

\end{frame}

\begin{frame}
\frametitle{Recherche d'information}
\frametitle{Moteur de recherche}

\end{frame}

\begin{frame}
\frametitle{Recherche d'information}
\frametitle{Moteur de recherche}

\end{frame}


\begin{frame}
\frametitle{Recherche d'information}
\framesubtitle{Quelque moteurs de recherche}

\begin{itemize}
	\item Google
	\item Bing 
	\item Yahoo
	\item DuckDuckGo
\end{itemize}

\end{frame}


\begin{frame}
\frametitle{Recherche d'information}
\framesubtitle{Exploration: points négatifs}

Les explorateurs (spiders) ont quelques points négatifs; ils:
\begin{itemize}
	\item peuvent télécharger des informations privées ou classifiées
	\item ne prennent pas en considération les termes d'utilisation des sites web. 
	Par exemple, un site web qui donne droit de lecture sans recopier les informations. 
	\item surchargent les serveurs par des requêtes, ce qui entraine leur plantage ou les rend lourds.
\end{itemize}

\end{frame}

\begin{frame}[fragile]
\frametitle{Recherche d'information}
\framesubtitle{Exploration: Bloquer l'accès à votre contenu}

% https://support.google.com/webmasters/answer/6062608?hl=fr

En créant un fichier texte "\textbf{robot.txt}" dans la racine de votre site web, on peut bloquer des agents (robots) d'exploration.

%\begin{block}{Syntaxe simple de "robot.txt"}
%	\scriptsize\bfseries
%	\begin{lstlisting}
%	# commentaire
%	User-agent: [nom d'agent]
%	Disallow: [URL de ce qu'on veut bloquer] 
%	
%	# il faut laisser une ligne vide pour un autre groupe
%	\end{lstlisting}
%\end{block}

\begin{exampleblock}{exemple d'un fichier "robot.txt"}
	\scriptsize\bfseries
	\begin{lstlisting}
	# empecher Google a explorer un dossier et un fichier
	User-agent: Googlebot
	Disallow: /spam/ 
	Disallow: /utilisateurs.html
	
	# empecher Google a explorer les fichiers .doc 
	User-agent: Bingbot
	Disallow: /*.doc$ 
	
	# empecher tous les agents a explorer un dossier
	User-agent: *
	Disallow: /prive/
	\end{lstlisting}
\end{exampleblock}


\end{frame}

\begin{frame}[fragile]
\frametitle{Recherche d'information}
\frametitle{Exploration: Bloquer l'accès à votre contenu}

% https://developers.google.com/search/reference/robots_meta_tag
En l'indiquant dans le code HTML de nos pages

\begin{exampleblock}{exemple d'un fichier HTML qui bloque tous les robots}
	\scriptsize\bfseries
	\begin{lstlisting}
	<!DOCTYPE html>
	<html>
	    <head>
	        <meta name="robots" content="noindex" />
	    </head>
	    <body>
	    </body>
	</html>
	\end{lstlisting}
\end{exampleblock}

\begin{exampleblock}{exemple d'un meta qui bloque Google}
	\scriptsize\bfseries
	\begin{lstlisting}
	<meta name="googlebot" content="noindex" />
	\end{lstlisting}
\end{exampleblock}

\end{frame}


\section{Rechercher avec Google}

\begin{frame}
\frametitle{Rechercher avec Google}

\end{frame}

\section{Messagerie électronique}

\begin{frame}
\frametitle{Messagerie électronique}

\end{frame}

\begin{frame}
\frametitle{Messagerie électronique}
\framesubtitle{Caractéristique d'un moteur de recherche}

\end{frame}

\begin{frame}
\frametitle{Messagerie électronique}
\framesubtitle{Aspect légal en Algérie}


\begin{block}{Article 323 du code civil}
	L'écrit sous forme électronique est admis en tant que preuve au même titre que l'écrit sur support papier, à la condition que puisse être dûment identifiée la personne dont il émane et qu’il soit établi et conservé dans des conditions de nature à en garantir l'intégrité.
\end{block}

\end{frame}

\begin{frame}
\frametitle{Messagerie électronique}
\framesubtitle{Accès à la messagerie (Navigateur)}

\end{frame}

\begin{frame}
\frametitle{Messagerie électronique}
\framesubtitle{Accès à la messagerie (Client de messagerie)}

\end{frame}

\begin{frame}
\frametitle{Messagerie électronique}
\framesubtitle{Accès à la messagerie (comparaison)}

\begin{tabular}{p{.08\textwidth}p{.41\textwidth}p{.41\textwidth}}
	\hline\hline
	& Navigateur web & Client de messagerie \\
	\hline\hline
	
	Gain &
	+  
	
	+ 
	
	& 
	+ 
	
	+ 
	\\
	
	\hline
	Perte &
	- 
	
	- 
	&
	- 
	
	- 
	\\
	\hline\hline
\end{tabular}

\end{frame}



\section{Messagerie Gmail}

\begin{frame}
\frametitle{Messagerie Gmail}

\end{frame}



%\subsection{Bibliography}
%\frame[allowframebreaks]%
%{\frametitle{Bibliography}
%\tiny
%\bibliography{biblio}
%\bibliographystyle{apalike} 
%}


\end{document}

