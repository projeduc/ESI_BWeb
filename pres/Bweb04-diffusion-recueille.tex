% !TEX TS-program = pdflatex
% !TeX program = pdflatex
% !TEX encoding = UTF-8
% !TEX spellcheck = fr

\documentclass[xcolor=table]{beamer}


%\usepackage{fullpage}
%\usepackage[left=2.8cm,right=2.2cm,top=2 cm,bottom=2 cm]{geometry}
\setbeamersize{text margin left=10pt,text margin right=10pt}
\usepackage{amsmath,amssymb} 
\usepackage[T1]{fontenc}
\usepackage[utf8]{inputenc}
\usepackage[english,french]{babel}
\usepackage{txfonts}
\usepackage[]{graphicx}
\usepackage{multirow}
\usepackage{hyperref}
\usepackage{colortbl}
\usepackage{listings}
\usepackage{wrapfig}
\usepackage{multicol}

\hypersetup{
	colorlinks,
	urlcolor = blue
}

%\renewcommand{\baselinestretch}{1.5}

\def\supit#1{\raisebox{0.8ex}{\small\it #1}\hspace{0.05em}}

\AtBeginSection{%
	\begin{frame}
		\sectionpage
	\end{frame}
}

\newcommand{\rottext}[2]{%
	\rotatebox{90}{%
	\begin{minipage}{#1}%
		\raggedleft#2%
	\end{minipage}%
	}%
}

\usepackage{longtable}
\usepackage{tabu}


\institute{ %
École  nationale Supérieure d'Informatique (ESI, ex. INI), Algérie
}
\author[ \textbf{\footnotesize  \insertframenumber/\inserttotalframenumber} \hspace*{\fill} ESI (2019-2020)] %
{ARIES Abdelkrime}
%\titlegraphic{\includegraphics[height=1cm]{../img/esi-logo.png}%\hspace*{4.75cm}~


\date{Année unniversitaire: 2019/2020} %\today

\usetheme{Warsaw} % Antibes Boadilla Warsaw

\beamertemplatenavigationsymbolsempty

%\setbeamertemplate{headline}{}

\definecolor{lightblue}{HTML}{D0D2FF}
\definecolor{lightyellow}{HTML}{FFFFAA}
\definecolor{darkblue}{HTML}{0000BB}
\definecolor{olivegreen}{HTML}{006600}
\definecolor{violet}{HTML}{6600CC}

\newcommand{\keyword}[1]{\textcolor{red}{\bfseries\itshape #1}}
\newcommand{\expword}[1]{\textcolor{olivegreen}{#1}}
\newcommand{\optword}[1]{\textcolor{violet}{\bfseries #1}}

\makeatletter
\newcommand\mysphere{%
	\parbox[t]{10pt}{\raisebox{0.2pt}{\beamer@usesphere{item projected}{bigsphere}}}}
\makeatother

%\let\oldtabular\tabular
%\let\endoldtabular\endtabular
%\renewenvironment{tabular}{\rowcolors{2}{white}{lightblue}\oldtabular\rowcolor{blue}}{\endoldtabular}


\NoAutoSpacing %french autospacing after ":"

\title[BWEB: 03- Diffusion et recueille] %
{Bureautique et Web \\Chapitre 04: Diffusion et recueille des informations}  

\changegraphpath{../img/Bweb04-diffusion-recueille/}

\begin{document}

\begin{frame}
\frametitle{Diffusion et recueille des informations}
\framesubtitle{Introduction: Motivation}


\end{frame}

%===================================================================================
\section{Publipostage}
%===================================================================================

\begin{frame}
\frametitle{Publipostage}

\end{frame}

\subsection{Création d'une source de données}

\begin{frame}
\frametitle{Publipostage}
\framesubtitle{Création d'une source de données}

\end{frame}

\subsection{Création du masque de publipostage}

\begin{frame}
\frametitle{Publipostage}
\framesubtitle{Création du masque de publipostage}

\end{frame}

\subsection{Fusion}

\begin{frame}
\frametitle{Publipostage}
\framesubtitle{Fusion}

\end{frame}

\begin{frame}
\frametitle{Publipostage}
\framesubtitle{Un peu d'humeur}
\begin{center}
	\vgraphpage{mailing-humour.jpg}
\end{center}
\end{frame}


%===================================================================================
\section{Formulaires}
%===================================================================================

\begin{frame}
\frametitle{Formulaires}

\begin{minipage}{0.50\textwidth}
	D'après Larousse, un formulaire est 
	{
	\setlength{\textwidth}{.9\textwidth}%
	\begin{definition}
		Imprimé sur lequel figure une série de questions administratives auxquelles l'intéressé doit répondre ; questionnaire.
	\end{definition}
    }

	Un formulaire électronique contient:
	\begin{itemize}
		\item du texte explicatif et des étiquettes
		\item des champs à saisir 
	\end{itemize}
\end{minipage}
\begin{minipage}{0.49\textwidth}
	\hgraphpage{forms.png}
\end{minipage}

\end{frame}

\begin{frame}
\frametitle{Formulaires}

\begin{itemize}
	\item supprimer les informations non nécessaires. Par exemple, le champs \expword{age} si le champs \expword{date de naissance} existe
	\item toujours utiliser des étiquettes pour indiquer le sens des champs
	\item l'insertion des étiquettes au dessus des champs améliore la lisibilité
	\item éviter l'insertion d'une question à coté de l'autre
	\item regrouper les champs en sections 
	\item les tailles des champs doivent refléter la taille de leurs contenus. Par exemple, le champs \expword{code postal} doit être moins long que le champs \expword{adresse}
\end{itemize}

\end{frame}

\subsection{Les champs}

%\begin{frame}
%\frametitle{Formulaires}
%\framesubtitle{Les champs}
%
%\begin{itemize}
%	\item champs de texte 
%	\item zone de texte
%	\item bouton radio
%	\item case à couché 
%	\item liste déroulante  
%	\item sélecteur de date 
%\end{itemize}
%
%\end{frame}

\begin{frame}
\frametitle{Formulaires: Les champs}
\framesubtitle{champs texte}

\begin{minipage}{0.69\textwidth}
	\begin{itemize}
		\item pour saisir du texte
		\item sur une seule ligne 
		\item exemple: \expword{nom}, \expword{prénom}, \expword{adresse}, etc.
	\end{itemize}
\end{minipage}
\begin{minipage}{0.3\textwidth}
	\hgraphpage{input.png}
\end{minipage}

\end{frame}

\begin{frame}
\frametitle{Formulaires: Les champs}
\framesubtitle{zone de texte}

\begin{minipage}{0.50\textwidth}
	\begin{itemize}
		\item pour saisir du texte
		\item sur plusieurs lignes
		\item exemple: \expword{message}, \expword{commentaire}, etc.
	\end{itemize}
\end{minipage}
\begin{minipage}{0.49\textwidth}
	\hgraphpage{zonetexte.png}
\end{minipage}

\end{frame}


\begin{frame}
\frametitle{Formulaires: Les champs}
\framesubtitle{Bouton radio}

\begin{minipage}{0.59\textwidth}
	\begin{itemize}
		\item pour choisir une seule option parmi plusieurs
%		\item sur plusieurs lignes
%		\item exemple: \expword{message}, \expword{commentaire}, etc.
	\end{itemize}
\end{minipage}
\begin{minipage}{0.40\textwidth}
	\hgraphpage{radiobutton.png}
\end{minipage}

\begin{itemize}
	\item les boutons qui ont le même sens doivent avoir le même nom ou le même groupe
	\item de préférence, ils sont alignés verticalement 
	\item on les utilise : 
	\begin{itemize}
		\item lorsqu'il y a moins de 5 options 
		\item quand on veut comparer les options. Par exemple, \expword{la liste des prix}
		\item quand la visibilité et la rapidité de réponse sont prioritaires
	\end{itemize}
\end{itemize}

\end{frame}

\begin{frame}
\frametitle{Formulaires: Les champs}
\framesubtitle{Case à couché}

\begin{minipage}{0.59\textwidth}
	\begin{itemize}
		\item pour choisir plusieurs options
	\end{itemize}
\end{minipage}
\begin{minipage}{0.40\textwidth}
	\hgraphpage{checkbox.png}
\end{minipage}

\begin{itemize}
	\item de préférence, elles sont alignées verticalement 
	\item on les utilise : 
	\begin{itemize}
		\item lorsqu'il y a moins de 5 options 
		\item quand la visibilité et la rapidité de réponse sont prioritaires
		\item lorsqu'on a une question avec réponse booléenne (oui/non)
	\end{itemize}
\end{itemize}

\end{frame}


\begin{frame}
\frametitle{Formulaires: Les champs}
\framesubtitle{Liste déroulante}

\begin{minipage}{0.69\textwidth}
	\begin{itemize}
		\item pour choisir une seule option parmi plusieurs
		\item on les utilise : 
		\begin{itemize}
			\item lorsqu'il y a plus de 5 options 
			\item quand la première option est celle par défaut
		\end{itemize}
		\item des fois, la liste déroulante avec plusieurs options est trop gênante. C'est mieux d'utiliser un champs de texte avec la auto-complétion. Par exemple, \expword{choix du pays}.
	\end{itemize}
\end{minipage}
\begin{minipage}{0.30\textwidth}
	\hgraphpage{combobox.png}
\end{minipage}

\end{frame}

\begin{frame}
\frametitle{Formulaires: Les champs}
\framesubtitle{Sélecteur de date}

\begin{minipage}{0.50\textwidth}
	\begin{itemize}
		\item pour choisir une date
		\item exemples, \expword{date de naissance}, \expword{date de départ}, etc.
		\item par défaut, la date est celle d'aujourd'hui (du système)
		\item on peut fixer une date par défaut et des dates limites
	\end{itemize}
\end{minipage}
\begin{minipage}{0.49\textwidth}
	\hgraphpage{date.png}
\end{minipage}


\end{frame}

\subsection{Formulaires Word}

\begin{frame}
\frametitle{Formulaires}
\framesubtitle{Formulaires Word}

\begin{itemize}
	\item 
\end{itemize}

\end{frame}

\begin{frame}
\frametitle{Formulaires}
\framesubtitle{Formulaires Word}

\end{frame}

\subsection{Formulaires Google}

\begin{frame}
\frametitle{Formulaires}
\framesubtitle{Formulaires Google}

\end{frame}


\begin{frame}
\frametitle{Formulaires}
\framesubtitle{Un peu d'humeur}
\begin{center}
	\vgraphpage{forms-humour.jpg}
\end{center}
\end{frame}


\insertbibliography{Bweb04}{*}


\end{document}

