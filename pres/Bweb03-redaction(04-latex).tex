% !TEX TS-program = pdflatex
% !TeX program = pdflatex
% !TEX encoding = UTF-8
% !TEX spellcheck = fr

\documentclass[xcolor=table]{beamer}


%\usepackage{fullpage}
%\usepackage[left=2.8cm,right=2.2cm,top=2 cm,bottom=2 cm]{geometry}
\setbeamersize{text margin left=10pt,text margin right=10pt}
\usepackage{amsmath,amssymb} 
\usepackage[T1]{fontenc}
\usepackage[utf8]{inputenc}
\usepackage[english,french]{babel}
\usepackage{txfonts}
\usepackage[]{graphicx}
\usepackage{multirow}
\usepackage{hyperref}
\usepackage{colortbl}
\usepackage{listings}
\usepackage{wrapfig}
\usepackage{multicol}

\hypersetup{
	colorlinks,
	urlcolor = blue
}

%\renewcommand{\baselinestretch}{1.5}

\def\supit#1{\raisebox{0.8ex}{\small\it #1}\hspace{0.05em}}

\AtBeginSection{%
	\begin{frame}
		\sectionpage
	\end{frame}
}

\newcommand{\rottext}[2]{%
	\rotatebox{90}{%
	\begin{minipage}{#1}%
		\raggedleft#2%
	\end{minipage}%
	}%
}

\usepackage{longtable}
\usepackage{tabu}


\institute{ %
École  nationale Supérieure d'Informatique (ESI, ex. INI), Algérie
}
\author[ \textbf{\footnotesize  \insertframenumber/\inserttotalframenumber} \hspace*{\fill} ESI (2019-2020)] %
{ARIES Abdelkrime}
%\titlegraphic{\includegraphics[height=1cm]{../img/esi-logo.png}%\hspace*{4.75cm}~


\date{Année unniversitaire: 2019/2020} %\today

\usetheme{Warsaw} % Antibes Boadilla Warsaw

\beamertemplatenavigationsymbolsempty

%\setbeamertemplate{headline}{}

\definecolor{lightblue}{HTML}{D0D2FF}
\definecolor{lightyellow}{HTML}{FFFFAA}
\definecolor{darkblue}{HTML}{0000BB}
\definecolor{olivegreen}{HTML}{006600}
\definecolor{violet}{HTML}{6600CC}

\newcommand{\keyword}[1]{\textcolor{red}{\bfseries\itshape #1}}
\newcommand{\expword}[1]{\textcolor{olivegreen}{#1}}
\newcommand{\optword}[1]{\textcolor{violet}{\bfseries #1}}

\makeatletter
\newcommand\mysphere{%
	\parbox[t]{10pt}{\raisebox{0.2pt}{\beamer@usesphere{item projected}{bigsphere}}}}
\makeatother

%\let\oldtabular\tabular
%\let\endoldtabular\endtabular
%\renewenvironment{tabular}{\rowcolors{2}{white}{lightblue}\oldtabular\rowcolor{blue}}{\endoldtabular}


\NoAutoSpacing %french autospacing after ":"
\usepackage{lstautogobble}

\title[BWEB: 03- Rédaction (\LaTeX)] %
{Bureautique et Web \\Chapitre 03: Rédaction d'un document numérique \\ Introduction au \LaTeX}  

\changegraphpath{..//img/Bweb03-redaction/}

%\lstset{
%	basicstyle=\scriptsize\ttfamily,language=[LaTeX]Tex,breaklines=true,
%	breakautoindent=true,breakindent=2ex,
%}

\begin{document}
	
	\newcolumntype{L}[2]{%
		>{\vbox to #2\bgroup\vfill\flushleft}%
		p{#1}%
		<{\egroup}} 

\begin{frame}[plain]
	\maketitle
\end{frame}

\begin{frame}
\frametitle{Rédaction d'un document numérique}
\framesubtitle{\LaTeX}

\end{frame}

\begin{frame}
\frametitle{Rédaction d'un document numérique}
\framesubtitle{\LaTeX : Plan}

\begin{multicols}{2}
%	\small
\tableofcontents
\end{multicols}
\end{frame}

%===================================================================================
\section{Préparer un document}
%===================================================================================

\begin{frame}[fragile]
\frametitle{Préparer un document}
Un document \LaTeX se compose de ces parties :
\begin{itemize}
	\item préambule, contenant la classe du document (comme un modèle), appels des extensions et des commandes
	\item contenu entre \keyword{\textbackslash begin\{document\}} et \keyword{\textbackslash end\{document\}}
\end{itemize}

\begin{minipage}{0.60\textwidth} 
	Parmi les classes disponibles
	\begin{itemize}
		\item \optword{book} : pour rédiger des livres
%		\item memoir
		\item \optword{report} : pour rédiger des rapports
		\item \optword{article} : pour rédiger des articles
		\item \optword{letter} : pour rédiger des lettres
		\item \optword{beamer} : pour créer des présentations (comme celle-ci)
	\end{itemize}
\end{minipage}
%
\begin{minipage}{0.38\textwidth}
\begin{exampleblock}{exemple: hello.tex}
\scriptsize\bfseries
\begin{lstlisting}[language={[LaTeX]TeX}]
\documentclass{book}
% espace pour les packages
\begin{document}
    % le contenu
    Hello World!
\end{document}
\end{lstlisting}
\end{exampleblock}
\end{minipage}

\end{frame}

\subsection{Sections}

\begin{frame}
\frametitle{Préparer un document}
\framesubtitle{Sections}

\rowcolors{2}{lightblue}{lightyellow}

\begin{tabular}{p{.3\textwidth}cp{.6\textwidth}}
	%	\hline\hline
	\rowcolor{darkblue}
	\textcolor{white}{Section} && \textcolor{white}{Fonction} \\
	%	\hline\hline
	\textbackslash part\{...\} && La partie; ex. \expword{\'Etude bibliographique}. 
	
	Elle n'existe pas dans \optword{letter}.\\
	
	\textbackslash chapter\{...\} && Un chapitre; 
	
	Elle existe seulement dans \optword{book} et \optword{report}.\\
	
	\textbackslash section\{...\} && Une section (un titre de niveau 1); 
	
	Elle n'existe pas dans \optword{letter}.\\
	
	\textbackslash subsection\{...\} && Une sous-section (un titre de niveau 2); 
	
	Elle n'existe pas dans \optword{letter}.\\
	
	\textbackslash subsubsection\{...\} && Une subdivision (un titre de niveau 3); 
	
	Elle n'existe pas dans \optword{letter}.\\
\end{tabular}

Si on ajoute un étoile, la section ne sera pas numérotée. 
Par exemple, \expword{\textbackslash chapter*\{Introduction\}}
\end{frame}

%\begin{frame}[fragile]
%\frametitle{Préparer un document: Sections}
%\framesubtitle{Parties et Chapitres}
%
%\begin{exampleblock}{exemple: titre.tex}
%\scriptsize\bfseries
%\begin{lstlisting}
%\documentclass{book}
%
%\title{Introduction LaTeX}
%\author{Abdelkrime Aries}
%\date{2020}
%
%\begin{document}
%
%    \maketitle
%	\chapter*{Introduction}
%	mot mot mot mot
%
%    \chapter{Un chapite numerote}
%    mot mot mot mot
%    
%    \section{une section}
%	
%    \chapter{Un chapite numerote}
%    mot mot mot mot
%    
%    
%
%\end{document}
%\end{lstlisting}
%\end{exampleblock}
%
%\end{frame}

\subsection{Pages}

%\begin{frame}
%\frametitle{Préparer un document}
%\framesubtitle{Pages}
%
%\end{frame}

\begin{frame}[fragile]
\frametitle{Préparer un document: Pages}
\framesubtitle{Titre (Page de garde)}

\begin{minipage}{0.59\textwidth} 
	\begin{itemize}
		\item On doit introduire le titre, le nom de l'auteur, la date et d'autres informations selon la classe utilisée. 
		\item Pour générer le titre à partir de ces informations, on utilise la commande \keyword{\textbackslash maketitle}
		\item Dans \optword{book} et \optword{report}, \LaTeX\ crée une page de garde (une nouvelle page avec les informations). 
		Dans \optword{article}, \LaTeX\ insère ces informations en première page suivies par un texte 
	\end{itemize}
\end{minipage}
%
\begin{minipage}{0.40\textwidth}
\begin{exampleblock}{exemple: titre.tex}
\scriptsize\bfseries
\begin{lstlisting}
\documentclass{article}
	
\title{Introduction LaTeX}
\author{Abdelkrime Aries}
\date{2020}
	
\begin{document}
	
    \maketitle
	
    Hello World!
	
\end{document}
\end{lstlisting}
\end{exampleblock}
\end{minipage}

\end{frame}

\begin{frame}[fragile]
\frametitle{Préparer un document: Pages}
\framesubtitle{Taille et format}

\begin{itemize}
	\item Dans les options de la classe, on peut définir la taille de la page. 
	Parmi les tailles, on peut citer: \optword{a4paper}, \optword{a5paper} et \optword{letterpaper}.
	\item On peut, aussi, définir la taille des caractères pour le texte normal. 
	Dans les rapports de l'ESI, la taille utilisée est \optword{12pt}.
	\item Si on veut que le document soit recto-verso, on utilise l'option \optword{twoside}
\end{itemize}

\begin{exampleblock}{exemple: la taille de la page}
\scriptsize\bfseries
\begin{lstlisting}
\documentclass[a4paper, 12pt, twoside]{book}
\end{lstlisting}
\end{exampleblock}

\end{frame}

\begin{frame}[fragile]
\frametitle{Préparer un document: Pages}
\framesubtitle{Orientation (tout le document)}

\begin{itemize}
	\item Par défaut, l'orientation \optword{Portrait}
	\item Pour utiliser l'orientation \optword{Paysage}, on fait appel à l'extension \optword{geometry} avec l'option \optword{landscape}
\end{itemize}

\begin{exampleblock}{exemple: orientation paysage}
\scriptsize\bfseries
\begin{lstlisting}
\documentclass[a4paper, 12pt]{book}
\usepackage[landscape]{geometry}
\end{lstlisting}
\end{exampleblock}

\end{frame}


\begin{frame}[fragile]
\frametitle{Préparer un document: Pages}
\framesubtitle{Orientation (une partie du document)}

\begin{minipage}{0.30\textwidth} 
\begin{itemize}
	\item Importer l'extension \optword{pdflscape}
	\item Mettre le contenu dans l'environnement \optword{landscape}
\end{itemize}
\end{minipage}
%
\begin{minipage}{0.69\textwidth}
\begin{exampleblock}{exemple: paysage.tex}
\scriptsize\bfseries
\begin{lstlisting}
\documentclass[a4, 12pt]{article}
\usepackage{pdflscape}

\begin{document}
    page 1 avec orientation portrait

    \begin{landscape}
        page 2 avec orientation paysage 
        \newpage
        page 3 avec orientation paysage
    \end{landscape}

    page 4 avec orientation portrait
\end{document}
\end{lstlisting}
\end{exampleblock}
\end{minipage}

\end{frame}


\begin{frame}[fragile]
\frametitle{Préparer un document: Pages}
\framesubtitle{Marges}

\begin{itemize}
	\item On définit les marges en utilisant l'extension \optword{geometry}.
	\item Dans ces options, on peut fixer les marges gauche (\optword{left}), droite (\optword{right}), tête (\optword{top}) et pied (\optword{bottom}).
	\item Si on veut appliquer la même marge pour les quatre coins: \optword{margin}
	\item Si le document est recto-verso, on utilise \optword{inner} pour la marge intérieure (envers le reliure) et \optword{outer} pour la marge extérieure.
\end{itemize}

\begin{exampleblock}{exemple: les marges de la page}
\scriptsize\bfseries
\begin{lstlisting}
\documentclass[a4paper, 12pt]{book}
\usepackage[left=2.8cm, right=2.2cm, top=2cm, bottom=2cm]{geometry}
\end{lstlisting}
\end{exampleblock}

\end{frame}

\begin{frame}[fragile]
\frametitle{Préparer un document: Pages}
\framesubtitle{Colonnes (tout le document)}

\begin{itemize}
	\item Il existe des articles avec deux colonnes
	\item Utiliser l'option \optword{twocolumn} de la classe.
\end{itemize}

\begin{exampleblock}{exemple: les marges de la page}
\scriptsize\bfseries
\begin{lstlisting}
\documentclass[a4paper, 12pt, twocolumn]{article}
\end{lstlisting}
\end{exampleblock}

\end{frame}

\begin{frame}[fragile]
\frametitle{Préparer un document: Pages}
\framesubtitle{Colonnes (une partie du document)}

\begin{minipage}{0.30\textwidth} 
\begin{itemize}
	\item Importer l'extension \optword{multicol}
	\item Mettre le texte dans l'environnement \optword{multicols} en spécifiant le nombre des colonnes
\end{itemize}
\end{minipage}
%
\begin{minipage}{0.69\textwidth}
\begin{exampleblock}{exemple: colonnes.tex}
\scriptsize\bfseries
\begin{lstlisting}
\documentclass{article}
\usepackage{multicol}
	
\begin{document}
\begin{multicols}{3}
[Un texte qui s'affiche sur une seule colonne.]
Ceci est la premiere phrase. Ceci est la phrase 2.
Ceci est la phrase 3. Ceci est la phrase 4.
Ceci est la phrase 5. Ceci est la phrase 6.
Ceci est la phrase 7. Ceci est la phrase 8.
Ceci est la phrase 9. Ceci est la phrase 10.
\end{multicols}
Un texte qui s'affiche sur une seule colonne.
	
\end{document}
\end{lstlisting}
\end{exampleblock}
\end{minipage}

\end{frame}

\begin{frame}[fragile]
\frametitle{Préparer un document: Pages}
\framesubtitle{Saute de page}

Pour insérer un saut de page
\begin{itemize}
	\item on utilise la commande \keyword{\textbackslash newpage}
	\item avant un chapitre, on utilise la commande \keyword{\textbackslash clearpage}
	\item avant un chapitre dans un document recto-verso, on utilise la commande \keyword{\textbackslash cleardoublepage}
\end{itemize}

\begin{exampleblock}{exemple: nouvpage.tex}
\scriptsize\bfseries
\begin{lstlisting}
\documentclass{article}
\begin{document}
    Avant la nouvelle page
    \newpage
    Apres la nouvelle page
\end{document}
\end{lstlisting}
\end{exampleblock}

\end{frame}

%\begin{frame}
%\frametitle{Préparer un document: Pages}
%\framesubtitle{Bordures}
%
%\end{frame}

%\begin{frame}
%\frametitle{Préparer un document: Pages}
%\framesubtitle{Filigrane}
%
%\end{frame}

\begin{frame}[fragile]
\frametitle{Préparer un document: Pages}
\framesubtitle{Arrière-plan}

\begin{minipage}{0.49\textwidth}
Pour définir une couleur d'arrière plan
\begin{itemize}
	\item on utilise la commande \keyword{\textbackslash pagecolor} de l'extension \optword{xcolor}
	\item pour utiliser une image comme arrière-plan, on fait appel à l'extension \optword{background} (voir le document \expword{pageimage.tex})
\end{itemize}
\end{minipage}
%
\begin{minipage}{0.50\textwidth}
\begin{exampleblock}{exemple: pagecouleur.tex}
\scriptsize\bfseries
\begin{lstlisting}
\documentclass{article}
\usepackage{xcolor}
	
\begin{document}
    \pagecolor{yellow}
    Je suis une page jaune

    \newpage

    \pagecolor{green}
    Je suis une page verte
	
\end{document}
\end{lstlisting}
\end{exampleblock}
\end{minipage}

\end{frame}

\begin{frame}
\frametitle{Préparer un document: Pages}
\framesubtitle{En-tête et pied de page}



\end{frame}


\subsection{Paragraphes}

\begin{frame}[fragile]
\frametitle{Préparer un document}
\framesubtitle{Paragraphes}

\begin{minipage}{0.49\textwidth}
	\begin{itemize}
		\item ...
		\item 
	\end{itemize}
\end{minipage}
%
\begin{minipage}{0.50\textwidth}
\begin{exampleblock}{exemple: paragraphes.tex}
\scriptsize\bfseries
\begin{lstlisting}
\documentclass{article}
	
\begin{document}
    Phrase 1 Paragraphe 1. 
    Phrase 2 Paragraphe 1. 
    
    Phrase 1 Paragraphe 2. 
    Phrase 2 Paragraphe 2.		
\end{document}
\end{lstlisting}
\end{exampleblock}
\end{minipage}

\end{frame}


%\begin{frame}[fragile]
%\frametitle{Préparer un document: Paragraphes}
%\framesubtitle{Internationalisation}
%
%\begin{minipage}{0.49\textwidth}
%	\begin{itemize}
%		\item ...
%		\item 
%	\end{itemize}
%\end{minipage}
%%
%\begin{minipage}{0.50\textwidth}
%\begin{exampleblock}{exemple: i18n.tex}
%\lstset{escapeinside=``}
%\scriptsize\bfseries
%\begin{lstlisting}
%\documentclass{article}
%\usepackage[utf8]{inputenc}
%\usepackage[arabic,french]{babel}
%
%\begin{document}
%Les caractères é, è, à, ô, etc. ne causeront aucun problème. 
%
%\selectlanguage{arabic}`\<
%هذا نص بالعربية
%>`\selectlanguage{french}`\<
%
%Un texte en français avec un autre en arabe 
%
%>`\begin{otherlanguage}{arabic}
%هذا نص بالعربية
%\end{otherlanguage}
%
%\end{document}
%\end{lstlisting}
%\end{exampleblock}
%\end{minipage}
%
%\end{frame}

\begin{frame}
\frametitle{Préparer un document: Paragraphes}
\framesubtitle{Alignement}

\end{frame}

\begin{frame}
\frametitle{Préparer un document: Paragraphes}
\framesubtitle{Espacement vertical}

\end{frame}

\begin{frame}
\frametitle{Préparer un document: Paragraphes}
\framesubtitle{Interligne}

\end{frame}

\begin{frame}
\frametitle{Préparer un document: Paragraphes}
\framesubtitle{Retrait}

\end{frame}

\begin{frame}
\frametitle{Préparer un document: Paragraphes}
\framesubtitle{Retrait de 1ère ligne}

\end{frame}


\begin{frame}
\frametitle{Préparer un document: Paragraphes}
\framesubtitle{Listes}

\end{frame}


\subsection{Mise en forme et styles}

\begin{frame}
\frametitle{Préparer un document}
\framesubtitle{Mise en forme et styles}

\end{frame}

\begin{frame}
\frametitle{Préparer un document: Mise en forme et styles}
\framesubtitle{Police}

\end{frame}

\begin{frame}
\frametitle{Préparer un document: Mise en forme et styles}
\framesubtitle{Couleur}

\end{frame}

\begin{frame}
\frametitle{Préparer un document: Mise en forme et styles}
\framesubtitle{Casse}

\end{frame}

\begin{frame}
\frametitle{Préparer un document: Mise en forme et styles}
\framesubtitle{Style des caractères}

\end{frame}

\begin{frame}
\frametitle{Préparer un document: Mise en forme et styles}
\framesubtitle{Effets de texte}

\end{frame}


%===================================================================================
\section{Enrichir un document}
%===================================================================================

\begin{frame}
\frametitle{Enrichir un document}

\end{frame}

\subsection{Tableaux}

\begin{frame}
\frametitle{Enrichir un document}
\framesubtitle{Tableaux}

\end{frame}

\begin{frame}
\frametitle{Enrichir un document: Tableaux}
\framesubtitle{Formatage: Bordures}

\end{frame}


\subsection{Images et dessins}

\begin{frame}
\frametitle{Enrichir un document}
\framesubtitle{Images}

\end{frame}

\begin{frame}
\frametitle{Enrichir un document: Images}
\framesubtitle{Insérer une image}

\end{frame}

\begin{frame}
\frametitle{Enrichir un document: Images et dessins}
\framesubtitle{Modifier une image}

\end{frame}

\begin{frame}
\frametitle{Enrichir un document: Images}
\framesubtitle{Habillage}

\end{frame}

\begin{frame}
\frametitle{Enrichir un document: Images}
\framesubtitle{Positionnement}

\end{frame}

\subsection{Liens et renvoi}

\begin{frame}
\frametitle{Enrichir un document}
\framesubtitle{Liens et renvoi}

\end{frame}

\begin{frame}
\frametitle{Enrichir un document: Liens et renvoi}
\framesubtitle{Insérer un lien hypertexte}

\end{frame}

\begin{frame}
\frametitle{Enrichir un document: Liens et renvoi}
\framesubtitle{Renvoi vers un emplacement dans le document}

\end{frame}

\subsection{Formules et symboles}

\begin{frame}
\frametitle{Enrichir un document}
\framesubtitle{Formules et symboles}

\end{frame}

\begin{frame}
\frametitle{Enrichir un document: Formules et symboles}
\framesubtitle{Insérer une formule}

\end{frame}

\begin{frame}
\frametitle{Enrichir un document: Formules et symboles}
\framesubtitle{Insérer des symboles}

\end{frame}

%===================================================================================
\section{Références}
%===================================================================================

\begin{frame}
\frametitle{Références}

\end{frame}

\subsection{Notes de bas de page}

\begin{frame}
\frametitle{Références}
\framesubtitle{Notes de bas de page}

\end{frame}

\subsection{Table de matières}

\begin{frame}
\frametitle{Références}
\framesubtitle{Table de matières}

\end{frame}

\subsection{Légendes}

\begin{frame}
\frametitle{Références}
\framesubtitle{Légendes}

\end{frame}

\subsection{Index}

\begin{frame}
\frametitle{Références}
\framesubtitle{Index}

\end{frame}

%%===================================================================================
%\section{Révision et partage}
%%===================================================================================
%
%\begin{frame}
%\frametitle{Révision et partage}
%
%\end{frame}
%
%\subsection{Vérification}
%
%\begin{frame}
%\frametitle{Révision et partage}
%\framesubtitle{Vérification}
%
%\end{frame}
%
%\subsection{Commentaires et suivi}
%
%\begin{frame}
%\frametitle{Révision et partage}
%\framesubtitle{Commentaires et suivi}
%
%\end{frame}
%
%\subsection{Comparaison}
%
%\begin{frame}
%\frametitle{Révision et partage}
%\framesubtitle{Comparaison}
%
%\end{frame}
%
%\subsection{Partage}
%
%\begin{frame}
%\frametitle{Révision et partage}
%\framesubtitle{Partage}
%
%
%\end{frame}
%
%\begin{frame}
%\frametitle{Révision et partage: Partage}
%\framesubtitle{Protection}
%
%\end{frame}
%
%\begin{frame}
%\frametitle{Révision et partage: Partage}
%\framesubtitle{Impression}
%
%\end{frame}
%
%\begin{frame}
%\frametitle{Révision et partage: Partage}
%\framesubtitle{Sauvegarde}
%
%\end{frame}
%
%\begin{frame}
%\frametitle{Révision et partage: Partage}
%\framesubtitle{Sauvegarde sur cloud}
%
%
%\end{frame}

\nocite{*}
%\subsection{Bibliography}
%\frame[allowframebreaks]%
%{\frametitle{Références}
%\tiny
%\bibliography{Bweb01}
%\bibliographystyle{plain} 
%}

\begin{multicols*}{2}[\usebeamertemplate*{frametitle}\frametitle{Références}]%
	\tiny
	\bibliography{Bweb03}
	\bibliographystyle{acm}
\end{multicols*}


\end{document}

