% !TEX TS-program = pdflatex
% !TeX program = pdflatex
% !TEX encoding = UTF-8
% !TEX spellcheck = fr

\documentclass[xcolor=table]{beamer}


%\usepackage{fullpage}
%\usepackage[left=2.8cm,right=2.2cm,top=2 cm,bottom=2 cm]{geometry}
\setbeamersize{text margin left=10pt,text margin right=10pt}
\usepackage{amsmath,amssymb} 
\usepackage[T1]{fontenc}
\usepackage[utf8]{inputenc}
\usepackage[english,french]{babel}
\usepackage{txfonts}
\usepackage[]{graphicx}
\usepackage{multirow}
\usepackage{hyperref}
\usepackage{colortbl}
\usepackage{listings}
\usepackage{wrapfig}
\usepackage{multicol}

\hypersetup{
	colorlinks,
	urlcolor = blue
}

%\renewcommand{\baselinestretch}{1.5}

\def\supit#1{\raisebox{0.8ex}{\small\it #1}\hspace{0.05em}}

\AtBeginSection{%
	\begin{frame}
		\sectionpage
	\end{frame}
}

\newcommand{\rottext}[2]{%
	\rotatebox{90}{%
	\begin{minipage}{#1}%
		\raggedleft#2%
	\end{minipage}%
	}%
}

\usepackage{longtable}
\usepackage{tabu}


\institute{ %
École  nationale Supérieure d'Informatique (ESI, ex. INI), Algérie
}
\author[ \textbf{\footnotesize  \insertframenumber/\inserttotalframenumber} \hspace*{\fill} ESI (2019-2020)] %
{ARIES Abdelkrime}
%\titlegraphic{\includegraphics[height=1cm]{../img/esi-logo.png}%\hspace*{4.75cm}~


\date{Année unniversitaire: 2019/2020} %\today

\usetheme{Warsaw} % Antibes Boadilla Warsaw

\beamertemplatenavigationsymbolsempty

%\setbeamertemplate{headline}{}

\definecolor{lightblue}{HTML}{D0D2FF}
\definecolor{lightyellow}{HTML}{FFFFAA}
\definecolor{darkblue}{HTML}{0000BB}
\definecolor{olivegreen}{HTML}{006600}
\definecolor{violet}{HTML}{6600CC}

\newcommand{\keyword}[1]{\textcolor{red}{\bfseries\itshape #1}}
\newcommand{\expword}[1]{\textcolor{olivegreen}{#1}}
\newcommand{\optword}[1]{\textcolor{violet}{\bfseries #1}}

\makeatletter
\newcommand\mysphere{%
	\parbox[t]{10pt}{\raisebox{0.2pt}{\beamer@usesphere{item projected}{bigsphere}}}}
\makeatother

%\let\oldtabular\tabular
%\let\endoldtabular\endtabular
%\renewenvironment{tabular}{\rowcolors{2}{white}{lightblue}\oldtabular\rowcolor{blue}}{\endoldtabular}


\NoAutoSpacing %french autospacing after ":"
\usepackage{lstautogobble}

\title[BWEB: 03- Rédaction (\LaTeX)] %
{Bureautique et Web \\Chapitre 03: Rédaction d'un document numérique \\ Introduction au \LaTeX}  

\changegraphpath{../img/Bweb03-redaction/latex/}

%\lstset{
%	basicstyle=\scriptsize\ttfamily,language=[LaTeX]Tex,breaklines=true,
%	breakautoindent=true,breakindent=2ex,
%}

\begin{document}
	
	\newcolumntype{L}[2]{%
		>{\vbox to #2\bgroup\vfill\flushleft}%
		p{#1}%
		<{\egroup}} 

\begin{frame}[plain]
	\maketitle
\end{frame}

\insertlicence

\begin{frame}
\frametitle{Rédaction d'un document numérique}
\framesubtitle{\LaTeX}

\end{frame}

\begin{frame}
\frametitle{Rédaction d'un document numérique}
\framesubtitle{\LaTeX : Plan}

\begin{multicols}{2}
%	\small
\tableofcontents
\end{multicols}
\end{frame}

%===================================================================================
\section{Préparer un document}
%===================================================================================

\begin{frame}[fragile]
\frametitle{Préparer un document}
Un document \LaTeX se compose de ces parties :
\begin{itemize}
	\item préambule, contenant la classe du document (comme un modèle), appels des extensions et des commandes
	\item contenu entre \keyword{\textbackslash begin\{document\}} et \keyword{\textbackslash end\{document\}}
\end{itemize}

\begin{minipage}{0.60\textwidth} 
	Parmi les classes disponibles
	\begin{itemize}
		\item \optword{book} : pour rédiger des livres
%		\item memoir
		\item \optword{report} : pour rédiger des rapports
		\item \optword{article} : pour rédiger des articles
		\item \optword{letter} : pour rédiger des lettres
		\item \optword{beamer} : pour créer des présentations (comme celle-ci)
	\end{itemize}
\end{minipage}
%
\begin{minipage}{0.38\textwidth}
\begin{exampleblock}{exemple: hello.tex}
\scriptsize\bfseries
\begin{lstlisting}[language={[LaTeX]TeX}]
\documentclass{book}
% espace pour les packages
\begin{document}
    % le contenu
    Hello World!
\end{document}
\end{lstlisting}
\end{exampleblock}
\end{minipage}

\end{frame}

\subsection{Sections}

\begin{frame}
\frametitle{Préparer un document}
\framesubtitle{Sections}

\rowcolors{2}{lightblue}{lightyellow}

\begin{tabular}{p{.3\textwidth}cp{.6\textwidth}}
	%	\hline\hline
	\rowcolor{darkblue}
	\textcolor{white}{Section} && \textcolor{white}{Fonction} \\
	%	\hline\hline
	\textbackslash part\{...\} && La partie; ex. \expword{\'Etude bibliographique}. 
	
	Elle n'existe pas dans \optword{letter}.\\
	
	\textbackslash chapter\{...\} && Un chapitre; 
	
	Elle existe seulement dans \optword{book} et \optword{report}.\\
	
	\textbackslash section\{...\} && Une section (un titre de niveau 1); 
	
	Elle n'existe pas dans \optword{letter}.\\
	
	\textbackslash subsection\{...\} && Une sous-section (un titre de niveau 2); 
	
	Elle n'existe pas dans \optword{letter}.\\
	
	\textbackslash subsubsection\{...\} && Une subdivision (un titre de niveau 3); 
	
	Elle n'existe pas dans \optword{letter}.\\
\end{tabular}

Si on ajoute un étoile, la section ne sera pas numérotée. 
Par exemple, \expword{\textbackslash chapter*\{Introduction\}}
\end{frame}

%\begin{frame}[fragile]
%\frametitle{Préparer un document: Sections}
%\framesubtitle{Parties et Chapitres}
%
%\begin{exampleblock}{exemple: titre.tex}
%\scriptsize\bfseries
%\begin{lstlisting}
%\documentclass{book}
%
%\title{Introduction LaTeX}
%\author{Abdelkrime Aries}
%\date{2020}
%
%\begin{document}
%
%    \maketitle
%	\chapter*{Introduction}
%	mot mot mot mot
%
%    \chapter{Un chapite numerote}
%    mot mot mot mot
%    
%    \section{une section}
%	
%    \chapter{Un chapite numerote}
%    mot mot mot mot
%    
%    
%
%\end{document}
%\end{lstlisting}
%\end{exampleblock}
%
%\end{frame}

\subsection{Pages}

%\begin{frame}
%\frametitle{Préparer un document}
%\framesubtitle{Pages}
%
%\end{frame}

\begin{frame}[fragile]
\frametitle{Préparer un document: Pages}
\framesubtitle{Titre (Page de garde)}

\begin{minipage}{0.59\textwidth} 
	\begin{itemize}
		\item On doit introduire le titre, le nom de l'auteur, la date et d'autres informations selon la classe utilisée. 
		\item Pour générer le titre à partir de ces informations, on utilise la commande \keyword{\textbackslash maketitle}
		\item Dans \optword{book} et \optword{report}, \LaTeX\ crée une page de garde (une nouvelle page avec les informations). 
		Dans \optword{article}, \LaTeX\ insère ces informations en première page suivies par un texte 
	\end{itemize}
\end{minipage}
%
\begin{minipage}{0.40\textwidth}
\begin{exampleblock}{exemple: titre.tex}
\scriptsize\bfseries
\begin{lstlisting}
\documentclass{article}
	
\title{Introduction LaTeX}
\author{Abdelkrime Aries}
\date{2020}
	
\begin{document}
	
    \maketitle
	
    Hello World!
	
\end{document}
\end{lstlisting}
\end{exampleblock}
\end{minipage}

\end{frame}

\begin{frame}[fragile]
\frametitle{Préparer un document: Pages}
\framesubtitle{Taille et format}

\begin{itemize}
	\item Dans les options de la classe, on peut définir la taille de la page. 
	Parmi les tailles, on peut citer: \optword{a4paper}, \optword{a5paper} et \optword{letterpaper}.
	\item On peut, aussi, définir la taille des caractères pour le texte normal. 
	Dans les rapports de l'ESI, la taille utilisée est \optword{12pt}.
	\item Si on veut que le document soit recto-verso, on utilise l'option \optword{twoside}
\end{itemize}

\begin{exampleblock}{exemple: la taille de la page}
\scriptsize\bfseries
\begin{lstlisting}
\documentclass[a4paper, 12pt, twoside]{book}
\end{lstlisting}
\end{exampleblock}

\end{frame}

\begin{frame}[fragile]
\frametitle{Préparer un document: Pages}
\framesubtitle{Orientation (tout le document)}

\begin{itemize}
	\item Par défaut, l'orientation \optword{Portrait}
	\item Pour utiliser l'orientation \optword{Paysage}, on fait appel à l'extension \optword{geometry} avec l'option \optword{landscape}
\end{itemize}

\begin{exampleblock}{exemple: orientation paysage}
\scriptsize\bfseries
\begin{lstlisting}
\documentclass[a4paper, 12pt]{book}
\usepackage[landscape]{geometry}
\end{lstlisting}
\end{exampleblock}

\end{frame}


\begin{frame}[fragile]
\frametitle{Préparer un document: Pages}
\framesubtitle{Orientation (une partie du document)}

\begin{minipage}{0.30\textwidth} 
\begin{itemize}
	\item Importer l'extension \optword{pdflscape}
	\item Mettre le contenu dans l'environnement \optword{landscape}
\end{itemize}
\end{minipage}
%
\begin{minipage}{0.69\textwidth}
\begin{exampleblock}{exemple: paysage.tex}
\scriptsize\bfseries
\begin{lstlisting}
\documentclass[a4paper, 12pt]{article}
\usepackage{pdflscape}

\begin{document}
    page 1 avec orientation portrait

    \begin{landscape}
        page 2 avec orientation paysage 
        \newpage
        page 3 avec orientation paysage
    \end{landscape}

    page 4 avec orientation portrait
\end{document}
\end{lstlisting}
\end{exampleblock}
\end{minipage}

\end{frame}


\begin{frame}[fragile]
\frametitle{Préparer un document: Pages}
\framesubtitle{Marges}

\begin{itemize}
	\item On définit les marges en utilisant l'extension \optword{geometry}.
	\item Dans ces options, on peut fixer les marges gauche (\optword{left}), droite (\optword{right}), tête (\optword{top}) et pied (\optword{bottom}).
	\item Si on veut appliquer la même marge pour les quatre coins: \optword{margin}
	\item Si le document est recto-verso, on utilise \optword{inner} pour la marge intérieure (envers le reliure) et \optword{outer} pour la marge extérieure.
\end{itemize}

\begin{exampleblock}{exemple: les marges de la page}
\scriptsize\bfseries
\begin{lstlisting}
\documentclass[a4paper, 12pt]{book}
\usepackage[left=2.8cm, right=2.2cm, top=2cm, bottom=2cm]{geometry}
\end{lstlisting}
\end{exampleblock}

\end{frame}

\begin{frame}[fragile]
\frametitle{Préparer un document: Pages}
\framesubtitle{Colonnes (tout le document)}

\begin{itemize}
	\item Il existe des articles avec deux colonnes
	\item Utiliser l'option \optword{twocolumn} de la classe.
\end{itemize}

\begin{exampleblock}{exemple: les marges de la page}
\scriptsize\bfseries
\begin{lstlisting}
\documentclass[a4paper, 12pt, twocolumn]{article}
\end{lstlisting}
\end{exampleblock}

\end{frame}

\begin{frame}[fragile]
\frametitle{Préparer un document: Pages}
\framesubtitle{Colonnes (une partie du document)}

\begin{minipage}{0.30\textwidth} 
\begin{itemize}
	\item Importer l'extension \optword{multicol}
	\item Mettre le texte dans l'environnement \optword{multicols} en spécifiant le nombre des colonnes
\end{itemize}
\end{minipage}
%
\begin{minipage}{0.69\textwidth}
\begin{exampleblock}{exemple: colonnes.tex}
\scriptsize\bfseries
\begin{lstlisting}
\documentclass{article}
\usepackage{multicol}
	
\begin{document}
\begin{multicols}{3}
[Un texte qui s'affiche sur une seule colonne.]
Ceci est la premiere phrase. Ceci est la phrase 2.
Ceci est la phrase 3. Ceci est la phrase 4.
Ceci est la phrase 5. Ceci est la phrase 6.
Ceci est la phrase 7. Ceci est la phrase 8.
Ceci est la phrase 9. Ceci est la phrase 10.
\end{multicols}
Un texte qui s'affiche sur une seule colonne.
	
\end{document}
\end{lstlisting}
\end{exampleblock}
\end{minipage}

\end{frame}

\begin{frame}[fragile]
\frametitle{Préparer un document: Pages}
\framesubtitle{Saute de page}

Pour insérer un saut de page
\begin{itemize}
	\item on utilise la commande \keyword{\textbackslash newpage}
	\item avant un chapitre, on utilise la commande \keyword{\textbackslash clearpage}
	\item avant un chapitre dans un document recto-verso, on utilise la commande \keyword{\textbackslash cleardoublepage}
\end{itemize}

\begin{exampleblock}{exemple: nouvpage.tex}
\scriptsize\bfseries
\begin{lstlisting}
\documentclass{article}
\begin{document}
    Avant la nouvelle page
    \newpage
    Apres la nouvelle page
\end{document}
\end{lstlisting}
\end{exampleblock}

\end{frame}

%\begin{frame}
%\frametitle{Préparer un document: Pages}
%\framesubtitle{Bordures}
%
%\end{frame}

%\begin{frame}
%\frametitle{Préparer un document: Pages}
%\framesubtitle{Filigrane}
%
%\end{frame}

\begin{frame}[fragile]
\frametitle{Préparer un document: Pages}
\framesubtitle{Arrière-plan}

\begin{minipage}{0.49\textwidth}
Pour définir une couleur d'arrière plan
\begin{itemize}
	\item on utilise la commande \keyword{\textbackslash pagecolor} de l'extension \optword{xcolor}
	\item pour utiliser une image comme arrière-plan, on fait appel à l'extension \optword{background} (voir le document \expword{pageimage.tex})
\end{itemize}
\end{minipage}
%
\begin{minipage}{0.50\textwidth}
\begin{exampleblock}{exemple: pagecouleur.tex}
\scriptsize\bfseries
\begin{lstlisting}
\documentclass{article}
\usepackage{xcolor}
	
\begin{document}
    \pagecolor{yellow}
    Je suis une page jaune

    \newpage

    \pagecolor{green}
    Je suis une page verte
	
\end{document}
\end{lstlisting}
\end{exampleblock}
\end{minipage}

\end{frame}

\begin{frame}
\frametitle{Préparer un document: Pages}
\framesubtitle{En-tête et pied de page}

\begin{itemize}
	\item Pour définir l'entête et les pied d'une page comme vide, utiliser la commande \keyword{\textbackslash pagestyle\{empty\}}
	\item Utiliser l'extension \optword{fancyhdr}
	\item 
\end{itemize}


\end{frame}


\subsection{Paragraphes}

\begin{frame}[fragile]
\frametitle{Préparer un document}
\framesubtitle{Paragraphes}

\begin{minipage}{0.49\textwidth}
	\begin{itemize}
		\item Si on laisse une ligne vide, \LaTeX créera un nouveau paragraphe. 
		\item On peut écrire les phrases d'un même paragraphe sur multiples lignes.
		\item Pour sauter la ligne sans créer un nouveau paragraphe, utiliser la commande \keyword{\textbackslash newline} ou simplement \keyword{\textbackslash \textbackslash}
	\end{itemize}
\end{minipage}
%
\begin{minipage}{0.50\textwidth}
\begin{exampleblock}{exemple: paragraphes.tex}
\scriptsize\bfseries
\begin{lstlisting}
\documentclass{article}
	
\begin{document}
    Phrase 1 Paragraphe 1. 
    Phrase 2 Paragraphe 1. 
    
    Phrase 1 Paragraphe 2. 
    Phrase 2 Paragraphe 2.	
    \newline	
    Phrase 3 Paragraphe 2.
\end{document}
\end{lstlisting}
\end{exampleblock}
\end{minipage}

\end{frame}


%\begin{frame}[fragile]
%\frametitle{Préparer un document: Paragraphes}
%\framesubtitle{Internationalisation}
%
%\begin{minipage}{0.49\textwidth}
%	\begin{itemize}
%		\item ...
%		\item 
%	\end{itemize}
%\end{minipage}
%%
%\begin{minipage}{0.50\textwidth}
%\begin{exampleblock}{exemple: i18n.tex}
%\lstset{escapeinside=``}
%\scriptsize\bfseries
%\begin{lstlisting}
%\documentclass{article}
%\usepackage[utf8]{inputenc}
%\usepackage[arabic,french]{babel}
%
%\begin{document}
%Les caractères é, è, à, ô, etc. ne causeront aucun problème. 
%
%\selectlanguage{arabic}`\<
%هذا نص بالعربية
%>`\selectlanguage{french}`\<
%
%Un texte en français avec un autre en arabe 
%
%>`\begin{otherlanguage}{arabic}
%هذا نص بالعربية
%\end{otherlanguage}
%
%\end{document}
%\end{lstlisting}
%\end{exampleblock}
%\end{minipage}
%
%\end{frame}

\begin{frame}[fragile]
\frametitle{Préparer un document: Paragraphes}
\framesubtitle{Alignement}

\begin{minipage}{0.49\textwidth}
	Les environnements:
	\begin{itemize}
		\item \optword{flushleft} pour aligner à gauche
		\item \optword{center} pour centrer
		\item \optword{flushright} pour aligner à droite
	\end{itemize}
\end{minipage}
%
\begin{minipage}{0.50\textwidth}
\begin{exampleblock}{exemple: alignement.tex}
\scriptsize\bfseries
\begin{lstlisting}
\documentclass{article}

\begin{document}
    \begin{flushleft}
        Aligner a gauche
    \end{flushleft}

    \begin{center}
        Centrer
    \end{center} 

    \begin{flushright}
        Aligner a droite
    \end{flushright}
\end{document}
\end{lstlisting}
\end{exampleblock}
\end{minipage}

\end{frame}


\begin{frame}[fragile]
\frametitle{Préparer un document: Paragraphes}
\framesubtitle{Espacement vertical, interligne et retrait de 1ère ligne}

\begin{minipage}{0.49\textwidth}
	Les paramètres :
	\begin{itemize}
		\item \keyword{\textbackslash parskip} : espacement entre deux paragraphes
		\item \keyword{\textbackslash parindent} : retrait de 1ère ligne
		\item \keyword{\textbackslash baselineskip} : interligne
	\end{itemize}

	Avant un paragraphe :
	\begin{itemize}
		\item \keyword{\textbackslash indent} : forcer le retrait
		\item \keyword{\textbackslash noindent} : ignorer le retrait
	\end{itemize}
\end{minipage}
%
\begin{minipage}{0.50\textwidth}
\begin{exampleblock}{exemple: espacement.tex}
\scriptsize\bfseries
\begin{lstlisting}
\documentclass{article}

\setlength{\parskip}{1cm}
\setlength{\baselineskip}{12pt}
\setlength{\parindent}{2cm}

\begin{document}

    Phrase 1 Paragraphe 1. \newline
    Phrase 2 Paragraphe 1. 

    Phrase 1 Paragraphe 2. 
    Phrase 2 Paragraphe 2.		

\end{document}
\end{lstlisting}
\end{exampleblock}
\end{minipage}

\end{frame}

\begin{frame}[fragile]
\frametitle{Préparer un document: Paragraphes}
\framesubtitle{Listes}

\begin{minipage}{0.49\textwidth}
	Les environnements :
	\begin{itemize}
		\item \optword{itemize} : listes à puces
		\item \optword{enumerate} : listes numérotées
	\end{itemize}
	
	caractérisés par :
	\begin{itemize}
		\item un élément est défini par \keyword{\textbackslash item}
		\item on peut créer des sous-listes
	\end{itemize}
\end{minipage}
%
\begin{minipage}{0.50\textwidth}
\begin{exampleblock}{exemple: listes.tex}
\scriptsize\bfseries
\begin{lstlisting}
\documentclass{article}

\begin{document}
    \begin{enumerate}
        \item etape 1
        \item etape 2:
        \begin{itemize}
            \item un element
            \item un autre element
        \end{itemize}
    \end{enumerate}
\end{document}
\end{lstlisting}
\end{exampleblock}
\end{minipage}

\end{frame}


\subsection{Mise en forme}

\begin{frame}[fragile]
\frametitle{Préparer un document: Mise en forme}
\framesubtitle{Styles de la police}

\begin{minipage}{0.49\textwidth}
	Les commandes :
	\begin{itemize}
		\item \keyword{\textbackslash textbf} : gras
		\item \keyword{\textbackslash textit} : italique
		\item \keyword{\textbackslash underline} : souligné
		\item \keyword{\textbackslash uppercase} : majuscule
		\item \keyword{\textbackslash lowercase} : minuscule
		\item \keyword{\textbackslash textrm} : famille roman 
		\item \keyword{\textbackslash textsf} : famille sans serif
		\item \keyword{\textbackslash texttt} : famille teletype
	\end{itemize}
\end{minipage}
%
\begin{minipage}{0.50\textwidth}
\begin{exampleblock}{exemple: styles.tex}
\scriptsize\bfseries
\begin{lstlisting}
\documentclass{article}
		
\begin{document}
    \textbf{Texte En Gras}. 
    \textit{Texte En Italique}. 
    \underline{Texte Souligne}. 
    \uppercase{Texte En Majuscule}. 
    \lowercase{Texte En Minuscule}.
    \textrm{texte en roman}. 
    \textsf{texte en sans serif}. 
    \texttt{texte en teletype}. 
\end{document}
\end{lstlisting}
\end{exampleblock}
\end{minipage}


\end{frame}

\begin{frame}[fragile]
\frametitle{Préparer un document: Mise en forme}
\framesubtitle{Tailles de la police}

\begin{minipage}{0.39\textwidth}
	\begin{itemize}
		\item \keyword{\textbackslash tiny}
		\item \keyword{\textbackslash scriptsize}
		\item \keyword{\textbackslash footnotesize}
		\item \keyword{\textbackslash small}
		\item \keyword{\textbackslash normalsize}
		\item \keyword{\textbackslash large}
		\item \keyword{\textbackslash Large}
		\item \keyword{\textbackslash LARGE}
		\item \keyword{\textbackslash huge}
		\item \keyword{\textbackslash Huge}
	\end{itemize}
\end{minipage}
%
\begin{minipage}{0.60\textwidth}
\begin{exampleblock}{exemple: taillespolice.tex}
\scriptsize\bfseries
\begin{lstlisting}
\documentclass[12pt]{article}
		
\begin{document}
    {\tiny texte en tiny.} 
    {\scriptsize texte en scriptsize.} 
    {\footnotesize texte en footnotesize.} 
    {\small texte en small.} 
    {\normalsize texte en normalsize.} 
    {\large texte en large.} 
    {\Large texte en Large.} 
    {\LARGE texte en LARGE.} 
    {\huge texte en huge.} 
    {\Huge texte en Huge.} 
\end{document}
\end{lstlisting}
\end{exampleblock}
\end{minipage}

\end{frame}

\begin{frame}[fragile]
\frametitle{Préparer un document: Mise en forme}
\framesubtitle{Couleur du texte}

\begin{minipage}{0.39\textwidth}
	\begin{itemize}
		\item utiliser la commande \keyword{\textbackslash textcolor} de l'extension \optword{xcolor}
		\item les couleurs prédéfinies sont: black, blue, brown, cyan, darkgray, gray, green, lightgray, lime, magenta, olive, orange, pink, purple, red, teal, violet, white, yellow.
		\item pour définir d'autres couleurs, utiliser la commande \keyword{\textbackslash definecolor}
	\end{itemize}
\end{minipage}
%
\begin{minipage}{0.60\textwidth}
\begin{exampleblock}{exemple: taillespolice.tex}
\scriptsize\bfseries
\begin{lstlisting}
\documentclass{article}
\usepackage{xcolor}
\definecolor{couleur1}{RGB}{180, 80, 255}
\begin{document}

    Un texte normal. 

    \textcolor{red}{Un texte en rouge}

    \textcolor{couleur1}{Un texte en couleur1}

\end{document}
\end{lstlisting}
\end{exampleblock}
\end{minipage}

\end{frame}


%===================================================================================
\section{Enrichir un document}
%===================================================================================

\begin{frame}
\frametitle{Enrichir un document}

\end{frame}

\subsection{Tableaux}

\begin{frame}[fragile]
\frametitle{Enrichir un document : Tableaux}
\framesubtitle{Insérer un tableau}

\begin{minipage}{0.59\textwidth}
	\begin{itemize}
		\item utiliser l'environnement \optword{tabular}
		\item l'argument définit le nombre des colonnes en utilisant les lettres: \optword{l} (aligner à gauche), \optword{c} (center) et \optword{r} (aligner à droite).
		\item la barre (\optword{|}) trace un trait vertical  
		\item \keyword{\textbackslash hline} trace un trait horizontal
		\item les colonnes sont séparés par \optword{\&} et les lignes par \optword{\textbackslash \textbackslash}
	\end{itemize}
\end{minipage}
%
\begin{minipage}{0.40\textwidth}
\begin{exampleblock}{exemple: tableaux1.tex}
\scriptsize\bfseries
\begin{lstlisting}
\documentclass{article}

\begin{document}
    \begin{tabular}{|l|c|r|}
        \hline
        l1c1 & l1c2 & l1c3 \\
        \hline 
        l2c1 & l2c2 & l2c3 \\
        \hline
    \end{tabular}
\end{document}
\end{lstlisting}
\end{exampleblock}
\end{minipage}

\end{frame}


\begin{frame}[fragile]
\frametitle{Enrichir un document : Tableaux}
\framesubtitle{Positionner un tableau}

\begin{minipage}{0.59\textwidth}
	\begin{itemize}
		\item utiliser l'environnement \optword{table}
		\item l'argument définit le positionnement en utilisant les lettres:
		\begin{itemize}
			\item \optword{!} demande à \LaTeX\ de faire tout son possible pour respecter l'ordre indiqué
			ensuite ;
			\item \optword{h} (here) ici, si possible ;
			\item \optword{t} (top) en haut d’une page (celle-ci ou la suivante) ;
			\item \optword{b} (bottom) en bas d’une page (celle-ci ou la suivante) ;
			\item \optword{p} (page of floats) sur une page spéciale ne contenant pas de texte mais
			uniquement des tableaux et des figures.
			
		\end{itemize}
		
	\end{itemize}
\end{minipage}
%
\begin{minipage}{0.40\textwidth}
\begin{exampleblock}{exemple: tableaux2.tex}
\scriptsize\bfseries
\begin{lstlisting}
\documentclass{article}
		
\begin{document}
   \begin{table}[!htp]
      \begin{tabular}{|l|c|r|}
         \hline
         l1c1 & l1c2 & l1c3 \\
         \hline 
         l2c1 & l2c2 & l2c3 \\
         \hline
      \end{tabular}
   \end{table}
\end{document}
\end{lstlisting}
\end{exampleblock}
\end{minipage}

\end{frame}

\subsection{Images}

\begin{frame}[fragile]
\frametitle{Enrichir un document: Images}
\framesubtitle{Insérer une image}

\begin{itemize}
	\item utiliser la commande \keyword{\textbackslash includegraphics} de l'extension \optword{graphicx}
	\item on peut fixer la largeur (\optword{width}) et la hauteur (\optword{height})
	\item pour récupérer la largeur du texte, utiliser la commande \keyword{\textbackslash textwidth}
	\item pour récupérer la hauteur du texte, utiliser la commande \keyword{\textbackslash textheight}
\end{itemize}

\begin{exampleblock}{exemple: image1.tex}
\scriptsize\bfseries
\begin{lstlisting}
\documentclass{article}
\usepackage{graphicx}
\begin{document}
    \includegraphics[width=4cm, height=2cm]{banner.jpg}
\end{document}
\end{lstlisting}
\end{exampleblock}

\end{frame}

\begin{frame}[fragile]
\frametitle{Enrichir un document: Images}
\framesubtitle{Positionner une figure}

\begin{minipage}{0.54\textwidth}
	\begin{itemize}
		\item utiliser l'environnement \optword{figure}
		\item l'argument définit le positionnement en utilisant les lettres:
		\begin{itemize}
			\item \optword{!} demande à \LaTeX\ de faire tout son possible pour respecter l'ordre indiqué
			ensuite ;
			\item \optword{h} (here) ici, si possible ;
			\item \optword{t} (top) en haut d’une page (celle-ci ou la suivante) ;
			\item \optword{b} (bottom) en bas d’une page (celle-ci ou la suivante) ;
			\item \optword{p} (page of floats) sur une page spéciale ne contenant pas de texte mais
			uniquement des tableaux et des figures.
			
		\end{itemize}
		
	\end{itemize}
\end{minipage}
%
\begin{minipage}{0.45\textwidth}
\begin{exampleblock}{exemple: image2.tex}
\scriptsize\bfseries
\begin{lstlisting}
\documentclass{article}
\usepackage{graphicx}

\begin{document}
   \begin{figure}[!htp]
      \includegraphics{banner.jpg}
   \end{figure}
\end{document}
\end{lstlisting}
\end{exampleblock}
\end{minipage}

\end{frame}

\subsection{Liens et renvoi}

\begin{frame}[fragile]
\frametitle{Enrichir un document: Liens et renvoi}
\framesubtitle{Insérer un lien hypertexte}

\begin{itemize}
	\item importer l'extension \optword{hyperref}
	\item utiliser la commande \keyword{\textbackslash url} pour afficher un URL
	\item utiliser la commande \keyword{\textbackslash href} pour afficher une description avec un lien vers un URL
\end{itemize}

\begin{exampleblock}{exemple: liens.tex}
\scriptsize\bfseries
\begin{lstlisting}
\documentclass{article}
\usepackage{hyperref}
		
\begin{document}
   Le site de l'ESI : \url{http://esi.dz/}
   
   \href{http://esi.dz/}{Cliquer ici pour ouvrir le site de l'ESI}
\end{document}
\end{lstlisting}
\end{exampleblock}


\end{frame}

\begin{frame}[fragile]
\frametitle{Enrichir un document: Liens et renvoi}
\framesubtitle{Renvoi vers un emplacement dans le document}

\begin{itemize}
	\item importer l'extension \optword{hyperref}
	\item utiliser la commande \keyword{\textbackslash hypertarget} pour marquer la destination
	\item utiliser la commande \keyword{\textbackslash hyperlink} pour ajouter un lien vers la destination
\end{itemize}

\begin{exampleblock}{exemple: liens.tex}
\scriptsize\bfseries
\begin{lstlisting}
\documentclass{article}
\usepackage{hyperref}

\begin{document}
    Aller vers \hyperlink{mapage2}{la page suivante}
    \newpage
    \hypertarget{mapage2}{Ici} c'est la destination
\end{document}
\end{lstlisting}
\end{exampleblock}

\end{frame}

\subsection{Formules et symboles}

\begin{frame}
\frametitle{Enrichir un document: Formules et symboles}
\framesubtitle{Insérer une formule}

\end{frame}

\begin{frame}
\frametitle{Enrichir un document: Formules et symboles}
\framesubtitle{Insérer des symboles}

\end{frame}

%===================================================================================
\section{Références}
%===================================================================================

\begin{frame}
\frametitle{Références}

\end{frame}

\subsection{Notes de bas de page}

\begin{frame}
\frametitle{Références}
\framesubtitle{Notes de bas de page}

\end{frame}

\subsection{Table de matières}

\begin{frame}
\frametitle{Références}
\framesubtitle{Table de matières}

\end{frame}

\subsection{Légendes}

\begin{frame}
\frametitle{Références}
\framesubtitle{Légendes}

\end{frame}

\subsection{Index}

\begin{frame}
\frametitle{Références}
\framesubtitle{Index}

\end{frame}

%%===================================================================================
%\section{Révision et partage}
%%===================================================================================
%
%\begin{frame}
%\frametitle{Révision et partage}
%
%\end{frame}
%
%\subsection{Vérification}
%
%\begin{frame}
%\frametitle{Révision et partage}
%\framesubtitle{Vérification}
%
%\end{frame}
%
%\subsection{Commentaires et suivi}
%
%\begin{frame}
%\frametitle{Révision et partage}
%\framesubtitle{Commentaires et suivi}
%
%\end{frame}
%
%\subsection{Comparaison}
%
%\begin{frame}
%\frametitle{Révision et partage}
%\framesubtitle{Comparaison}
%
%\end{frame}
%
%\subsection{Partage}
%
%\begin{frame}
%\frametitle{Révision et partage}
%\framesubtitle{Partage}
%
%
%\end{frame}
%
%\begin{frame}
%\frametitle{Révision et partage: Partage}
%\framesubtitle{Protection}
%
%\end{frame}
%
%\begin{frame}
%\frametitle{Révision et partage: Partage}
%\framesubtitle{Impression}
%
%\end{frame}
%
%\begin{frame}
%\frametitle{Révision et partage: Partage}
%\framesubtitle{Sauvegarde}
%
%\end{frame}
%
%\begin{frame}
%\frametitle{Révision et partage: Partage}
%\framesubtitle{Sauvegarde sur cloud}
%
%
%\end{frame}

\begin{frame}
\frametitle{Rédaction d'un document numérique: \LaTeX}
\framesubtitle{Un peu d'humeur}

\vgraphpage{latex-humour.png}

\end{frame}

\insertbibliography{Bweb03}{*}


\end{document}

