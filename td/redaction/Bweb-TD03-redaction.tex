% !TEX TS-program = pdflatex
% !TeX program = pdflatex
% !TEX encoding = UTF-8
% !TEX spellcheck = fr

\documentclass[11pt, a4paper]{article}
%\usepackage{fullpage}
\usepackage[left=1cm,right=1cm,top=1cm,bottom=2cm]{geometry}
\usepackage[fleqn]{amsmath}
\usepackage{amssymb}

\usepackage[T1]{fontenc}
\usepackage[utf8]{inputenc}
\usepackage[french,english]{babel}
\usepackage{txfonts} 
\usepackage[]{graphicx}
\usepackage{multirow}
\usepackage{hyperref}
\usepackage{parskip}
\usepackage{multicol}

\usepackage{turnstile}%Induction symbole

\renewcommand{\baselinestretch}{1}

\setlength{\parindent}{0pt}


\begin{document}

\selectlanguage {french}
%\pagestyle{empty} 

\noindent
\begin{tabular}{ll}
\multirow{3}{*}{\includegraphics[width=2cm]{../esi-logo.png}} & \'Ecole national Supérieure d'Informatique\\
& 1ère année cycle préparatoire\\
& Bureautique et Web
\end{tabular}\\[.25cm]
\noindent\rule{\textwidth}{1pt}\\%[-0.25cm]
\begin{center}
{\LARGE \textbf{T.D. Rédaction d'un document numérique}}
%\begin{flushright}
%	ARIES Abdelkrime
%\end{flushright}
\end{center}
\noindent\rule{\textwidth}{1pt}

\section*{Exercice 1 : Rédaction d'un rapport avec Word}

\vspace{-12pt}
\begin{tabular}{|p{\textwidth}|}
	\hline\\
	Objectif : l'étudiant doit pouvoir utiliser Word pour rédiger un rapport complet \\\\
	\hline
\end{tabular}

\subsection*{Préparation du document}

\begin{enumerate}
	\item Créer un nouveau document avec les marges : 3cm gauche et 2cm pour le reste 
	\item Mettre à jour le style "Normal" selon cette spécification : La police utilisée par défaut est "Times New Roman", 12pt.
	\item Ajouter un saut de section de page. La 1ère page sera la page de garde.  
	\begin{multicols}{2}
	\begin{itemize}
		\item Elle doit contenir l'entête de l'ESI (entete.png). 
		\item Le mot "Rapport" en gras, 16pt, centré. 
		\item Ajouter le titre "Introduction à la programmation" dans un cadre. 
		\begin{itemize}
			\item Le cadre est un tableau avec une cellule ayant des marges 0.25cm partout, et un contour de 1.25pt
			\item le titre doit être en gras, 20pt, centré.
		\end{itemize}
		\item Insérer un tableau avec deux cellules: la première contient l'expression "Encadré par :" en gras, retour à la ligne et le nom de votre enseignant du TD ; la deuxième contient l'expression "Réalisé par :" en gras, retour à la ligne et les noms des étudiants. Effacer les bordures.
		\item En bas écrire "Année : " suivi de l'année, en gras, 14 pt.
	\end{itemize}
	\end{multicols}

	\item Ajouter des sauts de pages pour créer :  
	\begin{multicols}{3}
		\begin{enumerate}
			\item une page pour le remerciement, 
			\item une pour la dédicace, 
			\item une pour le résumé, 
			\item une pour la table de matière, 
			\item une pour la liste des figures, 
			\item une pour la liste des tableaux 
			\item et une pour la liste des abréviations.
		\end{enumerate}
	\end{multicols}
	
	\item Numéroter les pages en utilisant les nombres romans (I, II, etc.). La première page ne doit pas être numérotée (différente).
	\item Sauter la section (page suivante) et copier-coller le contenu du rapport (rapport.txt)
	\item Dans la page où il y a "Introduction [CHAPITRE*]", modifier le numéro de page de cette section : 
	\begin{multicols}{2}
	\begin{enumerate}
		\item Découcher "Lier au précédent" 
		\item Laisser "Première page différente" couchée (pour ne pas avoir une entête dans la page du chapitre) 
		\item Format des numéros de page : numéros arabes (1, 2, etc.) et numérotation à partir de 1.
		\item Ajouter un numéro de page à la première page du chapitre (s'il n'existe pas)
	\end{enumerate}
	\end{multicols}
	
	\item Les chapitres sont marqués par "[CHAPITRE]". Insérer un saut de section (page suivante) avant le deuxième chapitre après l'introduction. 
	\begin{itemize}
		\item Format des numéros de page : numérotation à la suite de la section précédente
		\item Insérer des sauts de section (page suivante) avant le reste des chapitres. 
	\end{itemize}

	\item Choisir Le premier paragraphe et appliquer les mises en forme suivantes : 
	\begin{itemize}
		\item Texte justifié 
		\item Options paragraphe : interligne 1.5, retrait de première ligne 1cm, espacement avant et après 6pt
	\end{itemize}

	\item Mettre à jour le style "Paragraphe" selon ces mises en forme
	\item Sélectionner le texte copié et appliquer le style "Paragraphe"  
	\item Créer des styles pour les chapitres, les sections et les sous section. Pour ce faire, suivre les étapes suivantes : 
	\begin{multicols}{2}
	\begin{enumerate}
		\item Sélectionner "[CHAPITRE]" et appliquer le style "Normal" 
		\item Appliquer les options : Arial, 20pt, gras, centré, espacement avant 0pt après 24pt
		\item Créer un nouveau style appelé : "monChapitre*", puis un autre : "monChapitre". Si on clique sur modifier le style, on remarque que "monChapitre" est basé sur "monChapitre*".
		\item Sélectionner "[TITRE1]" et appliquer le style "Normal" 
		\item Appliquer les options : Arial, 16pt, gras, aligné à gauche, espacement avant 12pt après 6pt
		\item Créer un nouveau style appelé : "maSection*", puis un autre : "maSection"
		\item Chercher "[TITRE2]", sélectionner l'expression et appliquer le style "Normal"
		\item Appliquer les options : Arial, 14pt, gras, aligné à gauche, espacement avant 6pt après 6pt
		\item Créer un nouveau style appelé : "maSousSection"
	\end{enumerate}
	\end{multicols}
	
	\item Créer des styles des listes à plusieurs niveaux (pour numéroter les chapitres, les sections et les sous-sections). Pour ce faire, suivre les étapes suivantes : 
	\begin{multicols}{2}
	\begin{enumerate}
		\item Cliquer sur "liste à plusieurs niveaux" (Onglet : Accueil, Groupe : Paragraphe)
		\item Choisir "définir un nouveau style de liste". Nommer le style "monStyleListe"
		\item Cliquer sur "Format" et choisir "Numérotation"
		\item Cliquer sur "Définir pour tous les niveaux" et mettre 0cm partout
		\item Cliquer sur "Plus \textgreater \textgreater" pour afficher plus d'options
		\item Style de nombre pour ce niveau : I, II, III, ...
		\item Mise en forme de la numérotation : supprimer le parenthèse pour avoir \textbf{I}
		\item Style à appliquer à ce niveau : monChapitre
		\item Choisir le niveau 2
		\item Style de nombre pour ce niveau : 1, 2, 3, ...
		\item Mise en forme de la numérotation : supprimer le parenthèse pour avoir \textbf{1}
		\item Style à appliquer à ce niveau : maSection
		\item Niveau à afficher dans la galerie : Niveau 3
		\item Choisir le niveau 3
		\item Style de nombre pour ce niveau : 1, 2, 3, ...
		\item Mise en forme de la numérotation : supprimer le parenthèse pour avoir \textbf{1} ; mettre le curseur avant le chiffre ; Inclure le numéro de niveau à partir de : Niveau 2 ; ajouter un point pour avoir \textbf{1.1}
		\item Style à appliquer à ce niveau : maSousSection
		\item Niveau à afficher dans la galerie : Niveau 3
	\end{enumerate}
	\end{multicols}

	\item Appliquer les styles sur les sections. Utiliser la recherche pour chercher les mots suivants : 
	\begin{multicols}{2}
	\begin{itemize}
		\item "[CHAPITRE]" appliquer le style "monChapitre"
		\item "[TITRE1]" appliquer le style "maSection"
		\item "[TITRE2]" appliquer le style "maSousSection"
		\item "[CHAPITRE*]" appliquer le style "monChapitre*"
		\item "[TITRE1*]" appliquer le style "maSection*"
	\end{itemize}
	\end{multicols}
	
	\item Cliquer sur la page du chapitre 1 et ajouter une entête vide 
	\item Découcher "Lier au précédent", Coucher "Première page différente" et "Pages paires et impaires différentes"
	\item Cliquer sur "Suivant" pour passer à l'entête de la page suivante.
	\item Cliquer sur "QuickPart" et sélectionner "Champs", ensuite chercher "RéfStyle" dans les noms des champs et sélectionner "maSection" dans le nom de style. Retour à la ligne ; écrire trois tirets "- - -" ; retour à la ligne pour avoir un trait.
	\item Cliquer "Suivant ; écrire "Chapitre : " ; ajouter une référence style "monChapitre" (comme l'étape précédente) ; retour à la ligne ; écrire trois tirets ; retour à la ligne pour avoir un trait.
	\item On remarque qu'il y a des pages qui n'ont pas des  numéros de pages. Choisir une et ajouter le numéro de page
	
	\item Chercher "[GTABLEAU]" : c'est un grand tableau dont la page ne peut pas contenir. 
	\begin{enumerate}
		\item Avant ce tableau, ajouter un saut de section (Page suivante) et après aussi
		\item Revenir à la page du tableau et sélectionner "Paysage" comme orientation.
		\item Dans l'entête de cette page et celle suivante, découcher "Première page différente"
	\end{enumerate}
	
	\item Chercher "[PUCES]" ; sélectionner les lignes suivantes marquées par "-." et transformer les en une liste à puces
	\item Chercher "[CODE]" ; sélectionner la ligne ; police "Courier New" ; créer un nouveau style appelé "monCode" ; appliquer le sur le reste des codes.
	
\end{enumerate}

\subsection*{Enrichissement du document}

\begin{enumerate}
	\item Chercher "[TABLEAU]" et convertir le texte vers un tableau. 
	\item Ajouter la légende indiquée entre les crochets. Elle doit être au-dessus du tableau. 
	\item Avant le tableau, il y a un texte "[REF]". Remplacer le par une renvoi vers ce tableau :
	\begin{enumerate}
		\item Aller vers l'onglet "Références", groupe "Légendes", et choisir "Renvoi"
		\item Choisir la catégorie "Tableau", sélectionner la légende du tableau, insérer un renvoi à : Texte et numéro uniquement.
	\end{enumerate}
	\item Chercher "[GTABLEAU]" et le remplacer par le tableau dans "gtableau.docx"
	\begin{enumerate}
		\item Fusionner les entêtes du tableau (ex. la cellule "Texte" doit être fusionnée avec la suivante)
		\item Effacer les bordures en haut à gauche
		\item Appliquer la même chose que "[TABLEAU]"
	\end{enumerate}
	\item Chercher "[IMAGE]" et insérer l'image "compile.png". 
	\item Régler la luminosité de l'image afin que l'arrière plan soit en blanc.
	\item Ajouter la légende "Les étapes de compilation". Elle doit être au-dessous de la figure. 
	\item Avant la figure, il y a un texte "[REF]". Remplacer le par une renvoi vers cette figure :
	\begin{enumerate}
		\item Aller vers l'onglet "Références", groupe "Légendes", et choisir "Renvoi"
		\item Choisir la catégorie "Figure", sélectionner la légende de la figure, insérer un renvoi à : Texte et numéro uniquement.
	\end{enumerate}
	\item Chercher "[SCHEMA]" et dessiner (insérer une zone de dessin pour contenir le schéma) le schéma indiqué par "interprete.png"
	\item Ajouter la légende de figure "Les étapes de l'interprétation"
	\item Avant la figure, il y a un texte "[REF]". Remplacer le par une renvoi vers cette figure
	\item Chercher "[EQ1]" et remplacer ce texte par une équation $ (765)_{10} = 7 \times 10^2 + 6 \times 10^1 + 5 \times 10^0 $
	\item Chercher "[EQ2]" et remplacer ce texte par une équation $ (11001)_{2} = 1 \times 2^4 + 1 \times 2^3 + 0 \times 2^2 + 0 \times 2^1 + 1 \times 2^0 = (29)_{10} $
	\item Chercher "[NOTEBP]" et mettre ce qui est entre les deuxièmes crochets comme note de bas de page. 
	\item  Chercher "[LIEN]" et ajouter un lien hypertexte : le texte est dans les deuxièmes crochets et le lien est dans les troisièmes.
	\item Dans la page "liste des abréviations", insérer un tableau de 2 colonnes et 5 lignes
	\item Ajouter les abréviations suivantes : 
	\begin{multicols}{2}
		\begin{itemize}
			\item ENIAC : Electronic Numerical Integrator And Computer
			\item UTF-8 : Unicode Transformation Format - 8 bits
			\item ASCII : American Standard Code for Information Interchange
			\item DOS : Disk Operating System
		\end{itemize}
	\end{multicols}
	\item Trier le tableau selon la première colonne
\end{enumerate}

\subsection*{Références}

\begin{enumerate}
	\item Chercher "[NOTEBP]" et mettre ce qui est entre les deuxièmes crochets comme note de bas de page. 
	\item Dans la page "liste des figures", insérer la liste des figures
	\item Dans la page "liste des tableaux", insérer la liste des tableaux
	\item Dans la page "table de matières", insérer une table de matière personnalisée
	\begin{itemize}
		\item Dans "Options" affecter le niveau 1 aux styles "monChapitre" et "monChapitre*" ; affecter le niveau 2 au style "maSection" ; affecter le niveau 3 au style "maSousSection"
	\end{itemize}
	\item Indexer les mots suivants : calculateur, processeur, programmation. Ajouter un renvoi pour avoir \textbf{UCT ... voir processeur}. Insérer l'index dans une page à part à la fin du rapport. 
	\item Ajouter les ressources bibliographiques indiquées dans le fichier "bib.text"
	\item Chercher "[BIB]" et insérer la citation référencée par un numéro (deuxièmes crochets)
	\item Insérer une table bibliographique en fin du document (nouvelle page). 
	\item Choisir le style "APA" 
\end{enumerate}

\subsection*{Finalisation}

\begin{enumerate}
	\item Si ce n'est pas déjà fait, nettoyer le document des marqueurs "[CHAPITRE]", "[CHAPITRE*]", "[TITRE1]", "[TITRE1*]", "[TITRE2]", "[REF]", etc. Utiliser l'outil de remplacement pour les remplacer par un vide. 
	\item Choisir des remerciements depuis le fichier "remerciement.txt" et insérer les dans la page de remerciement (avec modification). 
	\item Choisir des dédicaces depuis le fichier "dedicace.txt" et insérer les dans la page de dédicaces
	\item Insérer le résumé depuis le fichier "resume.txt" dans la page de résumé 
	\item Vérifier l'orthographe
	\item Sauvegarder le document s'il n'est pas déjà sauvegardé et exporter le comme PDF.
\end{enumerate}

\newpage
\section*{Exercice 2 : Rédaction d'un CV avec Word}

\vspace{-12pt}
\begin{tabular}{|p{\textwidth}|}
	\hline\\
	Objectif : l'étudiant doit pouvoir rédiger son CV avec Word \\\\
	\hline
\end{tabular}

\begin{enumerate}
	\item Choisir un modèle de CV parmi ceux fournis avec ce TD, ou télécharger un autre 
	\item Rédiger votre CV
\end{enumerate}

\section*{Exercice 3 : Rédaction d'un rapport avec \LaTeX}

\vspace{-12pt}
\begin{tabular}{|p{\textwidth}|}
	\hline\\
	Objectif : l'étudiant doit pouvoir utiliser \LaTeX pour rédiger un rapport complet \\\\
	\hline
\end{tabular}

\begin{enumerate}
	\item L'exercice sera complété ultérieurement
\end{enumerate}

\section*{Exercice 4 : Rédaction d'un CV avec \LaTeX}

\vspace{-12pt}
\begin{tabular}{|p{\textwidth}|}
	\hline\\
	Objectif : l'étudiant doit pouvoir rédiger son CV avec \LaTeX \\\\
	\hline
\end{tabular}

\begin{enumerate}
	\item Utiliser "moderncv" pour créer votre CV
\end{enumerate}

\end{document}
