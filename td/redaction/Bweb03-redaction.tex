\documentclass[11pt, a4paper]{article}
%\usepackage{fullpage}
\usepackage[left=1cm,right=1cm,top=1cm,bottom=2cm]{geometry}
\usepackage[fleqn]{amsmath}
\usepackage{amssymb}

\usepackage[T1]{fontenc}
\usepackage[utf8]{inputenc}
\usepackage[french,english]{babel}
\usepackage{txfonts} 
\usepackage[]{graphicx}
\usepackage{multirow}
\usepackage{hyperref}
\usepackage{parskip}

\usepackage{turnstile}%Induction symbole

\renewcommand{\baselinestretch}{1}

\setlength{\parindent}{0pt}


\begin{document}

\selectlanguage {french}
%\pagestyle{empty} 

\noindent\rule{\textwidth}{1pt}\\[0.25cm]
%\noindent
%\begin{tabular}{ll}
%\multirow{2}{*}{\includegraphics[width=2cm]{img/esi-logo.png}} & \'Ecole national Supérieure d'Informatique\\
%& 3ième année cycle commun\\
%\end{tabular}\\[.25cm]
\noindent
\begin{center}
{\LARGE \textbf{T.D. sur les grammaires}}\\
\begin{flushright}
	Abdelkrime Aries
\end{flushright}
\end{center}
\noindent\rule{\textwidth}{1pt}

\section*{Exercice 1 : Rédaction d'un rapport avec Word}

\subsection*{Préparation du document}

\begin{enumerate}
	\item Créer un nouveau document avec les marges : 3cm gauche et 2cm pour le reste 
	\item Mettre à jour le style "Normal" selon cette spécification : La police utilisée par défaut est "Times New Roman", 12pt.
	\item Ajouter un saut de section de page. La 1ère page sera la page de garde.  
	\begin{itemize}
		\item Elle doit contenir l'entête de l'ESI (entete.png). 
		\item Le mot "Rapport" en gras, 16pt, centré. 
		\item Ajouter le titre "Introduction à la programmation" dans un cadre (tableau avec une cellule), le titre doit être en gras, 18pt, centré. 
		\item Insérer un tableau avec deux cellules: la première contient l'expression "Encadré par :" en gras, retour à la ligne et le nom de votre enseignant du TD ; la deuxième contient l'expréssion "Réalisé par :" en gras, retour à la ligne et les noms des étudiants. Effacer les bordures.
		\item En bas écrire "Année : " suivi de l'année, en gras, 14 pt.
	\end{itemize}

	\item Ajouter des sauts de pages pour créer :  
	\begin{itemize}
		\item une page pour le remerciement, 
		\item une pour la dédicace, 
		\item une pour le résumé, 
		\item une pour la table de matière, 
		\item une pour la liste des figures, 
		\item une pour la liste des tableaux 
		\item et une pour la liste des abbréviations.
	\end{itemize}

	\item Numéroter les pages en utilisant les nombres romans (I, II, etc.). La première page ne doit pas être numérotée (différente).
	\item Sauter la section (page suivante) et copier-coller le contenu du rapport (rapport.txt)
	\item Dans la page où il y a "Introduction [CHAPITRE]", modifier le numéro de page de cette section : 
	\begin{itemize}
		\item Découcher "Lier au précédent" 
		\item Laisser "Première page différente" couchée (pour ne pas avoir une entête dans la page du chapitre) 
		\item Format des numéros de page : numéros arabes (1, 2, etc.) et numérotation à partir de 1.
		\item Ajouter un numéro de page à la première page du chapitre (s'il n'existe pas)
	\end{itemize}
	
	\item Les chapitres sont marqués par [CHAPITRE]. Insérer un saut de section (page suivante) avant le deuxième chapitre après l'introduction. 
	\begin{itemize}
		\item Format des numéros de page : numérotation à la suite de la section précédente
		\item Insérer des sauts de section (page suivante) avant le reste des chapitres. 
	\end{itemize}

	\item Choisir Le premier paragraphe et appliquer les mises en forme suivantes : 
	\begin{itemize}
		\item Texte justifié 
		\item Options paragraphe : interligne 1.5, retrait de première ligne 1cm, espacement avant et après 6pt
	\end{itemize}

	\item Mettre à jour le style "Paragraphe" selon ces mises en forme
	\item Sélectionner le texte copié et appliquer le style "Paragraphe"  
	\item Créer un style pour les chapitres. Pour ce faire, suivre les étapes suivantes :  
	\begin{itemize}
		\item Chercher "Notions [CHAPITRE]" et cliquer sur la ligne où il se trouve cette expression 
		\item Cliquer sur "liste à plusieurs niveaux" (Onglet : Acceuil, Groupe : Paragraphe)
		\item Choisir "définir un nouveau format de numérotation 
	\end{itemize}
\end{enumerate}


\end{document}
