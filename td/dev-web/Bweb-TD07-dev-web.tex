% !TEX TS-program = pdflatex
% !TeX program = pdflatex
% !TEX encoding = UTF-8
% !TEX spellcheck = fr

\documentclass[11pt, a4paper]{article}
%\usepackage{fullpage}
\usepackage[left=1cm,right=1cm,top=1cm,bottom=2cm]{geometry}
\usepackage[fleqn]{amsmath}
\usepackage{amssymb}
%\usepackage{indentfirst}
\usepackage[T1]{fontenc}
\usepackage[utf8]{inputenc}
\usepackage[french,english]{babel}
\usepackage{txfonts} 
\usepackage[]{graphicx}
\usepackage{multirow}
\usepackage{hyperref}
\usepackage{parskip}
\usepackage{multicol}
\usepackage{wrapfig}

\usepackage{turnstile}%Induction symbole

\renewcommand{\baselinestretch}{1}

\setlength{\parindent}{24pt}


\begin{document}

\selectlanguage {french}
%\pagestyle{empty} 

\noindent
\begin{tabular}{ll}
\multirow{3}{*}{\includegraphics[width=2cm]{../esi-logo.png}} & \'Ecole national Supérieure d'Informatique\\
& 1ère année cycle préparatoire\\
& Bureautique et Web
\end{tabular}\\[.25cm]
\noindent\rule{\textwidth}{1pt}\\%[-0.25cm]
\begin{center}
{\LARGE \textbf{T.D. Développement web}}
%\begin{flushright}
%	ARIES Abdelkrime
%\end{flushright}
\end{center}
\noindent\rule{\textwidth}{1pt}

\section*{Exercice : HTML}

\vspace{-12pt}
\begin{tabular}{|p{\textwidth}|}
	\hline\\
	Objectif : l'étudiant doit pouvoir : 
	\begin{itemize}
		\item créer un site web statique 
		\item comprendre et utiliser les éléments HTML
		\item structurer les pages en utilisant les éléments sémantiques HTML5
		\item créer des formulaires en utilisant HTML
	\end{itemize}\\
	\hline
\end{tabular}

Votre rapport "Introduction à la programmation" a été transformé à un livre. 
Pour promouvoir votre livre, vous avez décidé de créer un petit site web statique. 
Afin d'enrichir le site avec des photos, vous avez envisagé de présenter quelques photos de votre wilaya.

\begin{enumerate}
	\item Choisir un éditeur de texte 
	\item Créer un dossier \textbf{prog} et recopier le dossier \textbf{img} dedans
	\item Dans le dossier \textbf{prog}, créer une page \textbf{index.html}
	\item Cette page doit avoir l'entête suivante: 
\begin{verbatim}
<head>
  <meta charset="utf-8">
  <meta name="author" content="[VOTRE NOM]">
  <meta name="description" content="Livre sur la programmation">
  <meta name="keywords" content="livre, programmation">
  <title>Introduction à la programmation</title>
</head>	
\end{verbatim}
	\item La page doit contenir une entête (la bannière et le titre), un menu, un contenu principale et le pied de page. 
	\begin{itemize}
		\item L'entête est un élément \textbf{<header>} qui contient un \textbf{<div id="banner">} contenant une image \textbf{img/banner.png}. 
		L'élément \textbf{<div>} est suivi par un élément de type \textbf{<h1>} contenant le titre "Introduction à la programmation"
		\item le menu est un élément \textbf{<nav>}. 
		Il contient des listes à puce avec 3 éléments. 
		Chaque élément de liste contient un lien hypertexte vers une page. 
		Donc, on doit avoir 3 liens : un vers le menu principal (\textbf{index.html}), 
		un vers la page de présentation de l'auteur (\textbf{auteur.html})
		et un vers la page de contact (\textbf{contact.html})
		\item le contenu principal est un élément de type \textbf{<main>}. 
		Laisser le vide pour l'instant.
		\item le pied de page est un élément de type \textbf{<footer>}. 
		Il contient le mot \textbf{Copyright} suivi par la marque déposée (\textbf{\&copy;}) suivi par l'année suivi par votre nom complet. 
		Un utilisateur qui clique sur votre nom doit pouvoir envoyer un mail vers votre boite mail.
	\end{itemize}
	
	\item dupliquer la page \textbf{index.html} pour avoir deux autres pages : \textbf{auteur.html} et \textbf{contact.html}
	\item dans chaque page, remplacer la référence vers elle-même dans le menu par \textbf{\#}. 
	Ceci pour empêcher le rechargement de la même page. 
	\item En plus, ajouter l'attribut \textbf{class="actif"} à l'élément de liste \textbf{<li>} de la page en cours.
	\item Ouvrir les pages dans un navigateur et à chaque modification dans la source actualiser la page pour voir le résultat.
	
\end{enumerate}

\subsection*{La page "index.html"}

Dans l'élément \textbf{<main>}, insérer le contenu principale de la page  
\begin{enumerate}
	\item Créer une section \textbf{Information}. Une section est un élément de type \textbf{<section>} avec un titre de niveau 2.
	\item Dans cette section, créer le tableau suivant : 
	\begin{tabular}{|l|l|l|}
		\hline
		\multirow{2}{*}{\textbf{Information}} & \multicolumn{2}{c|}{\textbf{Edition}} \\
		\cline{2-3}
		& \textbf{Edition 1} & \textbf{Edition 2} \\
		\hline
		\textbf{Titre} & \multicolumn{2}{l|}{Introduction à la programmation} \\
		\hline
		\textbf{auteur} & \multicolumn{2}{l|}{VOTRE NOM COMPLET} \\
		\hline
		\textbf{année} & 2019 & 2020 \\
		\hline
		\textbf{\#pages} & 30 & 50 \\
		\hline
	\end{tabular}
	\item Les textes en gras sont des titres. Les bordures du tableau ne sont pas visibles (on va utiliser CSS après).
	\item Créer une autre section \textbf{Description} et recopier le contenu à partir de \textbf{description.text}
	\item mettre le nom \textbf{Rick Cook} en emphase fort. 
	\item définir le titre "The Wizardry Compiled" comme liens hypertexte vers \\ \url{https://www.amazon.com/Wizardry-Compiled-Rick-Cook/dp/0671698567}\\
	La page du lien doit s'ouvrir dans une nouvelle onglet. 
	\item définir ces paroles comme citation en bloc.
	\item créer une deuxième section \textbf{Contenu} et recopier le contenu à partir de \textbf{contenu.txt}. 
	La description de chaque chapitre doit être dans un nouveau paragraphe.
	\item avant la deuxième paragraphe, ajouter le code suivant : 
\begin{verbatim}
<aside>
  Plus de concepts seront ajoutées dans des versions ultérieurs.
</aside>
\end{verbatim}
	
\end{enumerate}

\subsection*{La page "auteur.html"}

Dans l'élément \textbf{<main>}, insérer le contenu principale de la page  
\begin{enumerate}
	\item ajouter une section \textbf{A propos de l'auteur}. Mettre votre nom en emphase fort et ajouter une petite description. 
	Par exemple : 
\begin{verbatim}
<section id="moi">
  <h2>A propos de l'auteur</h2>
  <p>
    <strong>ARIES Abdelkrime</strong> :
    ingénieur en informatique de l'université de Jijel,
    magister en informatique de l'école nationale supérieure d'informatique d'Alger.
  </p>
</section>	
\end{verbatim}
	\item chercher des images sur votre wilaya (au moins 3 images). Si leurs tailles sont grandes, redimensionner les (de préférence, les images doivent avoir la même largeur et hauteur). 
	Insérer les dans le dossier \textbf{prog/img/}
	\item ajouter une deuxième section \textbf{Quelques images de WILAYA}. 
	\item dans cette  section, après le titre, ajouter le code suivant : 
\begin{verbatim}
<div id="slideshow">
  <div class="slide-wrapper">
    <div class="slide">...</div>
    <div class="slide">...</div>
    <div class="slide">...</div>
  </div>
</div>	
\end{verbatim}
	\item remplacer les trois points par une image (la balise \textbf{<img>}). 
	\item si vous avez plus de trois images, recopier la ligne du \textbf{<div class="slide">}
\end{enumerate}

\subsection*{La page "contact.html"}

Dans l'élément \textbf{<main>}, insérer le contenu principale de la page  
\begin{enumerate}
	\item ajouter un formulaire vers le site \url{https://www.monsite.com} avec la méthode \textbf{POST}. 
	\item dans le formulaire, ajouter une section \textbf{Informations personnelles}. 
	Elle doit contenir : 
	\begin{itemize}
		\item le nom et le prénom séparés. Ces champs doivent être obligatoires. 
		\item la date de naissance. Le contact doit être âgé entre 17 ans et 80 ans.
		\item l'émail 
		\item niveau d'étude : 1CP, 2CP, 1CS, 2CS, 3CS, Licence, Master ou Autre (liste déroulante)
		\item expérience : débutant, intermédiaire ou avancé (boutons radio). Par défaut, "débutant" est choisi.
	\end{itemize}
	\item ajouter une autre section \textbf{Opinion}. Cette section doit contenir : 
	\begin{itemize}
		\item Quelles sont les éléments à améliorer : les commentaires, les variables ou les instructions (sous forme de cases à coucher).
		\item Choisir une couleur pour la page de garde (un champ de couleur). La valeur par défaut est bleu (\textbf{\#0000ff})
		\item Votre opinion (un élément \textbf{<textarea>} avec 3 lignes)
		\item Joindre un fichier. Les fichiers acceptés sont de type PDF (\textbf{application/pdf})
	\end{itemize}
	\item ajouter un bouton pour envoyer le formulaire
\end{enumerate}




\section*{Exercice : CSS}

\vspace{-12pt}
\begin{tabular}{|p{\textwidth}|}
	\hline\\
	Objectif : l'étudiant doit pouvoir utiliser CSS afin d'améliorer la présentation de son site web \\\\
	\hline
\end{tabular}

\begin{enumerate}
	\item 
\end{enumerate}

\end{document}
