% !TEX TS-program = pdflatex
% !TeX program = pdflatex
% !TEX encoding = UTF-8
% !TEX spellcheck = fr

\documentclass[11pt, a4paper]{article}
%\usepackage{fullpage}
\usepackage[left=1cm,right=1cm,top=1cm,bottom=2cm]{geometry}
\usepackage[fleqn]{amsmath}
\usepackage{amssymb}
%\usepackage{indentfirst}
\usepackage[T1]{fontenc}
\usepackage[utf8]{inputenc}
\usepackage[french,english]{babel}
\usepackage{txfonts} 
\usepackage[]{graphicx}
\usepackage{multirow}
\usepackage{hyperref}
\usepackage{parskip}
\usepackage{multicol}
\usepackage{wrapfig}

\usepackage{turnstile}%Induction symbole

\renewcommand{\baselinestretch}{1}

\setlength{\parindent}{24pt}


\begin{document}

\selectlanguage {french}
%\pagestyle{empty} 

\noindent
\begin{tabular}{ll}
\multirow{3}{*}{\includegraphics[width=2cm]{../esi-logo.png}} & \'Ecole national Supérieure d'Informatique\\
& 1ère année cycle préparatoire\\
& Bureautique et Web
\end{tabular}\\[.25cm]
\noindent\rule{\textwidth}{1pt}\\%[-0.25cm]
\begin{center}
{\LARGE \textbf{T.D. Développement web}}
%\begin{flushright}
%	ARIES Abdelkrime
%\end{flushright}
\end{center}
\noindent\rule{\textwidth}{1pt}

\section*{Exercice : HTML}

\vspace{-12pt}
\begin{tabular}{|p{\textwidth}|}
	\hline\\
	Objectif : l'étudiant doit pouvoir : 
	\begin{itemize}
		\item créer un site web statique 
		\item comprendre et utiliser les éléments HTML
		\item structurer les pages en utilisant les éléments sémantiques HTML5
		\item créer des formulaires en utilisant HTML
	\end{itemize}\\
	\hline
\end{tabular}

Votre rapport "Introduction à la programmation" a été transformé à un livre. 
Pour promouvoir votre livre, vous avez décidé de créer un petit site web statique. 
Afin d'enrichir le site avec des photos, vous avez envisagé de présenter quelques photos de votre wilaya.

\begin{enumerate}
	\item Choisir un éditeur de texte 
	\item Créer un dossier \textbf{prog} et recopier le dossier \textbf{img} dedans
	\item Dans le dossier \textbf{prog}, créer une page \textbf{index.html}
	\item Cette page doit avoir l'entête suivante: 
\begin{verbatim}
<head>
  <meta charset="utf-8">
  <meta name="author" content="[VOTRE NOM]">
  <meta name="description" content="Livre sur la programmation">
  <meta name="keywords" content="livre, programmation">
  <title>Introduction à la programmation</title>
</head>	
\end{verbatim}
	\item La page doit contenir une entête (la bannière et le titre), un menu, un contenu principale et le pied de page. 
	\begin{itemize}
		\item L'entête est un élément \textbf{<header>} qui contient un \textbf{<div id="banner">} contenant une image \textbf{img/banner.png}. 
		L'élément \textbf{<div>} est suivi par un élément de type \textbf{<h1>} contenant le titre "Introduction à la programmation"
		\item le menu est un élément \textbf{<nav>}. 
		Il contient des listes à puce avec 3 éléments. 
		Chaque élément de liste contient un lien hypertexte vers une page. 
		Donc, on doit avoir 3 liens : un vers le menu principal (\textbf{index.html}), 
		un vers la page de présentation de l'auteur (\textbf{auteur.html})
		et un vers la page de contact (\textbf{contact.html})
		\item le contenu principal est un élément de type \textbf{<main>}. 
		Laisser le vide pour l'instant.
		\item le pied de page est un élément de type \textbf{<footer>}. 
		Il contient le mot \textbf{Copyright} suivi par la marque déposée (\textbf{\&copy;}) suivi par l'année suivi par votre nom complet. 
		Un utilisateur qui clique sur votre nom doit pouvoir envoyer un mail vers votre boite mail.
	\end{itemize}
	
	\item dupliquer la page \textbf{index.html} pour avoir deux autres pages : \textbf{auteur.html} et \textbf{contact.html}
	\item dans chaque page, remplacer la référence vers elle-même dans le menu par \textbf{\#}. 
	Ceci pour empêcher le rechargement de la même page. 
	\item En plus, ajouter l'attribut \textbf{class="actif"} à l'élément de liste \textbf{<li>} de la page en cours.
	\item Ouvrir les pages dans un navigateur et à chaque modification dans la source actualiser la page pour voir le résultat.
	
\end{enumerate}

\subsection*{La page "index.html"}

Dans l'élément \textbf{<main>}, insérer le contenu principale de la page  
\begin{enumerate}
	\item Créer une section \textbf{Information}. Une section est un élément de type \textbf{<section>} avec un titre de niveau 2.
	\item Dans cette section, créer le tableau suivant : 
	\begin{tabular}{|l|l|l|}
		\hline
		\multirow{2}{*}{\textbf{Information}} & \multicolumn{2}{c|}{\textbf{Edition}} \\
		\cline{2-3}
		& \textbf{Edition 1} & \textbf{Edition 2} \\
		\hline
		\textbf{Titre} & \multicolumn{2}{l|}{Introduction à la programmation} \\
		\hline
		\textbf{auteur} & \multicolumn{2}{l|}{VOTRE NOM COMPLET} \\
		\hline
		\textbf{année} & 2019 & 2020 \\
		\hline
		\textbf{\#pages} & 30 & 50 \\
		\hline
	\end{tabular}
	\item Les textes en gras sont des titres. Les bordures du tableau ne sont pas visibles (on va utiliser CSS après).
	\item Créer une autre section \textbf{Description} et recopier le contenu à partir de \textbf{description.text}
	\item mettre le nom \textbf{Rick Cook} en emphase fort. 
	\item définir le titre "The Wizardry Compiled" comme liens hypertexte vers \\ \url{https://www.amazon.com/Wizardry-Compiled-Rick-Cook/dp/0671698567}\\
	La page du lien doit s'ouvrir dans une nouvelle onglet. 
	\item définir ces paroles comme citation en bloc.
	\item créer une deuxième section \textbf{Contenu} et recopier le contenu à partir de \textbf{contenu.txt}. 
	La description de chaque chapitre doit être dans un nouveau paragraphe.
	\item avant la deuxième paragraphe, ajouter le code suivant : 
\begin{verbatim}
<aside>
  Plus de concepts seront ajoutées dans des versions ultérieurs.
</aside>
\end{verbatim}
	
\end{enumerate}

\subsection*{La page "auteur.html"}

Dans l'élément \textbf{<main>}, insérer le contenu principale de la page  
\begin{enumerate}
	\item ajouter une section \textbf{A propos de l'auteur}. Mettre votre nom en emphase fort et ajouter une petite description. 
	Par exemple : 
\begin{verbatim}
<section id="moi">
  <h2>A propos de l'auteur</h2>
  <p>
    <strong>ARIES Abdelkrime</strong> :
    ingénieur en informatique de l'université de Jijel,
    magister en informatique de l'école nationale supérieure d'informatique d'Alger.
  </p>
</section>	
\end{verbatim}
	\item chercher des images sur votre wilaya (au moins 3 images). Si leurs tailles sont grandes, redimensionner les (de préférence, les images doivent avoir la même largeur et hauteur). 
	Insérer les dans le dossier \textbf{prog/img/}
	\item ajouter une deuxième section \textbf{Quelques images de WILAYA}. 
	\item dans cette  section, après le titre, ajouter le code suivant : 
\begin{verbatim}
<div id="slideshow">
  <div class="slide-wrapper">
    <div class="slide">...</div>
    <div class="slide">...</div>
    <div class="slide">...</div>
  </div>
</div>	
\end{verbatim}
	\item remplacer les trois points par une image (la balise \textbf{<img>}). 
	\item si vous avez plus de trois images, recopier la ligne du \textbf{<div class="slide">}
\end{enumerate}

\subsection*{La page "contact.html"}

Dans l'élément \textbf{<main>}, insérer le contenu principale de la page  
\begin{enumerate}
	\item ajouter un formulaire vers le site \url{https://www.monsite.com} avec la méthode \textbf{POST}. 
	\item dans le formulaire, ajouter une section \textbf{Informations personnelles}. 
	Elle doit contenir : 
	\begin{itemize}
		\item le nom et le prénom séparés. Ces champs doivent être obligatoires. 
		\item la date de naissance. Le contact doit être âgé entre 17 ans et 80 ans.
		\item l'émail 
		\item niveau d'étude : 1CP, 2CP, 1CS, 2CS, 3CS, Licence, Master ou Autre (liste déroulante)
		\item expérience : débutant, intermédiaire ou avancé (boutons radio). Par défaut, "débutant" est choisi.
	\end{itemize}
	\item ajouter une autre section \textbf{Opinion}. Cette section doit contenir : 
	\begin{itemize}
		\item Quelles sont les éléments à améliorer : les commentaires, les variables ou les instructions (sous forme de cases à coucher).
		\item Choisir une couleur pour la page de garde (un champ de couleur). La valeur par défaut est bleu (\textbf{\#0000ff})
		\item Votre opinion (un élément \textbf{<textarea>} avec 3 lignes)
		\item Joindre un fichier. Les fichiers acceptés sont de type PDF (\textbf{application/pdf})
	\end{itemize}
	\item ajouter un bouton pour envoyer le formulaire
\end{enumerate}




\section*{Exercice : CSS}

\vspace{-12pt}
\begin{tabular}{|p{\textwidth}|}
	\hline\\
	Objectif : l'étudiant doit pouvoir utiliser CSS afin d'améliorer la présentation de son site web \\\\
	\hline
\end{tabular}

\begin{enumerate}
	\item Ajouter le méta \textbf{viewport} aux trois fichiers HTML
	\item Créer un fichier "style.css"
	\item Définir ce fichier comme feuille de style externe aux trois fichiers HTML
	\item Dans ce qui vient, le choix des couleurs et des tailles est laissé à l'étudiant
\end{enumerate}

\subsection*{Entête et pied de page}

Définir les règles 
\begin{enumerate}
	\item Pour les éléments \textbf{<header>} :
	\begin{multicols}{2}
	\begin{itemize}
		\item Un gradient de couleurs comme arrière-plan
		\item La disposition \textbf{flex} (afin d'afficher le banner et le titre côte à côte)
		\item Sans marges externes (afin que le menu soit attaché à l'entête)
	\end{itemize} 
	\end{multicols}
	\item Pour les titres de niveau 1 descendants d'un élément \textbf{<header>} :
	\begin{itemize}
		\item Définir les marges \textbf{auto} (afin de centrer le titre horizontalement et verticalement)
	\end{itemize} 
	\item Pour l'élément dont l'identifiant est \textbf{banner} :
	\begin{itemize}
		\item Des marges externes de 5px
	\end{itemize} 
	\item Pour l'image descendante de l'élément dont l'identifiant est \textbf{banner} :
	\begin{itemize}
		\item Une hauteur de 100px
	\end{itemize} 
	\item Pour les éléments \textbf{<footer>} :
	\begin{multicols}{2}
	\begin{itemize}
		\item Choisir deux couleurs de fond et d'écriture
		\item Centrer le texte horizontalement
		\item Fixer les marges internes à 5px
		\item Définir la marge externe en haut par 10px
		\item Rendre les coins arrondis : 5px (sans afficher les bordures)
	\end{itemize} 
	\end{multicols}
\end{enumerate}

\subsection*{Menu}

Définir les règles 
\begin{enumerate}
	\item Pour les éléments \textbf{<nav>} :
	\begin{multicols}{2}
	\begin{itemize}
		\item La position \textbf{sticky} (afin que le menu défile jusqu'à une position et s'arrête ; reste affiché en haut)
		\item La distance en haut doit être 0 
		\item Sans marges externes (afin que le menu soit attaché à l'entête)
		\item Choisir une couleur de fond 
		\item En défilant la page, on remarqué que le contenu passe sur le menu. 
		On veut qu'il passe sous le menu (ce dernier doit être proche de l'œil: axe Z)
	\end{itemize} 
	\end{multicols}
	\item Pour les éléments \textbf{<ul>} descendants de \textbf{<nav>} :
	\begin{itemize}
		\item Sans marges externes et internes
	\end{itemize} 
	\item Pour les éléments \textbf{<li>} descendants de \textbf{<nav>} :
	\begin{itemize}
		\item disposition en \textbf{inline-block} (afin que les éléments du menu soient positionnés un après l'autre)
	\end{itemize} 
	\item Pour les liens hypertextes descendants des éléments \textbf{<ul>} qui sont à leur tour descendants de \textbf{<nav>}  :
	\begin{multicols}{2}
	\begin{itemize}
		\item Disposition en \textbf{inline-block} ou \textbf{block} (afin d'appliquer les marges)
		\item Marges internes et externes de 5px
		\item Choisir des couleurs de fond et d'écriture
		\item Éliminer les décorations des liens (souligné) en utilisant la valeur \textbf{none}
		\item Mettre le texte en gras
		\item Rendre les coins arrondis : 5px (sans afficher les bordures)
	\end{itemize} 
	\end{multicols}
	\item Pour les liens hypertextes descendants des éléments \textbf{<li>} dont la classe est \textbf{actif} et qui sont à leur tour descendants de \textbf{<nav>}  :
	\begin{itemize}
		\item Définir une couleur de fond (Afin de préciser que l'élément du menu est déjà sélectionné)
	\end{itemize} 

	\item Lorsqu'on met le curseur sur les liens hypertextes descendants des éléments \textbf{<li>} qui sont à leur tour descendants de \textbf{<nav>}  :
	\begin{multicols}{2}
	\begin{itemize}
		\item Définir une couleur de fond différente de l'originale
		\item Définir une transition de la couleur du fond (durée 3s)
	\end{itemize} 
	\end{multicols}
	\item Lorsqu'on met le curseur sur les liens hypertextes descendants des éléments \textbf{<li>} dont la classe est \textbf{actif} et qui sont à leur tour descendants de \textbf{<nav>}  :
	\begin{multicols}{2}
	\begin{itemize}
		\item La même couleur de fond que la règle avant précédente
		\item Définir la transition à \textbf{none} (afin de ne pas avoir une transition pour les éléments sélectionnés)
	\end{itemize} 
	\end{multicols}

\end{enumerate}

\subsection*{Les sections}

Définir les règles 
\begin{enumerate}
	\item Pour les éléments \textbf{<section>} :
	\begin{multicols}{2}
	\begin{itemize}
		\item Une bordure à ligne continue avec une taille de 3px et une couleur au choix
		\item Rendre les coins arrondis : 10px
		\item Ajouter une marge en haut de 30px (afin de faire ressortir le titre)
		\item Définir la position comme \textbf{relative} (Afin qu'on puisse positionner le titre par rapport à la section)
	\end{itemize} 
	\end{multicols}
	\item Pour les titres de niveau 2 qui sont fils directes de \textbf{<section>} :
	\begin{multicols}{2}
	\begin{itemize}
		\item Marges internes de 0.2em
		\item Bordures formatées comme celles de \textbf{<section>}
		\item La position \textbf{absolute} avec positionnement en haut de -1.7em et à gauche de 10px
		\item Couleur de fond blanche 
		\item Transformer le texte en majuscule 
		\item Taille de la police : 1em
	\end{itemize} 
	\end{multicols}
	\item Ajouter du contenu avant celui \textbf{<section>} en utilisant \textbf{::before} (afin de laisser un espace entre le  titre qu'on a déplacé et le reste du contenu de la section) :
	\begin{multicols}{2}
	\begin{itemize}
		\item Le contenu est vide
		\item De type block
		\item La hauteur est de 1em
		\item La marge en bas : 5px
	\end{itemize} 
	\end{multicols}
\end{enumerate}

\subsection*{Les tableaux}

Définir les règles 
\begin{enumerate}
	\item La largeur d'un tableau doit être 100\% de la largeur de la fenêtre
	\item Pour les cellules d'un tableau (normales et de titre) :
	\begin{multicols}{2}
	\begin{itemize}
		\item La marge interne : 5px
		\item Rendre les coins arrondis : 5px (sans afficher les bordures)
	\end{itemize} 
	\end{multicols}
	\item Pour les cellules de titres de colonnes :
	\begin{multicols}{2}
	\begin{itemize}
		\item Choisir une couleur foncée de fond
		\item Choisir la couleur blanche pour l'écriture
	\end{itemize}  
	\end{multicols}
	\item Pour les cellules de titres de lignes :
	\begin{multicols}{2}
	\begin{itemize}
		\item Choisir une autre couleur foncée de fond
		\item Choisir la couleur blanche pour l'écriture
	\end{itemize} 
	\end{multicols}
	\item Pour les cellules normales descendantes des lignes \textbf{<tr>} impaires :
	\begin{itemize}
		\item Choisir une couleur claire de fond
	\end{itemize} 
	\item Pour les cellules normales descendantes des lignes \textbf{<tr>} paires :
	\begin{itemize}
		\item Choisir une autre couleur claire de fond
	\end{itemize} 
\end{enumerate}

\subsection*{Format de texte}

Définir les règles 
\begin{enumerate}
	\item Pour chaque paragraphe (\textbf{<p>}) :
	\begin{multicols}{2}
	\begin{itemize}
		\item Définir une indentation de première ligne : 2em;
		\item Justifier le texte
	\end{itemize} 
	\end{multicols}
	\item Pour la première lettre d'un paragraphe :
	\begin{multicols}{2}
	\begin{itemize}
		\item Définir une couleur d'écriture au choix
		\item Justifier le texte
		\item Attribuer la valeur 3em à la taille de la police
		\item La lettre ne doit pas être en gras (\textit{font-weight: normal;})
		\item La police doit être \textbf{AngloText} accompagnée comme ressource
	\end{itemize} 
	\end{multicols}
	\item Pour les citations en bloque :
	\begin{multicols}{2}
	\begin{itemize}
		\item Définir les marges gauches et droite par 2em (ou plus)
		\item Fixer les marges internes à 5px;
		\item Attribuer une bordure de 1px avec ligne continue et une couleur aux choix
		\item Rendre les coins arrondis : 10px
		\item Afficher le texte en italique 
		\item Définir un ombre de boîte (choisir des dimensions et la couleur)
	\end{itemize} 
	\end{multicols}
	\item Pour les éléments \textbf{<aside>} :
	\begin{multicols}{2}
	\begin{itemize}
		\item Choisir une couleur de fond claire
		\item Le texte doit être gras et en italique
		\item Fixer la marge externe gauche à 15px
		\item Définir une marge interne de 5px
		\item Flotter à droite
		\item Définir une grande bordure gauche (10px)
	\end{itemize} 
	\end{multicols}
\end{enumerate}

\subsection*{Diaporama des images}

Ici, on va créer un diaporama des images de la page "\textbf{auteur.html}". 
Dans ce qui suit, $N$ réfère au nombre des images. 
L'idée est de stocké les images avec des largeurs similaires l'une après l'autre horizontalement dans un \textbf{<div>}. 
Ce \textbf{<div>} doit être inclus dans un autre qui a la taille d'une seule image (pour afficher une seule à la fois).
L'animation consiste à décaler le \textbf{<div>} interne sur plusieurs étapes.
\begin{enumerate}
	\item Définir des étapes d'animation :
	\begin{multicols}{2}
	\begin{itemize}
		\item Chaque étape a une distance de $\frac{100}{N-1}\%$. 
		Si nous avons 4 images, les étapes sont : 0\%, 33\%, 67\% et 100\%.
		Si nous avons 3 images, les étapes sont : 0\%, 50\%, 100\%.
		\item Pour chaque étape, on diminue la position gauche de 100\% par rapport à l'étape précédente.
	\end{itemize} 
	\end{multicols}
\begin{verbatim}
@keyframes slideshow {
  0% {left: 0;}
  33% {left: -100%;}
  67% {left: -200%;}
  100% {left: -300%;}
}
\end{verbatim}
	\item Pour les éléments ayant une classe \textbf{slide-wrapper} :
	\begin{multicols}{2}
	\begin{itemize}
		\item Cacher le débordement du contenu (\textbf{overflow})
		\item Définir la largeur par 50\%;
		\item Spécifier les marges par \textbf{auto} (pour centrer le diaporama)
		\item Rendre les coins arrondis : 20px (sans afficher les bordures)
	\end{itemize} 
	\end{multicols}
	\item Pour éléments ayant une classe \textbf{slide} et descendant des éléments de classe \textbf{slide-wrapper} :
	\begin{multicols}{2}
	\begin{itemize}
		\item Définir la position comme \textbf{relative}
		\item Fixer la largeur à $N * 100\%$ (Si on a 4 images, donc 400\%)
		\item Éliminer les marges externes
		\item Rendre les coins arrondis : 10px
		\item Initialiser la position gauche à 0
		\item Définir une animation de 20s (ou au choix) en utilisant les étapes définies précédemment. 
		L'animation doit boucler infiniment et à l'alternance (va et vient) 
	\end{itemize} 
	\end{multicols}
	\item Pour les images \textbf{<img>} descendantes d'un éléments ayant une classe \textbf{slide} et descendant des éléments de classe \textbf{slide-wrapper} :
	\begin{multicols}{2}
	\begin{itemize}
		\item Flotter à gauche
		\item Définir une largeur de $\frac{100}{N}\%$. Par exemple, pour 4 images, la largeur est 25\%.
	\end{itemize} 
	\end{multicols}
\end{enumerate}

\subsection*{Les formulaires}

Définir les règles 
\begin{enumerate}
	\item Pour les éléments \textbf{<input>}, \textbf{<select>} et \textbf{<textarea>} :
	\begin{multicols}{2}
	\begin{itemize}
		\item Définir une bordure de 1px avec couleur au choix 
		\item Rendre les coins arrondis : 5px
		\item Fixer les marges internes et externes à 5px
	\end{itemize} 
	\end{multicols}
	\item Pour les éléments \textbf{<input>} qui sont obligatoires :
	\begin{itemize}
		\item Définir une couleur de fond
	\end{itemize} 
	
\end{enumerate}

\subsection*{Le smartphone}

Dans les médias de type \textbf{screen} avec une largeur maximale de \textbf{600px}, on définit les règles suivantes.

\begin{enumerate}
	\item Pour les étiquettes, ajouter un contenu après leurs contenus en utilisant \textbf{::after} :
	\begin{multicols}{2}
	\begin{itemize}
		\item Son contenu est vide
		\item disposition en bloque (pour que les champs de saisi retournent à la ligne)
	\end{itemize} 
	\end{multicols}
	\item Pour les éléments de type \textbf{<select>}, \textbf{<textarea>} et \textbf{<input>} qui ne sont pas de type \textbf{checkbox} ou \textbf{radio} :
	\begin{itemize}
		\item Largeur de 90\%
	\end{itemize}
	\item Pour les éléments dont la classe est \textbf{slide-wrapper} :
	\begin{itemize}
		\item Largeur de 90\%
	\end{itemize}  
\end{enumerate}

\end{document}
