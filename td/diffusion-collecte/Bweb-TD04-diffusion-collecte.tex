% !TEX TS-program = pdflatex
% !TeX program = pdflatex
% !TEX encoding = UTF-8
% !TEX spellcheck = fr

\documentclass[11pt, a4paper]{article}
%\usepackage{fullpage}
\usepackage[left=1cm,right=1cm,top=1cm,bottom=2cm]{geometry}
\usepackage[fleqn]{amsmath}
\usepackage{amssymb}
%\usepackage{indentfirst}
\usepackage[T1]{fontenc}
\usepackage[utf8]{inputenc}
\usepackage[french,english]{babel}
\usepackage{txfonts} 
\usepackage[]{graphicx}
\usepackage{multirow}
\usepackage{hyperref}
\usepackage{parskip}
\usepackage{multicol}
\usepackage{wrapfig}

\usepackage{turnstile}%Induction symbole

\renewcommand{\baselinestretch}{1}

\setlength{\parindent}{24pt}


\begin{document}

\selectlanguage {french}
%\pagestyle{empty} 

\noindent
\begin{tabular}{ll}
\multirow{3}{*}{\includegraphics[width=2cm]{../esi-logo.png}} & \'Ecole national Supérieure d'Informatique\\
& 1ère année cycle préparatoire\\
& Bureautique et Web
\end{tabular}\\[.25cm]
\noindent\rule{\textwidth}{1pt}\\%[-0.25cm]
\begin{center}
{\LARGE \textbf{T.D. Diffusion et collecte d'informations}}
%\begin{flushright}
%	ARIES Abdelkrime
%\end{flushright}
\end{center}
\noindent\rule{\textwidth}{1pt}

\section*{Exercice 1 : Formulaires}

\vspace{-12pt}
\begin{tabular}{|p{\textwidth}|}
	\hline\\
	Objectif : l'étudiant doit pouvoir créer des formulaires afin de collecter les informations   \\\\
	\hline
\end{tabular}

Après avoir publié votre rapport "Introduction à la programmation" sur les réseaux sociaux, plusieurs étudiants ont été intéressés par des formations dans cet axe. 
Afin de satisfaire leurs demandes, vous avez décidé d'organiser une série de formations intitulé \textbf{ESI.prog} en collaboration avec un groupe de vos collègues.

Pour bien organiser les formations, vous avez opté pour une inscription via des formulaires. 
Vous vous êtes mis d'accord à utiliser des formulaires Google puisqu'ils sont plus pratiques lors de l'extraction des données. 
Les formulaires Word sont une alternative au cas des exceptions. 

Tout d'abord, vous devez collecter les informations personnelles et de contact. 
Vous voulez avoir le titre (M., Mme, Mlle.), le nom, le prénom, la date de naissance, l'age, l'adresse et l'émail du candidat. 
Aussi, vous voulez savoir le niveau de l'étude : 1CP, 2CP, 1CS, 2CS, 3CS ou Autre.
Ce dernier choix est réservé aux étudiants hors ESI. 
Dans ce cas, vous devez récupérer le nom de l'université ou l'école d'origine. 

Après avoir introduit ses information personnelles, le candidat passe vers les informations des formations. 
Les formations proposées sont : Introduction à la programmation, C, C\#, Java, Javascript et Python. 
Le candidat peut choisir plusieurs formations à la fois.
Pour bien organiser des sessions de formation, le candidat doit fournir son niveau d'expérience : Débutant, Intermédiaire ou Avancé. 
Pour ce faire : 
\begin{itemize}
	\item Si vous utilisez Word, vous devez seulement récupérer son niveau d'expérience dans la programmation en général.
	\item Si vous utilisez Google Forms, vous devez récupérer son niveaux d'expérience dans chaque type de formation.
\end{itemize}
Vous voulez fournir une version papier de la formation où le candidat doit payer les frais d'impression. 
Le formulaire doit donner la possibilité de choisir si on veut une version papier ou non.

Vous voulez avoir des informations supplémentaires qui ne sont pas obligatoires. 
Le formulaire doit permettre au candidat d'exprimer son opinion concernant le rapport (en terme de quantité d'information). 
Vous devez fournir une échelle de 1 (insuffisant) jusqu'à 5 (très bon). 
Aussi, vous devez fournir un moyen pour que le candidat puisse exprimer son opinion librement. 

Avant d'envoyer le formulaire aux destinataires, vous devez le tester pour évaluer le temps de remplissage et la facilité d'usage. 
Pour ce faire, il faut l'envoyer à quelques membres du groupe et récupérer leurs opinions pour mieux améliorer le formulaire. 

\subsection*{Besoins fonctionnelles}

\begin{multicols}{2}
\begin{itemize}
	\item Le formulaire doit être facile à utiliser : divisé en sections, avec moins de cliques, etc. 
	\item Le formulaire doit utiliser les bons éléments et fournir des descriptions utiles
	\item L'utilisateur doit pouvoir modifier sa réponse (en cas de Google Forms)
	\item On doit recevoir une réponse par utilisateur (en cas de Google Forms)
\end{itemize}
\end{multicols}

\newpage
\subsection*{Besoins techniques}

\begin{multicols}{2}
\begin{itemize}
	\item Le formulaire doit être implémenté en utilisant Google Forms
	\item Si on ne peut pas utiliser Google Forms, on se contente par Word
	\item On doit utiliser "nomUniv.xslx" pour la liste des universités (Pour le test, on peut insérer seulement 7 choix). \textbf{Astuce :} Vous pouvez copier la liste et coller dans la première option dans Google Forms.
\end{itemize}
\end{multicols}

\section*{Exercice 2 : Publipostage}

\vspace{-12pt}
\begin{tabular}{|p{\textwidth}|}
	\hline\\
	Objectif : l'étudiant doit pouvoir appliquer le publipostage afin de diffuser les informations   \\\\
	\hline
\end{tabular}

Vous avez reçu plusieurs demandes de candidature pour les formations de programmation (\textbf{ESI.prog}). 
Afin d'organiser une première session de formation, en tant que chargé de communication, vous avez préparé un fichier "formation.xlsx". 
Ce fichier contient le titre, le nom, le prénom, l'émail, le niveau (ou l'université), l'adresse, le type de la formation et le niveau d'expérience. 
Pour permettre aux externes de se préparer (voyage, etc.), vous voulez programmer cette première session seulement pour les étudiants de l'ESI.

La journée de la formation, les étudiants doivent apporter la carte d'étudiant (ou une carte qui preuve l'identité) et une invitation. 
Pour imprimer les différentes invitations, vous avez opté pour une solution par publipostage. 
Aux début de la lettre on doit fournir les informations suivantes : 
\begin{multicols}{2}
\begin{itemize}
	\item Les informations de l'expéditeur (le chargé de communication dans le groupe)
	\item Les informations du destinataire (le candidat)
	\item La date (Insertion -> Texte -> QuickPart -> Champ -> Date ) et lieu
	\item L'objet de la lettre
	\item Une formule d'appel: "\textbf{Cher NOM Prénom,}" ou "\textbf{Chère NOM Prénom,}"
\end{itemize}
\end{multicols}

Les formations de cette session auront lieu la même journée (qui sera fixée lors de l'envoi), mais dans des heurs différentes selon le niveau d'expérience : 
\begin{multicols}{3}
\begin{itemize}
	\item \textbf{Débutant} : 10h00-12h00
	\item \textbf{Intermédiaire} : 13h30-15h30
	\item \textbf{Avancé} : 16h30-18h30
\end{itemize}
\end{multicols}
Les salles sont attribuées selon le type de formation : 
\begin{multicols}{3}
\begin{itemize}
	\item \textbf{Introduction à la programmation} : CP1
	\item \textbf{C} : CP2
	\item \textbf{C\#} : CP3
	\item \textbf{Java} : CP4
	\item \textbf{Javascript} : CP5
	\item \textbf{Python} : CP6
\end{itemize}
\end{multicols}

Afin d'aider les étudiants des classes préparatoires à bien suivre la formation, vous leur fournissez des brochures à la rentré. Dans ce cas, on ajoute la phrase "\textbf{En tant que étudiant en classes préparatoires, vous pouvez demander vos brochures lors de l’enregistrement.}" à la lettre en cas des 1CP et des 2CP.

\subsection*{Besoins fonctionnelles}

\begin{itemize}
	\item Les lettres doivent être envoyées seulement aux étudiants concernés
	\item Lorsqu'on fusionne, Word doit vous demander la date qui sera inséré dans toutes les invitations (les deux emplacements dans la lettre)
\end{itemize}

\subsection*{Besoins techniques}

\begin{itemize}
	\item Le publipostage doit être implémenté en utilisant Word
	\item On doit utiliser la première ligne du fichier "formation.xlsx" comme noms des champs
\end{itemize}

\subsection*{Modèle}

\begin{center}
	\fbox{\includegraphics[]{publi-example.pdf}}
\end{center}


\end{document}
