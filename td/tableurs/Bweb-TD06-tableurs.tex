% !TEX TS-program = pdflatex
% !TeX program = pdflatex
% !TEX encoding = UTF-8
% !TEX spellcheck = fr

\documentclass[11pt, a4paper]{article}
%\usepackage{fullpage}
\usepackage[left=1cm,right=1cm,top=1cm,bottom=2cm]{geometry}
\usepackage[fleqn]{amsmath}
\usepackage{amssymb}
%\usepackage{indentfirst}
\usepackage[T1]{fontenc}
\usepackage[utf8]{inputenc}
\usepackage[french,english]{babel}
\usepackage{txfonts} 
\usepackage[]{graphicx}
\usepackage{multirow}
\usepackage{hyperref}
\usepackage{parskip}
\usepackage{multicol}
\usepackage{wrapfig}

\usepackage{turnstile}%Induction symbole

\renewcommand{\baselinestretch}{1}

\setlength{\parindent}{24pt}


\begin{document}

\selectlanguage {french}
%\pagestyle{empty} 

\noindent
\begin{tabular}{ll}
\multirow{3}{*}{\includegraphics[width=2cm]{../esi-logo.png}} & \'Ecole national Supérieure d'Informatique\\
& 1ère année cycle préparatoire\\
& Bureautique et Web
\end{tabular}\\[.25cm]
\noindent\rule{\textwidth}{1pt}\\%[-0.25cm]
\begin{center}
{\LARGE \textbf{T.D. Tableurs}}
%\begin{flushright}
%	ARIES Abdelkrime
%\end{flushright}
\end{center}
\noindent\rule{\textwidth}{1pt}

\section*{Exercice : Excel (fiche de notes)}

\vspace{-12pt}
\begin{tabular}{|p{\textwidth}|}
	\hline\\
	Objectif : l'étudiant doit pouvoir créer une fiche de note avec des graphiques et des statistiques  \\\\
	\hline
\end{tabular}

Après avoir achevé la série de formations \textbf{ESI.prog} avec succès, vous avez pensé à sélectionner des étudiants pour passer au niveau avancé des formations. 
Pour ce faire, vous avez organisé une série de tests : un contrôle continu (CC), un contrôle intermédiaire (CI), un contrôle final (CF) et des travaux pratiques (TP). 
Les travaux pratiques se composent de : conception, suivi et implémentation. 

Vous avez reçu les différentes notes indiquées dans le fichier \textbf{fiche\_note.xlsx}. 
Vous voulez organiser cette fiche et faire quelques statistiques.

\begin{enumerate}
	\item Ouvrir le fichier \textbf{fiche\_note.xlsx}. Si vous n'avez pas Excel, utiliser \textbf{Google Sheets}
	\item Renommer la feuille \textbf{Feuil1} par \textbf{TP} et \textbf{Feuil2} par \textbf{Total}
	\item Aller à la feuille \textbf{TP}
	\item Dans la colonne \textbf{F}, calculer la note du TP sur 20. Sachant que la conception est sur 12, le suivi est sur 10 et l'implémentation est sur 15 ; ce qui donne une note sur 37.
	\item Dans la colonne \textbf{G}, calculer la valeur arrondie de la note dans \textbf{F}. L'arrondissement doit être avec 2 chiffres après la virgule. Il doit être vers la valeur supérieure (utiliser la fonction \textbf{ARRONDI.SUP}). Par exemple, l'arrondie de la valeur 10,27027027 doit être 10,28.
	\item Aller à la feuille \textbf{Total}
	\item Dans la colonne \textbf{F}, vous devez récupérer les notes de TP arrondies à partir le la feuille \textbf{TP}. Pour ce faire, utiliser la fonction \textbf{CHERCHERV} pour chercher le nom et récupérer la note. 
	\item Vous avez remarqué qu'il y a des notes de TP non trouvables et d'autres erronées. Après 2h de recherche sur le net, vous avez constaté que la fonction \textbf{CHERCHERV} a besoin d'une liste ordonnée pour bien fonctionner. Sinon, si on veut garder la liste non triées, on doit ajouter un autre argument \textbf{FAUX} à la fonction \textbf{CHERCHERV} pour dire "chercher la valeur exacte". La première solution est meilleure en terme de temps d'exécution.
	\item Aller à la feuille \textbf{TP}
	\item Trier le tableau selon le nom des étudiants 
	\item Aller à la feuille \textbf{Total}. Le problème est réglé. 
	\item Dans la colonne \textbf{G}, calculer la note finale où les coefficients de chaque épreuve est donnée dans la feuille (plage \textbf{N3:Q3}). On peut changer les coefficients ; donc il ne faut pas les utiliser comme des constantes dans la formule.
	\item Dans la colonne \textbf{H}, vous devez afficher "non admis" si la note finale est inférieure à 10 ; "très bien" si elle est supérieure ou égale à 16 ; "bien" si elle est entre 10 et 15 (10 inclus). 
	\item On veut faire des statistiques sur les moyennes et les médianes des notes des étudiants dans chaque épreuve. Compléter ces statistiques en bas du tableau. 
	\item Dans les cellules \textbf{J7}, \textbf{K7} et \textbf{L7} on veut avoir le nombre des étudiants ayant les mentions en haut de chaque cellule. Utiliser la fonction \textbf{NB.SI} qui prend une plage de donnée comme premier argument et une condition sous forme d'un texte comme deuxième argument. Par exemple, \textbf{NB.SI(G2:G21; "<10")}. Vous voulez utiliser les textes dans \textbf{J6}, \textbf{K6} et \textbf{L6}. Donc, pour générer la condition, vous avez utilisé la fonction \textbf{CONCATENER} (ex. \textbf{CONCATENER("="; J6)})
	\item Insérer un graphique qui indique le pourcentage de chaque mention. 
	\item Formater le tableau : par exemple les titres avec une couleur, etc. 
	\item Dans la colonne \textbf{G}, colorer automatiquement les cellules avec une valeur supérieure ou égale à 16 en vert. Aussi, colorier les cellules avec une valeur inférieure à 10 en rouge.
	\item Dans la mise en page, choisir la taille "Enveloppe US n\degres10" et l'orientation "Paysage"
	\item Définir seulement le tableau des notes et les statistiques en bas comme zone d'impression 
	\item La première ligne (les titres) doit se figurer dans chaque page (si le tableau prend plus d'une page)
	\item Protéger la feuille \textbf{Total}
\end{enumerate}


\end{document}
