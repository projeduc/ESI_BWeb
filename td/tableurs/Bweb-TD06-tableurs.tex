% !TEX TS-program = pdflatex
% !TeX program = pdflatex
% !TEX encoding = UTF-8
% !TEX spellcheck = fr

\documentclass[11pt, a4paper]{article}
%\usepackage{fullpage}
\usepackage[left=1cm,right=1cm,top=1cm,bottom=2cm]{geometry}
\usepackage[fleqn]{amsmath}
\usepackage{amssymb}
%\usepackage{indentfirst}
\usepackage[T1]{fontenc}
\usepackage[utf8]{inputenc}
\usepackage[french,english]{babel}
\usepackage{txfonts} 
\usepackage[]{graphicx}
\usepackage{multirow}
\usepackage{hyperref}
\usepackage{parskip}
\usepackage{multicol}
\usepackage{wrapfig}

\usepackage{turnstile}%Induction symbole

\renewcommand{\baselinestretch}{1}

\setlength{\parindent}{24pt}


\begin{document}

\selectlanguage {french}
%\pagestyle{empty} 

\noindent
\begin{tabular}{ll}
\multirow{3}{*}{\includegraphics[width=2cm]{../esi-logo.png}} & \'Ecole national Supérieure d'Informatique\\
& 1ère année cycle préparatoire\\
& Bureautique et Web
\end{tabular}\\[.25cm]
\noindent\rule{\textwidth}{1pt}\\%[-0.25cm]
\begin{center}
{\LARGE \textbf{T.D. Présentation}}
%\begin{flushright}
%	ARIES Abdelkrime
%\end{flushright}
\end{center}
\noindent\rule{\textwidth}{1pt}

\section*{Exercice : Powerpoint}

\vspace{-12pt}
\begin{tabular}{|p{\textwidth}|}
	\hline\\
	Objectif : l'étudiant doit pouvoir créer une présentation Powerpoint (temps estimé : 1h pour la réalisation et 15mn pour la présentation)   \\\\
	\hline
\end{tabular}

Après avoir publié votre rapport "Introduction à la programmation" sur les réseaux sociaux, plusieurs étudiants ont été intéressés par des formations dans cet axe. 
Afin de satisfaire leurs demandes, vous avez décidé d'organiser une série de formations intitulé \textbf{ESI.prog} en collaboration avec un groupe de vos collègues.

\begin{enumerate}
	\item Créer une présentation Powerpoint vide 
	\item Modifier le diapositive de titre 
	\begin{multicols}{3}
	\begin{enumerate}
		\item S'il n'existe pas, insérer le
		\item Ajouter le titre "Introduction à la programmation"
		\item Ajouter votre nom et prénom 
		\item Ajouter la date de présentation : 18 Avril 2020
	\end{enumerate}
	\end{multicols}

	\item Afficher le pied des diapositives 
	\begin{multicols}{3}
	\begin{enumerate}
		\item Dans Acceuil, Texte, Entête/Pied
		\item Fixer la date à 18/04/2020
		\item Activer le numéro de diapositive
		\item Ajouter un pied de page "ESI- " suivi de votre nom
		\item Ne pas afficher sur la diapostive de titre
		\item Appliquer partout
	\end{enumerate}
	\end{multicols}
	
	\item Créer 3 diapositives de type "Titre et contenu" avec les titres
	\begin{multicols}{3}
	\begin{enumerate}
		\item Contexte
		\item Problématique et objectif
		\item Plan
	\end{enumerate} 
	\end{multicols}
	
	\item Ajouter le contenu à partir du fichier "diapostives.txt"
	\begin{itemize}
		\item Compléter les trois diapositives précédentes
		\item Là où il y a \textbf{[SECTION]} créer un diapositive de type "Titre de section" avec le titre indiqué après \textbf{[SECTION]}
		\item Là où il y a \textbf{[DIAPO]} créer un diapositive de type "Titre et contenu" avec le titre indiqué après \textbf{[DIAPO]} et insérer le contenu
		\item Sauf \textbf{[DIAPO]} Technologies d'exécution. Insérer un diapositive de type "Comparaison", insérer les deux images "compile.png" et "interprete.png" et les titres "Compilation" et "Interprétation"
	\end{itemize}
	
	\item Ajouter une transition aux diapositives
	\begin{multicols}{3}
	\begin{enumerate}
		\item Aller à l'onglet "TRANSITIONS"
		\item Choisir une transition de votre choix (elle doit être simple)
		\item Dans le groupe "Minutage", appuyer sur "Appliquer partout" 
	\end{enumerate}
	\end{multicols}
	
	\newpage
	\item Appliquer des animation
	\begin{multicols}{3}
	\begin{enumerate}
		\item Aller vers le dernier diapostive
		\item Sélectionner le contenu (les puces)
		\item Aller vers l'onglet "ANIMATIONS"
		\item Choisir une animation d'apparition (elle doit être simple)
		\item Afficher le volet d'animation
		\item Sélectionner la deuxième animation
		\item Dans le groupe "minutage", Démarrer au clic
		\item Refaire la même chose pour la troisième animation
	\end{enumerate}
	\end{multicols}
	
	\item Dans l'onglet "CREATION"
	\begin{multicols}{2}
	\begin{itemize}
		\item Choisir un thème
		\item Modifier la taille des diapositives (4:3)
	\end{itemize}
	\end{multicols}
	
	\item Modifier le masque de la diapositive 
	%\begin{multicols}{3}
	\begin{enumerate}
		\item Onglet "AFFICHAGE" ; Masque des diapositives;
		\item Mettre un petit logo de l'ESI dans le masque supérieure pour l'afficher sur le reste des masques
		\item Dans le diapositive de titre, le logo doit être grand
		\item Désactiver le mode Masque 
	\end{enumerate}
	%\end{multicols}
	
	\item Deux étudiants doivent passer au tableau et présenter (Pas applicable dans le cas d'apprentissage à distance)
\end{enumerate}


\end{document}
