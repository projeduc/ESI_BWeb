% !TEX TS-program = pdflatex
% !TeX program = pdflatex
% !TEX encoding = UTF-8
% !TEX spellcheck = fr

\documentclass[11pt, a4paper]{article}
%\usepackage{fullpage}
\usepackage[left=1cm,right=1cm,top=1cm,bottom=2cm]{geometry}
\usepackage[fleqn]{amsmath}
\usepackage{amssymb}
%\usepackage{indentfirst}
\usepackage[T1]{fontenc}
\usepackage[utf8]{inputenc}
\usepackage[french,english]{babel}
\usepackage{txfonts} 
\usepackage[]{graphicx}
\usepackage{multirow}
\usepackage{hyperref}
\usepackage{parskip}
\usepackage{multicol}
\usepackage{wrapfig}

\usepackage{turnstile}%Induction symbole

\renewcommand{\baselinestretch}{1}

\setlength{\parindent}{24pt}


\begin{document}

\selectlanguage {french}
%\pagestyle{empty} 

\noindent
\begin{tabular}{ll}
\multirow{3}{*}{\includegraphics[width=2cm]{../esi-logo.png}} & \'Ecole national Supérieure d'Informatique\\
& 1ère année cycle préparatoire\\
& Bureautique et Web
\end{tabular}\\[.25cm]
\noindent\rule{\textwidth}{1pt}\\%[-0.25cm]
\begin{center}
{\LARGE \textbf{T.D. Tableurs}}
%\begin{flushright}
%	ARIES Abdelkrime
%\end{flushright}
\end{center}
\noindent\rule{\textwidth}{1pt}

\section*{Exercice : Excel (fiche de notes)}

\vspace{-12pt}
\begin{tabular}{|p{\textwidth}|}
	\hline\\
	Objectif : l'étudiant doit pouvoir créer une fiche de note avec des graphiques et des statistiques  \\\\
	\hline
\end{tabular}

Après avoir achevé la série de formations \textbf{ESI.prog} avec succès, vous avez pensé à sélectionner des étudiants pour passer au niveau avancé des formations. 
Pour ce faire, vous avez organisé une série de tests : un contrôle continu (CC), un contrôle intermédiaire (CI), un contrôle final (CF) et des travaux pratiques (TP). 
Les travaux pratiques se composent de : conception, suivi et implémentation. 

Vous avez reçu les différentes notes indiquées dans le fichier \textbf{fiche\_note.xlsx}. 
Vous voulez organiser cette fiche et faire quelques statistiques.

\begin{enumerate}
	\item Ouvrir le fichier \textbf{fiche\_note.xlsx}. Si vous n'avez pas Excel, utiliser \textbf{Google Sheets}
	\item Renommer la feuille \textbf{Feuil1} par \textbf{TP} et \textbf{Feuil2} par \textbf{Total}
	\item Aller à la feuille \textbf{TP}
	\item Dans la colonne \textbf{F}, calculer la note du TP sur 20. Sachant que la conception est sur 12, le suivi est sur 10 et l'implémentation est sur 15 ; ce qui donne une note sur 37.
	\item Dans la colonne \textbf{G}, calculer la valeur arrondie de la note dans \textbf{F}. L'arrondissement doit être avec 2 chiffres après la virgule. Il doit être vers la valeur supérieure (utiliser la fonction \textbf{ARRONDI.SUP}). Par exemple, l'arrondie de la valeur 10,27027027 doit être 10,28.
	\item Aller à la feuille \textbf{Total}
	\item Dans la colonne \textbf{F}, vous devez récupérer les notes de TP arrondies à partir le la feuille \textbf{TP}. Pour ce faire, utiliser la fonction \textbf{CHERCHERV} pour chercher le nom et récupérer la note. 
	\item Vous avez remarqué qu'il y a des notes de TP non trouvables et d'autres erronées. Après 2h de recherche sur le net, vous avez constaté que la fonction \textbf{CHERCHERV} a besoin d'une liste ordonnée pour bien fonctionner. Sinon, si on veut garder la liste non triées, on doit ajouter un autre argument \textbf{FAUX} à la fonction \textbf{CHERCHERV} pour dire "chercher la valeur exacte". La première solution est meilleure en terme de temps d'exécution.
	\item Aller à la feuille \textbf{TP}
	\item Trier le tableau selon le nom des étudiants 
	\item Aller à la feuille \textbf{Total}. Le problème est réglé. 
	\item Dans la colonne \textbf{G}, calculer la note finale où les coefficients de chaque épreuve est donnée dans la feuille (plage \textbf{N3:Q3}). On peut changer les coefficients ; donc il ne faut pas les utiliser comme des constantes dans la formule.
	\item Dans la colonne \textbf{H}, vous devez afficher "non admis" si la note finale est inférieure à 10 ; "très bien" si elle est supérieure ou égale à 16 ; "bien" si elle est entre 10 et 15 (10 inclus). 
	\item On veut faire des statistiques sur les moyennes et les médianes des notes des étudiants dans chaque épreuve. Compléter ces statistiques en bas du tableau. 
	\item Dans les cellules \textbf{J7}, \textbf{K7} et \textbf{L7} on veut avoir le nombre des étudiants ayant les mentions en haut de chaque cellule. Utiliser la fonction \textbf{NB.SI} qui prend une plage de donnée comme premier argument et une condition sous forme d'un texte comme deuxième argument. Par exemple, \textbf{NB.SI(G2:G21; "<10")}. Vous voulez utiliser les textes dans \textbf{J6}, \textbf{K6} et \textbf{L6}. Donc, pour générer la condition, vous avez utilisé la fonction \textbf{CONCATENER} (ex. \textbf{CONCATENER("="; J6)})
	\item Insérer un graphique qui indique le pourcentage de chaque mention. 
	\item Formater le tableau : par exemple les titres avec une couleur, etc. 
	\item Dans la colonne \textbf{G}, colorer automatiquement les cellules avec une valeur supérieure ou égale à 16 en vert. Aussi, colorier les cellules avec une valeur inférieure à 10 en rouge.
	\item Dans la mise en page, choisir la taille "Enveloppe US n\degres10" et l'orientation "Paysage"
	\item Définir seulement le tableau des notes et les statistiques en bas comme zone d'impression 
	\item La première ligne (les titres) doit se figurer dans chaque page (si le tableau prend plus d'une page)
	\item Protéger la feuille \textbf{Total}
\end{enumerate}


\section*{Exercice : TCD (analyse de participants)}

\vspace{-12pt}
\begin{tabular}{|p{\textwidth}|}
	\hline\\
	Objectif : l'étudiant doit pouvoir répondre à des questions en utilisant les tableaux croisés dynamiques  \\\\
	\hline
\end{tabular}

\textit{\textbf{Note : } les données accompagnées sont fictives ; elles sont générées automatiquement en utilisant un programme Python fourni avec les ressources. Toute ressemblance avec des personnes existantes ou ayant existé est purement fortuite.}

Après avoir achevé les tests concernant la série de formations \textbf{ESI.prog}, votre tâche dans l'équipe est de faire des analyses sur les participants. 
On vous a fourni un fichier \textbf{participants\_analyse.xlsx} contenant une liste de 200 participants dont 150 sont de l'ESI. 
La liste contient les informations suivantes : le nom, le prénom, la wilaya, l'université, le niveaux (seulement au cas de \textit{École nationale supérieure d'informatique d'Alger}), le type de formation, l'expérience et la note finale. 

Pour ce faire, vous avez pensé à utiliser les tableaux croisés dynamiques. 
On vous a demandé de répondre aux questions suivantes, où le titre en gras est le nom de la feuille (chaque TCD dans une nouvelle feuille) : 

\begin{enumerate}
	\item \textbf{Wilaya-NBR} : cette feuille doit fournir le nombre des participants par wilaya. Elle doit, aussi, contenir un histogramme représentant le TCD. 
	\item \textbf{Univ-NBR} : cette feuille doit fournir le nombre des participants par université. Les universités doivent être regroupées en : Centre, École, Faculté, Institue et Université.
	\item \textbf{Type-NBR} : cette feuille doit fournir le nombre des participants par type de formation. Elle doit être accompagnée par un graphique de type "secteurs" indiquant le pourcentage des participants dans chaque formation.
	\item \textbf{Niveau-Formation} : cette feuille doit fournir les moyennes de chaque type de formation par niveau. Puisque le niveau est fourni seulement pour l'ESI, donc on doit considérer seulement les participants de l'ESI.
	Ce TCD doit être accompagné par un histogramme. 
	\item \textbf{Formation-Experience} : cette feuille doit fournir les moyennes des notes et le nombre de participants par type de formation et expérience.
\end{enumerate}


\section*{Exercice : TCD (analyse des patients de COVID19)}

\vspace{-12pt}
\begin{tabular}{|p{\textwidth}|}
	\hline\\
	Objectif : l'étudiant doit pouvoir répondre à des questions en utilisant les tableaux croisés dynamiques  \\\\
	\hline
\end{tabular}

\textit{\textbf{Note : } les données accompagnées sont à l'origine de \url{https://www.kaggle.com/sudalairajkumar/novel-corona-virus-2019-dataset}}

Vous voulez faire des analyses sur les patients de COVID19 de 3 pays : la Chine, la Singapour et la Corée du sud.
Vous avez trouvé le fichier \textbf{covid19\_analyse.xlsx} contenant une liste de 404 patients. 
La liste contient les informations suivantes : la date de déclaration, le pays, le sex, l'age, l'état (hospitalisé, mort, rétabli).

Pour ce faire, vous avez pensé à utiliser les tableaux croisés dynamiques. 
Vous avez l'intention de répondre au questions suivantes, où le titre en gras est le nom de la feuille (chaque TCD dans une nouvelle feuille) : 

\begin{enumerate}
	\item \textbf{Date-NBR} : cette feuille doit fournir le nombre des patients par date. Elle doit, aussi, contenir une courbe représentant le le nombre de cas par date. 
	\item \textbf{Age-NBR} : cette feuille doit fournir le nombre des patients par age. Les ages doivent être regroupées en : Jeune(de 0 à 19 ans), Adulte (de 20 à 59 ans) et Vieux (de 60 ans et plus). Il faut filtrer les ages vides.
	\item \textbf{Etat-NBR} : cette feuille doit fournir le nombre des patients par pays. Elle doit être accompagnée par un graphique de type "secteurs" indiquant le pourcentage des patients dans chaque pays.
	\item \textbf{Etat-Pays} : cette feuille doit fournir les moyennes des ages de chaque état par pays. Puisque il y a des patients où on n'a pas fourni l'age, donc on doit considérer seulement les patients avec un age non vide.
	Ce TCD doit être accompagné par un histogramme. 
	\item \textbf{Sexe-Etat} : cette feuille doit fournir les moyennes des ages et le nombre des patients par sexe et état. Ici, ce n'ai pas la peine de filtrer l'age vide.
\end{enumerate}


\end{document}
